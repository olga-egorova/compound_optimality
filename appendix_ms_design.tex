% Appendix B.  The detailed design search for a multistratum experiment.

Then the optimal multistratum design search is implemented following the steps below:
\begin{enumerate}
\item Starting from the first stratum, if there are any factors applied at this level, a candidate set of treatments is formed together with the fitted model matrix comprising the primary terms and the matrix of potential terms. The optimal unblocked design $X_1$ is then obtained using the usual point-exchange algorithm, by minimising (\ref{eq::MSE_D_ms}) or (\ref{eq::MSE_L_ms}). The labels of the treatments applied at the current stratum are saved at this stage as well in order to calculate the number of pure error degrees of freedom at the lower strata.

If there are no factors applied at the first stratum, we move to the second one, and conduct the optimal design search for a blocked experiment, where the number of blocks is equal to the number of units in the first stratum $n_1$, with $n_2$ runs per block; in this case criterion function (\ref{eq::MSE_D_B_ms}) or (\ref{eq::MSE_L_B_ms}) is used, and number of pure error degrees of freedom $d_B$ is calculated according to the usual blocked experiment framework. Treatment labels are saved as well. 
 
\item When moving from stratum $i-1$ to stratum $i$, all factors applied in the higher strata are now treated as ``whole-plot'' factors. The corresponding model matrix $\bm{Xw.m}$ containing terms inherited from the higher strata is expanded accordingly, as treatments applied at each unit in stratum $i-1$ are now applied to $n_i$ units of the current stratum nested within it. A similar expansion procedure is carried out for the vector (or matrix, if there are two or more higher strata with factors applied) of the treatment labels, so that for each current unit we are able to see what treatment has been applied to it at each stratum.

Blocking with no factors applied may occur at any stratum, not only at the first one. In such cases the procedure remains the same: skipping to the next stratum with some treatment applied, expanding the design matrices and vectors with treatment labels corresponding to the higher strata.

\item Once there is a model matrix with the ``whole-plot'' terms is formed, the search procedure might be started for the current stratum. The candidate set of treatments is set for the factors applied in the current stratum; parameters of interest include not only the ones formed by these factors but also their interactions with the higher strata terms. As nesting within the previous strata is treated as fixed block effects, the criteria used are the ones given in (\ref{eq::MSE_D_B_ms}) and (\ref{eq::MSE_L_B_ms}). However, there are a few features worth noting:
\begin{itemize}
\item For each design under consideration during the extensive search procedure, its model matrix $\bm{X}_1$ is now constructed by binding the ``whole-plot'' model matrix $\bm{Xw.m}$, the model comprising the terms formed from the factors applied at the current stratum $\bm{Xi.m}$, and the matrix formed of the interaction terms (if any are to be included) between the two. The same relates to the construction of the potential terms matrix $\bm{X}_2$: it needs to be recalculated every time a design point is swapped between the candidate set of the current stratum terms and the current design if it contains any interactions involving terms inherited from the previous strata. If not, it then only comprises terms from the current stratum factors.
\item Presence of the potential terms matrix in the criteria also implies that each stratum $i$ will ``have'' its own number of potential terms $q_i$ and, therefore, in the cases when the value of the variance scaling parameter $\tau^2$ depends on it, at each stratum the criterion function will be evaluated with the respective values of $\tau^2_i$ instead of some common one for all levels. In this work we consider common values of $\tau^2$, however, it is a case-sensitive parameter, and it is to be discussed in each particular case.
\item As the numbers of primary and potential terms vary from stratum to stratum, so do the significance levels $\alpha_{LP}$ and $\alpha_{LoF}$ in the case of trace-based criterion (\ref{eq::MSE_L_B_ms}):
\begin{align*}
\alpha_{LP}&=1-(1-\alpha_1)^{\frac{1}{p}},\\
\alpha_{LoF}&=1-(1-\alpha_2)^{\frac{1}{q}},
\end{align*}
as the corrected confidence levels depend on the dimension of the confidence regions (as in (\ref{eq::Sidak})). 
\item We use the same values of weights in the criteria for all strata; however, the flexibility of the algorithm allows changing weights (and even criteria) between the strata.
\end{itemize}
\item If there are at least $3$ strata with some factors applied, and when the current stratum number is $3$ or more, an additional swapping procedure is performed (the same as that described by \cite{Trinca2001multistratum}). By looking at the $i-2$ stratum units that have the same treatments applied to them, and interchange the $i-1$ stratum units within those, the performance of the design evaluated with respect to the performance at the current stratum $i$. The same swapping is performed for all the higher strata up to the first one.
\item It is all then repeated from step number $2$, until current stratum $i$ reaches the lowest stratum $s$. 
\end{enumerate}