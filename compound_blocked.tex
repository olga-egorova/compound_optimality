
We will start by reviewing the `full' model for blocked experiments, similar to (\ref{eq::blocked_model}):
\begin{align*}
\bm{Y}=\bm{Z\beta}_{B}+\bm{X}_{1}\bm{\beta}_{1}+\bm{X}_{2}\bm{\beta}_{2}+\bm{\varepsilon}.
\end{align*}
Denote the $n\times(b+p)$ model matrix of the block and primary terms by $\tilde{\bm{X}}_{1}=[\bm{Z},\bm{X}_{1}]$, where the columns of $\bm{Z}$ contain block indicators, and the columns of $\bm{X}_{1}$ correspond to primary terms. Let also $\bm{\tilde{\beta}}_1=[\bm{\beta}_{B},\bm{\beta}_{1}]'$ be the joint vector of fixed block effects and primary model terms, and by $\bm{\hat{\tilde{\beta}}}_1$ we denote the vector of the corresponding estimates. It is worth noting that the number of primary terms $p$ does not include the intercept, as it is aliased with the block effects.

%The $DP$-, $LP$- and $LoF(DP)$- and $LoF(LP)$-components are derived as was shown in Section \ref{sec::gen_blocked}.
Then the overall mean square error matrix is
\begin{align}
\label{eq::MSE_b}
\mbox{MSE}(\bm{\hat{\tilde{\beta}}}_1|\bm{\tilde{\beta}})=&\mathtt{E}_{\bm{Y}|\bm{\beta}}[(\bm{\hat{\tilde{\beta}}}_1-\bm{\tilde{\beta}}_1)(\bm{\hat{\tilde{\beta}}}_1-\bm{\tilde{\beta}}_1)'] \notag\\=&\sigma^2(\bm{\tilde{X}}_1^{'}\bm{\tilde{X}}_1)^{-1}+\bm{\tilde{A}}\bm{\beta}_2\bm{\beta}_2'\bm{\tilde{A}}', 
\end{align}
where $\bm{\tilde{A}}=(\bm{\tilde{X}}_1^{'}\bm{\tilde{X}}_1)^{-1}\bm{\tilde{X}}_1^{'}\bm{X}_2$ is the alias matrix in the case of a blocked experiment.

Now consider the partition of the MSE matrix with respect to block and primary effects:
\begin{align}
\label{eq::mse_b_matrix}
&\mathtt{E}_{\bm{Y}|\bm{\beta}}[(\bm{\hat{\tilde{\beta}}}_1-\bm{\tilde{\beta}}_1)(\bm{\hat{\tilde{\beta}}}_1-\bm{\tilde{\beta}}_1)']= \notag\\ &\mathtt{E}_{\bm{Y}|\bm{\beta}}\{[\hat{\tilde{\beta}}_{11}-\tilde{\beta}_{11},\ldots,
\hat{\tilde{\beta}}_{1b}-\tilde{\beta}_{1b}, \hat{\tilde{\beta}}_{1 b+1}-\tilde{\beta}_{1 b+1},\ldots, \hat{\tilde{\beta}}_{1 b+p}-\tilde{\beta}_{1 b+p}]\times \notag\\ &[\hat{\tilde{\beta}}_{11}-\tilde{\beta}_{11},\ldots,
\hat{\tilde{\beta}}_{1b}-\tilde{\beta}_{1b}, \hat{\tilde{\beta}}_{1 b+1}-\tilde{\beta}_{1 b+1},\ldots, \hat{\tilde{\beta}}_{1 b+p}-\tilde{\beta}_{1 b+p}]'\}=\notag\\
&\mathtt{E}_{\bm{Y}|\bm{\beta}}\{[\bm{\hat{\beta}}_B-\bm{\beta}_B, \bm{\hat{\beta}}_1-\bm{\beta}_1][\bm{\hat{\beta}}_B-\bm{\beta}_B, \bm{\hat{\beta}}_1-\bm{\beta}_1]'\}=\notag\\
&\begin{bmatrix}
\mathtt{E}_{\bm{Y}|\bm{\beta}}(\bm{\hat{\beta}}_B-\bm{\beta}_B)(\bm{\hat{\beta}}_B-\bm{\beta}_B)' & \mathtt{E}_{\bm{Y}|\bm{\beta}}(\bm{\hat{\beta}}_B-\bm{\beta}_B)(\bm{\hat{\beta}}_1-\bm{\beta}_1)'\\
\mathtt{E}_{\bm{Y}|\bm{\beta}}(\bm{\hat{\beta}}_1-\bm{\beta}_1)(\bm{\hat{\beta}}_B-\bm{\beta}_B)' & \mathtt{E}_{\bm{Y}|\bm{\beta}}(\bm{\hat{\beta}}_1-\bm{\beta}_1)(\bm{\hat{\beta}}_1-\bm{\beta}_1)'
\end{bmatrix}.
\end{align}

The $[2,2]$-submatrix (bottom-right) represents the entity we are interested in -- the bias of the estimators of the primary terms. Denote it by $\mbox{MSE}(\bm{\hat{\tilde{\beta}}}_1|\bm{\tilde{\beta}})_{22}$. Then the corresponding submatrix of the first summand in (\ref{eq::MSE_b}) is
\begin{equation*}
[\sigma^2(\bm{\tilde{X}}_1^{'}\bm{\tilde{X}}_1)^{-1}]_{22}=\sigma^2(\bm{X}'_{1}
\bm{QX}_{1})^{-1}, \mbox { where } \bm{Q}=\bm{I}-\bm{Z}(\bm{Z}'\bm{Z})^{-1}\bm{Z}'. 
\end{equation*}

Using the matrix inversion rule for block matrices \citep{Harville2006matrix}:
\begin{align*}
\begin{bmatrix}
\bm{A} & \bm{B}\\
\bm{C} & \bm{D}
\end{bmatrix}
^{-1}=
\begin{bmatrix}
(\bm{A}-\bm{BD}^{-1}\bm{C})^{-1} & -(\bm{A}-\bm{BD}^{-1}\bm{C})^{-1}\bm{BD}^{-1} \\
-\bm{D}^{-1}\bm{C}(\bm{A}-\bm{BD}^{-1}\bm{C})^{-1} & \bm{D}^{-1}+\bm{D}^{-1}\bm{C}(\bm{A}-\bm{BD}^{-1}\bm{C})^{-1}\bm{BD}^{-1}
\end{bmatrix},
\end{align*}

under the conditions of invertability of $\bm{A}$, $\bm{D}$ and $\bm{A}-\bm{BD}^{-1}\bm{C}$, we now consider $\bm{\tilde{A}}$:
\begin{align*}
\bm{\tilde{A}}=&\left(
\begin{bmatrix}
\bm{Z}'\\
\bm{X}'_{1}
\end{bmatrix}
\begin{bmatrix}
\bm{Z} & \bm{X}_{1}
\end{bmatrix}\right)^{-1}
\begin{bmatrix}
\bm{Z}'\\
\bm{X}'_{1}
\end{bmatrix}\bm{X}_{2}=
\begin{pmatrix}
\bm{Z}'\bm{Z} & \bm{Z}'\bm{X}_{1}\\
\bm{X}'_{1}\bm{Z} & \bm{X}'_{1}\bm{X}_{1}
\end{pmatrix}^{-1}
\begin{bmatrix}
\bm{Z}'\\
\bm{X}'_{1}
\end{bmatrix}\bm{X}_{2}=
\\ &
\begin{bmatrix}
(\bm{Z}'\bm{PZ})^{-1} & -(\bm{Z}'\bm{PZ})^{-1}\bm{Z}'\bm{X}_{1}(\bm{X}'_{1}\bm{X}_{1})^{-1} \\
-(\bm{X}'_{1}\bm{QX}_{1})^{-1}\bm{X}'_{1}\bm{Z}(\bm{Z}'\bm{Z})^{-1} & (\bm{X}'_{1}\bm{QX}_{1})^{-1}
\end{bmatrix}
\begin{bmatrix}
\bm{Z}'\\
\bm{X}'_{1}
\end{bmatrix}\bm{X}_{2}=\\ &
\begin{bmatrix}
(\bm{Z}'\bm{PZ})^{-1}\bm{Z}'\bm{PX}_{2}\\
(\bm{X}'_{1}\bm{QX}_{1})^{-1}\bm{X}'_{1}\bm{QX}_{2}
\end{bmatrix}, 
\end{align*}
where $\bm{P}=\bm{I}-\bm{X}_{1}(\bm{X}'_{1}\bm{X}_{1})^{-1}\bm{X}'_{1}$ is symmetric.

Here $\bm{ZZ}'$, $\bm{X}'_{1}\bm{X}_{1}$ and $\bm{Z}'\bm{PZ}$ are all invertible and, therefore, the operations are legitimate. 

Now consider the second summand in (\ref{eq::MSE_b}):
\begin{align*}
&\bm{\tilde{A}}\bm{\beta}_2\bm{\beta}_2'\bm{\tilde{A}}'=
\begin{bmatrix}
(\bm{Z}'\bm{PZ})^{-1}\bm{Z}'\bm{PX}_{2}\bm{\beta}_2 \\
(\bm{X}'_{1}\bm{QX}_{1})^{-1}\bm{X}'_{1}\bm{QX}_{2}\bm{\beta}_2
\end{bmatrix}
\begin{bmatrix}
\bm{\beta}'_2\bm{X}'_{2}\bm{PZ}(\bm{Z}'\bm{PZ})^{-1} & \bm{\beta}'_2\bm{X}'_{2}\bm{QX}_{1}(\bm{X}'_{1}\bm{QX}_{1})^{-1}
\end{bmatrix}=\\
&\begin{bmatrix}
(\bm{Z}'\bm{PZ})^{-1}\bm{Z}'\bm{PX}_{2}\bm{\beta}_2\bm{\beta}'_2\bm{X}'_{2}\bm{PZ}(\bm{Z}'\bm{PZ})^{-1} & (\bm{Z}'\bm{PZ})^{-1}\bm{Z}'\bm{PX}_{2}\bm{\beta}_2\bm{\beta}'_2\bm{X}'_{2}\bm{QX}_{1}(\bm{X}'_{1}\bm{QX}_{1})^{-1} \\
(\bm{X}'_{1}\bm{QX}_{1})^{-1}\bm{X}'_{1}\bm{QX}_{2}\bm{\beta}_2\bm{\beta}'_2\bm{X}'_{2}\bm{PZ}(\bm{Z}'\bm{PZ})^{-1} & (\bm{X}'_{1}\bm{QX}_{1})^{-1}\bm{X}'_{1}\bm{QX}_{2}\bm{\beta}_2\bm{\beta}'_2\bm{X}'_{2}\bm{QX}_{1}(\bm{X}'_{1}\bm{QX}_{1})^{-1}
\end{bmatrix}.
\end{align*}

Therefore, the submatrix of (\ref{eq::MSE_b}) corresponding to the primary terms is
\begin{align}
\label{eq::mesb_submatrix}
\mbox{MSE}(\bm{\hat{\tilde{\beta}}}_1|\bm{\tilde{\beta}})_{22}=&\sigma^2(\bm{X}'_{1}\bm{QX}_{1})^{-1}+(\bm{X}'_{1}\bm{QX}_{1})^{-1}\bm{X}'_{1}\bm{QX}_{2}\bm{\beta}_2\bm{\beta}'_2\bm{X}'_{2}\bm{QX}_{1}(\bm{X}'_{1}\bm{QX}_{1})^{-1}=\notag\\& \sigma^2\bm{\tilde{M}}^{-1}+\bm{\tilde{M}}^{-1}\bm{X}'_{1}\bm{QX}_{2}\bm{\beta}_2\bm{\beta}'_2\bm{X}'_{2}\bm{QX}_{1}\bm{\tilde{M}},^{-1}
\end{align}
where $\bm{\tilde{M}}=\bm{X}'_{1}\bm{QX}_{1}.$
\subsection{Determinant-based criterion}
As in the unblocked case (\ref{eq::MSE_det}), we construct the determinant-based criterion function to be minimised as the expectation of the logarithm of the determinant of the submatrix of the MSE matrix which was outlined above. First we see how its determinant looks:
\begin{align}
\label{eq::MSE_B_det}
\det[\mbox{MSE}(\bm{\hat{\tilde{\beta}}}_1|\bm{\tilde{\beta}})_{22}]=&\det[\sigma^2\bm{\tilde{M}}^{-1}+\bm{\tilde{M}}^{-1}\bm{X}'_{1}\bm{QX}_{2}\bm{\beta}_2\bm{\beta}'_2\bm{X}'_{2}\bm{QX}_{1}\bm{\tilde{M}}^{-1}]=\notag\\& \sigma^{2p}\det[\bm{\tilde{M}}^{-1}+\bm{\tilde{M}}^{-1}\bm{X}'_{1}\bm{QX}_{2}\bm{\tilde{\beta}}_2\bm{\tilde{\beta}}'_2\bm{X}'_{2}\bm{QX}_{1}\bm{\tilde{M}}^{-1}]=\notag\\& \sigma^{2p}\det[\bm{\tilde{M}}^{-1}](1+\bm{\tilde{\beta}}'_2\bm{X}'_{2}\bm{QX}'_{1}\bm{\tilde{M}}^{-1}\bm{X}'_{1}\bm{QX}_{2}\bm{\tilde{\beta}}_2).
\end{align}

The expression in (\ref{eq::MSE_B_det}) is similar to the one we have in the unblocked case (\ref{eq::mse_det_dec}):
\begin{equation*}
\sigma^{2p}\det[\bm{M}^{-1}](1+\bm{\tilde{\beta}}_2'\bm{X}_2^{'}\bm{X}_1\bm{M}^{-1}\bm{X}_1^{'}\bm{X}_2\bm{\tilde{\beta}}_2).
\end{equation*}

The only two differences are the amended, ``blocked'' information matrix $\bm{\tilde{M}}$ for the primary terms, and the presence of matrix $\bm{Q}$ which indicates the inclusion of the blocked effects in the fitted model. The $q$-dimensional random vector $\bm{\tilde{\beta}}_2$, as before, follows $\mathcal{N}(\bm{0},\tau^{2}\bm{I}_{q})$, so that this prior does not depend on the error variance $\sigma_{.}^2$.

Taking the expectation of the logarithm of (\ref{eq::MSE_B_det}) over the set prior distribution is completely identical to the derivations shown on page \pageref{eq::mse_det_dec}; and the $MSE$-component in the determinant criterion is
\begin{equation}
\label{eq::mse_b_component}
\log(\det[\bm{\tilde{M}}^{-1}])+\mathtt{E}_{\bm{\tilde{\beta}}_2}\log(1+\bm{\tilde{\beta}}_2'\bm{X}_2^{'}\bm{QX}_1\bm{\tilde{M}}^{-1}\bm{X}_1^{'}\bm{QX}_2\bm{\tilde{\beta}}_2).
\end{equation}

The estimation of the second part of the expression above will be carried out using Monte Carlo sampling, as in the unblocked case. Therefore, the resulting determinant-based compound criterion for a blocked experiments (with the $DP$- and $LoF(DP)$-optimality as the first two components) is
\begin{align}
\label{eq::MSE_D_B}
\mbox{minimise }&\left[\left|(\bm{X}'_{1}\bm{Q}\bm{X}_{1})^{-1}\right|^{1/p}F_{p,d_B;1-\alpha_{DP}}\right]^{\kappa_{DP}} \times \notag \\ &\left[\left|\bm{\tilde{L}}+\frac{\bm{I}_{q}}{\tau^{2}}\right|^{-1/q}F_{q,d_B;1-\alpha_{LoF}}\right]^{\kappa_{LoF}}\times \notag\\ & \left[|\bm{X}'_{1}\bm{QX}_{1}|^{-1}\exp\left(\frac{1}{N}\sum_{i=1}^{N}\log(1+\bm{\tilde{\beta}}_{2i}'\bm{X}_2^{'}\bm{QX}_1\bm{\tilde{M}}^{-1}\bm{X}_1^{'}\bm{QX}_2\bm{\tilde{\beta}}_{2i})\right)\right]_{.}^{\kappa_{MSE}/p}
\end{align}
We also consider a point prior estimate of the MSE-component as was done in Section \ref{sec::mse_point}: the vector of potential parameters $\bm{\beta}_2$ will be set to $\sigma\tau\bm{1}_q$, so that the scaled version of it incorporated in (\ref{eq::MSE_D_B}) $\bm{\tilde{\beta}}_2$ is equal to $\tau\bm{1}_q$. Then the determinant-based criterion function with the point prior estimate, which is referred to as ``MSE(D)P'', is  
\begin{align}
\label{eq::MSE_DP_B}
\mbox{minimise }&\left[\left|(\bm{X}'_{1}\bm{Q}\bm{X}_{1})^{-1}\right|^{1/p}F_{p,d_B;1-\alpha_{DP}}\right]^{\kappa_{DP}} \times \notag \\ &\left[\left|\bm{\tilde{L}}+\frac{\bm{I}_{q}}{\tau^{2}}\right|^{-1/q}F_{q,d_B;1-\alpha_{LoF}}\right]^{\kappa_{LoF}}\times \notag\\ & \left[|\bm{X}'_{1}\bm{QX}_{1}|^{-1}\left(1+\tau^2\sum_{i,j=1}^{q}[\bm{X}_2^{'}\bm{QX}_1\bm{\tilde{M}}^{-1}\bm{X}_1^{'}\bm{QX}_2]_{(i,j)}\right)\right]_{.}^{\kappa_{MSE}/p}
\end{align}
 
When constructing the trace-based criterion, we take the expectation of the trace of the submatrix of the mean square error matrix, corresponding to the primary terms (\ref{eq::mesb_submatrix}):
\begin{align}
\label{eq::MSE_B_tr}
\mathtt{E}_{\beta_2}\mbox{trace}[\mbox{MSE}(\bm{\hat{\tilde{\beta}}}_1|\bm{\tilde{\beta}})_{22}]=& \mbox{trace}[\mathtt{E}_{\beta_2}\mbox{MSE}(\bm{\hat{\tilde{\beta}}}_1|\bm{\tilde{\beta}})_{22}]=\notag\\&\mbox{trace}[\sigma^2\bm{\tilde{M}}^{-1}_{22}+\mathtt{E}_{\beta_2}(\bm{\tilde{A}}\bm{\beta}_2\bm{\beta}_2'\bm{\tilde{A}})_{22}]=\notag \\& \sigma^2\mbox{trace}[\bm{\tilde{M}}^{-1}_{22}+\tau^2\{\bm{\tilde{A}}\bm{\tilde{A}}'\}_{22}]=\notag \\&\sigma^2[\mbox{trace}(\bm{X}'_{1}\bm{QX}_{1})^{-1}+\tau^2\mbox{trace}\{\bm{\tilde{A}}\bm{\tilde{A}}'\}_{22}],
\end{align}

where, as in the previous section, $\bm{\tilde{M}}=\bm{X}'_{1}\bm{QX}_{1}$ and $\bm{\tilde{A}}=(\bm{\tilde{X}}_1^{'}\bm{\tilde{X}}_1)^{-1}\bm{\tilde{X}}_1^{'}\bm{X}_2$. 

Only the amended forms of the information and alias matrices make the expression above different from the similar one in the ``unblocked'' case:
\begin{equation*}
\sigma^2[\mbox{trace}\{(\bm{X}_1^{'}\bm{X}_1)^{-1}\}+\tau^2\mbox{trace}\bm{A}\bm{A}'].
\end{equation*}

So the final trace-based compound criterion function for a blocked experiment is
\begin{align}
\label{eq::MSE_L_B}
\mbox{minimise }&\left[\frac{1}{p}\mbox{trace}(\bm{WX}'_{1}\bm{Q}\bm{X}_{1})^{-1}F_{1,d_B;1-\alpha_{LP}}\right]^{\kappa_{LP}}\times \notag \\&\left[\frac{1}{q}\mbox{trace}\left(\bm{\tilde{L}}+\frac{\bm{I}_{q}}{\tau^{2}}\right)^{-1}F_{1,d_B;1-\alpha_{LoF}}\right]^{\kappa_{LoF}}\times \notag \\& \left[\frac{1}{p}\mbox{trace}\{(\bm{X}'_{1}\bm{QX}_{1})^{-1}+\tau^2[\bm{\tilde{A}}\bm{\tilde{A}}']_{22}\}\right]^{\kappa_{MSE}}_{.}
\end{align}

In both criteria $d_B$ stands for the number of pure error degrees of freedom in the case of units gathered into equal-sized blocks. It can be calculated as $d_B=n-\mbox{rank}(\bm{Z}:\bm{T})$ (as in Section \ref{sec::back_blocked}), where $\bm{T}$ is the matrix containing treatment indicators for each of the experimental runs. In other words, the total number of replicates needs to be adjusted for the number of comparisons between blocks that are to be taken into account.