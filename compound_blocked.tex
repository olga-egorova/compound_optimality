
In some experiments, where the number of runs is relatively large, and/or the variability between units is quite high, experimental units are allocated in blocks such that within each block the units are similar in some way, e.g.~the runs carried out within the same day. Such formalisation contributes to 
controlling the variability by separating variation coming from the difference between blocks and the variability arising from within the blocks \citep{Bailey2008design}.

Under the assumption of additivity of $b$ fixed block effects the polynomial model can be presented as:
\begin{equation}
\label{eq::blocked_model}
\bm{Y}=\bm{Z\beta}_B+\bm{X\beta}+\bm{\varepsilon},
\end{equation}
which, in addition to the same terms as in (\ref{eq::primary_model}) comprises $\bm{Z}$ -- the $n\times b$ matrix, such that its $(i,j)^{th}$ element is equal to $1$ if unit $i$ is in block $j$ and to $0$ otherwise, and $\bm{\beta}_B$ is the vector of block effects. 

The full information matrix has the form
$%\begin{equation*}
\bm{M_B}=
\begin{pmatrix}
\bm{Z}'\bm{Z} & \bm{Z}'\bm{X}\\
\bm{X}'\bm{Z} & \bm{X}'\bm{X}
\end{pmatrix}.     
$%\end{equation*} 

Using the rules of inverting blocked matrices (\citealp{Harville2006matrix}), we can isolate the variance of the polynomial coefficients' estimates:
\begin{align*}
\mbox{Var}(\hat{\bm{\beta}})&=\sigma^2(\bm{M_B}^{-1})_{22}=\sigma^2(\bm{X}'\bm{QX}),^{-1}\\
\mbox{where }&\bm{Q}=\bm{I}-\bm{Z}(\bm{Z}'\bm{Z})^{-1}\bm{Z}'.
\end{align*}
The $DP$- and $LP$-criteria become $DP_S$ and $LP_S$ in the context of a blocked experiment, and they can be straightforwardly defined as
\begin{align}
\label{eq::DPs_blocked}
DP_S: &(F_{p,d_B;1-\alpha_{DP}})^{p}\vert (\bm{X}'\bm{Q}\bm{X})^{-1}\vert \rightarrow \mbox{min,}\\
\label{eq::LPs_blocked}
LP_S: &F_{1,d_B;1-\alpha_{LP}}\mbox{trace}\{\bm{W}(\bm{X}'\bm{Q}\bm{X})^{-1}\} \rightarrow \mbox{min.}.
\end{align}

The number of pure error degrees of freedom is now calculated as $d_B=n-\mbox{rank}[\bm{Z}:\bm{T}]$, where $\bm{T}$ is the $n\times t$ matrix whose elements indicate the treatments (\citealp{GilmourTrinca2012}), providing the number of replications after subtracting the ones taken for the estimation of block contrasts.  

To adapt the derivation of the lack-of-fit and MSE-based criteria to the blocked experiments, we start by formulating the model comprising both primary terms and possible contamination in the form of potential terms, now for blocked experiments: 
\begin{align*}
\bm{Y}=\bm{Z\beta}_{B}+\bm{X}_{p}\bm{\beta}_{p}+\bm{X}_{q}\bm{\beta}_{q}+\bm{\varepsilon}.
\end{align*}
Denote the $n\times(b+p)$ model matrix of the block and primary terms by $\tilde{\bm{X}}_{p}=[\bm{Z},\bm{X}_{p}]$ and let $\bm{\tilde{\beta}}_p=[\bm{\beta}_{B},\bm{\beta}_{p}]'$ be the joint vector of fixed block effects and primary model terms, and by $\bm{\hat{\tilde{\beta}}}_p$ we denote the vector of the corresponding estimates. It is worth noting that the number of primary terms $p$ does not include the intercept, as it is aliased with the block effects.

\subsection{Lack-of-fit criteria}
The information matrix for the model (\ref{eq::blocked_model}), up to a multiple of $\sigma^2$, is:
\begin{align*}
\bm{M_B}=
\begin{pmatrix}
\bm{Z}^T\bm{Z} & \bm{Z}^T\bm{X}_{p} & \bm{Z}^T\bm{X}_{q}\\
\bm{X}^T_{p}\bm{Z} & \bm{X}^T_{p}\bm{X}_{p} & \bm{X}^T_{p}\bm{X}_{q}\\
\bm{X}^T_{q}\bm{Z} & \bm{X}^T_{q}\bm{X}_{p} & \bm{X}^T_{q}\bm{X}_{q}+\bm{I}_{q}/\tau^2
\end{pmatrix}
=
\begin{pmatrix}
\bm{\tilde{X}}^T_{p}\bm{\tilde{X}}_{p} & \bm{\tilde{X}}^T_{p}\bm{X}_{q}\\
\bm{X}^T_{q}\bm{\tilde{X}}_{p} & \bm{X}^T_{q}\bm{X}_{q}+\bm{I}_{q}/\tau^2
\end{pmatrix}_{.}
\end{align*} 

Assuming the same normal prior on $\bm{\beta}_q$ $\sim \mathcal{N}(\bm{0}, \tau^2\sigma^2\bm{I}_q)$, we can construct the variance-covariance matrix corresponding to the potential terms, which would be the inverse of the lower right submatrix of $\bm{M_B}$: $\bm{\tilde{\Sigma}}_{qq}=\sigma^2[\bm{M}^{-1}_B]_{22}$.
\begin{align*}
\bm{\tilde{\Sigma}}_{qq}&=\sigma^2([\bm{M_B}]_{22}-[\bm{M_B}]_{21}([\bm{M_B}]_{11})^{-1}[\bm{M_B}]_{12})^{-1}\\
&=\sigma^2(\bm{X}^T_{q}\bm{X}_{q}+\bm{I}_{q}/\tau^2-\bm{X}^T_{q}\bm{\tilde{X}}_{p}(\bm{\tilde{X}}^T_{p}\bm{\tilde{X}}_{p})^{-1}\bm{\tilde{X}}^T_{p}\bm{X}_{q})^{-1}\\&=\sigma^2\left(\bm{\tilde{L}}+\frac{\bm{I}_{q}}{\tau^2}\right), \mbox{ where }\bm{\tilde{L}}=\bm{X}^T_{q}\bm{X}_{q}-\bm{X}^T_{q}\bm{\tilde{X}}_{p}(\bm{\tilde{X}}^T_{p}\bm{\tilde{X}}_{p})^{-1}\bm{\tilde{X}}^T_{p}\bm{X}_{q}.
\end{align*}

Therefore, the lack-of-fit criteria in (\ref{eq::LoFDP_criterion}) and (\ref{eq::LoFLP_criterion}) are adjusted for blocked experiments by replacing the primary terms matrix $\bm{X}_{p}$ by the extended matrix $\bm{\tilde{X}}_{p}$ and the dispersion matrix $\bm{L}$ -- by $\bm{\tilde{L}}$ as obtained above.

\subsection{MSE-based criteria}

As for the MSE-based measure of the shift in the primary terms estimates, we first consider the overall mean square matrix  
\begin{align}
\label{eq::MSE_b}
\mbox{MSE}(\bm{\hat{\tilde{\beta}}}_p|\bm{\tilde{\beta}})=&\mathtt{E}_{\bm{Y}|\bm{\beta}}[(\bm{\hat{\tilde{\beta}}}_p-\bm{\tilde{\beta}}_p)(\bm{\hat{\tilde{\beta}}}_p-\bm{\tilde{\beta}}_p)'] \notag\\=&\sigma^2(\bm{\tilde{X}}_p^{'}\bm{\tilde{X}}_p)^{-1}+\bm{\tilde{A}}\bm{\beta}_q\bm{\beta}_q^T\bm{\tilde{A}}^T, 
\end{align}
with $\bm{\tilde{A}}=(\bm{\tilde{X}}_p^T\bm{\tilde{X}}_p)^{-1}\bm{\tilde{X}}_p^{'T}\bm{X}_q$ being the alias matrix, 
and its partition with respect to block and primary effects:
\begin{align*}
%\label{eq::mse_b_matrix}
&\mathtt{E}_{\bm{Y}|\bm{\beta}}[(\bm{\hat{\tilde{\beta}}}_p-\bm{\tilde{\beta}}_p)(\bm{\hat{\tilde{\beta}}}_p-\bm{\tilde{\beta}}_p)^T]= \notag\\ &\mathtt{E}_{\bm{Y}|\bm{\beta}}\{[\hat{\tilde{\beta}}_{p1}-\tilde{\beta}_{p1},\ldots,
\hat{\tilde{\beta}}_{pb}-\tilde{\beta}_{pb}, \hat{\tilde{\beta}}_{p b+1}-\tilde{\beta}_{p b+1},\ldots, \hat{\tilde{\beta}}_{p b+p}-\tilde{\beta}_{p b+p}]\times \notag\\ &[\hat{\tilde{\beta}}_{p 1}-\tilde{\beta}_{p1},\ldots,
\hat{\tilde{\beta}}_{pb}-\tilde{\beta}_{pb}, \hat{\tilde{\beta}}_{p b+1}-\tilde{\beta}_{p b+1},\ldots, \hat{\tilde{\beta}}_{p b+p}-\tilde{\beta}_{p b+p}]^T\}=\notag\\
&\mathtt{E}_{\bm{Y}|\bm{\beta}}\{[\bm{\hat{\beta}}_B-\bm{\beta}_B, \bm{\hat{\beta}}_p-\bm{\beta}_p][\bm{\hat{\beta}}_B-\bm{\beta}_B, \bm{\hat{\beta}}_p-\bm{\beta}_p]^T\}=\notag\\
&\begin{bmatrix}
\mathtt{E}_{\bm{Y}|\bm{\beta}}(\bm{\hat{\beta}}_B-\bm{\beta}_B)(\bm{\hat{\beta}}_B-\bm{\beta}_B)^T & \mathtt{E}_{\bm{Y}|\bm{\beta}}(\bm{\hat{\beta}}_B-\bm{\beta}_B)(\bm{\hat{\beta}}_p-\bm{\beta}_p)^T\\
\mathtt{E}_{\bm{Y}|\bm{\beta}}(\bm{\hat{\beta}}_p-\bm{\beta}_p)(\bm{\hat{\beta}}_B-\bm{\beta}_B)^T & \mathtt{E}_{\bm{Y}|\bm{\beta}}(\bm{\hat{\beta}}_p-\bm{\beta}_p)(\bm{\hat{\beta}}_p-\bm{\beta}_p)^T
\end{bmatrix}.
\end{align*}

The part  corresponding to the bias of the primary terms $\bm{\beta}_p$ is the lower right $p \times p$ submatrix, and we can extract it from the MSE expression in (\ref{eq::MSE_b}). The respective submatrix of the first summand  is
\begin{equation*}
[\sigma^2(\bm{\tilde{X}}_p^{T}\bm{\tilde{X}}_p)^{-1}]_{22}=\sigma^2(\bm{X}^T_{p}
\bm{QX}_{p})^{-1}, \mbox { where } \bm{Q}=\bm{I}-\bm{Z}(\bm{Z}^T\bm{Z})^{-1}\bm{Z}^T_{.} 
\end{equation*}

Using the matrix inversion rule for block matrices \citep{Harville2006matrix}, we now consider $\bm{\tilde{A}}$:
\begin{align*}
\bm{\tilde{A}}=&\left(
\begin{bmatrix}
\bm{Z}^T\\
\bm{X}^T_{p}
\end{bmatrix}
\begin{bmatrix}
\bm{Z} & \bm{X}_{p}
\end{bmatrix}\right)^{-1}
\begin{bmatrix}
\bm{Z}^T\\
\bm{X}^T_{p}
\end{bmatrix}\bm{X}_{q}=
\begin{pmatrix}
\bm{Z}^T\bm{Z} & \bm{Z}^T\bm{X}_{p}\\
\bm{X}^T_{p}\bm{Z} & \bm{X}^T_{p}\bm{X}_{p}
\end{pmatrix}^{-1}
\begin{bmatrix}
\bm{Z}^T\\
\bm{X}^T_{p}
\end{bmatrix}\bm{X}_{q}=
\\ &
\begin{bmatrix}
(\bm{Z}^T\bm{PZ})^{-1} & -(\bm{Z}^T\bm{PZ})^{-1}\bm{Z}^T\bm{X}_{p}(\bm{X}^T_{p}\bm{X}_{p})^{-1} \\
-(\bm{X}^T_{p}\bm{QX}_{1})^{-1}\bm{X}^T_{p}\bm{Z}(\bm{Z}^T\bm{Z})^{-1} & (\bm{X}^T_{1}\bm{QX}_{p})^{-1}
\end{bmatrix}
\begin{bmatrix}
\bm{Z}^T\\
\bm{X}^T_{p}
\end{bmatrix}\bm{X}_{q}=\\ &
\begin{bmatrix}
(\bm{Z}^T\bm{PZ})^{-1}\bm{Z}^T\bm{PX}_{q}\\
(\bm{X}^T_{p}\bm{QX}_{1})^{-1}\bm{X}^T_{p}\bm{QX}_{q}
\end{bmatrix}, 
\end{align*}
where $\bm{P}=\bm{I}-\bm{X}_{p}(\bm{X}^T_{p}\bm{X}_{p})^{-1}\bm{X}^T_{p}$; $\bm{ZZ}'$, $\bm{X}'_{1}\bm{X}_{1}$ and $\bm{Z}'\bm{PZ}$ are all invertible and, therefore, the operations are legitimate. Now consider the second summand in (\ref{eq::MSE_b}):
\begin{align*}
&\bm{\tilde{A}}\bm{\beta}_q\bm{\beta}_q^T\bm{\tilde{A}}^T=
\begin{bmatrix}
(\bm{Z}^T\bm{PZ})^{-1}\bm{Z}^T\bm{PX}_{q}\bm{\beta}_q \\
(\bm{X}^T_{p}\bm{QX}_{p})^{-1}\bm{X}^T_{p}\bm{QX}_{q}\bm{\beta}_q
\end{bmatrix}
\begin{bmatrix}
\bm{\beta}^T_q\bm{X}^T_{2}\bm{PZ}(\bm{Z}^T\bm{PZ})^{-1} & \bm{\beta}^T_q\bm{X}^T_{q}\bm{QX}_{p}(\bm{X}^T_{p}\bm{QX}_{p})^{-1}
\end{bmatrix}=\\
&\begin{bmatrix}
(\bm{Z}^T\bm{PZ})^{-1}\bm{Z}^T\bm{PX}_{q}\bm{\beta}_q\bm{\beta}^T_q\bm{X}^T_{q}\bm{PZ}(\bm{Z}^T\bm{PZ})^{-1} & (\bm{Z}^T\bm{PZ})^{-1}\bm{Z}^T\bm{PX}_{q}\bm{\beta}_q\bm{\beta}^T_q\bm{X}^T_{q}\bm{QX}_{p}(\bm{X}^T_{p}\bm{QX}_{p})^{-1} \\
(\bm{X}^T_{p}\bm{QX}_{p})^{-1}\bm{X}^T_{p}\bm{QX}_{q}\bm{\beta}_q\bm{\beta}^T_q\bm{X}^T_{q}\bm{PZ}(\bm{Z}^T\bm{PZ})^{-1} & (\bm{X}^T_{p}\bm{QX}_{p})^{-1}\bm{X}^T_{p}\bm{QX}_{q}\bm{\beta}_q\bm{\beta}^T_q\bm{X}^T_{q}\bm{QX}_{p}(\bm{X}^T_{p}\bm{QX}_{p})^{-1}
\end{bmatrix}.
\end{align*}

Then the submatrix of (\ref{eq::MSE_b}) corresponding to the primary terms is
\begin{align}
\label{eq::mesb_submatrix}
\mbox{MSE}(\bm{\hat{\tilde{\beta}}}_p|\bm{\tilde{\beta}})_{pp}=& \sigma^2(\bm{X}^T_{p}\bm{QX}_{p})^{-1}+(\bm{X}^T_{p}\bm{QX}_{p})^{-1}\bm{X}^T_{p}\bm{QX}_{q}\bm{\beta}_q\bm{\beta}^T_q\bm{X}^T_{q}\bm{QX}_{p}(\bm{X}^T_{p}\bm{QX}_{p})^{-1}\notag\\=& \sigma^2\bm{\tilde{M}}^{-1}+\bm{\tilde{M}}^{-1}\bm{X}^T_{p}\bm{QX}_{q}\bm{\beta}_q\bm{\beta}^T_q\bm{X}^T_{q}\bm{QX}_{p}\bm{\tilde{M}},^{-1}
\end{align}
where $\bm{\tilde{M}}=\bm{X}^T_{p}\bm{QX}_{p}.$

As in the unblocked case, we first look at the determinant of the corresponding submatrix (\ref{eq::mesb_submatrix}): 
\begin{align}
\label{eq::MSE_B_det}
\det[\mbox{MSE}(\bm{\hat{\tilde{\beta}}}_p|\bm{\tilde{\beta}})_{pp}]=&\det[\sigma^2\bm{\tilde{M}}^{-1}+\bm{\tilde{M}}^{-1}\bm{X}^T_{p}\bm{QX}_{q}\bm{\beta}_q\bm{\beta}^T_q\bm{X}^T_{q}\bm{QX}_{p}\bm{\tilde{M}}^{-1}]=\notag\\& \sigma^{2p}\det[\bm{\tilde{M}}^{-1}+\bm{\tilde{M}}^{-1}\bm{X}^T_{p}\bm{QX}_{q}\bm{\tilde{\beta}}_q\bm{\tilde{\beta}}^T_q\bm{X}^T_{q}\bm{QX}_{p}\bm{\tilde{M}}^{-1}]=\notag\\& \sigma^{2p}\det[\bm{\tilde{M}}^{-1}](1+\bm{\tilde{\beta}}^T_q\bm{X}^T_{q}\bm{QX}^T_{p}\bm{\tilde{M}}^{-1}\bm{X}^T_{p}\bm{QX}_{q}\bm{\tilde{\beta}}_q).
\end{align}

The $q$-dimensional random vector $\bm{\tilde{\beta}}_q$, as before, follows $\mathcal{N}(\bm{0},\tau^{2}\bm{I}_{q})$, so that this prior does not depend on the error variance $\sigma_{.}^2$. Next, taking the expectation of the logarithm of (\ref{eq::MSE_B_det}) over the set prior distribution is completely identical to the derivations leading to (\ref{eq::MSE_D}). The $MSE(D)$-component then becomes:
\begin{equation}
\label{eq::mse_b_component}
\log(\det[\bm{\tilde{M}}^{-1}])+\mathtt{E}_{\bm{\tilde{\beta}}_2}\log(1+\bm{\tilde{\beta}}_2'\bm{X}_2^{'}\bm{QX}_1\bm{\tilde{M}}^{-1}\bm{X}_1^{'}\bm{QX}_2\bm{\tilde{\beta}}_2)
\end{equation}

and the resulting determinant-based compound criterion for a blocked experiments is
\begin{align}
\label{eq::MSE_D_B}
&\left[\left|(\bm{X}^T_{p}\bm{Q}\bm{X}_{p})^{-1}\right|^{1/p}F_{p,d_B;1-\alpha_{DP}}\right]^{\kappa_{DP}} \times \notag \\ &\left[\left|\bm{\tilde{L}}+\frac{\bm{I}_{q}}{\tau^{2}}\right|^{-1/q}F_{q,d_B;1-\alpha_{LoF}}\right]^{\kappa_{LoF}}\times \\ & \left[|\bm{X}^T_{p}\bm{QX}_{p}|^{-1}\exp\left(\frac{1}{N}\sum_{i=1}^{N}\log(1+\bm{\tilde{\beta}}_{2i}'\bm{X}_q^{T}\bm{QX}_p\bm{\tilde{M}}^{-1}\bm{X}_p^{T}\bm{QX}_q\bm{\tilde{\beta}}_{2i})\right)\right]^{\kappa_{MSE}/p} \longrightarrow \mbox{ min.}\notag
\end{align}

Taking the expectation of the trace of (\ref{eq::mesb_submatrix}) makes up the trace-based component criterion:
\begin{align}
\label{eq::MSE_B_tr}
\mathtt{E}_{\beta_q}\mbox{trace}[\mbox{MSE}(\bm{\hat{\tilde{\beta}}}_p|\bm{\tilde{\beta}})_{pp}]&= \mbox{trace}[\mathtt{E}_{\beta_q}\mbox{MSE}(\bm{\hat{\tilde{\beta}}}_p|\bm{\tilde{\beta}})_{pp}]\notag \\&=\mbox{trace}[\sigma^2\bm{\tilde{M}}^{-1}_{pp}+\mathtt{E}_{\beta_q}(\bm{\tilde{A}}\bm{\beta}_q\bm{\beta}_q^T\bm{\tilde{A}})_{pp}]\notag\\&=  \sigma^2\mbox{trace}[\bm{\tilde{M}}^{-1}_{pp}+\tau^2\{\bm{\tilde{A}}\bm{\tilde{A}}^T\}_{pp}]\notag \\&=\sigma^2[\mbox{trace}(\bm{X}^T_{p}\bm{QX}_{p})^{-1}+\tau^2\mbox{trace}\{\bm{\tilde{A}}\bm{\tilde{A}}^T\}_{pp}].
\end{align}

The $MSE(LP)$-compound criterion for a blocked experiment is
\begin{align}
\label{eq::MSE_L_B}
&\left[\frac{1}{p}\mbox{trace}(\bm{WX}^T_{p}\bm{Q}\bm{X}_{p})^{-1}F_{1,d_B;1-\alpha_{LP}}\right]^{\kappa_{LP}}\times \notag \\&\left[\frac{1}{q}\mbox{trace}\left(\bm{\tilde{L}}+\bm{I}_{q}/\tau^{2}\right)^{-1}F_{1,d_B;1-\alpha_{LoF}}\right]^{\kappa_{LoF}}\times \notag \\& \left[\frac{1}{p}\mbox{trace}\{(\bm{X}^T_{p}\bm{QX}_{p})^{-1}+\tau^2[\bm{\tilde{A}}\bm{\tilde{A}}^T]_{pp}\}\right]^{\kappa_{MSE}} \longrightarrow \mbox{ min.}
\end{align}

The probability levels $\alpha$, weights $\kappa$ hold the same meanings as in the unblocked case; and as was noted before, the number of pure error degrees of freedom $d_B$ accounts for the comparisons between blocks.