Standard design optimality theory is developed under the assumption that the fitted model represents the true relationship of interest and that there is no misspecification. However, it is obviously quite a strong belief and in reality we need to take into account at least the possibility that the model might not provide a good fit to the data. In Section \ref{sec::back_misspecification} a brief overview of various types of model misspecification is presented. 

In this chapter we consider the case when the fitted polynomial model is nested within a larger model that is assumed to provide a better fit for the data. %Being interested in both the precision of the fitted model estimates and lack-of-fit testing in the direction of the bigger model, we start with the %generalised optimality criteria developed by \cite{Goos2005model} and amend them by treating the components from the objective of inference based on %replicates (`pure error'). 
Being interested in
\begin{itemize}
\item the precision of the fitted model estimates,
\item lack-of-fit testing in the direction of the bigger model, and
\item prediction bias,
\end{itemize}
we start with the generalised optimality criteria developed by \cite{Goos2005model} and amend them by treating the components from the objective of inference based on replicates (`pure error'). We then follow through two examples introduced previously and explore the performances and features of the resulting designs and the relationships between the criteria components.

\section{Model-Robust and Model-Sensitive Approach}
In order to get a representation of the relationship of interest between the explanatory factors and the response, a polynomial regression model (\ref{eq::back_model}) is fitted:
\begin{equation}
\label{eq::fitted_model}
\bm{Y}=\bm{X}_1\bm{\beta}_1+\bm{\varepsilon},
\end{equation}
where $\bm{X}_1$ is the $n\times p$ model matrix, each column of which contains primary terms -- powers and products of the explanatory factors and $\bm{\beta}_1$ is the corresponding $p$-dimensional vector of parameters and $\bm{\varepsilon}\sim \mathcal{N}(\bm{0}_{n},\sigma^{2}\bm{I}_{n})$ is the vector of independent identically distributed errors.

Allowing for the anticipated model contamination, we say that an extended (`true', `full') model is essentially the one that describes the true nature of the relationship between the response and the factors of interest and, therefore, might be expected to provide a better fit for the data:
\begin{equation}
\label{eq::full_model}
\bm{Y}=\bm{X}_1\bm{\beta}_1+\bm{X}_2\bm{\beta}_2+\bm{\varepsilon},
\end{equation}
where $\bm{X}_2$ is an $n\times q$ is an extension of the initial model matrix representing those extra $q$ potential terms that might represent the fitted model disturbance. The vector $\bm{\beta}_2$ denotes the corresponding parameters. This larger model has $p+q$ parameters in total and not all of them are estimable when the experiment is relatively small: $n<p+q$; this is the case we consider here.

\subsection{Model-robust approach}
The question of model misspecification was first discussed by \cite{Box1959}. Supposing that the true model consists of both primary and potential terms and only the primary model is fitted, they aimed at developing a search strategy for the designs that would allow precise estimates of the primary terms and some detectability of the potential terms (which was later named the `model-robust' approach). So the mean squared error for the prediction of responses at a factors' settings $\bm{x}_1$ integrated over the design region (IMSE) is considered as a sum of the expected squared bias and the expected prediction variance:
\begin{equation}
\label{eq::IMSE}
\mbox{IMSE}=\textbf{E}_{\mu}\textbf{E}_{\varepsilon}[y(\bm{x})-\textbf{E}_{\varepsilon}[\hat{y}(\bm{x}_1)]]^{2}+\textbf{E}_{\mu}\textbf{E}_{\varepsilon}[\textbf{E}_{\varepsilon}[\hat{y}(\bm{x}_1)]-\hat{y}(\bm{x}_1)]^{2}.  
\end{equation}
Here integration is performed with respect to two measures. The first, $\mu$, is defined on the design space. \cite{Box1959} proposed the uniform measure, that is just averaging over the region of interest (i.e. across all candidate experimental points). The other one is the usual probability measure which is determined by the error distribution and used to calculate expectation in its regular meaning.

By $y(\bm{x})=\bm{x^{'}}_{1}\bm{\beta}_1+\bm{x^{'}}_{2}\bm{\beta}_2+\varepsilon$ the true response value for a certain combination of explanatory factors $\bm{x}'=[\bm{x}^{'}_{1}\mbox{ } \bm{x}^{'}_{2}]$ is denoted, and $\hat{y}(\bm{x}_{1})$ is the fitted response value, according to the first model (\ref{eq::fitted_model}).

For further derivations \citet{Kobilinsky1998} proposed to transform the columns of the extended design matrix $\bm{X}=[\bm{X}_1,\bm{X}_2]$ in terms of the orthogonal polynomials with respect to measure $\mu$ on the design region. If $N$ is the number of candidate points, then each column of the $N\times(p+q)$-matrix of the candidate set of the full model terms is an $N$-dimensional vector, and the orthonormalisation procedure turns it into one of the orthonormal basis vectors of a $(p+q)$-dimensional subspace. This is carried out using the Gram-Schmidt orthonormalisation procedure (\citealp{Meyer2000}). Such a process guarantees the separability of effects, hence, eases the interpretation of the effects of the fitted model, and also allows the use of a simple prior distribution for the potential terms. 

After the orhonormalisation and a sequence of transitions (\citealp{Goos2005model}) the IMSE defined in (\ref{eq::IMSE}) can be presented as:
\begin{equation*}
%\label{IMSE_2}
\mbox{IMSE}=\bm{\beta^{'}}_{2}[\bm{A}'\bm{A}+\bm{I}_{q}]\bm{\beta}_2+\sigma^{2}\mbox{trace}(\bm{X}'_{1}\bm{X}_1)^{-1},
\end{equation*}   
where $\bm{A}=(\bm{X}'_{1}\bm{X}_1)^{-1}\bm{X}'_{1}\bm{X}_2$ is the alias matrix and $\bm{I}_{q}$ is the $q\times q$ identity matrix. It is worth noting that from the derivation point of view the orthonormalisaton is only necessary for the simplification of the first summand in the equation above, i.e. of the prediction bias component.

In order to obtain designs minimising the IMSE value, some information on the potential terms is needed. So it was decided by \citet{Kobilinsky1998} to put a normal prior on the potential terms, with zero mean as they are unlikely to be large and the variance proportional to the error variance $\sigma^2$, i.e. $\bm{\beta}_2\sim \mathcal{N}(\bm{0},\tau^{2}\sigma^{2}\bm{I}_{q})$, and then minimise the expectation of the IMSE with respect to this prior density:
\begin{equation*}
E_{\bm{\beta}}[\mbox{IMSE}]=\tau^{2}\sigma^{2}\mbox{trace}(\bm{A}'\bm{A}+\bm{I}_{q})+\sigma^{2}\mbox{trace}(\bm{X}'_{1}\bm{X}_{1})^{-1}.
\end{equation*}
As potential terms are orthonormalised, this makes it a reasonable assumption that components of $\bm{\beta}_2$ are independent and have equal variances. Choice of the parameter $\tau^{2}$ which determines how large the magnitude of each of the potential terms is assumed to possibly be in comparison to the error (residual) variance is quite arbitrary. So due to the orthogonalisation and scaling \citet{DuMouchel1994} suggested using $\tau^{2}=1$ that would signify the magnitude of the potential effects should not exceed the residual standard error. \citet{Kobilinsky1998} advocated $\tau^{2}=1/q$, so that this variance scaling parameter is inversely proportional to the number of potential terms; and the variation of the overall potential model contamination is comparable with the error disturbance.  

\subsection{Model-sensitive approach}
In contrast to the model-robust approach, the model-sensitive approach aims at making it possible to determine whether there is lack of fit of the fitted model in the direction of the potential terms. Testing for the lack of fit and precise estimation of the primary terms were put together in the criterion developed in \citet{Atkinson2007}:
\begin{equation}
\label{eq::model_sensitive}
\mbox{maximise} \left[ \frac{\alpha}{p}\log|\bm{X}'_{1}\bm{X}_1|+\frac{1-\alpha}{q}\log|\bm{X}'_{2}\bm{X}_2-\bm{X}'_{2}\bm{X}_1(\bm{X}'_{1}\bm{X}_1)^{-1}\bm{X}'_{1}\bm{X}_2|\right].
\end{equation}
The expression in (\ref{eq::model_sensitive}) is the weighted sum of two components: the first stands for $D$-optimality for the primary terms and the second one -- for $D_S$-optimality for the potential terms. Weights are determined by the `belief' parameter $\alpha\in [0,1]$ reflecting the extent of initial certainty in the primary model being true (i.e. $\alpha=1$ indicating that there is no misspecification to be accounted for) and scaled by the number of parameters of each submodel.

Checking whether the extended model provides a better fit for the data is directly related to the non-centrality parameter:
\begin{equation}
\label{eq::delta}
\delta=\frac{\bm{\beta}'_{2}[\bm{X}'_{2}\bm{X}_2-\bm{X}'_{2}\bm{X}_1(\bm{X}'_{1}\bm{X}_1)^{-1}\bm{X}'_{1}\bm{X}_2]\bm{\beta}_2}{\sigma^2}=\frac{\bm{\beta}'_{2}\bm{L}\bm{\beta}_2}{\sigma^2},
\end{equation}
maximising which maximises the power for the lack-of-fit test of the reduced model when the extended model is true. As it also depends on the (inestimable) values of potential terms, \citet{Goos2005model} applied the same idea as with the bias and variance components, i.e. maximise the expectation of the non-centrality parameter over the same prior distribution of $\bm{\beta}_2$:
\begin{equation*}
E_{\bm{\beta}}(\delta)=\tau^{2}\mbox{trace}[\bm{L}].
\end{equation*}
Finally, they combined $A$-optimality for the primary terms, lack-of-fit and bias components in one criterion and extended it to the case of inestimable potential terms (as in \citealp{DuMouchel1994}). The resulting Generalised A-criterion looks as follows:
\begin{equation}
\label{eq::GA}
\mbox{minimise} \left[ \frac{\gamma_{1}}{p}\mbox{trace}(\bm{X}'_{1}\bm{X}_1)^{-1}-\frac{\gamma_{2}}{q}\mbox{trace}\left(\bm{L}+\frac{\bm{I}_{q}}{\tau^{2}}\right)+\frac{\gamma_{3}}{q}\mbox{trace}(\bm{A}'\bm{A}+\bm{I}_{q})\right]_{.}
\end{equation}
In the next section it will be shown in detail, that the matrix $\bm{L}+\bm{I}_{q}/\tau^{2}$ in the second component is essentially the inverse posterior variance-covariance matrix of the vector of potential terms (up to a factor of $\sigma^2$). \\
Similarly, the $GD$-criterion (Generalised $D$-optimality criterion) is given by:
\begin{equation}
\label{eq::GD}
\mbox{minimise} \left[ \frac{\gamma_{1}}{p}\log|(\bm{X}'_{1}\bm{X}_1)^{-1}|+\frac{\gamma_{2}}{q}\log\left|\left(\bm{L}+\frac{\bm{I}_{q}}{\tau^{2}}\right)^{-1}\right|+\frac{\gamma_{3}}{q}\log|\bm{A}'\bm{A}+\bm{I}_{q}|\right]_{.}
\end{equation}
Weights $\gamma_{i}$ ($i=1..3$) are allocated to the different parts depending on which design properties are more or less desirable for a particular experiment; as before, $\gamma_{i} \geq 0$ and $\gamma_1+\gamma_2+\gamma_3=1$. \\
For the sake of consistency, we incorporate $L$-optimality for the primary terms in the GA-criterion (which becomes Generalised L-criterion) and reformulate both criteria in terms of the efficiencies, giving the following expressions to be minimised:
\begin{align}
\label{eq::GD_eff}
GD:\mbox{minimise }&|\bm{X}'_{1}\bm{X}_1|^{-\frac{\kappa_{D}}{p}}\times \left|\bm{L}+\frac{\bm{I}_{q}}{\tau^{2}}\right|^{-\frac{\kappa_{LoF}}{q}} \times |\bm{A}'\bm{A}+\bm{I}_{q}|^{\frac{\kappa_{bias}}{q}};
\end{align} 
\begin{align} 
\label{eq::GL_eff}
GL:\mbox{minimise }&\left[\frac{1}{p}\mbox{trace}\bm{W}(\bm{X}'_{1}\bm{X}_1)^{-1}\right]^{\kappa_{L}}\times \left[\frac{1}{q}\mbox{trace}\left(\bm{L}+\frac{\bm{I}_{q}}{\tau^{2}}\right)^{-1}\right]^{\kappa_{LoF}}\times\notag\\&\left[\frac{1}{q}\mbox{trace}(\bm{A}'\bm{A}+\bm{I}_{q})\right]_{.}^{\kappa_{bias}}
\end{align} 
For further examples we will still be treating the intercept as a nuisance parameter, and by $\bm{X}_1$ denote the matrix of primary terms without the intercept column (for example, as in (\ref{eq::ExtraFD_crit}) and (\ref{eq::ExtraFL_crit}); $\bm{Q}_0$ is as in (\ref{eq::s_infmatrix}). The criteria above can be straightforwardly adapted and so we will be using them in the following forms (for the shortness of notation, the standard subscript $S$ shall be omitted):
\begin{align}
\label{eq::GDs_eff}
GD:\mbox{minimise }&\vert\bm{X}'_1\bm{Q}_{0}\bm{X}_1\vert^{-\frac{\kappa_{D}}{(p-1)}}\times \left|\bm{L}+\frac{\bm{I}_{q}}{\tau^{2}}\right|^{-\frac{\kappa_{LoF}}{q}} \times |\bm{A}'\bm{A}+\bm{I}_{q}|^{\frac{\kappa_{bias}}{q}};
\end{align} 
\begin{align} 
\label{eq::GLs_eff}
GL:\mbox{minimise }&\left[\frac{1}{p-1}\mbox{trace}\{\bm{W}(\bm{X}'_{1}\bm{Q}_{0}\bm{X}_{1})^{-1}\}\right]^{\kappa_{L}}\times \left[\frac{1}{q}\mbox{trace}\left(\bm{L}+\frac{\bm{I}_{q}}{\tau^{2}}\right)^{-1}\right]^{\kappa_{LoF}}\times\notag\\&\left[\frac{1}{q}\mbox{trace}(\bm{A}'\bm{A}+\bm{I}_{q})\right]_{.}^{\kappa_{bias}}
\end{align} 

\subsection{Generalised and compound criteria: example}
We continue considering the setup of an experiment with five explanatory factors, each consisting of three levels, and $40$ runs. The model to be fitted is the second-order polynomial containing the intercept, all quadratic, interaction and linear terms, i.e. $p=21$. The potential terms are all of the third order: the products of three linear terms and products of quadratic and linear (pure cubic terms are not included since the factors have three levels), so that $q=30$. Thus, $n<p+q$, and the extended model cannot be fitted straightforwardly.

We looked at how the designs summarised in Tables \ref{Table_extraFD} and \ref{Table_extraFL} (Section \ref{sec::optimal_extraF}), that are optimal according to different determinant- and trace-based compound criteria, perform in terms of the $GD$ and $GL$-criteria defined above in (\ref{eq::GDs_eff}) and (\ref{eq::GLs_eff}) with all the weight allocated to the LoF component and with equal weights. We considered two values of the variance scaling parameter: $\tau^2=1$ and $\tau^2=1/q$ ($q$ being equal to $30$). The results are presented in Tables \ref{Table::GD-eff} and \ref{Table::GL-eff}.

\begin{table}[h]
\begin{center}
\caption{GD-efficiencies of designs optimal in terms of determinant-based compound criteria}
\label{Table::GD-eff}
\resizebox{\textwidth}{!}} & \multicolumn{2}{c}{$\bm{\tau^2=1}$} & \multicolumn{2}{c}{$\bm{\tau^2=1/q}$} \\
\textbf{} & \textbf{D} & \textbf{DP} & \textbf{DF} & \textbf{LoF} & \textbf{PE} & \textbf{LoF} & \textbf{D} & \textbf{DP} & $\bm{\kappa_{LoF}=1}$ & $\bm{\kappa_{i}=1/3}$ & $\bm{\kappa_{LoF}=1}$ & $\bm{\kappa_{i}=1/3}$ \\
1 & 1 & 0 & 0 & 0 & \multicolumn{1}{|r}{0} & 19 & \multicolumn{1}{|r}{100.00} & 0.00 & \multicolumn{1}{|r}{96.80} & 93.47 & \multicolumn{1}{|r}{99.90} & 94.60 \\
2 & 0 & 1 & 0 & 0 & \multicolumn{1}{|r}{18} & 1 & \multicolumn{1}{|r}{93.00} & 100.00 & \multicolumn{1}{|r}{86.08} & 76.77 & \multicolumn{1}{|r}{99.45} & 80.68 \\
3 & 0.5 & 0.5 & 0 & 0 & \multicolumn{1}{|r}{16} & 3 & \multicolumn{1}{|r}{95.29} & 98.64 & \multicolumn{1}{|r}{87.26} & 79.77 & \multicolumn{1}{|r}{99.50} & 83.46 \\
4 & 0.5 & 0 & 0.5 & 0 & \multicolumn{1}{|r}{0} & 19 & \multicolumn{1}{|r}{100.00} & 0.00 & \multicolumn{1}{|r}{96.66} & 93.23 & \multicolumn{1}{|r}{99.90} & 94.60 \\
5 & 0.5 & 0 & 0 & 0.5 & \multicolumn{1}{|r}{12} & 7 & \multicolumn{1}{|r}{97.59} & 90.38 & \multicolumn{1}{|r}{89.72} & 84.06 & \multicolumn{1}{|r}{99.61} & 87.17 \\
6 & 0 & 0.5 & 0.5 & 0 & \multicolumn{1}{|r}{12} & 7 & \multicolumn{1}{|r}{97.62} & 90.40 & \multicolumn{1}{|r}{89.40} & 83.69 & \multicolumn{1}{|r}{99.59} & 86.89 \\
7 & 0 & 0.5 & 0 & 0.5 & \multicolumn{1}{|r}{14} & 5 & \multicolumn{1}{|r}{96.93} & 95.61 & \multicolumn{1}{|r}{88.32} & 81.72 & \multicolumn{1}{|r}{99.55} & 85.17 \\
8 & 0 & 0 & 0.5 & 0.5 & \multicolumn{1}{|r}{10} & 9 & \multicolumn{1}{|r}{49.77} & 42.27 & \multicolumn{1}{|r}{88.56} & 63.37 & \multicolumn{1}{|r}{99.55} & 65.99 \\
9 & 1/3 & 1/3 & 1/3 & 0 & \multicolumn{1}{|r}{12} & 7 & \multicolumn{1}{|r}{97.60} & 90.38 & \multicolumn{1}{|r}{88.93} & 82.69 & \multicolumn{1}{|r}{99.57} & 85.99 \\
10 & 1/3 & 1/3 & 0 & 1/3 & \multicolumn{1}{|r}{14} & 5 & \multicolumn{1}{|r}{96.93} & 95.61 & \multicolumn{1}{|r}{88.32} & 81.72 & \multicolumn{1}{|r}{99.55} & 85.17 \\
11 & 0 & 1/3 & 1/3 & 1/3 & \multicolumn{1}{|r}{12} & 7 & \multicolumn{1}{|r}{97.60} & 90.38 & \multicolumn{1}{|r}{88.93} & 82.69 & \multicolumn{1}{|r}{99.57} & 85.99 \\
12 & 0.25 & 0.25 & 0.25 & 0.25 & \multicolumn{1}{|r}{12} & 7 & \multicolumn{1}{|r}{97.62} & 89.40 & \multicolumn{1}{|r}{89.40} & 83.69 & \multicolumn{1}{|r}{99.59} & 86.89 
\end{tabular}
}
\end{center}
\end{table}

The most general tendency is that these previously obtained optimal designs perform better in terms of the generalised `lack-of-fit-only' criteria, with $\kappa_{LoF}=1$, and the corresponding efficiencies considerably increase when the value of $\tau^2$ is decreased (up to almost $100\%$ for most of the designs). As for the optimality with respect to all components, with all $\kappa$-s being equal to $1/3$, this increase is smaller, but still quite evident. 

Overall the efficiency values are not too bad, in both determinant- and trace-based cases, even for the almost `random' design $\#8$, where only degrees of freedom matter. $DP-$ and $LP-$ optimal designs perform slightly worse with respect to both generalised criteria considered, however, the deviation is too small cause any concern.

\begin{table}[h]
\begin{center}
\caption{GL-efficiencies of designs optimal in terms of trace-based compound criteria}
\label{Table::GL-eff}
\resizebox{\textwidth}{!}} & \multicolumn{2}{c}{$\bm{\tau^2=1}$} & \multicolumn{2}{c}{$\bm{\tau^2=1/q}$} \\
\textbf{} & \textbf{L} & \textbf{LP} & \textbf{DF} & \textbf{LoF} & \textbf{PE} & \textbf{LoF} & \textbf{L} & \textbf{LP} & $\bm{\kappa_{LoF}=1}$ & $\bm{\kappa_{i}=1/3}$ & $\bm{\kappa_{LoF}=1}$ & $\bm{\kappa_{i}=1/3}$ \\
1  & 1    & 0    & 0    & 0    & \multicolumn{1}{|r}{0}  & 19 & \multicolumn{1}{|r}{100.00} & 0.00   & \multicolumn{1}{|r}{97.50} & 92.46 & \multicolumn{1}{|r}{99.90} & 93.45 \\
2  & 0    & 1    & 0    & 0    & \multicolumn{1}{|r}{18} & 1  & \multicolumn{1}{|r}{88.88}  & 100.00 & \multicolumn{1}{|r}{89.29} & 76.85 & \multicolumn{1}{|r}{99.58} & 79.90 \\
3  & 0.5  & 0.5  & 0    & 0    & \multicolumn{1}{|r}{16} & 3  & \multicolumn{1}{|r}{92.83}  & 99.18  & \multicolumn{1}{|r}{89.97} & 80.65 & \multicolumn{1}{|r}{99.61} & 83.65 \\
4  & 0.5  & 0    & 0.5  & 0    & \multicolumn{1}{|r}{0}  & 19 & \multicolumn{1}{|r}{100.00} & 0.00   & \multicolumn{1}{|r}{97.50} & 92.46 & \multicolumn{1}{|r}{99.90} & 93.45 \\
5  & 0.5  & 0    & 0    & 0.5  & \multicolumn{1}{|r}{12} & 7  & \multicolumn{1}{|r}{91.71}  & 99.99  & \multicolumn{1}{|r}{89.43} & 78.83 & \multicolumn{1}{|r}{99.59} & 81.92 \\
6  & 0    & 0.5  & 0.5  & 0    & \multicolumn{1}{|r}{12} & 7  & \multicolumn{1}{|r}{95.61}  & 93.06  & \multicolumn{1}{|r}{92.95} & 85.32 & \multicolumn{1}{|r}{99.73} & 87.57 \\
7  & 0    & 0.5  & 0    & 0.5  & \multicolumn{1}{|r}{14} & 5  & \multicolumn{1}{|r}{90.11}  & 99.93  & \multicolumn{1}{|r}{89.35} & 77.73 & \multicolumn{1}{|r}{99.59} & 80.80 \\
8  & 0    & 0    & 0.5  & 0.5  & \multicolumn{1}{|r}{10} & 9  & \multicolumn{1}{|r}{25.19}  & 26.26  & \multicolumn{1}{|r}{88.50} & 42.98 & \multicolumn{1}{|r}{99.55} & 44.82 \\
9 & 1/3  & 1/3  & 1/3  & 0    & \multicolumn{1}{|r}{12} & 7  & \multicolumn{1}{|r}{95.35}  & 92.80  & \multicolumn{1}{|r}{92.87} & 84.81 & \multicolumn{1}{|r}{99.72} & 87.07 \\
10 & 1/3  & 1/3  & 0    & 1/3  & \multicolumn{1}{|r}{14} & 5  & \multicolumn{1}{|r}{91.71}  & 99.99  & \multicolumn{1}{|r}{89.43} & 78.83 & \multicolumn{1}{|r}{99.59} & 81.92 \\
11 & 0    & 1/3  & 1/3  & 1/3  & \multicolumn{1}{|r}{12} & 7  & \multicolumn{1}{|r}{92.74}  & 99.09  & \multicolumn{1}{|r}{90.37} & 80.62 & \multicolumn{1}{|r}{99.63} & 83.50 \\
12 & 0.25 & 0.25 & 0.25 & 0.25 & \multicolumn{1}{|r}{12} & 7  & \multicolumn{1}{|r}{93.99}  & 97.98  & \multicolumn{1}{|r}{90.87} & 82.02 & \multicolumn{1}{|r}{99.65} & 84.79 
\end{tabular}
}
\end{center}
\end{table} 

\section{Generalised Determinant- and Trace-based Criteria}
The lack-of-fit components in the generalised criteria (\ref{eq::GD_eff}) (and, hence, in (\ref{eq::GDs_eff}) and (\ref{eq::GLs_eff})) were derived from the expected values of the non-centrality parameter, i.e. point estimates. In this section we treat the components of these criteria from the `pure error' perspective and define criteria that would have the same desirable properties as before, and would be suitable (i.e. provide the most useful designs) when the further data analysis is to be carried out in terms of hypothesis testing and estimating confidence intervals.

\subsection{Generalised DP-criterion}
Recall the $GD$-criterion presented previously in (\ref{eq::GD_eff}):
\begin{equation*}
\mbox{minimise }|\bm{X}'_{1}\bm{X}_1|^{-\frac{\kappa_{D}}{p}}\times \left|\bm{L}+\frac{\bm{I}_{q}}{\tau^{2}}\right|^{-\frac{\kappa_{LoF}}{q}} \times |\bm{A}'\bm{A}+\bm{I}_{q}|.^{\frac{\kappa_{bias}}{q}}
\end{equation*}

First the `pure' estimation of the error variance based on replicates leads to replacing the first component by $DP$-optimality: $\vert(\bm{X}'_{1}\bm{X}_{1})^{-1/p}F_{p,d;1-\alpha_{DP}}\vert$; or $DPs$-optimality which we use throughout this work: $\vert(\bm{X}'_{1}\bm{Q}_{0}\bm{X}_{1})^{-1/(p-1)}F_{p-1,d;1-\alpha_{DP}}\vert$.

Then we amend the second component. A diffuse prior shall be put on primary terms (as was done by \cite{DuMouchel1994}) -- an arbitrary mean and a variance going to infinity, and the prior on potential terms is the one specified before: $\bm{\beta}_2\sim\mathcal{N}(0,\bm{\Sigma}_{0})$,
$\bm{\Sigma}_{0}=\sigma^{2}\tau^{2}\bm{I}_{q}$, then the posterior
distribution of the coefficients is (as in \cite{Koch2007introduction} and \cite{DuMouchel1994}):
$$\bm{\beta}|\bm{Y}\sim \mathcal{N}(\bm{b},\bm{\Sigma}),\mbox{ where }\bm{b}=\bm{\Sigma X}'\bm{Y}, \bm{X}=[\bm{X}_1, \bm{X}_2],$$
$$\bm{\Sigma}=\left[\frac{\bm{K}}{\sigma^{2}\tau^{2}}+\sigma^{-2}(\bm{X'X})\right]^{-1}=\sigma^{2}[\bm{K}/\tau^{2}+\bm{X'}\bm{X}],^{-1}$$
and 
\begin{equation*}
\bm{K}=\begin{pmatrix}
\bm{0}_{p\times p} & \bm{0}_{p\times q}\\
\bm{0}_{q\times p} & \bm{I}_{q\times q}
\end{pmatrix}.
\end{equation*}
The marginal posterior of $\bm{\beta}_2$ is $\mathcal{N}(\bm{b}_{2},\bm{\Sigma}_{22})$, so using the general formula for the inverse of a block matrix
$$\begin{bmatrix}
 \bm{A}& \bm{B}\\
 \bm{C}& \bm{D}
\end{bmatrix}^{-1}=\begin{bmatrix}
\ldots & \ldots\\
\ldots & (\bm{D}-\bm{CA}^{-1}\bm{B})^{-1}
\end{bmatrix},$$
we can obtain the expression for $\bm{\Sigma}_{22}$:
\begin{align*}
[\bm{K}/\tau^{2}+\bm{X}'\bm{X}]^{-1}&=\begin{bmatrix}
 \bm{X}'_1\bm{X}_1& \bm{X}'_1\bm{X}_2 \\
 \bm{X}'_2\bm{X}_1& \bm{X}'_2\bm{X}_2+\bm{I}_{q}/\tau^{2}
\end{bmatrix}^{-1}\\&=\begin{bmatrix}
\ldots & \ldots\\
\ldots &
(\bm{X}'_2\bm{X}_2+\bm{I}_{q}/\tau^{2}-\bm{X}'_2\bm{X}_1(\bm{X}'_1\bm{X}_1)^{-1}\bm{X}'_1\bm{X}_2)^{-1}
\end{bmatrix},\\
\bm{\Sigma}_{22}&=\sigma^{2}[(\bm{K}/\tau^{2}+\bm{X}^{'}\bm{X})^{-1}]_{22}=\sigma^{2}\left(\bm{L}+\frac{\bm{I}_{q}}{\tau^{2}}\right)^{-1},
\end{align*} 
and $$(\bm{\Sigma}_{22})^{-1}=\frac{1}{\sigma^{2}}\left(\bm{L}+\frac{\bm{I}_{q}}{\tau^{2}}\right).$$

Therefore, the $(1-\alpha)\times100\%$ confidence region for the potential terms over the posterior distribution, when $\sigma^2$ is estimated by $s^2$ on $\nu$ degrees of freedom, is given by \citep{Draper1998}:

%$$(\bm{\beta}_{2}-\bm{b}_{2})^{'}(\bm{\Sigma}_{22})^{-1}(\bm{\beta}_{2}-\bm{b}_{2})\leq \chi^{2}(q,\alpha),$$
%where $\chi^{2}(q,\alpha)$ is the $\alpha$-quantile of chi-square distribution with $q$ degrees of freedom.\\ 
%Equivalently,
%$$(\bm{\beta}_{2}-\bm{b}_{2})^{'}(\bm{L}+\bm{I}_{q}/\tau^{2})(\bm{\beta}_{2}-\bm{b}_{2})\leq \sigma^{2}\chi^{2}(q,\alpha).$$
%If we use the estimate $s^2$ of $\sigma^2$ on $\nu$ degrees of freedom ($\nu=d$ in case of pure error estimate), then:
$$(\bm{\beta}_{2}-\bm{b}_{2})^{'}(\bm{L}+\bm{I}_{q}/\tau^{2})(\bm{\beta}_{2}-\bm{b}_{2})\leq qs^{2}F_{q,\nu;1-\alpha},$$
where $\mathrm{F}(q,\nu,\alpha)$ is the $\alpha$-quantile of F-distribution with $q$ and $\nu$ degrees of freedom.\\ 
The volume of this confidence region is proportional to
\begin{equation*}
\sqrt{\vert(\bm{L}+\bm{I}_{q}/\tau^{2})^{-1}\vert}F^{q/2}_{q,\nu;1-\alpha}.
\end{equation*}
Minimising the volume is equivalent to minimising
\begin{equation*}
\vert(\bm{L}+\bm{I}_{q}/\tau^{2})^{-1/q}\vert\times F_{q,\nu;1-\alpha},
\end{equation*}
which is how we define the new lack-of-fit component. Note that when the mean square error estimation is used, $\nu=n-p$ and does not depend on the design and, therefore, this component is reduced to the lack-of-fit component of the $GD$-criterion (\ref{eq::GD_eff}). 

As for the bias component, it remains unchanged, as it does not depend on the way the error variance is being estimated. We also add the standard $D$-component in order to allow for the inferences associated with the point estimates.

So the generalised $DP$-criterion (or $GDP$-criterion) is:
\begin{align*}
%\label{eq::GDP}
\mbox{minimise }&\left[\frac{\kappa_{D}}{p}\log|(\bm{X}'_{1}\bm{X}_{1})^{-1}|+\frac{\kappa_{DP}}{p}\log\left\{|(\bm{X}'_{1}\bm{X}_{1})^{-1}|F^{p}_{p,d;1-\alpha_{DP}}\right\} + \right.\\ &\left.\frac{\kappa_{LoF}}{q}\log\left\{\left|\left(\bm{L}+\frac{\bm{I}_{q}}{\tau^{2}}\right)^{-1}\right|F^{q}_{q,d;1-\alpha_{LoF}}\right\}+\frac{\kappa_{bias}}{q}\log|\bm{A}'\bm{A}+\bm{I}_{q}|\right]_{.} 
\end{align*}

If we present the same criterion as the weighted product of corresponding efficiencies, then we have to minimise:
\begin{align}
\label{eq::GDP_eff}
\vert\bm{X}'_{1}\bm{X}_{1}\vert^{-\frac{\kappa_{D}}{p}}\times & \left[\left|\bm{X}'_{1}\bm{X}_{1}\right|^{-1/p}F_{p,d;1-\alpha_{DP}}\right]^{\kappa_{DP}} \times \notag \\ &\left[\left|\bm{L}+\frac{\bm{I}_{q}}{\tau^{2}}\right|^{-1/q}F_{q,d;1-\alpha_{LoF}}\right]^{\kappa_{LoF}} \times |\bm{A}'\bm{A}+\bm{I}_{q}|^{\frac{\kappa_{bias}}{q}}_{,}
\end{align} 

and when the intercept is a nuisance parameter, the generalised $DPs$ criterion is then:
\begin{align}
\label{eq::GDPs_eff}
\vert\bm{X}'_1\bm{Q}_0\bm{X}_1\vert^{-\frac{\kappa_{D}}{p}}\times & \left[\left|\bm{X}'_1\bm{Q}_0\bm{X}_1\right|^{-1/(p-1)}F_{p-1,d;1-\alpha_{DP}}\right]^{\kappa_{DP}} \times \notag \\ &\left[\left|\bm{L}+\frac{\bm{I}_{q}}{\tau^{2}}\right|^{-1/q}F_{q,d;1-\alpha_{LoF}}\right]^{\kappa_{LoF}} \times |\bm{A}'\bm{A}+\bm{I}_{q}|^{\frac{\kappa_{bias}}{q}},
\end{align} 
where $\bm{X}_1$ is the model matrix not containing the intercept column.

\subsection{Generalised LP-criterion}
As was shown in the previous section, the posterior variance-covariance matrix for the potential terms is 
\begin{equation*}
\bm{\Sigma}_{22}=\sigma^{2}\left(\bm{L}+\frac{\bm{I}_{q}}{\tau^{2}}\right)_{.}^{-1}
\end{equation*}

Following the logic of the trace-based criteria, the mean of the variances of linear functions of $\bm{\beta}_2$ defined by matrix $\bm{J}$ can be calculated as the trace of $\sigma^2\bm{JJ}'\left(\bm{L}+\frac{\bm{I}_{q}}{\tau^{2}}\right)_{,}^{-1}$ scaled by the number of potential terms. The mean of the squared lengths of the $(1-\alpha)\times100\%$ confidence intervals  for these linear functions is proportional to
\begin{equation*}
\frac{1}{q}\mbox{trace}\left[\bm{JJ}'\left(\bm{L}+\frac{\bm{I}_{q}}{\tau^{2}}\right)^{-1}\right]F_{1,d;1-\alpha}.
\end{equation*} 

Henceforth we mainly consider the case when $\bm{J}$ is the identity matrix, that is we work with the analogue of $AP$-optimality. In other words, the lack-of-fit component in the generalised $AP$-criterion  stands for the minimisation of the $\bm{L}_2$-norm of the $q$-dimensional vector of the posterior confidence intervals' lengths for the potential parameters. 

As before, we include both $L$- and $LP$-components for the primary terms, and the bias $GA$-component remains the same. The resulting criterion as a linear combination of traces is:
\begin{align*}
\mbox{minimise }&\left[ \frac{\gamma_{L}}{p}\mbox{trace}(\bm{WX}'_1\bm{X}_1)^{-1}+ \frac{\gamma_{LP}}{p}\mbox{trace}(\bm{WX}'_1\bm{X}_1)^{-1}F_{1,d;1-\alpha_{LP}}-\right. \\& \left.\frac{\gamma_{LoF}}{q}\mbox{trace}\left(\bm{L}+\frac{\bm{I}_{q}}{\tau^{2}}\right)F_{1,d;1-\alpha_{LoF}}+\frac{\gamma_{bias}}{q}\mbox{trace}(\bm{A}'\bm{A}+\bm{I}_{q})\right]_{.}
\end{align*} 

To keep the consistency of introducing the compound criteria as weighted products of the components' efficiencies, we need to minimise:
\begin{multline}
\label{eq::GLP_eff}
\mbox{minimise }\left[\frac{1}{p}\mbox{trace}(\bm{WX}'_1\bm{X}_1)^{-1}\right]^{\kappa_{L}}\times \left[\frac{1}{p}\mbox{trace}(\bm{WX}'_1\bm{X}_1)^{-1}F_{1,d;1-\alpha_{LP}}\right]^{\kappa_{LP}}\times \\ \left[\frac{1}{q}\mbox{trace}\left(\bm{L}+\frac{\bm{I}_{q}}{\tau^{2}}\right)^{-1}F_{1,d;1-\alpha_{LoF}}\right]^{\kappa_{LoF}}\times\left[\frac{1}{q}\mbox{trace}(\bm{A}'\bm{A}+\bm{I}_{q})\right]_{.}^{\kappa_{bias}}
\end{multline}

The generalised $LPs$ criterion, as it will be used further on, is:
\begin{multline}
\label{eq::GLPs_eff}
\mbox{minimise }\left[\frac{1}{p-1}\mbox{trace}\{\bm{W}(\bm{X}'_{1}\bm{Q}_{0}\bm{X}_{1})^{-1}\}\right]^{\kappa_{L}+\kappa_{LP}}\times \left[F_{1,d;1-\alpha_{LP}}\right]^{\kappa_{LP}}\times \\  \left[\frac{1}{q}\mbox{trace}\left(\bm{L}+\frac{\bm{I}_{q}}{\tau^{2}}\right)^{-1}F_{1,d;1-\alpha_{LoF}}\right]^{\kappa_{LoF}}\times\left[\frac{1}{q}\mbox{trace}(\bm{A'}\bm{A}+\bm{I}_{q})\right]_{.}^{\kappa_{bias}}
\end{multline}

Further we will consider examples of two experiments, and for each of them will obtain $GDP_S$- and $GLP_S$-optimal completely randomised designs with various combinations of weights allocated to different components. The points of the resulting design/model matrix are to be chosen from the orthonormalised candidate sets, which makes the bias component as we have it (as was justified by \cite{Goos2005model}). 

\subsection{Examples}
% Examples for Generalised DP- and LP- optimality criteria
In this section we consider several examples of factorial experiments, obtain optimal designs with respect to the generalised criteria developed and investigate their behaviour depending on various weight allocations, values of the scaling parameter $\tau^2$, number of potential terms in the model and the total size of the experiment. The results should provide some empirical evidence of any patterns and features of the relationships between the nature of the criteria, weights and characteristics of the resulting designs. 

\subsubsection{Example 1}
\label{sec::generalised_example}
First we continue exploring the example which was introduced earlier: a factorial experiment, with $5$ factors, each is at three levels. The small number of runs ($40$) allows estimation of the full-second order polynomial model ($p=21$), but we assume that the `true' model contains also all third-order terms (linear-by-linear-by-linear and quadratic-by-linear interactions), $q=30$ of them in total.

We considered several sets of weight allocations to the components of the generalised $D$-, $L$-, $DP$- and $LP$-criteria. We would like to see how the designs obtained differ, how efficient they are in terms of the original single criterion components and what their structure is in the sense of the relationship between the assigned weights and degrees of freedom allocated to the  pure error (i.e. the number of replicates) and lack-of-fit components.

Together with the original framework, we will also consider the same example, but with the value of the scaling parameter being inversely proportional to the number of potential terms: $\tau^2=1/q$. Further in this section we will investigate the case with only quadratic-by-linear potential terms, so that the total number is $q=20$ rather than $30$. 

These designs were obtained using the point exchange algorithm described previously, with $500$ random starts. The computational time was quite reasonable: for each design the search procedure took approximately $2.5$ to $7.5$ hours (i7-3770 CPU, $16.0$GB RAM). The optimal designs themselves are provided in the Supplementary material, together with the corresponding R code files.


%% GD-optimal designs, ex 11
\begin{table}[h]
\caption{Example 1.1. Properties of generalised D- and L-optimal designs. $\tau^2=1$}
\label{tab::GD_ex11}
\resizebox{\textwidth}{!}}                               \\
   & \textbf{D}       & \textbf{LoF(D)}    & \textbf{Bias(D)}   & \textbf{PE}        & \textbf{LoF}        & \textbf{D}   & \textbf{LoF(D)}   & \textbf{Bias(D)}  & \textbf{L}       & \textbf{LoF(L)}   & \textbf{Bias(L)}  \\
1 & 1 & 0 & 0 & \multicolumn{1}{|r}{0} & 19 & \multicolumn{1}{|r}{100.00} & 96.80 & 76.70 & \multicolumn{1}{|r}{98.96} & 96.27 & 71.09 \\
2 & 0 & 1 & 0 & \multicolumn{1}{|r}{0} & 19 & \multicolumn{1}{|r}{95.99} & 100.00 & 94.71 & \multicolumn{1}{|r}{92.97} & 100.00 & 92.32 \\
3 & 0 & 0 & 1 & \multicolumn{1}{|r}{1} & 18 & \multicolumn{1}{|r}{79.90} & 97.27 & 100.00 & \multicolumn{1}{|r}{68.75} & 97.06 & 100.00 \\
4 & 0.5 & 0.5 & 0 & \multicolumn{1}{|r}{0} & 19 & \multicolumn{1}{|r}{99.22} & 97.98 & 78.54 & \multicolumn{1}{|r}{98.46} & 97.64 & 73.10 \\
5 & 0.5 & 0 & 0.5 & \multicolumn{1}{|r}{0} & 19 & \multicolumn{1}{|r}{92.03} & 99.52 & 97.87 & \multicolumn{1}{|r}{85.88} & 99.48 & 97.20 \\
6 & 0 & 0.5 & 0.5 & \multicolumn{1}{|r}{0} & 19 & \multicolumn{1}{|r}{79.41} & 97.73 & 99.22 & \multicolumn{1}{|r}{68.48} & 97.54 & 98.99 \\
7 & 1/3 & 1/3 & 1/3 & \multicolumn{1}{|r}{0} & 19 & \multicolumn{1}{|r}{95.99} & 100.00 & 94.71 & \multicolumn{1}{|r}{92.97} & 100.00 & 92.32 \\
8 & 0.5 & 0.25 & 0.25 & \multicolumn{1}{|r}{0} & 19 & \multicolumn{1}{|r}{95.99} & 100.00 & 94.71 & \multicolumn{1}{|r}{92.97} & 100.00 & 92.32 \\
9 & 0.25 & 0.5 & 0.25 & \multicolumn{1}{|r}{0} & 19 & \multicolumn{1}{|r}{95.11} & 99.86 & 94.33 & \multicolumn{1}{|r}{90.33} & 99.84 & 91.91 \\
   & \multicolumn{3}{l}{\textbf{Criteria}} & \multicolumn{2}{l}{\textbf{DoF}} & \multicolumn{6}{l}{\textbf{Efficiency,\%}}                               \\
   & \textbf{L}       & \textbf{LoF(L)}    & \textbf{Bias(L)}   & \textbf{PE}        & \textbf{LoF}        & \textbf{D}   & \textbf{LoF(D)}   & \textbf{Bias(D)}  & \textbf{L}       & \textbf{LoF(L)}   & \textbf{Bias(L)}  \\
1 & 1 & 0 & 0 & \multicolumn{1}{|r}{0} & 19 & \multicolumn{1}{|r}{99.95} & 91.73 & 73.77 & \multicolumn{1}{|r}{100.00} & 90.66 & 70.01 \\
2 & 0 & 1 & 0 & \multicolumn{1}{|r}{0} & 19 & \multicolumn{1}{|r}{95.99} & 100.00 & 94.71 & \multicolumn{1}{|r}{92.97} & 100.00 & 92.32 \\
3 & 0 & 0 & 1 & \multicolumn{1}{|r}{1} & 18 & \multicolumn{1}{|r}{79.90} & 97.27 & 100.00 & \multicolumn{1}{|r}{68.75} & 97.06 & 100.00 \\
4 & 0.5 & 0.5 & 0 & \multicolumn{1}{|r}{0} & 19 & \multicolumn{1}{|r}{99.29} & 97.49 & 74.07 & \multicolumn{1}{|r}{99.83} & 97.16 & 66.83 \\
5 & 0.5 & 0 & 0.5 & \multicolumn{1}{|r}{0} & 19 & \multicolumn{1}{|r}{95.99} & 100.00 & 94.71 & \multicolumn{1}{|r}{92.97} & 100.00 & 92.32 \\
6 & 0 & 0.5 & 0.5 & \multicolumn{1}{|r}{0} & 19 & \multicolumn{1}{|r}{81.22} & 97.99 & 98.09 & \multicolumn{1}{|r}{69.20} & 97.85 & 97.39 \\
7 & 1/3 & 1/3 & 1/3 & \multicolumn{1}{|r}{0} & 19 & \multicolumn{1}{|r}{95.99} & 100.00 & 94.71 & \multicolumn{1}{|r}{92.97} & 100.00 & 92.32 \\
8 & 0.5 & 0.25 & 0.25 & \multicolumn{1}{|r}{0} & 19 & \multicolumn{1}{|r}{95.99} & 100.00 & 94.71 & \multicolumn{1}{|r}{92.97} & 100.00 & 92.32 \\
9 & 0.25 & 0.5 & 0.25 & \multicolumn{1}{|r}{0} & 19 & \multicolumn{1}{|r}{95.01} & 99.82 & 95.10 & \multicolumn{1}{|r}{91.00} & 99.86 & 92.96
\end{tabular}
}
\end{table}

Tables \ref{tab::GD_ex11} and \ref{tab::GDP_ex11} contain the summaries of the optimal designs: the weight allocations are given in the first columns (`Criteria'), followed by the distribution of the degrees of freedom (`DoF') and, finally, the efficiencies of the designs with respect to the individual determinant- and trace-based component criteria (`Efficiency, $\%$').

Almost all $GD$- and $GL$-optimal designs in Table \ref{tab::GD_ex11} have all $19$ available degrees of freedom allocated to the lack-of-fit and none to the pure error (except for the `bias'-optimal design \#$3$), so they contain no replicate points in contrast to the $GDP$- and $GLP$-optimal designs in Table \ref{tab::GDP_ex11}, where the imbalance tends to occur in favour of the pure error when some weight is put on either of the `pure error' components ($DP$, $LP$, $LoF(DP)$ and $LoF(LP)$), e.g.~designs \#$2$, \#$3$, \#$11$, \#$14$. The imbalance seems to be, however, less extreme in the case of trace-based than in the case of determinant-based optimality.

It is also quite notable that the same designs were obtained as optimal with regards to different weight allocation schemes and even with respect to different criteria, for example, the lack-of-fit optimal design (design \#$2$ in Table \ref{tab::GD_ex11}), which remains optimal if some weight is distributed among all three components: either equally (\#$7$) or with half of it allocated to the `primary terms' component (\#$8$). This particular design performs quite well in terms of all three components (all efficiency values are above $90\%$), and is also optimal with respect to $L-$ and Bias$(L)$- components taken with equal weights, e.g.~when no weight is actually allocated to the lack-of-fit part of the criterion. The design itself is given in Table \ref{tab::Ex11_LoFD_design}; whilst no points are replicated, they seem to be distributed quite uniformly in the design region, though there are no centre points and each factor takes its $0$-level value in only a few runs. Besides that, there are not any particular patterns of interest to be observed.
\begin{table}[h]
\centering
\caption{Example 1. LoF(D)- and LoF(L)-optimal design (\#$2$, \#$7$ and \#$8$)}
\label{tab::Ex11_LoFD_design}
\scalebox{0.8}{
\begin{tabular}{rrrrrr|r|rrrrrr}
1  & -1 & -1 & -1 & -1 & -1 &  & 21 & 0 & 1  & -1 & 1  & -1 \\
2  & -1 & -1 & -1 & -1 & 1  &  & 22 & 0 & 1  & 0  & 0  & 1  \\
3  & -1 & -1 & -1 & 1  & 0  &  & 23 & 0 & 1  & 1  & -1 & -1 \\
4  & -1 & -1 & 0  & 1  & 1  &  & 24 & 1 & -1 & -1 & -1 & 0  \\
5  & -1 & -1 & 1  & -1 & -1 &  & 25 & 1 & -1 & -1 & 0  & 1  \\
6  & -1 & -1 & 1  & 0  & 1  &  & 26 & 1 & -1 & -1 & 1  & -1 \\
7  & -1 & -1 & 1  & 1  & -1 &  & 27 & 1 & -1 & 0  & -1 & 1  \\
8  & -1 & 0  & -1 & 1  & -1 &  & 28 & 1 & -1 & 1  & -1 & -1 \\
9  & -1 & 0  & 0  & -1 & 0  &  & 29 & 1 & -1 & 1  & 1  & -1 \\
10 & -1 & 0  & 1  & 1  & 1  &  & 30 & 1 & -1 & 1  & 1  & 1  \\
11 & -1 & 1  & -1 & -1 & -1 &  & 31 & 1 & 0  & -1 & -1 & -1 \\
12 & -1 & 1  & -1 & -1 & 1  &  & 32 & 1 & 0  & 0  & 1  & 0  \\
13 & -1 & 1  & -1 & 1  & 1  &  & 33 & 1 & 0  & 1  & -1 & 1  \\
14 & -1 & 1  & 0  & 1  & -1 &  & 34 & 1 & 1  & -1 & -1 & 1  \\
15 & -1 & 1  & 1  & -1 & 1  &  & 35 & 1 & 1  & -1 & 0  & -1 \\
16 & -1 & 1  & 1  & 0  & -1 &  & 36 & 1 & 1  & -1 & 1  & 1  \\
17 & -1 & 1  & 1  & 1  & 0  &  & 37 & 1 & 1  & 0  & -1 & -1 \\
18 & 0  & -1 & -1 & 1  & 1  &  & 38 & 1 & 1  & 1  & -1 & 0  \\
19 & 0  & -1 & 0  & 0  & -1 &  & 39 & 1 & 1  & 1  & 1  & -1 \\
20 & 0  & -1 & 1  & -1 & 1  &  & 40 & 1 & 1  & 1  & 1  & 1 
\end{tabular}
}
\end{table}

The design that was found as a Bias$(L)$-optimal design turned out to be Bias$(D)$-optimal as well: it is the only one with $1$ pure-error degree of freedom, which occurred due to the replicated central point. This design is given in Table \ref{tab::Ex11_BiasD_design}, and it can be observed that, in contrast to the lack-of-fit-optimal design, it seems to be more space-filling an the sense that it contains more `not corner' points. 

It is worth noting that in general the bias component seems to be in opposition to the `primary terms' components, in both determinant- and trace-based criteria: the bias-optimal design is the worst in terms of $D-$ and $L-$ efficiencies (less than $80\%$ and $70\%$ respectively) with the minimum among others being consistently above $90\%$, and best designs in terms of $D-$ and $L-$optimality providing lower efficiencies for minimising the prediction bias. Further on, we shall see how this changes when we either consider a smaller set of potential terms or tune the scaling parameter. 

\begin{table}[h]
\centering
\caption{Example 1. Bias(D)- and Bias(L)-optimal design \#$3$}
\label{tab::Ex11_BiasD_design}
\scalebox{0.8}{
\begin{tabular}{rrrrrr|r|rrrrrr}
1  & -1 & -1 & -1 & 0  & 1  &  & 21 & 0 & 0  & 0  & 0  & 0  \\
2  & -1 & -1 & 0  & -1 & 0  &  & 22 & 0 & 0  & 1  & -1 & -1 \\
3  & -1 & -1 & 0  & 1  & 0  &  & 23 & 0 & 0  & 1  & 1  & -1 \\
4  & -1 & -1 & 1  & 0  & -1 &  & 24 & 0 & 1  & -1 & -1 & -1 \\
5  & -1 & 0  & -1 & -1 & -1 &  & 25 & 0 & 1  & -1 & 1  & -1 \\
6  & -1 & 0  & -1 & -1 & 0  &  & 26 & 0 & 1  & 1  & 0  & 1  \\
7  & -1 & 0  & -1 & 1  & -1 &  & 27 & 1 & -1 & -1 & -1 & 0  \\
8  & -1 & 0  & 1  & 0  & 1  &  & 28 & 1 & -1 & -1 & 1  & 0  \\
9  & -1 & 1  & -1 & 0  & 0  &  & 29 & 1 & -1 & 0  & -1 & -1 \\
10 & -1 & 1  & 0  & -1 & 1  &  & 30 & 1 & -1 & 0  & 0  & 1  \\
11 & -1 & 1  & 0  & 0  & -1 &  & 31 & 1 & -1 & 0  & 1  & -1 \\
12 & -1 & 1  & 0  & 1  & 1  &  & 32 & 1 & -1 & 1  & 0  & 0  \\
13 & -1 & 1  & 1  & -1 & 0  &  & 33 & 1 & 0  & -1 & 0  & -1 \\
14 & -1 & 1  & 1  & 1  & 0  &  & 34 & 1 & 0  & 1  & -1 & 1  \\
15 & 0  & -1 & -1 & 0  & -1 &  & 35 & 1 & 0  & 1  & 1  & 0  \\
16 & 0  & -1 & 1  & -1 & 1  &  & 36 & 1 & 0  & 1  & 1  & 1  \\
17 & 0  & -1 & 1  & 1  & 1  &  & 37 & 1 & 1  & -1 & 0  & 1  \\
18 & 0  & 0  & -1 & -1 & 1  &  & 38 & 1 & 1  & 0  & -1 & 0  \\
19 & 0  & 0  & -1 & 1  & 1  &  & 39 & 1 & 1  & 0  & 1  & 0  \\
20 & 0  & 0  & 0  & 0  & 0  &  & 40 & 1 & 1  & 1  & 0  & -1
\end{tabular}
}
\end{table}

\begin{landscape}
%% GDP- and GLP-optimal designs, ex 11
\begin{table}[p]
\caption{Example 1. Properties of generalised DP- and LP-optimal designs. $\tau^2=1$}
\label{tab::GDP_ex11}
\resizebox{\linewidth}{!}} \\
\multicolumn{1}{l}{} & \multicolumn{1}{l}{{\bf D}} & \multicolumn{1}{l}{{\bf DP}} & \multicolumn{1}{l}{{\bf LoF(DP)}} & \multicolumn{1}{l}{{\bf Bias(D)}} & \multicolumn{1}{l}{{\bf PE}} & \multicolumn{1}{l}{{\bf LoF}} & \multicolumn{1}{l}{{\bf D}} & \multicolumn{1}{l}{{\bf DP}} & \multicolumn{1}{l}{{\bf LoF(D)}} & \multicolumn{1}{l}{{\bf LoF(DP)}} & \multicolumn{1}{l}{{\bf Bias(D)}} & \multicolumn{1}{l}{{\bf L}} & \multicolumn{1}{l}{{\bf LP}} & \multicolumn{1}{l}{{\bf LoF(L)}} & \multicolumn{1}{l}{{\bf LoF(LP)}} & \multicolumn{1}{l}{{\bf Bias(L)}} \\
1 & 1 & 0 & 0 & 0 & \multicolumn{1}{|r}{0} & 19 & \multicolumn{1}{|r}{100.00} & 0.00 & 96.80 & 0.00 & 76.70 & \multicolumn{1}{|r}{98.96} & 0.00 & 96.27 & 0.00 & 71.09 \\
2 & 0 & 1 & 0 & 0 & \multicolumn{1}{|r}{18} & 1 & \multicolumn{1}{|r}{93.00} & 100.00 & 86.08 & 97.67 & 51.39 & \multicolumn{1}{|r}{86.47} & 98.52 & 84.94 & 98.04 & 37.87 \\
3 & 0 & 0 & 1 & 0 & \multicolumn{1}{|r}{17} & 2 & \multicolumn{1}{|r}{48.85} & 51.59 & 89.83 & 100.00 & 42.78 & \multicolumn{1}{|r}{9.72} & 10.87 & 90.47 & 98.97 & 7.22 \\
4 & 0 & 0 & 0 & 1 & \multicolumn{1}{|r}{1} & 18 & \multicolumn{1}{|r}{79.90} & 0.00 & 97.27 & 0.00 & 100.00 & \multicolumn{1}{|r}{68.75} & 0.00 & 97.06 & 0.00 & 100.00 \\
5 & 0.5 & 0.5 & 0 & 0 & \multicolumn{1}{|r}{16} & 3 & \multicolumn{1}{|r}{95.29} & 98.64 & 87.26 & 95.06 & 55.49 & \multicolumn{1}{|r}{91.31} & 100.00 & 86.21 & 95.11 & 45.81 \\
6 & 0.5 &  & 0.5 & 0 & \multicolumn{1}{|r}{18} & 1 & \multicolumn{1}{|r}{92.94} & 99.93 & 86.17 & 97.77 & 51.98 & \multicolumn{1}{|r}{87.37} & 99.54 & 85.04 & 98.12 & 37.88 \\
7 & 0.5 & 0 & 0 & 0.5 & \multicolumn{1}{|r}{0} & 19 & \multicolumn{1}{|r}{92.03} & 0.00 & 99.52 & 0.00 & 97.87 & \multicolumn{1}{|r}{85.88} & 0.00 & 99.48 & 0.00 & 97.20 \\
8 & 0 & 0.5 & 0.5 & 0 & \multicolumn{1}{|r}{18} & 1 & \multicolumn{1}{|r}{93.00} & 100.00 & 86.08 & 97.67 & 51.39 &\multicolumn{1}{|r}{86.47} & 98.52 & 84.94 & 98.04 & 37.87 \\
9 & 0 & 0.5 & 0 & 0.5 & \multicolumn{1}{|r}{11} & 8 & \multicolumn{1}{|r}{85.70} & 76.28 & 93.80 & 87.24 & 78.00 & \multicolumn{1}{|r}{74.51} & 69.02 & 93.69 & 84.48 & 68.63 \\
10 & 0 & 0 & 0.5 & 0.5 & \multicolumn{1}{|r}{9} & 10 & \multicolumn{1}{|r}{63.97} & 51.31 & 95.55 & 79.77 & 94.40 & \multicolumn{1}{|r}{32.96} & 26.97 & 95.57 & 75.09 & 92.02 \\
11 & 0.25 & 0.25 & 0.25 & 0.25 & \multicolumn{1}{|r}{15} & 4 & \multicolumn{1}{|r}{93.05} & 94.17 & 88.67 & 94.35 & 61.15 & \multicolumn{1}{|r}{85.26} & 91.15 & 87.84 & 93.95 & 49.62 \\
12 & 1/3 & 1/3 & 1/3 & 0 & \multicolumn{1}{|r}{18} & 1 & \multicolumn{1}{|r}{93.00} & 100.00 & 86.08 & 97.67 & 51.39 & \multicolumn{1}{|r}{86.47} & 98.52 & 84.94 & 98.04 & 37.87 \\
13 & 1/3 & 1/3 & 0 & 1/3 & \multicolumn{1}{|r}{12} & 7 & \multicolumn{1}{|r}{96.43} & 89.30 & 89.25 & 86.52 & 64.26 & \multicolumn{1}{|r}{93.02} & 90.19 & 88.37 & 85.10 & 56.46 \\
14 & 0 & 1/3 & 1/3 & 1/3 & \multicolumn{1}{|r}{15} & 4 & \multicolumn{1}{|r}{89.28} & 90.35 & 88.79 & 94.48 & 64.58 & \multicolumn{1}{|r}{79.84} & 85.36 & 88.05 & 94.02 & 53.77 \\
 &  &  &  &  &  &  &  &  &  &  &  &  &  &  &  &  \\
\multicolumn{1}{l}{} & \multicolumn{4}{l}{{\bf Criteria}} & \multicolumn{2}{l}{{\bf DoF}} & \multicolumn{10}{l}{{\bf Efficiency,\%}} \\
\multicolumn{1}{l}{} & \multicolumn{1}{l}{{\bf L}} & \multicolumn{1}{l}{{\bf LP}} & \multicolumn{1}{l}{{\bf LoF(LP)}} & \multicolumn{1}{l}{{\bf Bias(L)}} & \multicolumn{1}{l}{{\bf PE}} & \multicolumn{1}{l}{{\bf LoF}} & \multicolumn{1}{l}{{\bf D}} & \multicolumn{1}{l}{{\bf DP}} & \multicolumn{1}{l}{{\bf LoF(D)}} & \multicolumn{1}{l}{{\bf LoF(DP)}} & \multicolumn{1}{l}{{\bf Bias(D)}} & \multicolumn{1}{l}{{\bf L}} & \multicolumn{1}{l}{{\bf LP}} & \multicolumn{1}{l}{{\bf LoF(L)}} & \multicolumn{1}{l}{{\bf LoF(LP)}} & \multicolumn{1}{l}{{\bf Bias(L)}} \\
1 & 1 & 0 & 0 & 0 & \multicolumn{1}{|r}{0} & 19 & \multicolumn{1}{|r}{99.95} & 0.00 & 91.73 & 0.00 & 73.77 & \multicolumn{1}{|r}{100.00} & 0.00 & 90.66 & 0.00 & 70.01 \\
2 & 0 & 1 & 0 & 0 & \multicolumn{1}{|r}{16} & 3 & \multicolumn{1}{|r}{95.29} & 98.64 & 87.26 & 95.06 & 55.49 & \multicolumn{1}{|r}{91.31} & 100.00 & 86.21 & 95.11 & 45.81 \\
3 & 0 & 0 & 1 & 0 & \multicolumn{1}{|r}{18} & 1 & \multicolumn{1}{|r}{48.65} & 52.31 & 87.64 & 99.44 & 42.35 & \multicolumn{1}{|r}{17.82} & 20.30 & 87.37 & 100.00 & 11.60 \\
4 & 0 & 0 & 0 & 1 & \multicolumn{1}{|r}{1} & 18 & \multicolumn{1}{|r}{79.90} & 0.00 & 97.27 & 0.00 & 100.00 & \multicolumn{1}{|r}{68.75} & 0.00 & 97.06 & 0.00 & 100.00 \\
5 & 0.5 & 0.5 & 0 & 0 & \multicolumn{1}{|r}{14} & 5 & \multicolumn{1}{|r}{96.30} & 95.00 & 88.80 & 91.98 & 57.53 & \multicolumn{1}{|r}{94.46} & 98.24 & 87.91 & 92.19 & 47.22 \\
6 & 0.5 &  & 0.5 & 0 & \multicolumn{1}{|r}{15} & 4 & \multicolumn{1}{|r}{95.61} & 96.76 & 87.89 & 93.52 & 55.06 & \multicolumn{1}{|r}{92.98} & 99.41 & 86.91 & 94.09 & 44.06 \\
7 & 0.5 & 0 & 0 & 0.5 & \multicolumn{1}{|r}{0} & 19 & \multicolumn{1}{|r}{95.99} & 0.00 & 100.00 & 0.00 & 94.71 & \multicolumn{1}{|r}{92.97} & 0.00 & 100.00 & 0.00 & 92.32 \\
8 & 0 & 0.5 & 0.5 & 0 & \multicolumn{1}{|r}{18} & 1 & \multicolumn{1}{|r}{92.94} & 99.93 & 86.17 & 97.77 & 51.98 & \multicolumn{1}{|r}{87.37} & 99.54 & 85.04 & 98.93 & 37.88 \\
9 & 0 & 0.5 & 0 & 0.5 & \multicolumn{1}{|r}{12} & 7 & \multicolumn{1}{|r}{96.78} & 89.63 & 89.59 & 86.85 & 63.49 & \multicolumn{1}{|r}{93.56} & 90.71 & 88.70 & 86.23 & 56.34 \\
10 & 0 & 0 & 0.5 & 0.5 & \multicolumn{1}{|r}{9} & 10 & \multicolumn{1}{|r}{63.30} & 50.77 & 94.86 & 79.20 & 93.51 & \multicolumn{1}{|r}{39.89} & 32.63 & 94.69 & 75.47 & 90.52 \\
11 & 0.25 & 0.25 & 0.25 & 0.25 & \multicolumn{1}{|r}{13} & 6 & \multicolumn{1}{|r}{96.64} & 92.58 & 89.18 & 89.57 & 60.52 & \multicolumn{1}{|r}{93.87} & 94.53 & 88.27 & 89.44 & 52.90 \\
12 & 1/3 & 1/3 & 1/3 & 0 & \multicolumn{1}{|r}{16} & 3 & \multicolumn{1}{|r}{94.45} & 97.77 & 87.38 & 95.22 & 53.63 & \multicolumn{1}{|r}{91.08} & 99.75 & 86.35 & 96.03 & 41.78 \\
13 & 1/3 & 1/3 & 0 & 1/3 & \multicolumn{1}{|r}{12} & 7 & \multicolumn{1}{|r}{96.20} & 89.09 & 89.81 & 87.06 & 64.68 & \multicolumn{1}{|r}{92.76} & 89.93 & 88.98 & 86.39 & 57.04 \\
14 & 0 & 1/3 & 1/3 & 1/3 & \multicolumn{1}{|r}{14} & 5 & \multicolumn{1}{|r}{95.20} & 93.91 & 88.68 & 91.85 & 60.21 & \multicolumn{1}{|r}{91.11} & 94.75 & 87.79 & 92.07 & 52.81
\end{tabular}
}
\end{table}
\end{landscape}

%%% Comments on GDP and GLP ex1 tau2=1
In Table \ref{tab::GDP_ex11} the performance characteristics of the designs optimal with respect to generalised $DP$ and $LP$ criteria are given. Due to the larger number of components (compared with the $GD$ and $GL$ compound criteria), we considered $14$ instead of $9$ weight allocation sets, in order to have a chance to observe the relationships between the criteria components and between the weights and optimal designs' performances. Some efficiencies cannot be evaluated when there are no replicates, in such cases they are set to $0$.

In general it can be observed that all the designs have large $D$-efficiencies, the worst are those where no weight was put on either $D$- or $DP$-component; similar tendency seems to be true for generalised $LP$-optimal designs. When some weight is assigned to the $DP$- or $LP$-component, the resulting designs tend to be good in terms of $LoF$-efficiencies: the $DP$-optimal design (as given in Table \ref{tab::Ex11_DP_design}) retains optimality when half of the weight is put on the lack-of-fit component (\#$8$) or is equally distributed between the $D$-, $DP$- and $LoF(DP)$-components (\#$12$). All the replicates are single (i.e. there is no point being replicated more than once), and only $4$ of them out of $18$ are not `corner' points.

The opposite regarding the relationship between the `primary' and `potential' terms components, however, is not true: the $LoF(DP)$-optimal design \#$3$ is only $48.85\%$ $D$-efficient and $51.56\%$ $DP$-efficient (the same happens with the trace-based optimality). As for the $LoF(D)$- and $LoF(L)$- components, the optimal designs in this case lost around $10\%$ in the corresponding efficiencies compared to the generalised $D$- and $L$-optimal designs in Table \ref{tab::GD_ex11}.  

\begin{table}[h]
\centering
\caption{Example 1. DP-optimal design \#$2$}
\label{tab::Ex11_DP_design}
\scalebox{0.8}{
\begin{tabular}{rrrrrr|r|rrrrrr}
1  & -1 & -1 & -1 & -1 & -1 &  & 21 & 0 & 0  & 1  & 1  & 0  \\
2  & -1 & -1 & -1 & -1 & -1 &  & 22 & 0 & 1  & -1 & 0  & 1  \\
3  & -1 & -1 & -1 & 1  & 1  &  & 23 & 1 & -1 & -1 & -1 & 1  \\
4  & -1 & -1 & -1 & 1  & 1  &  & 24 & 1 & -1 & -1 & -1 & 1  \\
5  & -1 & -1 & 1  & -1 & 0  &  & 25 & 1 & -1 & -1 & 1  & -1 \\
6  & -1 & -1 & 1  & 1  & -1 &  & 26 & 1 & -1 & -1 & 1  & -1 \\
7  & -1 & -1 & 1  & 1  & -1 &  & 27 & 1 & -1 & 0  & 0  & 0  \\
8  & -1 & 0  & 1  & 0  & 1  &  & 28 & 1 & -1 & 0  & 0  & 0  \\
9  & -1 & 0  & 1  & 0  & 1  &  & 29 & 1 & -1 & 1  & -1 & -1 \\
10 & -1 & 1  & -1 & -1 & 1  &  & 30 & 1 & -1 & 1  & -1 & -1 \\
11 & -1 & 1  & -1 & -1 & 1  &  & 31 & 1 & -1 & 1  & 1  & 1  \\
12 & -1 & 1  & -1 & 1  & -1 &  & 32 & 1 & -1 & 1  & 1  & 1  \\
13 & -1 & 1  & -1 & 1  & -1 &  & 33 & 1 & 0  & 0  & 1  & 1  \\
14 & -1 & 1  & 1  & -1 & -1 &  & 34 & 1 & 1  & -1 & -1 & -1 \\
15 & -1 & 1  & 1  & -1 & -1 &  & 35 & 1 & 1  & -1 & -1 & -1 \\
16 & -1 & 1  & 1  & 1  & 1  &  & 36 & 1 & 1  & -1 & 1  & 1  \\
17 & -1 & 1  & 1  & 1  & 1  &  & 37 & 1 & 1  & 1  & -1 & 1  \\
18 & 0  & -1 & 0  & -1 & 1  &  & 38 & 1 & 1  & 1  & -1 & 1  \\
19 & 0  & -1 & 0  & -1 & 1  &  & 39 & 1 & 1  & 1  & 1  & -1 \\
20 & 0  & 0  & 1  & 1  & 0  &  & 40 & 1 & 1  & 1  & 1  & -1
\end{tabular}
}
\end{table}
The contradiction between the components corresponding to the prediction bias and to the primary terms remains, and might be even more dramatic: in general, bias efficiencies are much lower (just a few are above $90\%$, only when more than half of the weight is allocated to this component). Together with that, designs optimal in terms of determinant-based criteria sometimes have very low trace-based efficiencies, e.g.~the $LoF(DP)$-optimal design \#$3$ (Table \ref{tab::Ex11_LoFDP_design}), which can be seen to have many points being replicated twice and even three times (unlike the designs seen before).
On the other hand, trace-based optimal designs tend to perform quite well in terms of $GDP$-components. This encourages the idea of the necessity of a careful choice between a trace- and determinant-based approach in this case or, at least, of a double check when constructing a design to be used. 

\begin{table}[h]
\centering
\caption{Example 1. LoF(DP)-optimal design \#$3$}
\label{tab::Ex11_LoFDP_design}
\scalebox{0.8}{
\begin{tabular}{rrrrrr|r|rrrrrr}
1  & -1 & -1 & -1 & -1 & 0  &  & 21 & 0 & 1  & 1  & -1 & 0  \\
2  & -1 & -1 & -1 & -1 & 0  &  & 22 & 1 & -1 & -1 & -1 & 1  \\
3  & -1 & -1 & 0  & -1 & -1 &  & 23 & 1 & -1 & 1  & -1 & -1 \\
4  & -1 & -1 & 1  & -1 & 1  &  & 24 & 1 & -1 & 1  & -1 & -1 \\
5  & -1 & -1 & 1  & -1 & 1  &  & 25 & 1 & -1 & 1  & 1  & 1  \\
6  & -1 & -1 & 1  & -1 & 1  &  & 26 & 1 & 0  & -1 & -1 & -1 \\
7  & -1 & -1 & 1  & 1  & -1 &  & 27 & 1 & 0  & -1 & -1 & -1 \\
8  & -1 & -1 & 1  & 1  & -1 &  & 28 & 1 & 0  & -1 & -1 & -1 \\
9  & -1 & 0  & -1 & -1 & 1  &  & 29 & 1 & 0  & 1  & -1 & 1  \\
10 & -1 & 0  & -1 & 1  & -1 &  & 30 & 1 & 0  & 1  & -1 & 1  \\
11 & -1 & 0  & -1 & 1  & -1 &  & 31 & 1 & 0  & 1  & -1 & 1  \\
12 & -1 & 0  & -1 & 1  & -1 &  & 32 & 1 & 0  & 1  & 1  & 0  \\
13 & -1 & 0  & 1  & 1  & 1  &  & 33 & 1 & 1  & -1 & -1 & 1  \\
14 & -1 & 0  & 1  & 1  & 1  &  & 34 & 1 & 1  & -1 & -1 & 1  \\
15 & -1 & 0  & 1  & 1  & 1  &  & 35 & 1 & 1  & -1 & 0  & 1  \\
16 & -1 & 1  & -1 & -1 & -1 &  & 36 & 1 & 1  & -1 & 1  & -1 \\
17 & -1 & 1  & -1 & 1  & 1  &  & 37 & 1 & 1  & -1 & 1  & -1 \\
18 & -1 & 1  & -1 & 1  & 1  &  & 38 & 1 & 1  & -1 & 1  & -1 \\
19 & -1 & 1  & 0  & 1  & 0  &  & 39 & 1 & 1  & 1  & 1  & -1 \\
20 & 0  & 0  & 1  & 0  & 1  &  & 40 & 1 & 1  & 1  & 1  & 1 
\end{tabular}
}
\end{table}

Overall, lack-of-fit efficiencies  are quite large for all the designs (where they can be evaluated, of course), they are almost always higher than $85\%$ and only drop when the weight is not `fairly' distributed between the competing parts of the criteria. Designs performing well in terms of bias usually occurred when at least half of the weight was put on the components corresponding to the properties of the potential terms. This observation will later contribute to the motivation to substitute the prediction bias with the bias of the primary model coefficients' estimates (see Chapter \ref{ch::mse}). 

%%% Example 1. Tau2=1/q
%% Tables, comments
\paragraph{Example 1.1. $\tau^2=1/q$}\mbox{}\\
As the prior variance of the additional terms $\bm{\beta}_2$ in the full model (\ref{eq::full_model}) is proportional to the error variance $\sigma^2$ with the scaling parameter $\tau^2$, when the number of potential terms is large, and under the assumption of them being uncorrelated (i.e. independent in the case of the normal distribution), the total potential variability becomes quite substantial.

Therefore, as for $\tau^2=1$ in the first example we observed certain contradictions in the resulting performances between the criterion parts corresponding to the primary model terms and to the potential ones. The next step is to try a different value of this tuning parameter, specifically, to make it dependent on the number of potential terms, such that the total contamination from the model misspecification is comparable with the error disturbance. 

We set $\tau^2$ equal to $1/q$, i.e. the inverse of the number of potential terms, so that under the assumption of their independence their total prior variance would be 
\begin{equation}
\sum_{i=1}^{q} \sigma^{2} \tau^{2} = q\sigma^{2} \frac{1}{q}=\sigma^{2},
\end{equation}
that is exactly the error variance.

Tables \ref{tab::GD_ex1q} and \ref{tab::GDP_ex1q} contain the performances of the optimal designs, with the same layouts of weight allocations as before. The general observation is that the optimal designs seem to behave similarly to the ones obtained previously: generalised $D$- and $L$-optimal designs have no pure error degrees of freedom, and in the case of $GDP$- and $GLP$-optimality, on the contrary, most of the available residual degrees of freedom are allocated to the pure error component (with the imbalance being more evident in the case of determinant-based criteria). 

%%% GD- and GL-optimal designs
\begin{table}[h]
\caption{Example 1.1. Properties of generalised D- and L-optimal designs. $\tau^2=1/q$}
\label{tab::GD_ex1q}
\resizebox{\textwidth}{!}}                               \\
   & \textbf{D}       & \textbf{LoF(D)}    & \textbf{Bias(D)}   & \textbf{PE}        & \textbf{LoF}        & \textbf{D}   & \textbf{LoF(D)}   & \textbf{Bias(D)}  & \textbf{L}       & \textbf{LoF(L)}   & \textbf{Bias(L)}  \\
1 & 1 & 0 & 0 & \multicolumn{1}{|r}{0} & 19 & \multicolumn{1}{|r}{100.00} & 99.85 & 77.89 & \multicolumn{1}{|r}{98.96} & 99.85 & 72.51 \\
2 & 0 & 1 & 0 & \multicolumn{1}{|r}{0} & 19 & \multicolumn{1}{|r}{95.99} & 100.00 & 96.19 & \multicolumn{1}{|r}{92.97} & 100.00 & 94.16 \\
3 & 0 & 0 & 1 & \multicolumn{1}{|r}{0} & 19 & \multicolumn{1}{|r}{77.96} & 99.90 & 100.00 & \multicolumn{1}{|r}{66.24} & 99.90 & 100.00 \\
4 & 0.5 & 0.5 & 0 & \multicolumn{1}{|r}{0} & 19 & \multicolumn{1}{|r}{99.90} & 99.85 & 77.50 & \multicolumn{1}{|r}{98.74} & 99.85 & 71.90 \\
5 & 0.5 & 0 & 0.5 & \multicolumn{1}{|r}{0} & 19 & \multicolumn{1}{|r}{95.99} & 100.00 & 96.19 & \multicolumn{1}{|r}{92.97} & 100.00 & 94.16 \\
6 & 0 & 0.5 & 0.5 & \multicolumn{1}{|r}{0} & 19 & \multicolumn{1}{|r}{80.62} & 99.92 & 99.86 & \multicolumn{1}{|r}{68.88} & 99.92 & 99.69 \\
7 & 1/3 & 1/3 & 1/3 & \multicolumn{1}{|r}{0} & 19 & \multicolumn{1}{|r}{93.96} & 99.98 & 97.81 & \multicolumn{1}{|r}{88.15} & 99.98 & 96.62 \\
8 & 0.5 & 0.25 & 0.25 & \multicolumn{1}{|r}{0} & 19 & \multicolumn{1}{|r}{95.99} & 100.00 & 96.19 & \multicolumn{1}{|r}{92.97} & 100.00 & 94.16 \\
9 & 0.25 & 0.5 & 0.25 & \multicolumn{1}{|r}{0} & 19 & \multicolumn{1}{|r}{93.96} & 99.98 & 97.81 & \multicolumn{1}{|r}{88.15} & 99.98 & 96.62 \\
 &  &  &  &  &  &  &  &  &  &  &  \\
    & \multicolumn{3}{l}{\textbf{Criteria}} & \multicolumn{2}{l}{\textbf{DoF}} & \multicolumn{6}{l}{\textbf{Efficiency,\%}}                               \\
   & \textbf{L}       & \textbf{LoF(L)}    & \textbf{Bias(L)}   & \textbf{PE}        & \textbf{LoF}        & \textbf{D}   & \textbf{LoF(D)}   & \textbf{Bias(D)}  & \textbf{L}       & \textbf{LoF(L)}   & \textbf{Bias(L)}  \\
1 & 1 & 0 & 0 & \multicolumn{1}{|r}{0} & 19 & \multicolumn{1}{|r}{99.95} & 99.86 & 76.73 & \multicolumn{1}{|r}{100.00} & 99.86 & 70.37 \\
2 & 0 & 1 & 0 & \multicolumn{1}{|r}{0} & 19 & \multicolumn{1}{|r}{95.99} & 100.00 & 96.19 & \multicolumn{1}{|r}{92.97} & 100.00 & 94.16 \\
3 & 0 & 0 & 1 & \multicolumn{1}{|r}{0} & 19 & \multicolumn{1}{|r}{77.96} & 99.90 & 100.00 & \multicolumn{1}{|r}{66.24} & 99.90 & 100.00 \\
4 & 0.5 & 0.5 & 0 & \multicolumn{1}{|r}{0} & 19 & \multicolumn{1}{|r}{99.95} & 99.86 & 76.73 & \multicolumn{1}{|r}{100.00} & 99.86 & 70.37 \\
5 & 0.5 & 0 & 0.5 & \multicolumn{1}{|r}{0} & 19 & \multicolumn{1}{|r}{95.99} & 100.00 & 96.19 & \multicolumn{1}{|r}{92.97} & 100.00 & 94.16 \\
6 & 0 & 0.5 & 0.5 & \multicolumn{1}{|r}{0} & 19 & \multicolumn{1}{|r}{79.87} & 99.91 & 99.99 & \multicolumn{1}{|r}{65.72} & 99.91 & 99.97 \\
7 & 1/3 & 1/3 & 1/3 & \multicolumn{1}{|r}{0} & 19 & \multicolumn{1}{|r}{92.37} & 99.98 & 99.22 & \multicolumn{1}{|r}{87.39} & 99.98 & 98.95 \\
8 & 0.5 & 0.25 & 0.25 & \multicolumn{1}{|r}{0} & 19 & \multicolumn{1}{|r}{95.99} & 100.00 & 96.19 & \multicolumn{1}{|r}{92.97} & 100.00 & 94.16 \\
9 & 0.25 & 0.5 & 0.25 & \multicolumn{1}{|r}{0} & 19 & \multicolumn{1}{|r}{92.37} & 99.98 & 99.22 & \multicolumn{1}{|r}{87.39} & 99.98 & 98.95
\end{tabular}
}
\end{table}
As before, lack-of-fit determinant- and trace-based optimality are achieved by the same design (\#$2$ in Table \ref{tab::GD_ex1q}), and, moreover, it remains optimal when the weight is distributed between the $D$- ($L$-) and bias components (\#$5$) and it is also design \#$8$. 

The same holds for the bias-optimal designs. They are the same, and their $D$- and $L$- efficiencies are  $2-3\%$ lower than when $\tau^2=1$. These two designs are given in Tables \ref{tab::Ex1q_LoF_design} and \ref{tab::Ex1q_bias_design}; the bias-optimal design is quite different from the ones we have observed before, having just a few `corner' points, with the majority being more or less evenly distributed across the design region (i.e. the candidate set of points).

\begin{table}[h]
\centering
\caption{Example 1.1. LoF(D)- and LoF(L)-optimal design \#$2$ ($\tau^2=1/q$)}
\label{tab::Ex1q_LoF_design}
\scalebox{0.8}{
\begin{tabular}{rrrrrr|r|rrrrrr}
1  & -1 & -1 & -1 & -1 & 1  &  & 21 & 0 & 0  & 1  & -1 & 0  \\
2  & -1 & -1 & -1 & 1  & 1  &  & 22 & 0 & 1  & -1 & -1 & 1  \\
3  & -1 & -1 & -1 & 1  & -1 &  & 23 & 0 & 1  & 1  & 1  & 1  \\
4  & -1 & -1 & 0  & -1 & -1 &  & 24 & 1 & -1 & -1 & -1 & 0  \\
5  & -1 & -1 & 1  & -1 & 1  &  & 25 & 1 & -1 & -1 & 1  & 1  \\
6  & -1 & -1 & 1  & 0  & -1 &  & 26 & 1 & -1 & -1 & 1  & -1 \\
7  & -1 & -1 & 1  & 1  & 1  &  & 27 & 1 & -1 & 0  & -1 & 1  \\
8  & -1 & 0  & -1 & -1 & -1 &  & 28 & 1 & -1 & 1  & -1 & -1 \\
9  & -1 & 0  & 0  & 0  & 1  &  & 29 & 1 & -1 & 1  & 0  & 1  \\
10 & -1 & 0  & 1  & 1  & -1 &  & 30 & 1 & -1 & 1  & 1  & 0  \\
11 & -1 & 1  & -1 & -1 & 0  &  & 31 & 1 & 0  & -1 & -1 & 1  \\
12 & -1 & 1  & -1 & 0  & -1 &  & 32 & 1 & 0  & 0  & 0  & -1 \\
13 & -1 & 1  & -1 & 1  & 1  &  & 33 & 1 & 0  & 1  & 1  & 1  \\
14 & -1 & 1  & 0  & 1  & -1 &  & 34 & 1 & 1  & -1 & -1 & -1 \\
15 & -1 & 1  & 1  & -1 & -1 &  & 35 & 1 & 1  & -1 & 0  & 1  \\
16 & -1 & 1  & 1  & -1 & 1  &  & 36 & 1 & 1  & -1 & 1  & -1 \\
17 & -1 & 1  & 1  & 1  & 0  &  & 37 & 1 & 1  & 0  & 1  & 1  \\
18 & 0  & -1 & -1 & -1 & -1 &  & 38 & 1 & 1  & 1  & -1 & 1  \\
19 & 0  & -1 & 1  & 1  & -1 &  & 39 & 1 & 1  & 1  & -1 & -1 \\
20 & 0  & 0  & -1 & 1  & 0  &  & 40 & 1 & 1  & 1  & 1  & -1
\end{tabular}
}
\end{table}

\begin{table}[h]
\centering
\caption{Example 1.1. Bias(D)- and Bias(L)-optimal design \#$3$ ($\tau^2=1/q$)}
\label{tab::Ex1q_bias_design}
\scalebox{0.8}{
\begin{tabular}{rrrrrr|r|rrrrrr}
1  & -1 & -1 & -1 & 1  & 1  &  & 21 & 0 & 0  & 1  & 1  & -1 \\
2  & -1 & -1 & 0  & -1 & 1  &  & 22 & 0 & 1  & -1 & -1 & -1 \\
3  & -1 & -1 & 0  & 0  & -1 &  & 23 & 0 & 1  & -1 & 1  & 0  \\
4  & -1 & -1 & 0  & 1  & 0  &  & 24 & 0 & 1  & 1  & -1 & 0  \\
5  & -1 & -1 & 1  & 1  & 0  &  & 25 & 0 & 1  & 1  & 1  & 1  \\
6  & -1 & 0  & -1 & -1 & 0  &  & 26 & 1 & -1 & -1 & 0  & 0  \\
7  & -1 & 0  & -1 & 1  & -1 &  & 27 & 1 & -1 & 0  & -1 & -1 \\
8  & -1 & 0  & 0  & 0  & 1  &  & 28 & 1 & -1 & 0  & 0  & 1  \\
9  & -1 & 0  & 1  & -1 & -1 &  & 29 & 1 & -1 & 0  & 1  & -1 \\
10 & -1 & 0  & 1  & 0  & 1  &  & 30 & 1 & -1 & 1  & 0  & -1 \\
11 & -1 & 1  & -1 & 0  & 1  &  & 31 & 1 & 0  & -1 & 0  & -1 \\
12 & -1 & 1  & 0  & -1 & 1  &  & 32 & 1 & 0  & -1 & 1  & 1  \\
13 & -1 & 1  & 0  & 0  & -1 &  & 33 & 1 & 0  & 0  & 0  & -1 \\
14 & -1 & 1  & 0  & 1  & 1  &  & 34 & 1 & 0  & 1  & -1 & 1  \\
15 & -1 & 1  & 1  & 0  & 0  &  & 35 & 1 & 0  & 1  & 1  & 0  \\
16 & 0  & -1 & -1 & -1 & -1 &  & 36 & 1 & 1  & -1 & -1 & 0  \\
17 & 0  & -1 & -1 & 1  & 0  &  & 37 & 1 & 1  & 0  & -1 & 0  \\
18 & 0  & -1 & 1  & -1 & 0  &  & 38 & 1 & 1  & 0  & 0  & 1  \\
19 & 0  & -1 & 1  & 1  & 1  &  & 39 & 1 & 1  & 0  & 1  & -1 \\
20 & 0  & 0  & -1 & -1 & 1  &  & 40 & 1 & 1  & 1  & -1 & -1
\end{tabular}
}
\end{table}

%% GDP- and GLP-optimal designs. Example 1, tau2=1/q.
\begin{landscape}
\begin{table}[p]
\caption{Example 1.1. Properties of generalised DP- and LP-optimal designs. $\tau^2=1/q$}
\label{tab::GDP_ex1q}
\resizebox{\linewidth}{!}} \\
\multicolumn{1}{l}{} & \multicolumn{1}{l}{{\bf D}} & \multicolumn{1}{l}{{\bf DP}} & \multicolumn{1}{l}{{\bf LoF(DP)}} & \multicolumn{1}{l}{{\bf Bias(D)}} & \multicolumn{1}{l}{{\bf PE}} & \multicolumn{1}{l}{{\bf LoF}} & \multicolumn{1}{l}{{\bf D}} & \multicolumn{1}{l}{{\bf DP}} & \multicolumn{1}{l}{{\bf LoF(D)}} & \multicolumn{1}{l}{{\bf LoF(DP)}} & \multicolumn{1}{l}{{\bf Bias(D)}} & \multicolumn{1}{l}{{\bf L}} & \multicolumn{1}{l}{{\bf LP}} & \multicolumn{1}{l}{{\bf LoF(L)}} & \multicolumn{1}{l}{{\bf LoF(LP)}} & \multicolumn{1}{l}{{\bf Bias(L)}} \\
1 & 1 & 0 & 0 & 0 & \multicolumn{1}{|r}{0} & 19 & \multicolumn{1}{|r}{100.00} & 0.00 & 99.85 & 0.00 & 77.89 & \multicolumn{1}{|r}{98.96} & 0.00 & 99.85 & 0.00 & 72.51 \\
2 & 0 & 1 & 0 & 0 & \multicolumn{1}{|r}{18} & 1 & \multicolumn{1}{|r}{93.00} & 100.00 & 99.41 & 98.32 & 52.19 & \multicolumn{1}{|r}{86.47} & 98.52 & 99.41 & 98.26 & 38.62 \\
3 & 0 & 0 & 1 & 0 & \multicolumn{1}{|r}{19} & 0 & \multicolumn{1}{|r}{90.07} & 98.43 & 99.39 & 100.00 & 51.00 & \multicolumn{1}{|r}{79.39} & 91.96 & 99.38 & 100.00 & 37.11 \\
4 & 0 & 0 & 0 & 1 & \multicolumn{1}{|r}{0} & 19 & \multicolumn{1}{|r}{77.96} & 0.00 & 99.90 & 0.00 & 100.00 & \multicolumn{1}{|r}{66.24} & 0.00 & 99.90 & 0.00 & 100.00 \\
5 & 0.5 & 0.5 & 0 & 0 & \multicolumn{1}{|r}{16} & 3 & \multicolumn{1}{|r}{95.29} & 98.64 & 99.46 & 94.48 & 56.36 & \multicolumn{1}{|r}{91.31} & 100.00 & 99.46 & 94.19 & 46.72 \\
6 & 0.5 &  & 0.5 & 0 & \multicolumn{1}{|r}{18} & 1 & \multicolumn{1}{|r}{93.00} & 100.00 & 99.41 & 98.32 & 52.19 & \multicolumn{1}{|r}{86.47} & 98.52 & 99.41 & 98.26 & 38.62 \\
7 & 0.5 & 0 & 0 & 0.5 & \multicolumn{1}{|r}{0} & 19 & \multicolumn{1}{|r}{95.99} & 0.00 & 100.00 & 0.00 & 96.19 & \multicolumn{1}{|r}{92.97} & 0.00 & 100.00 & 0.00 & 94.16 \\
8 & 0 & 0.5 & 0.5 & 0 & \multicolumn{1}{|r}{19} & 0 & \multicolumn{1}{|r}{90.07} & 98.43 & 99.39 & 100.00 & 51.00 & \multicolumn{1}{|r}{79.39} & 91.96 & 99.38 & 100.00 & 37.11 \\
9 & 0 & 0.5 & 0 & 0.5 & \multicolumn{1}{|r}{12} & 7 & \multicolumn{1}{|r}{93.40} & 86.50 & 99.61 & 84.17 & 69.50 & \multicolumn{1}{|r}{85.03} & 82.44 & 99.61 & 82.68 & 62.11 \\
10 & 0 & 0 & 0.5 & 0.5 & \multicolumn{1}{|r}{10} & 9 & \multicolumn{1}{|r}{54.50} & 46.28 & 99.80 & 77.04 & 88.25 & \multicolumn{1}{|r}{13.05} & 11.44 & 99.80 & 74.25 & 77.38 \\
11 & 0.25 & 0.25 & 0.25 & 0.25 & \multicolumn{1}{|r}{15} & 4 & \multicolumn{1}{|r}{93.89} & 95.02 & 99.50 & 92.29 & 61.42 & \multicolumn{1}{|r}{88.13} & 94.22 & 99.50 & 91.82 & 52.26 \\
12 & 1/3 & 1/3 & 1/3 & 0 & \multicolumn{1}{|r}{18} & 1 & \multicolumn{1}{|r}{93.00} & 100.00 & 99.41 & 98.32 & 52.19 & \multicolumn{1}{|r}{86.47} & 98.52 & 99.41 & 98.26 & 38.62 \\
13 & 1/3 & 1/3 & 0 & 1/3 & \multicolumn{1}{|r}{12} & 7 & \multicolumn{1}{|r}{96.43} & 89.30 & 99.54 & 84.11 & 65.26 & \multicolumn{1}{|r}{93.02} & 90.19 & 99.54 & 82.63 & 57.58 \\
14 & 0 & 1/3 & 1/3 & 1/3 & \multicolumn{1}{|r}{16} & 3 & \multicolumn{1}{|r}{91.86} & 95.09 & 99.48 & 94.50 & 61.98 & \multicolumn{1}{|r}{84.85} & 92.92 & 99.48 & 94.21 & 50.69 \\
 &  &  &  &  &  &  &  &  &  &  &  &  &  &  &  &  \\
 \multicolumn{1}{l}{} & \multicolumn{4}{l}{{\bf Criteria}} & \multicolumn{2}{l}{{\bf DoF}} & \multicolumn{10}{l}{{\bf Efficiency,\%}} \\
\multicolumn{1}{l}{} & \multicolumn{1}{l}{{\bf L}} & \multicolumn{1}{l}{{\bf LP}} & \multicolumn{1}{l}{{\bf LoF(LP)}} & \multicolumn{1}{l}{{\bf Bias(L)}} & \multicolumn{1}{l}{{\bf PE}} & \multicolumn{1}{l}{{\bf LoF}} & \multicolumn{1}{l}{{\bf D}} & \multicolumn{1}{l}{{\bf DP}} & \multicolumn{1}{l}{{\bf LoF(D)}} & \multicolumn{1}{l}{{\bf LoF(DP)}} & \multicolumn{1}{l}{{\bf Bias(D)}} & \multicolumn{1}{l}{{\bf L}} & \multicolumn{1}{l}{{\bf LP}} & \multicolumn{1}{l}{{\bf LoF(L)}} & \multicolumn{1}{l}{{\bf LoF(LP)}} & \multicolumn{1}{l}{{\bf Bias(L)}} \\
1 & 1 & 0 & 0 & 0 & \multicolumn{1}{|r}{0} & 19 & \multicolumn{1}{|r}{99.95} & 0.00 & 99.86 & 0.00 & 76.73 & \multicolumn{1}{|r}{100.00} & 0.00 & 99.86 & 0.00 & 70.37 \\
2 & 0 & 1 & 0 & 0 & \multicolumn{1}{|r}{16} & 3 & \multicolumn{1}{|r}{95.29} & 98.64 & 99.46 & 94.48 & 56.36 & \multicolumn{1}{|r}{91.31} & 100.00 & 99.46 & 94.19 & 46.72 \\
3 & 0 & 0 & 1 & 0 & \multicolumn{1}{|r}{19} & 0 & \multicolumn{1}{|r}{90.07} & 98.43 & 99.39 & 100.00 & 51.00 & \multicolumn{1}{|r}{79.39} & 91.96 & 99.38 & 100.00 & 37.11 \\
4 & 0 & 0 & 0 & 1 & \multicolumn{1}{|r}{0} & 19 & \multicolumn{1}{|r}{77.96} & 0.00 & 99.90 & 0.00 & 100.00 & \multicolumn{1}{|r}{66.24} & 0.00 & 99.90 & 0.00 & 100.00 \\
5 & 0.5 & 0.5 & 0 & 0 & \multicolumn{1}{|r}{14} & 5 & \multicolumn{1}{|r}{96.30} & 95.00 & 99.53 & 89.86 & 58.43 & \multicolumn{1}{|r}{94.46} & 98.24 & 99.53 & 89.14 & 48.16 \\
6 & 0.5 &  & 0.5 & 0 & \multicolumn{1}{|r}{18} & 1 & \multicolumn{1}{|r}{92.94} & 99.93 & 99.41 & 98.32 & 52.79 & \multicolumn{1}{|r}{87.37} & 99.54 & 99.41 & 98.26 & 38.63 \\
7 & 0.5 & 0 & 0 & 0.5 & \multicolumn{1}{|r}{0} & 19 & \multicolumn{1}{|r}{95.99} & 0.00 & 100.00 & 0.00 & 96.19 & \multicolumn{1}{|r}{92.97} & 0.00 & 100.00 & 0.00 & 94.16\\
8 & 0 & 0.5 & 0.5 & 0 & \multicolumn{1}{|r}{18} & 1 & \multicolumn{1}{|r}{92.94} & 99.93 & 99.41 & 98.32 & 52.79 & \multicolumn{1}{|r}{87.37} & 99.54 & 99.41 & 98.26 & 38.63 \\
9 & 0 & 0.5 & 0 & 0.5 & \multicolumn{1}{|r}{12} & 7 & \multicolumn{1}{|r}{96.75} & 89.59 & 99.57 & 84.13 & 63.86 & \multicolumn{1}{|r}{93.87} & 91.01 & 99.57 & 82.65 & 56.04 \\
10 & 0 & 0 & 0.5 & 0.5 & \multicolumn{1}{|r}{10} & 9 & \multicolumn{1}{|r}{57.94} & 49.20 & 99.72 & 76.98 & 85.60 & \multicolumn{1}{|r}{36.49} & 31.98 & 99.72 & 74.19 & 75.86 \\
11 & 0.25 & 0.25 & 0.25 & 0.25 & \multicolumn{1}{|r}{14} & 5 & \multicolumn{1}{|r}{95.72} & 94.42 & 99.52 & 89.85 & 60.52 & \multicolumn{1}{|r}{91.98} & 95.66 & 99.52 & 89.13 & 52.93 \\
12 & 1/3 & 1/3 & 1/3 & 0 & \multicolumn{1}{|r}{16} & 3 & \multicolumn{1}{|r}{94.45} & 97.77 & 99.48 & 94.47 & 54.47 & \multicolumn{1}{|r}{91.08} & 99.75 & 99.46 & 94.20 & 42.62 \\
13 & 1/3 & 1/3 & 0 & 1/3 & \multicolumn{1}{|r}{12} & 7 & \multicolumn{1}{|r}{97.41} & 90.21 & 99.56 & 84.12 & 62.55 & \multicolumn{1}{|r}{94.94} & 92.05 & 99.55 & 82.64 & 55.03 \\
14 & 0 & 1/3 & 1/3 & 1/3 & \multicolumn{1}{|r}{14} & 5 & \multicolumn{1}{|r}{95.05} & 93.77 & 99.51 & 89.85 & 62.66 & \multicolumn{1}{|r}{90.60} & 94.22 & 99.51 & 89.13 & 54.76
\end{tabular}
}
\end{table}
\end{landscape}
On average, designs' performances in terms of both bias and lack-of-fit slightly increased compared to the case of $\tau^2=1$, which can be seen especially well in Table \ref{tab::GDP_ex1q}, where $LoF(D)-$ and $LoF(L)$-efficiencies are around $10\%$ larger than what was seen in Table \ref{tab::GDP_ex11}. However, the minimum efficiency values are still roughly the same and are achieved when no weight is allocated to the potential terms components.    

Designs that are optimal with respect to $GDP$-criteria still tend to have relatively low trace-based efficiency values, and the minimum values are even worse, i.e. when no weight was allocated to the $D$-component, the $GDP$-optimal design \#$10$ is only $13.05\%$ $L$-efficient and $11.44\%$ $LP$-efficient in comparison to $32.96\%$ and $26.97\%$ respectively when $\tau^2$ was set to $1$. The corresponding drops for the $D$- and $DP$-efficiencies are around $9\%$ and $2\%$ accordingly. However, it is worth noting that in general, in both cases, the contradiction between criterion components corresponding to the primary and potential terms is stronger for the trace-based criteria than for the determinant-based ones.

Overall, setting a different value of $\tau^2$ resulted in similar tendencies of the optimal designs; the most considerable is the larger average $LoF(D)-$ and $LoF(L)-$ efficiencies of all optimal designs: reducing the prior variance of the potential terms leads to most of them being consistently well above $99\%$ compared to previously observed average of around $85\%-90\%$ (for $GDP$- and $GLP$-optimal designs). Such consistently high performance may signify the lack-of-fit component redundancy for small enough values of $\tau^2$.

%%%%%%%%%%% Example 1.2
\paragraph{Example 1.2}\mbox{}\\
Now in the framework of the previous example $1$ (five three-level factors, the fitted model is full quadratic polynomial, prior variance scaling parameter $\tau^2=1$) we consider a smaller subset of the potential terms and take just $q=20$ of them: all linear-by-quadratic terms. Tables \ref{tab::GD_ex12} and \ref{tab::GDP_ex12} reflect the performances of the generalised $D$-, $L$-, $DP$- and $LP$-criteria together with the distribution of the degrees of freedom and efficiencies with respect to the individual components; the layouts of the weight distributions are the same as before. The point exchange search algorithm was implemented with $500$ random starts, and the computation took a little less time (around $2.5$ to $5.5$ hours depending on the number of components involved). 

%% GD- and GL-optimal designs, ex 12
\begin{table}[h]
\caption{Example 1.2. Properties of generalised D and L-optimal designs}
\label{tab::GD_ex12}
\resizebox{\textwidth}{!}}                               \\
   & \textbf{D}       & \textbf{LoF(D)}    & \textbf{Bias(D)}   & \textbf{PE}        & \textbf{LoF}        & \textbf{D}   & \textbf{LoF(D)}   & \textbf{Bias(D)}  & \textbf{L}       & \textbf{LoF(L)}   & \textbf{Bias(L)}  \\
1 & 1 & 0 & 0 & \multicolumn{1}{|r}{0} & 19 & \multicolumn{1}{|r}{100.00} & 97.50 & 76.51 & \multicolumn{1}{|r}{98.96} & 45.57 & 70.77 \\
2 & 0 & 1 & 0 & \multicolumn{1}{|r}{0} & 19 & \multicolumn{1}{|r}{71.31} & 100.00 & 100.00 & \multicolumn{1}{|r}{41.41} & 100.00 & 100.00 \\
3 & 0 & 0 & 1 & \multicolumn{1}{|r}{0} & 19 & \multicolumn{1}{|r}{71.31} & 100.00 & 100.00 & \multicolumn{1}{|r}{41.41} & 100.00 & 100.00 \\
4 & 0.5 & 0.5 & 0 & \multicolumn{1}{|r}{0} & 19 & \multicolumn{1}{|r}{98.85} & 94.10 & 77.59 & \multicolumn{1}{|r}{97.70} & 93.61 & 75.48 \\
5 & 0.5 & 0 & 0.5 & \multicolumn{1}{|r}{0} & 19 & \multicolumn{1}{|r}{92.51} & 97.61 & 96.51 & \multicolumn{1}{|r}{85.88} & 97.50 & 95.98 \\
6 & 0 & 0.5 & 0.5 & \multicolumn{1}{|r}{0} & 19 & \multicolumn{1}{|r}{75.56} & 99.46 & 98.06 & \multicolumn{1}{|r}{62.63} & 99.67 & 97.48 \\
7 & 1/3 & 1/3 & 1/3 & \multicolumn{1}{|r}{0} & 19 & \multicolumn{1}{|r}{92.14} & 97.69 & 97.55 & \multicolumn{1}{|r}{85.10} & 97.61 & 97.21 \\
8 & 0.5 & 0.25 & 0.25 & \multicolumn{1}{|r}{0} & 19 & \multicolumn{1}{|r}{94.35} & 97.35 & 93.72 & \multicolumn{1}{|r}{88.97} & 97.20 & 92.60 \\
9 & 0.25 & 0.5 & 0.25 & \multicolumn{1}{|r}{0} & 19 & \multicolumn{1}{|r}{92.79} & 97.56 & 96.06 & \multicolumn{1}{|r}{86.39} & 97.46 & 95.31 \\
 &  &  &  &  &  &  &  &  &  &  &  \\
   & \multicolumn{3}{l}{\textbf{Criteria}} & \multicolumn{2}{l}{\textbf{DoF}} & \multicolumn{6}{l}{\textbf{Efficiency,\%}}                               \\
   & \textbf{L}       & \textbf{LoF(L)}    & \textbf{Bias(L)}   & \textbf{PE}        & \textbf{LoF}        & \textbf{D}   & \textbf{LoF(D)}   & \textbf{Bias(D)}  & \textbf{L}       & \textbf{LoF(L)}   & \textbf{Bias(L)}  \\
1 & 1 & 0 & 0 & \multicolumn{1}{|r}{0} & 19 & \multicolumn{1}{|r}{99.95} & 92.39 & 73.60 & \multicolumn{1}{|r}{100.00} & 91.90 & 69.70 \\
2 & 0 & 1 & 0 & \multicolumn{1}{|r}{0} & 19 & \multicolumn{1}{|r}{71.31} & 100.00 & 100.00 & \multicolumn{1}{|r}{41.41} & 100.00 & 100.00 \\
3 & 0 & 0 & 1 & \multicolumn{1}{|r}{0} & 19 & \multicolumn{1}{|r}{71.31} & 100.00 & 100.00 & \multicolumn{1}{|r}{41.41} & 100.00 & 100.00 \\
4 & 0.5 & 0.5 & 0 & \multicolumn{1}{|r}{0} & 19 & \multicolumn{1}{|r}{99.32} & 93.23 & 75.69 & \multicolumn{1}{|r}{98.86} & 92.74 & 71.39 \\
5 & 0.5 & 0 & 0.5 & \multicolumn{1}{|r}{0} & 19 & \multicolumn{1}{|r}{93.32} & 97.46 & 95.78 & \multicolumn{1}{|r}{87.17} & 97.32 & 94.94 \\
6 & 0 & 0.5 & 0.5 & \multicolumn{1}{|r}{0} & 19 & \multicolumn{1}{|r}{78.00} & 98.81 & 98.07 & \multicolumn{1}{|r}{58.95} & 98.87 & 97.67 \\
7 & 1/3 & 1/3 & 1/3 & \multicolumn{1}{|r}{0} & 19 & \multicolumn{1}{|r}{92.68} & 97.64 & 96.75 & \multicolumn{1}{|r}{86.26} & 97.54 & 96.26 \\
8 & 0.5 & 0.25 & 0.25 & \multicolumn{1}{|r}{0} & 19 & \multicolumn{1}{|r}{99.11} & 94.46 & 86.58 & \multicolumn{1}{|r}{99.00} & 94.11 & 84.41 \\
9 & 0.25 & 0.5 & 0.25 & \multicolumn{1}{|r}{0} & 19 & \multicolumn{1}{|r}{92.37} & 98.12 & 96.52 & \multicolumn{1}{|r}{86.99} & 97.95 & 95.69 
\end{tabular}
}
\end{table}

Similar tendencies can be observed as in the initial example (Tables \ref{tab::GD_ex11} and \ref{tab::GDP_ex11}) regarding the distribution of degrees of freedom: $GD$- and $GL$-optimal designs do not allow estimation of the pure error, but the imbalance in the case of $GDP$-optimal designs is weaker than before. Now the lack-of-fit-optimal design is also bias-optimal in terms of generalised $D$- and $L$-criteria (designs \#$2$ and \#$3$ in Table \ref{tab::GD_ex12}, and the design is presented in Table \ref{tab::Ex12_LoF_design}), its $D$- and $L$-efficiencies are the lowest ($71.31\%$ and only $41.41\%$ respectively); these components also seem to be acting in the same direction, i.e. lack-of-fit-efficient designs are also bias-efficient and vice versa. 

The same relationship pattern holds for the bias and lack-of-fit parts in the determinant and trace forms: higher bias$(D)$-efficiency usually occurs together with higher bias$(L)$-efficiency (and the same for the lack-of-fit components). 

\begin{table}[h]
\centering
\caption{Example 1.2. LoF- and Bias-optimal design \#$2$ and \#$3$}
\label{tab::Ex12_LoF_design}
\scalebox{0.8}{
\begin{tabular}{rrrrrr|r|rrrrrr}
1  & -1 & -1 & -1 & 1  & 0  &  & 21 & 0 & 0  & 1  & -1 & 1  \\
2  & -1 & -1 & 0  & -1 & -1 &  & 22 & 0 & 0  & 1  & 1  & 1  \\
3  & -1 & -1 & 0  & 0  & 1  &  & 23 & 0 & 1  & -1 & 0  & 1  \\
4  & -1 & -1 & 1  & -1 & 0  &  & 24 & 0 & 1  & -1 & 1  & -1 \\
5  & -1 & 0  & -1 & -1 & 1  &  & 25 & 0 & 1  & 0  & -1 & -1 \\
6  & -1 & 0  & -1 & 1  & 1  &  & 26 & 0 & 1  & 1  & 0  & -1 \\
7  & -1 & 0  & 0  & 1  & 0  &  & 27 & 1 & -1 & -1 & -1 & 0  \\
8  & -1 & 0  & 1  & 0  & -1 &  & 28 & 1 & -1 & -1 & 0  & -1 \\
9  & -1 & 1  & -1 & -1 & 0  &  & 29 & 1 & -1 & 0  & -1 & 1  \\
10 & -1 & 1  & -1 & 0  & -1 &  & 30 & 1 & -1 & 0  & 1  & -1 \\
11 & -1 & 1  & 0  & -1 & 1  &  & 31 & 1 & -1 & 1  & 0  & 1  \\
12 & -1 & 1  & 0  & 1  & -1 &  & 32 & 1 & -1 & 1  & 1  & 0  \\
13 & -1 & 1  & 1  & 0  & 1  &  & 33 & 1 & 0  & -1 & 0  & 1  \\
14 & -1 & 1  & 1  & 1  & 0  &  & 34 & 1 & 0  & 0  & -1 & 0  \\
15 & 0  & -1 & -1 & 0  & 1  &  & 35 & 1 & 0  & 1  & -1 & -1 \\
16 & 0  & -1 & 0  & 1  & 1  &  & 36 & 1 & 0  & 1  & 1  & -1 \\
17 & 0  & -1 & 1  & -1 & 1  &  & 37 & 1 & 1  & -1 & 1  & 0  \\
18 & 0  & -1 & 1  & 0  & -1 &  & 38 & 1 & 1  & 0  & 0  & -1 \\
19 & 0  & 0  & -1 & -1 & -1 &  & 39 & 1 & 1  & 0  & 1  & 1  \\
20 & 0  & 0  & -1 & 1  & -1 &  & 40 & 1 & 1  & 1  & -1 & 0 
\end{tabular}
}
\end{table}
  


\begin{landscape}
%% GDP-optimal designs, ex 12
\begin{table}[p]
\caption{Example 1.2. Properties of generalised DP- and LP-optimal designs}
\label{tab::GDP_ex12}
\resizebox{\linewidth}{!}} \\
\multicolumn{1}{l}{} & \multicolumn{1}{l}{{\bf D}} & \multicolumn{1}{l}{{\bf DP}} & \multicolumn{1}{l}{{\bf LoF(DP)}} & \multicolumn{1}{l}{{\bf Bias(D)}} & \multicolumn{1}{l}{{\bf PE}} & \multicolumn{1}{l}{{\bf LoF}} & \multicolumn{1}{l}{{\bf D}} & \multicolumn{1}{l}{{\bf DP}} & \multicolumn{1}{l}{{\bf LoF(D)}} & \multicolumn{1}{l}{{\bf LoF(DP)}} & \multicolumn{1}{l}{{\bf Bias(D)}} & \multicolumn{1}{l}{{\bf L}} & \multicolumn{1}{l}{{\bf LP}} & \multicolumn{1}{l}{{\bf LoF(L)}} & \multicolumn{1}{l}{{\bf LoF(LP)}} & \multicolumn{1}{l}{{\bf Bias(L)}} \\
1 & 1 & 0 & 0 & 0 & \multicolumn{1}{|r}{0} & 19 & \multicolumn{1}{|r}{100.00} & 0.00 & 92.06 & 0.00 & 73.83 & \multicolumn{1}{|r}{98.96} & 0.00 & 91.58 & 0.00 & 69.87 \\
2 & 0 & 1 & 0 & 0 & \multicolumn{1}{|r}{18} & 1 & \multicolumn{1}{|r}{93.00} & 100.00 & 86.65 & 96.61 & 56.39 & \multicolumn{1}{|r}{86.47} & 98.52 & 86.02 & 97.91 & 39.67 \\
3 & 0 & 0 & 1 & 0 & \multicolumn{1}{|r}{17} & 2 & \multicolumn{1}{|r}{48.81} & 51.55 & 91.32 & 100.00 & 51.49 & \multicolumn{1}{|r}{12.78} & 14.29 & 92.27 & 100.00 & 24.94 \\
4 & 0 & 0 & 0 & 1 & \multicolumn{1}{|r}{0} & 19 & \multicolumn{1}{|r}{71.31} & 0.00 & 100.00 & 0.00 & 100.00 & \multicolumn{1}{|r}{41.41} & 0.00 & 100.00 & 0.00 & 100.00 \\
5 & 0.5 & 0.5 & 0 & 0 & \multicolumn{1}{|r}{16} & 3 & \multicolumn{1}{|r}{95.29} & 98.64 & 88.52 & 95.00 & 64.31 & \multicolumn{1}{|r}{91.31} & 100.00 & 88.04 & 95.97 & 50.12 \\
6 & 0.5 &  & 0.5 & 0 & \multicolumn{1}{|r}{16} & 3 & \multicolumn{1}{|r}{95.05} & 98.39 & 88.90 & 95.41 & 67.67 & \multicolumn{1}{|r}{90.98} & 99.64 & 88.49 & 96.32 & 53.37 \\
7 & 0.5 & 0 & 0 & 0.5 & \multicolumn{1}{|r}{0} & 19 & \multicolumn{1}{|r}{92.51} & 0.00 & 97.61 & 0.00 & 96.51 & \multicolumn{1}{|r}{85.88} & 0.00 & 97.50 & 0.00 & 95.98 \\
8 & 0 & 0.5 & 0.5 & 0 & \multicolumn{1}{|r}{18} & 1 & \multicolumn{1}{|r}{92.98} & 99.98 & 86.77 & 96.74 & 56.51 & \multicolumn{1}{|r}{86.32} & 98.34 & 86.16 & 98.03 & 41.23 \\
9 & 0 & 0.5 & 0 & 0.5 & \multicolumn{1}{|r}{14} & 5 & \multicolumn{1}{|r}{93.71} & 92.44 & 91.52 & 93.61 & 85.25 & \multicolumn{1}{|r}{87.43} & 90.93 & 91.38 & 93.91 & 79.33 \\
10 & 0 & 0 & 0.5 & 0.5 & \multicolumn{1}{|r}{14} & 5 & \multicolumn{1}{|r}{73.25} & 72.26 & 94.19 & 96.33 & 95.11 & \multicolumn{1}{|r}{64.26} & 66.83 & 94.79 & 96.02 & 92.74 \\
11 & 0.25 & 0.25 & 0.25 & 0.25 & \multicolumn{1}{|r}{14} & 5 & \multicolumn{1}{|r}{93.71} & 92.44 & 91.52 & 93.61 & 85.25 & \multicolumn{1}{|r}{87.43} & 90.93 & 91.38 & 93.91 & 79.33 \\
12 & 1/3 & 1/3 & 1/3 & 0 & \multicolumn{1}{|r}{17} & 2 & \multicolumn{1}{|r}{94.24} & 99.53 & 87.81 & 96.15 & 61.44 & \multicolumn{1}{|r}{89.21} & 99.77 & 87.29 & 97.27 & 45.08 \\
13 & 1/3 & 1/3 & 0 & 1/3 & \multicolumn{1}{|r}{14} & 5 & \multicolumn{1}{|r}{93.71} & 92.44 & 91.52 & 93.61 & 85.25 & \multicolumn{1}{|r}{87.43} & 90.93 & 91.38 & 93.91 & 79.33 \\
14 & 0 & 1/3 & 1/3 & 1/3 & \multicolumn{1}{|r}{14} & 5 & \multicolumn{1}{|r}{93.67} & 92.40 & 91.53 & 93.61 & 85.27 & \multicolumn{1}{|r}{87.55} & 91.06 & 91.39 & 93.92 & 79.42 \\
 &  &  &  &  &  &  &  &  &  &  &  &  &  &  &  &  \\
\multicolumn{1}{l}{} & \multicolumn{4}{l}{{\bf Criteria}} & \multicolumn{2}{l}{{\bf DoF}} & \multicolumn{10}{l}{{\bf Efficiency,\%}} \\
\multicolumn{1}{l}{} & \multicolumn{1}{l}{{\bf L}} & \multicolumn{1}{l}{{\bf LP}} & \multicolumn{1}{l}{{\bf LoF(LP)}} & \multicolumn{1}{l}{{\bf Bias(L)}} & \multicolumn{1}{l}{{\bf PE}} & \multicolumn{1}{l}{{\bf LoF}} & \multicolumn{1}{l}{{\bf D}} & \multicolumn{1}{l}{{\bf DP}} & \multicolumn{1}{l}{{\bf LoF(D)}} & \multicolumn{1}{l}{{\bf LoF(DP)}} & \multicolumn{1}{l}{{\bf Bias(D)}} & \multicolumn{1}{l}{{\bf L}} & \multicolumn{1}{l}{{\bf LP}} & \multicolumn{1}{l}{{\bf LoF(L)}} & \multicolumn{1}{l}{{\bf LoF(LP)}} & \multicolumn{1}{l}{{\bf Bias(L)}} \\ 
1 & 1 & 0 & 0 & 0 & \multicolumn{1}{|r}{0} & 19 & \multicolumn{1}{|r}{99.95} & 0.00 & 92.39 & 0.00 & 73.60 & \multicolumn{1}{|r}{100.00} & 0.00 & 91.90 & 0.00 & 69.70 \\
2 & 0 & 1 & 0 & 0 & \multicolumn{1}{|r}{16} & 3 & \multicolumn{1}{|r}{95.29} & 98.64 & 88.52 & 95.00 & 64.31 & \multicolumn{1}{|r}{91.31} & 100.00 & 88.04 & 95.97 & 50.12 \\
3 & 0 & 0 & 1 & 0 & \multicolumn{1}{|r}{17} & 2 & \multicolumn{1}{|r}{48.81} & 51.55 & 91.32 & 100.00 & 51.49 & \multicolumn{1}{|r}{12.78} & 14.29 & 92.27 & 100.00 & 24.94 \\
4 & 0 & 0 & 0 & 1 & \multicolumn{1}{|r}{0} & 19 & \multicolumn{1}{|r}{71.31} & 0.00 & 100.00 & 0.00 & 100.00 & \multicolumn{1}{|r}{41.41} & 0.00 & 100.00 & 0.00 & 100.00 \\
5 & 0.5 & 0.5 & 0 & 0 & \multicolumn{1}{|r}{14} & 5 & \multicolumn{1}{|r}{96.30} & 95.00 & 89.12 & 91.15 & 64.66 & \multicolumn{1}{|r}{94.46} & 98.24 & 88.60 & 91.79 & 53.71 \\
6 & 0.5 &  & 0.5 & 0 & \multicolumn{1}{|r}{16} & 3 & \multicolumn{1}{|r}{94.28} & 97.59 & 88.28 & 94.75 & 62.28 & \multicolumn{1}{|r}{90.90} & 99.54 & 87.78 & 95.75 & 50.13 \\
7 & 0.5 & 0 & 0 & 0.5 & \multicolumn{1}{|r}{0} & 19 & \multicolumn{1}{|r}{92.14} & 0.00 & 97.69 & 0.00 & 97.55 & \multicolumn{1}{|r}{85.10} & 0.00 & 97.61 & 0.00 & 97.21 \\
8 & 0 & 0.5 & 0.5 & 0 & \multicolumn{1}{|r}{18} & 1 & \multicolumn{1}{|r}{92.94} & 99.93 & 87.07 & 97.08 & 58.19 & \multicolumn{1}{|r}{87.37} & 99.54 & 86.51 & 98.32 & 41.91 \\
9 & 0 & 0.5 & 0 & 0.5 & \multicolumn{1}{|r}{14} & 5 & \multicolumn{1}{|r}{93.70} & 92.43 & 91.47 & 93.55 & 85.06 & \multicolumn{1}{|r}{88.04} & 91.57 & 91.32 & 93.87 & 79.11 \\
10 & 0 & 0 & 0.5 & 0.5 & \multicolumn{1}{|r}{14} & 5 & \multicolumn{1}{|r}{64.38} & 63.51 & 91.34 & 93.43 & 91.15 & \multicolumn{1}{|r}{42.59} & 44.29 & 91.19 & 93.75 & 88.64 \\
11 & 0.25 & 0.25 & 0.25 & 0.25 & \multicolumn{1}{|r}{14} & 5 & \multicolumn{1}{|r}{93.53} & 92.26 & 91.48 & 93.57 & 85.21 & \multicolumn{1}{|r}{87.85} & 91.36 & 91.36 & 84.18 & 79.08 \\
12 & 1/3 & 1/3 & 1/3 & 0 & \multicolumn{1}{|r}{16} & 3 & \multicolumn{1}{|r}{94.45} & 97.77 & 87.96 & 94.41 & 60.47 & \multicolumn{1}{|r}{91.08} & 99.75 & 87.41 & 95.45 & 46.71 \\
13 & 1/3 & 1/3 & 0 & 1/3 & \multicolumn{1}{|r}{12} & 7 & \multicolumn{1}{|r}{95.03} & 88.00 & 92.52 & 88.83 & 86.75 & \multicolumn{1}{|r}{90.37} & 87.62 & 92.40 & 88.46 & 81.82 \\
14 & 0 & 1/3 & 1/3 & 1/3 & \multicolumn{1}{|r}{14} & 5 & \multicolumn{1}{|r}{93.70} & 92.43 & 91.47 & 93.55 & 85.06 & \multicolumn{1}{|r}{88.04} & 91.57 & 91.32 & 93.87 & 79.11
\end{tabular}
}
\end{table}
\end{landscape}

Overall, designs perform a bit better in terms of the `potential' terms components than before, which is not a surprising consequence of reducing the number of potential terms by a third. If we look at $GLP$-optimal designs \#$9$ and \#$10$, when half of the weight is on the bias component, and the other half is moved from the $LP$ component to the lack-of-fit one, while in terms of all lack-of-fit and bias individual criteria the design still performs quite well, for example, the $LP$-efficiency decreases from $91.57\%$ to $44.29\%$ in comparison to falling from $90.71\%$ to $32.63\%$ in the case of $q=30$ and $\tau^2=1$ (and to similar values in the case of $\tau^2=1/q$). The contradiction between the two opposite groups of elementary criteria seems to be slightly less drastic when the number of potential terms is decreased. 

$GDP$-optimal design \#$5$ turns out to be also $LP$-optimal, $GLP$-optimal design \#$14$ is also optimal in terms of the $GLP$-criterion with equal importance of $LP$ and bias$(L)$ (\#$9$). In terms of performance it is quite similar to $GDP$-optimal designs \#$9$, \#$11$ and \#$13$ (given in Table \ref{tab::Ex12_GDP_design}), which are the same, even though they were obtained using quite different weight allocations. It also illustrates an example of a design providing a good compromise between various criterion components. In the previous case various efficiencies of the corresponding designs differed by up to $13\%$.  In general, in this `smaller' example, the performance of the optimal designs appears to be less sensitive to changes in the weight allocation scheme, especially when some reallocations occur within either the `primary' or the `potential' terms' elementary criterion group. 

\begin{table}[h]
\centering
\caption{Example 1.2. GDP-optimal design \#$9$, \#$11$ and \#$13$}
\label{tab::Ex12_GDP_design}
\scalebox{0.8}{
\begin{tabular}{rrrrrr|r|rrrrrr}
1  & -1 & -1 & -1 & -1 & -1 &  & 21 & 0 & 0  & 1  & 1  & 0  \\
2  & -1 & -1 & -1 & 1  & 1  &  & 22 & 0 & 1  & 0  & 1  & 1  \\
3  & -1 & -1 & -1 & 1  & 1  &  & 23 & 0 & 1  & 1  & 0  & -1 \\
4  & -1 & -1 & 0  & 0  & 0  &  & 24 & 1 & -1 & -1 & -1 & 1  \\
5  & -1 & -1 & 1  & -1 & 1  &  & 25 & 1 & -1 & -1 & -1 & 1  \\
6  & -1 & -1 & 1  & -1 & 1  &  & 26 & 1 & -1 & -1 & 1  & -1 \\
7  & -1 & -1 & 1  & 1  & -1 &  & 27 & 1 & -1 & -1 & 1  & -1 \\
8  & -1 & -1 & 1  & 1  & -1 &  & 28 & 1 & -1 & 1  & -1 & -1 \\
9  & -1 & 0  & 0  & 0  & -1 &  & 29 & 1 & -1 & 1  & -1 & -1 \\
10 & -1 & 1  & -1 & -1 & 1  &  & 30 & 1 & -1 & 1  & 1  & 1  \\
11 & -1 & 1  & -1 & -1 & 1  &  & 31 & 1 & -1 & 1  & 1  & 1  \\
12 & -1 & 1  & -1 & 1  & -1 &  & 32 & 1 & 0  & 0  & 0  & 1  \\
13 & -1 & 1  & -1 & 1  & -1 &  & 33 & 1 & 1  & -1 & -1 & -1 \\
14 & -1 & 1  & 1  & -1 & -1 &  & 34 & 1 & 1  & -1 & -1 & -1 \\
15 & -1 & 1  & 1  & -1 & -1 &  & 35 & 1 & 1  & -1 & 1  & 1  \\
16 & -1 & 1  & 1  & 1  & 1  &  & 36 & 1 & 1  & 0  & 0  & 0  \\
17 & -1 & 1  & 1  & 1  & 1  &  & 37 & 1 & 1  & 1  & -1 & 1  \\
18 & 0  & -1 & -1 & 0  & 1  &  & 38 & 1 & 1  & 1  & -1 & 1  \\
19 & 0  & -1 & 0  & -1 & -1 &  & 39 & 1 & 1  & 1  & 1  & -1 \\
20 & 0  & 0  & -1 & -1 & 0  &  & 40 & 1 & 1  & 1  & 1  & -1
\end{tabular}
}
\end{table}

\newpage
\subsubsection{Example 2}\mbox{}\\
Having observed a certain inflexibility of the designs regarding the response to the changes in the allocation of weights, we consider a larger factorial experiment. Now it has four factors, and each factor can take five levels, i.e. $625$ candidate points in total; the number of runs is chosen to be $60$. Fitting the full second-order polynomial model as the primary one means we have $p=15$ parameters and $45$ residual degrees of freedom in total. 

All third and fourth order terms are taken as potentially missed terms (so there are $q=55$ of them), and we assume that the `true' model is actually the full third-order polynomial, with the total $p+q=70$ parameters and, therefore, not all of them can be estimated. The variance scaling parameter of the prior assumed for the potential terms remains the same: $\tau^2=1$.

We ran the point exchange algorithm with $200$ random starts, the time needed for the computations is now around $4$ -- $13$ hours; the resulting designs are summarised in Tables \ref{tab::GD&GL_ex2} and \ref{tab::GDP_ex2} below and we see most of the features  observed in the previous example remain present here as well. However, some properties do differ.

The main one is that a few of $GD$- and $GL$-optimal  designs have some degrees of freedom allocated to the pure error component (fewer than to the lack-of-fit component though); even $D$- and $L$-optimal designs have $17$ and $19$ pure error degrees of freedom respectively, and designs with no weight allocated to these components perform worse than in the previous example (e.g.~$GD$-optimal designs \#$3$ and \#$6$ are just $65\%-67\%$ $D$-efficient, the corresponding $GL$-optimal designs are $59\%-61\%$ $L$-efficient. Most of the replicates are at corner points; however, for example, centre point in the $GD$-optimal design \#$4$ is replicated three times. 

Designs that are optimal in terms of bias are the same for $D$- and $L$-based criteria: design \#$3$ in Table \ref{tab::GD&GL_ex2}. The $LoF(L)$-optimal design also turned out to be optimal with respect to the determinant based component; it is shown in Table \ref{tab::Ex2_LoF_designs} together with the `original' $LoF(D)$-optimal design: although they are different, both have quite a lot of points with at least one coordinate being equal to $\pm 0.5$, which is a more general tendency for the designs that perform well in terms of either lack-of-fit component.

The $LoF(DP)$-optimal design, which is also $LoF(LP)$-optimal, turns out to be `robust' to the allocation of half of the weight to the $DP$-component (\#$9$ in Table \ref{tab::GDP_ex2}). The design itself is given in Table \ref{tab::Ex2_LoFDP_design}: it can be seen that in approximately half of the runs at least one factor is applied at either the $0.5$ or $-0.5$-level value (and most of these points are replicated more than once).


\begin{landscape}
%GD-opt. Ex 2
\begin{table}[p]
\centering
\caption{Example 2. Properties of generalised D- and L-optimal designs. $\tau^2=1$}
\label{tab::GD&GL_ex2}
\scalebox{0.8}{
%\resizebox{\linewidth}{!}} \\
\multicolumn{1}{l}{} & \multicolumn{1}{r}{{\bf D}} & \multicolumn{1}{r}{{\bf LoF(D)}} & \multicolumn{1}{r}{{\bf Bias(D)}} & \multicolumn{1}{r}{{\bf PE}} & \multicolumn{1}{r}{{\bf LoF}} & \multicolumn{1}{r}{{\bf D}} & \multicolumn{1}{r}{{\bf DP}} & \multicolumn{1}{r}{{\bf LoF(D)}} & \multicolumn{1}{r}{{\bf LoF(DP)}} & \multicolumn{1}{r}{{\bf Bias(D)}} & \multicolumn{1}{r}{{\bf L}} & \multicolumn{1}{r}{{\bf LP}} & \multicolumn{1}{r}{{\bf LoF(L)}} & \multicolumn{1}{r}{{\bf LoF(LP)}} & \multicolumn{1}{r}{{\bf Bias(L)}} \\
1 & 1 & 0 & 0 & \multicolumn{1}{|r}{17} & 28 & \multicolumn{1}{|r}{100.00} & 87.14 & 97.76 & 84.28 & 75.25 & \multicolumn{1}{|r}{99.35} & 87.66 & 98.19 & 84.21 & 62.27 \\
2 & 0 & 1 & 0 & \multicolumn{1}{|r}{0} & 45 & \multicolumn{1}{|r}{89.21} & 0.00 & 100.00 & 0.00 & 86.15 & \multicolumn{1}{|r}{87.80} & 0.00 & 99.96 & 0.00 & 80.61 \\
3 & 0 & 0 & 1 & \multicolumn{1}{|r}{0} & 45 & \multicolumn{1}{|r}{65.66} & 0.00 & 97.31 & 0.00 & 100.00 & \multicolumn{1}{|r}{59.61} & 0.00 & 97.15 & 0.00 & 100.00 \\
4 & 0.5 & 0.5 & 0 & \multicolumn{1}{|r}{11} & 34 & \multicolumn{1}{|r}{99.47} & 73.71 & 98.85 & 70.52 & 76.41 & \multicolumn{1}{|r}{98.01} & 72.60 & 99.13 & 67.87 & 65.07 \\
5 & 0.5 & 0 & 0.5 & \multicolumn{1}{|r}{0} & 45 & \multicolumn{1}{|r}{96.21} & 0.00 & 99.57 & 0.00 & 81.93 & \multicolumn{1}{|r}{93.38} & 0.00 & 99.62 & 0.00 & 74.81 \\
6 & 0 & 0.5 & 0.5 & \multicolumn{1}{|r}{0} & 45 & \multicolumn{1}{|r}{67.04} & 0.00 & 97.53 & 0.00 & 99.70 & \multicolumn{1}{|r}{60.89} & 0.00 & 97.36 & 0.00 & 99.68 \\
7 & 1/3 & 1/3 & 1/3 & \multicolumn{1}{|r}{0} & 45 & \multicolumn{1}{|r}{95.79} & 0.00 & 99.57 & 0.00 & 82.26 & \multicolumn{1}{|r}{93.00} & 0.00 & 99.62 & 0.00 & 75.28 \\
8 & 0.5 & 0.25 & 0.25 & \multicolumn{1}{|r}{4} & 41 & \multicolumn{1}{|r}{97.56} & 33.71 & 99.24 & 31.06 & 80.48 & \multicolumn{1}{|r}{94.87} & 25.52 & 99.36 & 17.80 & 72.32 \\
9 & 0.25 & 0.5 & 0.25 & \multicolumn{1}{|r}{0} & 45 & \multicolumn{1}{|r}{95.55} & 0.00 & 99.59 & 0.00 & 82.40 & \multicolumn{1}{|r}{92.79} & 0.00 & 99.63 & 0.00 & 75.58 \\
 &  &  &  &  &  &  &  &  &  &  &  &  &  &  & \\
\multicolumn{1}{l}{} & \multicolumn{3}{l}{{\bf Criteria}} & \multicolumn{2}{l}{{\bf DoF}} & \multicolumn{10}{l}{{\bf Efficiency,\%}} \\
\multicolumn{1}{l}{} & \multicolumn{1}{l}{{\bf L}} & \multicolumn{1}{l}{{\bf LoF(L)}} & \multicolumn{1}{l}{{\bf Bias(L)}} & \multicolumn{1}{l}{{\bf PE}} & \multicolumn{1}{l}{{\bf LoF}} & \multicolumn{1}{l}{{\bf D}} & \multicolumn{1}{l}{{\bf DP}} & \multicolumn{1}{l}{{\bf LoF(D)}} & \multicolumn{1}{l}{{\bf LoF(DP)}} & \multicolumn{1}{l}{{\bf Bias(D)}} & \multicolumn{1}{l}{{\bf L}} & \multicolumn{1}{l}{{\bf LP}} & \multicolumn{1}{l}{{\bf LoF(L)}} & \multicolumn{1}{l}{{\bf LoF(LP)}} & \multicolumn{1}{l}{{\bf Bias(L)}} \\
1 & 1 & 0 & 0 & \multicolumn{1}{|r}{19} & 26 & \multicolumn{1}{|r}{99.73} & 89.73 & 97.87 & 87.72 & 74.35 & \multicolumn{1}{|r}{100.00} & 91.16 & 98.41 & 87.86 & 59.89 \\
2 & 0 & 1 & 0 & \multicolumn{1}{|r}{0} & 45 & \multicolumn{1}{|r}{89.88} & 0.00 & 100.00 & 0.00 & 85.08 & \multicolumn{1}{|r}{88.83} & 0.00 & 100.00 & 0.00 & 79.12 \\
3 & 0 & 0 & 1 & \multicolumn{1}{|r}{0} & 45 & \multicolumn{1}{|r}{65.66} & 0.00 & 97.31 & 0.00 & 100.00 & \multicolumn{1}{|r}{59.61} & 0.00 & 97.15 & 0.00 & 100.00 \\
4 & 0.5 & 0.5 & 0 & \multicolumn{1}{|r}{15} & 30 & \multicolumn{1}{|r}{99.73} & 83.48 & 98.45 & 80.90 & 75.22 & \multicolumn{1}{|r}{99.73} & 84.39 & 98.81 & 80.34 & 62.73 \\
5 & 0.5 & 0 & 0.5 & \multicolumn{1}{|r}{0} & 45 & \multicolumn{1}{|r}{86.26} & 0.00 & 99.53 & 0.00 & 89.43 & \multicolumn{1}{|r}{83.81} & 0.00 & 99.45 & 0.00 & 86.19 \\
6 & 0 & 0.5 & 0.5 & \multicolumn{1}{|r}{0} & 45 & \multicolumn{1}{|r}{67.22} & 0.00 & 97.55 & 0.00 & 99.64 & \multicolumn{1}{|r}{61.16} & 0.00 & 97.39 & 0.00 & 99.59 \\
7 & 1/3 & 1/3 & 1/3 & \multicolumn{1}{|r}{0} & 45 & \multicolumn{1}{|r}{86.79} & 0.00 & 99.66 & 0.00 & 88.81 & \multicolumn{1}{|r}{84.55} & 0.00 & 99.59 & 0.00 & 85.53 \\
8 & 0.5 & 0.25 & 0.25 & \multicolumn{1}{|r}{0} & 45 & \multicolumn{1}{|r}{95.05} & 0.00 & 99.62 & 0.00 & 81.87 & \multicolumn{1}{|r}{93.72} & 0.00 & 99.69 & 0.00 & 75.10 \\
9 & 0.25 & 0.5 & 0.25 & \multicolumn{1}{|r}{0} & 45 & \multicolumn{1}{|r}{87.82} & 0.00 & 99.71 & 0.00 & 87.91 & \multicolumn{1}{|r}{85.86} & 0.00 & 99.66 & 0.00 & 84.23
\end{tabular}
}
\end{table}
\end{landscape}

\begin{table}[h]
\centering
\caption{Example 2. LoF(D)-optimal (left) and LoF(L)-optimal (right) designs}
\label{tab::Ex2_LoF_designs}
\scalebox{0.7}{
\begin{tabular}{rrrrr|rrrrr|r|rrrrr|rrrrr}
1  & -1   & -1   & -1   & -1   & 31 & 0   & 0    & 0    & 0    &  & 1  & -1   & -1   & -1   & -1   & 31 & 0   & 1    & -1   & 1    \\
2  & -1   & -1   & -1   & 0    & 32 & 0   & 1    & -1   & -1   &  & 2  & -1   & -1   & -1   & 1    & 32 & 0   & 1    & 1    & -1   \\
3  & -1   & -1   & -1   & 1    & 33 & 0   & 1    & 0.5  & 1    &  & 3  & -1   & -1   & -0.5 & 0.5  & 33 & 0.5 & -1   & -0.5 & 1    \\
4  & -1   & -1   & 0    & 0.5  & 34 & 0   & 1    & 1    & -1   &  & 4  & -1   & -1   & 0    & -1   & 34 & 0.5 & -1   & 0.5  & -1   \\
5  & -1   & -1   & 0.5  & -0.5 & 35 & 0.5 & -1   & -1   & 0.5  &  & 5  & -1   & -1   & 1    & -1   & 35 & 0.5 & -1   & 1    & -0.5 \\
6  & -1   & -1   & 1    & -1   & 36 & 0.5 & -1   & 0.5  & 1    &  & 6  & -1   & -1   & 1    & 0    & 36 & 0.5 & 0    & -1   & -1   \\
7  & -1   & -1   & 1    & 1    & 37 & 0.5 & -1   & 1    & -1   &  & 7  & -1   & -1   & 1    & 1    & 37 & 0.5 & 0.5  & 1    & 1    \\
8  & -1   & -0.5 & -0.5 & -1   & 38 & 0.5 & 1    & -1   & 1    &  & 8  & -1   & -0.5 & -1   & -0.5 & 38 & 0.5 & 1    & -1   & -0.5 \\
9  & -1   & -0.5 & 0.5  & 1    & 39 & 0.5 & 1    & -0.5 & -0.5 &  & 9  & -1   & -0.5 & 0    & 1    & 39 & 0.5 & 1    & 0.5  & 1    \\
10 & -1   & -0.5 & 1    & 0.5  & 40 & 0.5 & 1    & 1    & 0.5  &  & 10 & -1   & 0    & -1   & -1   & 40 & 1   & -1   & -1   & -1   \\
11 & -1   & 0    & -1   & 1    & 41 & 1   & -1   & -1   & -1   &  & 11 & -1   & 0    & 1    & 1    & 41 & 1   & -1   & -1   & 0    \\
12 & -1   & 0    & 1    & -1   & 42 & 1   & -1   & -1   & 1    &  & 12 & -1   & 0.5  & -1   & 1    & 42 & 1   & -1   & -1   & 1    \\
13 & -1   & 0.5  & -1   & -0.5 & 43 & 1   & -1   & -0.5 & -0.5 &  & 13 & -1   & 0.5  & 0.5  & -1   & 43 & 1   & -1   & 0    & -0.5 \\
14 & -1   & 1    & -1   & -1   & 44 & 1   & -1   & 0.5  & -1   &  & 14 & -1   & 0.5  & 1    & -0.5 & 44 & 1   & -1   & 0    & 1    \\
15 & -1   & 1    & -1   & 1    & 45 & 1   & -1   & 1    & 0    &  & 15 & -1   & 1    & -1   & -1   & 45 & 1   & -1   & 1    & -1   \\
16 & -1   & 1    & -0.5 & 0    & 46 & 1   & -1   & 1    & 1    &  & 16 & -1   & 1    & -1   & 0    & 46 & 1   & -1   & 1    & 0.5  \\
17 & -1   & 1    & 0    & -1   & 47 & 1   & -0.5 & -1   & 0    &  & 17 & -1   & 1    & -1   & 1    & 47 & 1   & -1   & 1    & 1    \\
18 & -1   & 1    & 0    & 1    & 48 & 1   & -0.5 & 0    & 1    &  & 18 & -1   & 1    & 0    & 1    & 48 & 1   & -0.5 & -0.5 & -1   \\
19 & -1   & 1    & 1    & -1   & 49 & 1   & -0.5 & 1    & -1   &  & 19 & -1   & 1    & 0.5  & -0.5 & 49 & 1   & -0.5 & 1    & 1    \\
20 & -1   & 1    & 1    & 0    & 50 & 1   & 0    & -1   & -1   &  & 20 & -1   & 1    & 1    & -1   & 50 & 1   & 0    & -1   & 1    \\
21 & -1   & 1    & 1    & 1    & 51 & 1   & 0    & 1    & 1    &  & 21 & -1   & 1    & 1    & 1    & 51 & 1   & 0    & 1    & -1   \\
22 & -0.5 & -1   & -0.5 & 1    & 52 & 1   & 0.5  & -1   & 1    &  & 22 & -0.5 & -1   & 0.5  & 1    & 52 & 1   & 0.5  & -1   & -0.5 \\
23 & -0.5 & -1   & 0    & -1   & 53 & 1   & 0.5  & 1    & -0.5 &  & 23 & -0.5 & -0.5 & -1   & 1    & 53 & 1   & 0.5  & 0.5  & 1    \\
24 & -0.5 & -1   & 1    & -0.5 & 54 & 1   & 1    & -1   & -1   &  & 24 & -0.5 & -0.5 & 1    & -1   & 54 & 1   & 1    & -1   & -1   \\
25 & -0.5 & 0    & -1   & -1   & 55 & 1   & 1    & -1   & 0    &  & 25 & -0.5 & 1    & -0.5 & -1   & 55 & 1   & 1    & -1   & 1    \\
26 & -0.5 & 0.5  & 1    & 1    & 56 & 1   & 1    & -0.5 & 1    &  & 26 & -0.5 & 1    & 1    & 0.5  & 56 & 1   & 1    & -0.5 & 0.5  \\
27 & -0.5 & 1    & -1   & 0.5  & 57 & 1   & 1    & 0    & -1   &  & 27 & 0    & -1   & -1   & -1   & 57 & 1   & 1    & 0    & -1   \\
28 & 0    & -1   & -1   & -1   & 58 & 1   & 1    & 0.5  & 0.5  &  & 28 & 0    & -1   & -1   & 0.5  & 58 & 1   & 1    & 1    & -1   \\
29 & 0    & -1   & 1    & 1    & 59 & 1   & 1    & 1    & -1   &  & 29 & 0    & -1   & 1    & 1    & 59 & 1   & 1    & 1    & 0    \\
30 & 0    & -0.5 & -1   & 1    & 60 & 1   & 1    & 1    & 1    &  & 30 & 0    & 0    & 0    & 0    & 60 & 1   & 1    & 1    & 1   
\end{tabular}
}
\end{table}

The $D$- and $DP$- optimal design ($GDP$-optimal design \#$5$ in Table \ref{tab::GDP_ex2}) is also $LP$-optimal, but it is different from the one that was obtained as a $GLP$-optimal design (Table \ref{tab::GDP_ex2}, design \#$2$); their efficiencies with respect to other components are quite similar, but the $GDP$-optimal design tends to perform slightly better in terms of determinant-based elementary criteria.

Other than that, there are no more repeated designs, and the distribution of the degrees of freedom between pure error and lack-of-fit components is less extreme than in the case of a smaller experiment.

Most of the designs are highly $D$-, $L$- and $LoF$-efficient, regardless of the corresponding weights. $GD$- and $GL$-optimal designs have moderate $DP$- and $LP$-efficiencies (where the positive numbers of pure error degrees of freedom allow their estimation). In general, the contradiction between optimising the criterion parts, corresponding to either primary terms or the potentially missed ones, is still in place; however, in this particular case it is a bit less drastic than before. For example, the minimum values of $D$- and $DP$- efficiency are achieved when the weight is distributed between lack-of-fit and bias components (designs \#$10$ in Tables \ref{tab::GDP_ex11} and \ref{tab::GDP_ex2}), and the same is true for trace-based optimality.  Bias efficiency values are overall lower than others, and they seem to depend on whether most of the weight is assigned to the criterion parts which correspond to the potential terms (i.e. lack-of-fit and bias itself). This is quite sensible, especially as this component minimises the expected prediction bias rather than dealing with the bias in the parameters' estimators. 
%In the next chapter we will introduce another form of 'bias component', which would arguably be of more interest in a case when a potential model contamination is assumed. 

\begin{landscape}
%% GDP- and GLP-optimal designs, ex 2
\begin{table}[p]
\caption{Example 2. Properties of generalised DP- and LP-optimal designs. $\tau^2=1$}
\label{tab::GDP_ex2}
\resizebox{\linewidth}{!}} \\
\multicolumn{1}{l}{} & \multicolumn{1}{l}{{\bf D}} & \multicolumn{1}{l}{{\bf DP}} & \multicolumn{1}{l}{{\bf LoF(DP)}} & \multicolumn{1}{l}{{\bf Bias(D)}} & \multicolumn{1}{l}{{\bf PE}} & \multicolumn{1}{l}{{\bf LoF}} & \multicolumn{1}{l}{{\bf D}} & \multicolumn{1}{l}{{\bf DP}} & \multicolumn{1}{l}{{\bf LoF(D)}} & \multicolumn{1}{l}{{\bf LoF(DP)}} & \multicolumn{1}{l}{{\bf Bias(D)}} & \multicolumn{1}{l}{{\bf L}} & \multicolumn{1}{l}{{\bf LP}} & \multicolumn{1}{l}{{\bf LoF(L)}} & \multicolumn{1}{l}{{\bf LoF(LP)}} & \multicolumn{1}{l}{{\bf Bias(L)}} \\
1 & 1 & 0 & 0 & 0 & \multicolumn{1}{|r}{17} & 28 & \multicolumn{1}{|r}{100.00} & 87.14 & 97.76 & 83.92 & 75.25 & \multicolumn{1}{|r}{99.35} & 87.66 & 98.19 & 84.21 & 62.27 \\
2 & 0 & 1 & 0 & 0 & \multicolumn{1}{|r}{35} & 10 & \multicolumn{1}{|r}{97.85} & 100.00 & 95.18 & 99.84 & 71.21 & \multicolumn{1}{|r}{95.85} & 98.82 & 96.02 & 99.90 & 52.12 \\
3 & 0 & 0 & 1 & 0 & \multicolumn{1}{|r}{37} & 8 & \multicolumn{1}{|r}{95.78} & 98.71 & 94.27 & 100.00 & 70.60 & \multicolumn{1}{|r}{92.66} & 96.26 & 95.03 & 100.00 & 49.96 \\
4 & 0 & 0 & 0 & 1 & \multicolumn{1}{|r}{0} & 45 & \multicolumn{1}{|r}{65.66} & 0.00 & 97.31 & 0.00 & 100.00 & \multicolumn{1}{|r}{59.61} & 0.00 & 97.15 & 0.00 & 100.00 \\
5 & 0.5 & 0.5 & 0 & 0 & \multicolumn{1}{|r}{31} & 14 & \multicolumn{1}{|r}{99.52} & 99.70 & 95.90 & 98.00 & 72.65 & \multicolumn{1}{|r}{98.53} & 99.72 & 96.62 & 98.38 & 56.10 \\
6 & 0.5 & 0 & 0.5 & 0 & \multicolumn{1}{|r}{36} & 9 & \multicolumn{1}{|r}{97.20} & 99.76 & 94.62 & 99.83 & 70.66 & \multicolumn{1}{|r}{94.58} & 97.89 & 95.38 & 99.90 & 50.44 \\
7 & 0.5 & 0 & 0 & 0.5 & \multicolumn{1}{|r}{0} & 45 & \multicolumn{1}{|r}{96.21} & 0.00 & 99.57 & 0.00 & 81.93 & \multicolumn{1}{|r}{93.38} & 0.00 & 99.62 & 0.00 & 74.81  \\
8 & 0 & 0.5 & 0.5 & 0 & \multicolumn{1}{|r}{37} & 8 & \multicolumn{1}{|r}{95.78} & 98.71 & 94.27 & 100.00 & 70.60 & \multicolumn{1}{|r}{92.66} & 96.26 & 95.03 & 100.00 & 49.96 \\
9 & 0 & 0.5 & 0 & 0.5 & \multicolumn{1}{|r}{30} & 15 & \multicolumn{1}{|r}{95.81} & 95.44 & 96.43 & 97.80 & 76.90 & \multicolumn{1}{|r}{92.86} & 93.48 & 97.09 & 98.25 & 63.00 \\
10 & 0 & 0 & 0.5 & 0.5 & \multicolumn{1}{|r}{26} & 19 & \multicolumn{1}{|r}{60.51} & 58.65 & 95.49 & 93.48 & 90.16 & \multicolumn{1}{|r}{53.62} & 52.59 & 95.58 & 94.53 & 85.86 \\
11 & 0.25 & 0.25 & 0.25 & 0.25 & \multicolumn{1}{|r}{33} & 12 & \multicolumn{1}{|r}{98.18} & 99.41 & 95.67 & 99.12 & 73.14 & \multicolumn{1}{|r}{95.20} & 97.30 & 96.41 & 99.36 & 56.89 \\
12 & 1/3 & 1/3 & 1/3 & 0 & \multicolumn{1}{|r}{35} & 10 & \multicolumn{1}{|r}{97.54} & 99.69 & 95.10 & 99.76 & 71.27 & \multicolumn{1}{|r}{94.91} & 97.84 & 95.89 & 99.86 & 52.66 \\
13 & 1/3 & 1/3 & 0 & 1/3 & \multicolumn{1}{|r}{29} & 16 & \multicolumn{1}{|r}{98.40} & 97.41 & 96.45 & 97.03 & 75.80 & \multicolumn{1}{|r}{95.81} & 95.89 & 97.11 & 97.55 & 62.20 \\
14 & 0 & 1/3 & 1/3 & 1/3 & \multicolumn{1}{|r}{32} & 13 & \multicolumn{1}{|r}{93.27} & 93.95 & 96.11 & 98.91 & 75.91 & \multicolumn{1}{|r}{89.89} & 91.44 & 96.82 & 99.22 & 61.64 \\
 &  &  &  &  &  &  &  &  &  &  &  &  &  &  &  &  \\
\multicolumn{1}{l}{} & \multicolumn{4}{l}{{\bf Criteria}} & \multicolumn{2}{l}{{\bf DoF}} & \multicolumn{10}{l}{{\bf Efficiency,\%}} \\
\multicolumn{1}{l}{} & \multicolumn{1}{l}{{\bf L}} & \multicolumn{1}{l}{{\bf LP}} & \multicolumn{1}{l}{{\bf LoF(LP)}} & \multicolumn{1}{l}{{\bf Bias(L)}} & \multicolumn{1}{l}{{\bf PE}} & \multicolumn{1}{l}{{\bf LoF}} & \multicolumn{1}{l}{{\bf D}} & \multicolumn{1}{l}{{\bf DP}} & \multicolumn{1}{l}{{\bf LoF(D)}} & \multicolumn{1}{l}{{\bf LoF(DP)}} & \multicolumn{1}{l}{{\bf Bias(D)}} & \multicolumn{1}{l}{{\bf L}} & \multicolumn{1}{l}{{\bf LP}} & \multicolumn{1}{l}{{\bf LoF(L)}} & \multicolumn{1}{l}{{\bf LoF(LP)}} & \multicolumn{1}{l}{{\bf Bias(L)}} \\
1 & 1 & 0 & 0 & 0 & \multicolumn{1}{|r}{19} & 26 & \multicolumn{1}{|r}{99.73} & 89.73 & 97.87 & 87.33 & 74.35 & \multicolumn{1}{|r}{100.00} & 91.16 & 98.41 & 87.86 & 59.89 \\
2 & 0 & 1 & 0 & 0 & \multicolumn{1}{|r}{31} & 14 & \multicolumn{1}{|r}{99.04} & 99.22 & 96.09 & 98.19 & 71.97 & \multicolumn{1}{|r}{98.81} & 100.00 & 96.73 & 98.62 & 54.22 \\
3 & 0 & 0 & 1 & 0 & \multicolumn{1}{|r}{37} & 8 & \multicolumn{1}{|r}{95.78} & 98.71 & 94.27 & 100.00 & 70.60 & \multicolumn{1}{|r}{92.66} & 96.26 & 95.03 & 100.00 & 49.96 \\
4 & 0 & 0 & 0 & 1 & \multicolumn{1}{|r}{0} & 45 & \multicolumn{1}{|r}{65.66} & 0.00 & 97.31 & 0.00 & 100.00 & \multicolumn{1}{|r}{59.61} & 0.00 & 97.15 & 0.00 & 100.00 \\
5 & 0.5 & 0.5 & 0 & 0 & \multicolumn{1}{|r}{31} & 14 & \multicolumn{1}{|r}{99.04} & 99.22 & 96.09 & 98.19 & 71.97 & \multicolumn{1}{|r}{98.81} & 100.00 & 96.73 & 98.62 & 54.22 \\
6 & 0.5 &  & 0.5 & 0 & \multicolumn{1}{|r}{31} & 14 & \multicolumn{1}{|r}{98.75} & 98.94 & 96.22 & 98.32 & 72.13 & \multicolumn{1}{|r}{98.52} & 99.71 & 96.89 & 98.71 & 54.15 \\
7 & 0.5 & 0 & 0 & 0.5 & \multicolumn{1}{|r}{0} & 45 & \multicolumn{1}{|r}{86.68} & 0.00 & 99.57 & 0.00 & 89.16 & \multicolumn{1}{|r}{84.22} & 0.00 & 99.49 & 0.00 & 85.90 \\
8 & 0 & 0.5 & 0.5 & 0 & \multicolumn{1}{|r}{34} & 11 & \multicolumn{1}{|r}{95.57} & 97.22 & 95.07 & 99.12 & 70.81 & \multicolumn{1}{|r}{95.69} & 98.24 & 95.64 & 99.42 & 51.68 \\
9 & 0 & 0.5 & 0 & 0.5 & \multicolumn{1}{|r}{25} & 20 & \multicolumn{1}{|r}{88.05} & 84.64 & 97.58 & 94.55 & 82.70 & \multicolumn{1}{|r}{85.29} & 83.00 & 98.10 & 95.28 & 74.70 \\
10 & 0 & 0 & 0.5 & 0.5 & \multicolumn{1}{|r}{25} & 20 & \multicolumn{1}{|r}{60.19} & 57.86 & 95.58 & 92.61 & 91.46 & \multicolumn{1}{|r}{52.38} & 50.97 & 95.64 & 93.69 & 88.62 \\
11 & 0.25 & 0.25 & 0.25 & 0.25 & \multicolumn{1}{|r}{25} & 20 & \multicolumn{1}{|r}{98.80} & 94.97 & 97.10 & 94.08 & 74.98 & \multicolumn{1}{|r}{98.18} & 95.54 & 97.75 & 94.77 & 61.57 \\
12 & 1/3 & 1/3 & 1/3 & 0 & \multicolumn{1}{|r}{31} & 14 & \multicolumn{1}{|r}{98.70} & 98.88 & 96.18 & 98.28 & 72.06 & \multicolumn{1}{|r}{98.50} & 99.69 & 96.88 & 98.66 & 53.98 \\
13 & 1/3 & 1/3 & 0 & 1/3 & \multicolumn{1}{|r}{23} & 22 & \multicolumn{1}{|r}{97.20} & 91.74 & 97.79 & 92.58 & 77.01 & \multicolumn{1}{|r}{95.96} & 91.73 & 98.34 & 93.30 & 65.26 \\
14 & 0 & 1/3 & 1/3 & 1/3 & \multicolumn{1}{|r}{28} & 17 & \multicolumn{1}{|r}{88.98} & 87.50 & 97.09 & 96.85 & 80.91 & \multicolumn{1}{|r}{86.52} & 86.05 & 97.66 & 97.44 & 70.55
\end{tabular}
}
\end{table}
\end{landscape}


\begin{table}[h]
\centering
\caption{Example 2. LoF(DP)- and LoF(LP)-optimal design}
\label{tab::Ex2_LoFDP_design}
\scalebox{0.8}{
\begin{tabular}{rrrrr|r|rrrrr|r|rrrrr|r|rrrrr}
1  & -1 & -1  & -1   & -1 &  & 16 & -0.5 & -1   & -1 & 1    &  & 31 & 0.5 & -1 & -1 & -0.5 &  & 46 & 1 & -1 & -1 & 0  \\
2  & -1 & -1  & -1   & -1 &  & 17 & -0.5 & -1   & -1 & 1    &  & 32 & 0.5 & -1 & -1 & -0.5 &  & 47 & 1 & -1 & -1 & 0  \\
3  & -1 & 0   & -1   & 1  &  & 18 & -0.5 & -1   & -1 & 1    &  & 33 & 0.5 & 0  & -1 & 1    &  & 48 & 1 & -1 & -1 & 1  \\
4  & -1 & 0   & -1   & 1  &  & 19 & -0.5 & -0.5 & -1 & -0.5 &  & 34 & 0.5 & 0  & -1 & 1    &  & 49 & 1 & -1 & -1 & 1  \\
5  & -1 & 0   & -1   & 1  &  & 20 & -0.5 & -0.5 & -1 & -0.5 &  & 35 & 0.5 & 0  & -1 & 1    &  & 50 & 1 & -1 & -1 & 1  \\
6  & -1 & 0   & -1   & 1  &  & 21 & -0.5 & -0.5 & -1 & -0.5 &  & 36 & 0.5 & 1  & -1 & 0    &  & 51 & 1 & -1 & 1  & -1 \\
7  & -1 & 0   & -0.5 & 0  &  & 22 & -0.5 & -0.5 & -1 & -0.5 &  & 37 & 0.5 & 1  & -1 & 0    &  & 52 & 1 & -1 & 1  & 1  \\
8  & -1 & 0.5 & -1   & 0  &  & 23 & 0    & -0.5 & -1 & 1    &  & 38 & 0.5 & 1  & -1 & 0    &  & 53 & 1 & 0  & -1 & -1 \\
9  & -1 & 0.5 & -1   & 0  &  & 24 & 0    & -0.5 & -1 & 1    &  & 39 & 0.5 & 1  & -1 & 0    &  & 54 & 1 & 0  & -1 & -1 \\
10 & -1 & 0.5 & -1   & 0  &  & 25 & 0    & -0.5 & -1 & 1    &  & 40 & 1   & -1 & -1 & -1   &  & 55 & 1 & 0  & -1 & -1 \\
11 & -1 & 0.5 & -0.5 & 0  &  & 26 & 0    & 0.5  & -1 & -1   &  & 41 & 1   & -1 & -1 & -1   &  & 56 & 1 & 1  & -1 & -1 \\
12 & -1 & 1   & -1   & 1  &  & 27 & 0    & 0.5  & -1 & -1   &  & 42 & 1   & -1 & -1 & -1   &  & 57 & 1 & 1  & -1 & -1 \\
13 & -1 & 1   & -1   & 1  &  & 28 & 0    & 0.5  & -1 & -1   &  & 43 & 1   & -1 & -1 & -1   &  & 58 & 1 & 1  & -1 & -1 \\
14 & -1 & 1   & 0.5  & -1 &  & 29 & 0.5  & -1   & -1 & -0.5 &  & 44 & 1   & -1 & -1 & 0    &  & 59 & 1 & 1  & -1 & 1  \\
15 & -1 & 1   & 0.5  & 1  &  & 30 & 0.5  & -1   & -1 & -0.5 &  & 45 & 1   & -1 & -1 & 0    &  & 60 & 1 & 1  & -1 & 1 
\end{tabular}
}
\end{table}




\newpage
\section{Blocked Experiments}
\label{sec::gen_blocked}
%% Generalised criteria for blocked experiments 
\subsection{Generalised criteria}
In order to adapt the generalised criteria (\ref{eq::GDP_eff}) and (\ref{eq::GLP_eff}) for blocked experiments, we need to consider each component taking into account that each potential design point as a set of factors' values is now also included in one of the blocks.

Reviewing the full model for a blocked experiment from Section \ref{sec::back_blocked}, equation (\ref{eq::blocked_model})
\begin{align*}
\bm{Y}=\bm{Z\beta}_{B}+\bm{X}_{1}\bm{\beta}_{1}+\bm{X}_{2}\bm{\beta}_{2}+\bm{\varepsilon},
\end{align*}
denote the $n\times(b+p)$ model matrix of the block and primary terms: 
$\tilde{\bm{X}}_{1}=[\bm{Z},\bm{X}_{1}]$, where  columns of $\bm{Z}$ contain block indicators, and columns of $\bm{X}_{1}$ correspond to primary terms. $\bm{\beta}_{B}$ is the vector of fixed block effects, and $\bm{\beta}_1$ and $\bm{\beta}_2$ contain, as before, primary and potential model terms.

It was shown in (\ref{eq::DPs_blocked}) and (\ref{eq::LPs_blocked}) how the $DP$- and $LP$-criteria are evaluated due to the adjustments:
\begin{align*}
DP_S: \mbox{minimise } &(F_{p-1,d_B;1-\alpha_{DP}})^{p-1}\vert (\bm{X}'_{1}\bm{Q}\bm{X}_{1})^{-1}\vert,\notag\\
LP_S: \mbox{minimise } &F_{1,d_B;1-\alpha_{LP}}\mbox{tr}\{\bm{W}(\bm{X}'_{1}\bm{Q}\bm{X}_{1})^{-1}\}.
\end{align*}
As was outlined before, the number of pure error degrees of freedom is now $d_B=n-\mbox{rank}(\bm{Z}:\bm{T})$, where $\bm{T}$ is the treatment matrix, i.e. the number of replicates adjusted for the inter-block comparisons.

We can construct the amended posterior information matrix (up to a multiple of $\sigma^2)$:
\begin{align*}
\bm{M_B}=
\begin{pmatrix}
\bm{Z}'\bm{Z} & \bm{Z}'\bm{X}_{1} & \bm{Z}'\bm{X}_{2}\\
\bm{X}'_{1}\bm{Z} & \bm{X}'_{1}\bm{X}_{1} & \bm{X}'_{1}\bm{X}_{2}\\
\bm{X}'_{2}\bm{Z} & \bm{X}'_{2}\bm{X}_{1} & \bm{X}'_{2}\bm{X}_{2}+\bm{I}_{q}/\tau^2
\end{pmatrix}
=
\begin{pmatrix}
\bm{\tilde{X}}'_{1}\bm{\tilde{X}}_{1} & \bm{\tilde{X}}'_{1}\bm{X}_{2}\\
\bm{X}'_{2}\bm{\tilde{X}}_{1} & \bm{X}'_{2}\bm{X}_{2}+\bm{I}_{q}/\tau^2
\end{pmatrix}_{.}
\end{align*} 

For the lack-of-fit components the submatrix of the posterior variance-covariance matrix corresponding to the potential terms, $\bm{\tilde{\Sigma}}_{22}=\sigma^2[\bm{M}^{-1}_B]_{22},$ needs to be derived: 
\begin{align*}
\bm{\tilde{\Sigma}}_{22}&=\sigma^2([\bm{M_B}]_{22}-[\bm{M_B}]_{21}([\bm{M_B}]_{11})^{-1}[\bm{M_B}]_{12})^{-1}\\
&=\sigma^2(\bm{X}'_{2}\bm{X}_{2}+\bm{I}_{q}/\tau^2-\bm{X}'_{2}\bm{\tilde{X}}_{1}(\bm{\tilde{X}}'_{1}\bm{\tilde{X}}_{1})^{-1}\bm{\tilde{X}}'_{1}\bm{X}_{2})^{-1}\\&=\sigma^2\left(\bm{\tilde{L}}+\frac{\bm{I}_{q}}{\tau^2}\right), \mbox{ where }\bm{\tilde{L}}=\bm{X}'_{2}\bm{X}_{2}-\bm{X}'_{2}\bm{\tilde{X}}_{1}(\bm{\tilde{X}}'_{1}\bm{\tilde{X}}_{1})^{-1}\bm{\tilde{X}}'_{1}\bm{X}_{2}.
\end{align*}

Therefore, these components can be adjusted for blocked experiments by replacing the primary terms matrix $\bm{X}_{1}$ by the extended matrix $\bm{\tilde{X}}_{1}$ and the dispersion matrix $\bm{L}$ -- by $\bm{\tilde{L}}$ as obtained above. 

In order to adapt the criterion component corresponding to the prediction bias, we need to state a clear notion of a point at which a prediction can be made; particularly, in the context of a blocked experiment a careful definition of what it means for any design point to be included in one of the blocks is required. 

Treating the blocks formed in the experiment as a sample from a certain population of blocks, for any combination of factor settings $\bm{x}_1$ (which is essentially a design point), we can define a vector $\bm{x}_0$. The $i$-th element of $\bm{x}_0$ is a random variable -- `indicator that this point $\bm{x}_1$ is in the $i$-th block', $\bm{x}_0=(x_{o1},...,x_{0b})'$. The expectation of such a random variable is $\textbf{E}(x_{0i})=1/b.$ Also denote $\tilde{\bm{x}}'_{1}=c(\bm{x}'_0,\bm{x}'_1)$ and $\tilde{\bm{\beta}}'_{1}=c(\bm{\beta}'_B,\bm{\beta}'_1).$

The true relationship of interest represented according to the `full' model form is
\begin{equation*}
\eta(\bm{x})=\bm{x}'_0\bm{\beta}_{B}+\bm{x}'_1\bm{\beta}_{1}+\bm{x}'_2\bm{\beta}_{2}=\tilde{\bm{x}}'_1\tilde{\bm{\beta}}_{1}+\bm{x}'_2\bm{\beta}_{2}.
\end{equation*}
The fitted value at $x_{1}$ and its expectation with respect to the error distribution are 
\begin{equation*}
\hat{y}(\bm{x}_1)=\tilde{\bm{x}}'_{1}\tilde{\bm{\beta}}_{1}=\tilde{\bm{x}}'_{1}(\tilde{\bm{X}}'_{1}\tilde{\bm{X}}_{1})^{-1}\tilde{\bm{X}}'_{1}\bm{Y} \\
\end{equation*}
and
\begin{equation*}
\textbf{E}_{\varepsilon}[\hat{y}(\bm{x}_1)]=\tilde{\bm{x}}'_1(\tilde{\bm{X}}'_{1}\tilde{\bm{X}}_{1})^{-1}\tilde{\bm{X}}'_{1}\textbf{E}_{\varepsilon}\bm{Y}=\tilde{\bm{x}}'_1(\tilde{\bm{X}}'_{1}\tilde{\bm{X}}_{1})^{-1}\tilde{\bm{X}}'_{1}(\tilde{\bm{X}}_{1}\tilde{\bm{\beta}}_{1}+\bm{X}_{2}\bm{\beta}_{2}).
\end{equation*}
Therefore, the bias component of the integrated mean squared error (\ref{eq::IMSE}) is 
\begin{equation*}
\textbf{E}_{\mu}\{\textbf{E}_{\varepsilon}[\eta(\bm{x})-\textbf{E}_{\varepsilon}[\hat{y}(\bm{x}_1)]]^2\}=\textbf{E}_{\mu}[\bm{x}'_2\bm{\beta}_{2}-\tilde{\bm{x}}'_{1}\tilde{\bm{A}}\bm{\beta}_{2}]^2,
\end{equation*}
where
\begin{equation*}
\tilde{\bm{A}}=(\tilde{\bm{X}}'_{1}\tilde{\bm{X}}_{1})^{-1}\tilde{\bm{X}}'_{1}\bm{X}_{2}.
\end{equation*}
Further:
\begin{align}
\label{eq::blocks_bias}
\textbf{E}_{\mu}[\bm{x}'_2\bm{\beta}_{2}-\tilde{\bm{x}}'_{1}\tilde{\bm{A}}\bm{\beta}_{2}]^2=&  \bm{\beta}'_{2}\textbf{E}_{\mu}[(\bm{x}'_{2}-\tilde{\bm{x}}'_{1}\tilde{\bm{A}})'(\bm{x}'_{2}-\tilde{\bm{x}}'_{1}\tilde{\bm{A}})]\bm{\beta}_{2}\notag \\=& \bm{\beta}'_{2}[\tilde{\bm{A}}'\bm{\mu_{11}}\tilde{\bm{A}}-\tilde{\bm{A}}'\bm{\mu_{12}}-\bm{\mu_{21}}\tilde{\bm{A}}+\bm{\mu_{22}}]\bm{\beta}_{2}\notag\\=& \bm{\beta}'_{2}[\tilde{\bm{A}}'\tilde{\bm{A}}+\bm{I}_{q}]\bm{\beta}_{2}.
\end{align}
The last transition takes place when columns of the full candidate matrix are orthonormalised: $\bm{\mu_{12}}=\bm{\mu_{21}}=\bm{0}$, $\bm{\mu_{11}}=\bm{I}_{p}$, $\bm{\mu_{22}}=\bm{I}_{q}$. That is equivalent to orthonormalising $[\bm{X}_{1},\bm{X}_{2}]$ as in the unblocked case and normalising the columns of the candidate matrix which correspond to blocks, i.e. setting non-zero elements to $1/\sqrt{N}$, where $N$ is the number of candidate points; the total number of rows in the `blocked' candidate matrix is equal to $N	\times b$.

Summarising all the components in the respective criteria, and presenting them in conformity with the notion of efficiency, we get the following generalised criteria:

Generalised $D$-criterion:
\begin{align}
\label{eq::GD_eff_blocks}
\mbox{ minimise } \vert (\bm{X}'_{1}\bm{Q}\bm{X}_{1})^{-1}\vert^{\frac{\kappa_D}{p}}\times \left|\left(\bm{\tilde{L}}+\frac{\bm{I}_{q}}{\tau^{2}}\right)^{-1}\right|^{\frac{\kappa_{LoF}}{q}}\times |\bm{\tilde{A}}'\bm{\tilde{A}}+\bm{I}_{q}|^{\frac{\kappa_{bias}}{q}};
\end{align}
Generalised $L$-criterion:
\begin{multline}
\label{eq::GL_eff_blocks}
\mbox{minimise } \left[\frac{1}{p}\mbox{trace}(\bm{W}\bm{X}'_{1}\bm{Q}\bm{X}_{1})^{-1}\right]^{\kappa_{L}}\times \left[\frac{1}{q}\mbox{trace}\left(\bm{\tilde{L}}+\frac{\bm{I}_{q}}{\tau^{2}}\right)^{-1}\right]^{\kappa_{LoF}}\times\\ \left[\frac{1}{q}\mbox{trace}(\bm{\tilde{A}}'\bm{\tilde{A}}+\bm{I}_{q})\right]^{\kappa_{bias}};
\end{multline}
Generalised $DP$-criterion:
\begin{multline}
\label{eq::GDP_eff_blocks}
\mbox{minimise } \vert (\bm{X}'_{1}\bm{Q}\bm{X}_{1})^{-1}\vert^{\frac{\kappa_D}{p}}\times \left[\left|(\bm{X}'_{1}\bm{Q}\bm{X}_{1})^{-1}\right|^{1/p}F_{p,d_B;1-\alpha_{DP}}\right]^{\kappa_{DP}} \times \\ \left[\left|\bm{\tilde{L}}+\frac{\bm{I}_{q}}{\tau^{2}}\right|^{-1/q}F_{q,d_B;1-\alpha_{LoF}}\right]^{\kappa_{LoF}} \times |\bm{\tilde{A}}'\bm{\tilde{A}}+\bm{I}_{q}|^{\frac{\kappa_{bias}}{q}};
\end{multline}
Generalised $LP$-criterion:
\begin{multline}
\label{eq::GLP_eff_blocks}
\mbox{minimise } \left[\frac{1}{p}\mbox{trace}(\bm{WX}'_{1}\bm{Q}\bm{X}_{1})^{-1}\right]^{\kappa_{L}}\times\left[\frac{1}{p}\mbox{trace}(\bm{WX}'_{1}\bm{Q}\bm{X}_{1})^{-1}F_{1,d_B;1-\alpha_{LP}}\right]^{\kappa_{LP}}\times \\ \left[\frac{1}{q}\mbox{trace}\left(\bm{\tilde{L}}+\frac{\bm{I}_{q}}{\tau^{2}}\right)^{-1}F_{1,d_B;1-\alpha_{LoF}}\right]^{\kappa_{LoF}}\times\left[\frac{1}{q}\mbox{trace}(\bm{\tilde{A}}'\bm{\tilde{A}}+\bm{I}_{q})\right]^{\kappa_{bias}}_{.}
\end{multline}
The non-negative weights $\kappa$ in each criterion are chosen such that the sum is equal to $1$. Significance levels $\alpha_{LP}$ and $\alpha_{LoF}$ in Generalised $LP$-criterion are corrected for multiple testing, as was described in (\ref{eq::Sidak}). Matrix $\bm{W}$ comprising weights for the $L$-criteria is the same as in the unblocked case (see Section \ref{seq::primary_criteria}, page \pageref{W_matrix}).

\subsection{Example}

We continue studying the example introduced earlier: five three-level factors in $40$ runs, the fitted model is the full quadratic polynomial, with no intercept due to the presence of block effects, so that the number of primary terms is $p=20$. Now we assume that the experimental units are organised into $b=5$ blocks of equal size, that is $8$ units per block. Therefore, the total number of available degrees of freedom is $n-(p+b)=15$ (as the intercept is not included in the fitted model). The potential model misspecification is represented by all third-order terms: linear-by-linear-by-linear terms and quadratic-by-linear; there are $q=30$ of them in total. The variance scaling parameter $\tau^2$ is set to be equal to $1$. The implemented point exchange algorithm with $200$ random starts took $5$--$11$ hours to provide the (near)-optimal designs. 

As in the unblocked case, $GD$- and $GL$-optimal designs have no pure error degrees of freedom (the designs' efficiencies are given in Table \ref{tab::GD_b}); what is notable is that the $D$-optimal design is also $L$-optimal, the $LoF(D)$-optimal design is also $LoF(L)$-optimal and the same equality holds for bias-optimality in terms of determinant- and trace-based criteria. Allocating all of the weight to lack-of-fit and bias components only results in the lowest $D$- and $L$-efficiencies (around $55\%$ and $10\%$ accordingly), which indicates quite strong conflict between the `primary' and `potential' parts of the compound criteria.

\begin{table}[h]
\caption{Properties of generalised D- and L-optimal blocked designs}
\label{tab::GD_b}
\resizebox{\textwidth}{!}}                               \\
   & \textbf{D}       & \textbf{LoF(D)}    & \textbf{Bias(D)}   & \textbf{PE}        & \textbf{LoF}        & \textbf{D}   & \textbf{LoF(D)}   & \textbf{Bias(D)}  & \textbf{L}       & \textbf{LoF(L)}   & \textbf{Bias(L)}  \\
1 & 1 & 0 & 0 & \multicolumn{1}{|r}{0} & 15 & \multicolumn{1}{|r}{100.00} & 96.29 & 59.23 & \multicolumn{1}{|r}{100.00} & 95.78 & 41.20 \\
2 & 0 & 1 & 0 & \multicolumn{1}{|r}{0} & 15 & \multicolumn{1}{|r}{86.09} & 100.00 & 85.72 & \multicolumn{1}{|r}{79.93} & 100.00 & 79.61 \\
3 & 0 & 0 & 1 & \multicolumn{1}{|r}{0} & 15 & \multicolumn{1}{|r}{59.16} & 99.41 & 100.00 & \multicolumn{1}{|r}{10.41} & 99.34 & 100.00 \\
4 & 0.5 & 0.5 & 0 & \multicolumn{1}{|r}{0} & 15 & \multicolumn{1}{|r}{99.22} & 96.67 & 59.91 & \multicolumn{1}{|r}{99.16} & 96.20 & 41.40 \\
5 & 0.5 & 0 & 0.5 & \multicolumn{1}{|r}{0} & 15 & \multicolumn{1}{|r}{86.83} & 99.94 & 85.36 & \multicolumn{1}{|r}{78.05} & 99.94 & 60.24 \\
6 & 0 & 0.5 & 0.5 & \multicolumn{1}{|r}{0} & 15 & \multicolumn{1}{|r}{55.17} & 97.89 & 90.89 & \multicolumn{1}{|r}{13.54} & 97.84 & 79.45 \\
7 & 1/3 & 1/3 & 1/3 & \multicolumn{1}{|r}{0} & 15 & \multicolumn{1}{|r}{85.74} & 99.17 & 81.43 & \multicolumn{1}{|r}{75.57} & 99.17 & 60.48 \\
8 & 0.5 & 0.25 & 0.25 & \multicolumn{1}{|r}{0} & 15 & \multicolumn{1}{|r}{92.99} & 99.53 & 76.52 & \multicolumn{1}{|r}{88.36} & 99.48 & 51.42 \\
9 & 0.25 & 0.5 & 0.25 & \multicolumn{1}{|r}{0} & 15 & \multicolumn{1}{|r}{86.48} & 99.06 & 81.97 & \multicolumn{1}{|r}{78.29} & 99.11 & 51.69 \\
 &  &  &  &  &  &  &  &  &  &  &  \\
 & \multicolumn{3}{l}{\textbf{Criteria}} & \multicolumn{2}{l}{\textbf{DoF}} & \multicolumn{6}{l}{\textbf{Efficiency,\%}}                               \\
   & \textbf{L}       & \textbf{LoF(L)}    & \textbf{Bias(L)}   & \textbf{PE}        & \textbf{LoF}        & \textbf{D}   & \textbf{LoF(D)}   & \textbf{Bias(D)}  & \textbf{L}       & \textbf{LoF(L)}   & \textbf{Bias(L)}  \\
1 & 1 & 0 & 0 & \multicolumn{1}{|r}{0} & 15 & \multicolumn{1}{|r}{98.88} & 96.52 & 58.39 & \multicolumn{1}{|r}{100.00} & 96.01 & 40.44 \\
2 & 0 & 1 & 0 & \multicolumn{1}{|r}{0} & 15 & \multicolumn{1}{|r}{86.09} & 100.00 & 85.72 & \multicolumn{1}{|r}{79.93} & 100.00 & 79.61 \\
3 & 0 & 0 & 1 & \multicolumn{1}{|r}{0} & 15 & \multicolumn{1}{|r}{59.16} & 99.41 & 100.00 & \multicolumn{1}{|r}{10.41} & 99.34 & 100.00 \\
4 & 0.5 & 0.5 & 0 & \multicolumn{1}{|r}{0} & 15 & \multicolumn{1}{|r}{98.23} & 96.65 & 58.27 & \multicolumn{1}{|r}{99.07} & 96.22 & 38.07 \\
5 & 0.5 & 0 & 0.5 & \multicolumn{1}{|r}{0} & 15 & \multicolumn{1}{|r}{85.71} & 99.53 & 83.49 & \multicolumn{1}{|r}{79.37} & 99.43 & 76.25 \\
6 & 0 & 0.5 & 0.5 & \multicolumn{1}{|r}{0} & 15 & \multicolumn{1}{|r}{51.20} & 98.19 & 97.44 & \multicolumn{1}{|r}{17.93} & 98.04 & 95.82 \\
7 & 1/3 & 1/3 & 1/3 & \multicolumn{1}{|r}{0} & 15 & \multicolumn{1}{|r}{86.47} & 98.54 & 77.00 & \multicolumn{1}{|r}{78.36} & 98.31 & 67.41 \\
8 & 0.5 & 0.25 & 0.25 & \multicolumn{1}{|r}{0} & 15 & \multicolumn{1}{|r}{93.09} & 97.62 & 66.27 & \multicolumn{1}{|r}{91.01} & 97.30 & 55.44 \\
9 & 0.25 & 0.5 & 0.25 & \multicolumn{1}{|r}{0} & 15 & \multicolumn{1}{|r}{86.09} & 100.00 & 85.72 & \multicolumn{1}{|r}{79.93} & 100.00 & 79.61
\end{tabular}
}
\end{table}

Some of the $GD$- and $GL$-optimal designs are similar in terms of efficiencies, which is true to some extent: e.g. designs \#$4$, \#$5$, \#$7$ and \#$8$ are alike, but their bias$(L)$-efficiency values tend to differ quite substantially. Again, as was previously observed, the designs that were obtained by assigning most of the weight to either bias or lack-of-fit parts are at least moderately bias-efficient.

As for $GDP$- and $GLP$-optimal designs, we can see designs which have all degrees of freedom allocated to the pure error: $GDP$-optimal designs \#$8$ and \#$12$ and $LoF(LP)$-optimal design in Table \ref{tab::GDP_b}. The first two are also $100\%$ $LoF(LP)$-efficient and at the same time they perform much better with respect to the other primary criteria ($D$, $L$, $DP$, $LP$) than the design that was obtained with all the weight put on the $LoF(LP)$-component. The latter is presented in Table \ref{tab::LoFLPB_design}: its points seem to be scattered around the experimental region without any particular pattern, more replicates occur between rather than within blocks; besides that there is not anything particular in this design's appearance that would suggest its poor performance with respect to both $L$/$LP$ and bias trace-based components.

%% GD- and GL-optimal designs

%%%%%%%%%

%GDP- and GLP-optimal designs, blocked
\begin{landscape}
\begin{table}[p]
\caption{Properties of generalised DP- and LP-optimal blocked designs}
\label{tab::GDP_b}
\resizebox{\linewidth}{!}} \\
\multicolumn{1}{l}{} & \multicolumn{1}{r}{{\bf D}} & \multicolumn{1}{r}{{\bf DP}} & \multicolumn{1}{r}{{\bf LoF(DP)}} & \multicolumn{1}{r}{{\bf Bias(D)}} & \multicolumn{1}{r}{{\bf PE}} & \multicolumn{1}{r}{{\bf LoF}} & \multicolumn{1}{r}{{\bf D}} & \multicolumn{1}{r}{{\bf DP}} & \multicolumn{1}{r}{{\bf LoF(D)}} & \multicolumn{1}{r}{{\bf LoF(DP)}} & \multicolumn{1}{r}{{\bf Bias(D)}} & \multicolumn{1}{r}{{\bf L}} & \multicolumn{1}{r}{{\bf LP}} & \multicolumn{1}{r}{{\bf LoF(L)}} & \multicolumn{1}{r}{{\bf LoF(LP)}} & \multicolumn{1}{r}{{\bf Bias(L)}} \\
1 & 1 & 0 & 0 & 0 & \multicolumn{1}{|r}{0} & 15 & \multicolumn{1}{|r}{100.00} & 0.00 & 96.29 & 0.00 & 59.23 & \multicolumn{1}{|r}{100.00} & 0.00 & 95.78 & 0.00 & 41.20 \\
2 & 0 & 1 & 0 & 0 & \multicolumn{1}{|r}{15} & 0 & \multicolumn{1}{|r}{89.58} & 100.00 & 87.68 & 97.66 & 47.71 & \multicolumn{1}{|r}{5.08} & 6.04 & 86.62 & 97.57 & 31.31 \\
3 & 0 & 0 & 1 & 0 & \multicolumn{1}{|r}{14} & 1 & \multicolumn{1}{|r}{38.44} & 42.92 & 89.78 & 100.00 & 39.17 & \multicolumn{1}{|r}{9.58} & 11.38 & 90.14 & 98.99 & 9.31 \\
4 & 0 & 0 & 0 & 1 & \multicolumn{1}{|r}{0} & 15 & \multicolumn{1}{|r}{59.16} & 0.00 & 99.41 & 0.00 & 100.00 & \multicolumn{1}{|r}{10.41} & 0.00 & 99.34 & 0.00 & 100.00 \\
5 & 0.5 & 0.5 & 0 & 0 & \multicolumn{1}{|r}{12} & 3 & \multicolumn{1}{|r}{92.03} & 96.45 & 88.53 & 92.29 & 47.90 & \multicolumn{1}{|r}{85.86} & 96.14 & 87.52 & 91.25 & 32.97 \\
6 & 0.5 &  & 0.5 & 0 & \multicolumn{1}{|r}{14} & 1 & \multicolumn{1}{|r}{89.10} & 99.47 & 87.43 & 97.39 & 48.66 & \multicolumn{1}{|r}{83.01} & 99.99 & 86.34 & 97.32 & 32.98 \\
7 & 0.5 & 0 & 0 & 0.5 & \multicolumn{1}{|r}{0} & 15 & \multicolumn{1}{|r}{86.83} & 0.00 & 99.94 & 0.00 & 85.36 & \multicolumn{1}{|r}{78.05} & 0.00 & 99.94 & 0.00 & 60.24 \\
8 & 0 & 0.5 & 0.5 & 0 & \multicolumn{1}{|r}{15} & 0 & \multicolumn{1}{|r}{84.20} & 96.43 & 87.18 & 99.76 & 44.23 & \multicolumn{1}{|r}{74.02} & 91.64 & 86.08 & 100.00 & 25.84 \\
9 & 0 & 0.5 & 0 & 0.5 & \multicolumn{1}{|r}{12} & 3 & \multicolumn{1}{|r}{81.14} & 85.04 & 89.82 & 93.64 & 55.47 & \multicolumn{1}{|r}{68.42} & 76.89 & 89.10 & 92.32 & 40.58 \\
10 & 0 & 0 & 0.5 & 0.5 & \multicolumn{1}{|r}{11} & 4 & \multicolumn{1}{|r}{48.84} & 49.19 & 90.72 & 90.74 & 60.53 & \multicolumn{1}{|r}{28.70} & 30.83 & 90.12 & 88.65 & 41.99 \\
11 & 0.25 & 0.25 & 0.25 & 0.25 & \multicolumn{1}{|r}{14} & 1 & \multicolumn{1}{|r}{86.25} & 96.29 & 87.68 & 97.67 & 51.04 & \multicolumn{1}{|r}{77.57} & 93.44 & 86.62 & 97.57 & 36.50 \\
12 & 1/3 & 1/3 & 1/3 & 0 & \multicolumn{1}{|r}{15} & 0 & \multicolumn{1}{|r}{86.92} & 99.55 & 87.18 & 99.76 & 44.00 & \multicolumn{1}{|r}{76.98} & 95.30 & 86.08 & 100.00 & 24.36 \\
13 & 1/3 & 1/3 & 0 & 1/3 & \multicolumn{1}{|r}{11} & 4 & \multicolumn{1}{|r}{89.39} & 90.04 & 89.42 & 89.44 & 54.63 & \multicolumn{1}{|r}{85.34} & 91.68 & 88.52 & 87.62 & 41.18 \\
14 & 0 & 1/3 & 1/3 & 1/3 & \multicolumn{1}{|r}{14} & 1 & \multicolumn{1}{|r}{83.78} & 93.53 & 87.91 & 97.92 & 52.55 & \multicolumn{1}{|r}{75.06} & 90.41 & 86.90 & 97.78 & 35.73 \\
 &  &  &  &  &  &  &  &  &  &  &  &  &  &  &  &  \\
\multicolumn{1}{l}{} & \multicolumn{4}{l}{{\bf Criteria}} & \multicolumn{2}{l}{{\bf DoF}} & \multicolumn{10}{l}{{\bf Efficiency,\%}} \\
\multicolumn{1}{l}{} & \multicolumn{1}{r}{{\bf L}} & \multicolumn{1}{r}{{\bf LP}} & \multicolumn{1}{r}{{\bf LoF(LP)}} & \multicolumn{1}{r}{{\bf Bias(L)}}  & \multicolumn{1}{r}{{\bf PE}} & \multicolumn{1}{r}{{\bf LoF}} & \multicolumn{1}{r}{{\bf D}} & \multicolumn{1}{r}{{\bf DP}} & \multicolumn{1}{r}{{\bf LoF(D)}} & \multicolumn{1}{r}{{\bf LoF(DP)}} & \multicolumn{1}{r}{{\bf Bias(D)}} & \multicolumn{1}{r}{{\bf L}} & \multicolumn{1}{r}{{\bf LP}} & \multicolumn{1}{r}{{\bf LoF(L)}} & \multicolumn{1}{r}{{\bf LoF(LP)}} & \multicolumn{1}{r}{{\bf Bias(L)}} \\
1 & 1 & 0 & 0 & 0 & \multicolumn{1}{|r}{0} & 15 & \multicolumn{1}{|r}{98.88} & 0.00 & 96.52 & 0.00 & 58.39 & \multicolumn{1}{|r}{100.00} & 0.00 & 96.01 & 0.00 & 40.44 \\
2 & 0 & 1 & 0 & 0 & \multicolumn{1}{|r}{14} & 1 & \multicolumn{1}{|r}{88.89} & 99.23 & 87.67 & 97.65 & 48.12 & \multicolumn{1}{|r}{84.20} & 100.00 & 86.61 & 97.56 & 32.66 \\
3 & 0 & 0 & 1 & 0 & \multicolumn{1}{|r}{15} & 0 & \multicolumn{1}{|r}{38.54} & 44.14 & 87.18 & 99.76 & 34.55 & \multicolumn{1}{|r}{0.09} & 0.11 & 86.08 & 100.00 & 0.07 \\
4 & 0 & 0 & 0 & 1 & \multicolumn{1}{|r}{0} & 15 & \multicolumn{1}{|r}{59.16} & 0.00 & 99.41 & 0.00 & 100.00 & \multicolumn{1}{|r}{10.41} & 0.00 & 99.34 & 0.00 & 100.00 \\
5 & 0.5 & 0.5 & 0 & 0 & \multicolumn{1}{|r}{11} & 4 & \multicolumn{1}{|r}{92.17} & 92.84 & 89.06 & 89.08 & 47.96 & \multicolumn{1}{|r}{90.02} & 95.23 & 88.09 & 87.34 & 34.03 \\
6 & 0.5 & 0 & 0.5 & 0 & \multicolumn{1}{|r}{14} & 1 & \multicolumn{1}{|r}{88.89} & 99.23 & 87.67 & 97.65 & 48.12 & \multicolumn{1}{|r}{84.20} & 100.00 & 86.61 & 97.56 & 32.66 \\
7 & 0.5 & 0 & 0 & 0.5 & \multicolumn{1}{|r}{0} & 15 & \multicolumn{1}{|r}{85.71} & 0.00 & 99.53 & 0.00 & 83.49 & \multicolumn{1}{|r}{79.37} & 0.00 & 99.43 & 0.00 & 76.25 \\
8 & 0 & 0.5 & 0.5 & 0 & \multicolumn{1}{|r}{14} & 1 & \multicolumn{1}{|r}{85.56} & 95.51 & 87.61 & 97.58 & 49.86 & \multicolumn{1}{|r}{77.64} & 92.21 & 86.54 & 97.50 & 32.65 \\
9 & 0 & 0.5 & 0 & 0.5 & \multicolumn{1}{|r}{11} & 4 & \multicolumn{1}{|r}{88.14} & 88.79 & 89.66 & 89.68 & 55.44 & \multicolumn{1}{|r}{82.72} & 87.51 & 88.82 & 87.82 & 45.28 \\
10 & 0 & 0 & 0.5 & 0.5 & \multicolumn{1}{|r}{11} & 4 & \multicolumn{1}{|r}{54.30} & 54.69 & 89.67 & 89.69 & 59.52 & \multicolumn{1}{|r}{36.23} & 38.32 & 88.80 & 87.85 & 50.37 \\
11 & 0.25 & 0.25 & 0.25 & 0.25 & \multicolumn{1}{|r}{12} & 3 & \multicolumn{1}{|r}{87.86} & 92.08 & 88.77 & 92.54 & 51.53 & \multicolumn{1}{|r}{83.55} & 92.50 & 87.82 & 91.45 & 40.57 \\
12 & 1/3 & 1/3 & 1/3 & 0 & \multicolumn{1}{|r}{14} & 1 & \multicolumn{1}{|r}{88.77} & 99.10 & 87.45 & 97.41 & 48.66 & \multicolumn{1}{|r}{83.71} & 99.41 & 86.36 & 97.34 & 31.81 \\
13 & 1/3 & 1/3 & 0 & 1/3 & \multicolumn{1}{|r}{10} & 5 & \multicolumn{1}{|r}{93.24} & 89.60 & 89.43 & 85.17 & 52.39 & \multicolumn{1}{|r}{91.31} & 91.39 & 88.48 & 82.56 & 41.53 \\
14 & 0 & 1/3 & 1/3 & 1/3 & \multicolumn{1}{|r}{12} & 3 & \multicolumn{1}{|r}{82.16} & 81.65 & 89.11 & 92.90 & 55.21 & \multicolumn{1}{|r}{65.70} & 72.74 & 88.20 & 91.75 & 44.54
\end{tabular}
}
\end{table}
\end{landscape}
In general, we observe a more unbalanced distribution of available degrees of freedom (especially in $GDP$-optimal designs); the overall $D$- and $L$-efficiencies are certainly lower than for the same weight allocations in the previous examples, e.g. $D$-efficiency is around $80-85\%$ in comparison to $90-95\%$ in Example 1 (Table \ref{tab::GDP_ex11}). Bias efficiencies slightly decreased as well; with lack-of-fit components no such tendency has been noticed. 


\begin{table}[h]
\centering
\caption{LoF(LP)-optimal blocked design}
\label{tab::LoFLPB_design}
\scalebox{0.8}{
\begin{tabular}{rrrrrr|r|rrrrrr}
1  & -1 & 1  & 0  & 0  & -1 &  & 25 & -1 & -1 & -1 & 1  & -1 \\
2  & 0  & 1  & -1 & 1  & 0  &  & 26 & -1 & -1 & -1 & 1  & -1 \\
3  & 0  & 1  & -1 & 1  & 0  &  & 27 & -1 & 1  & 0  & 0  & -1 \\
4  & 0  & 1  & 1  & 1  & -1 &  & 28 & -1 & 1  & 0  & 1  & 1  \\
5  & 0  & 1  & 1  & 1  & -1 &  & 29 & 0  & 0  & -1 & -1 & -1 \\
6  & 0  & 1  & 1  & 1  & -1 &  & 30 & 0  & 1  & -1 & 1  & 0  \\
7  & 1  & 1  & 1  & -1 & -1 &  & 31 & 1  & 1  & -1 & -1 & 1  \\
8  & 1  & 1  & 1  & -1 & -1 &  & 32 & 1  & 1  & 1  & -1 & -1 \\ \cline{1-6}\cline{8-13}
9  & -1 & -1 & -1 & 1  & -1 &  & 33 & -1 & -1 & 1  & 0  & 1  \\
10 & -1 & -1 & -1 & 1  & 1  &  & 34 & -1 & 0  & 0  & 1  & 0  \\
11 & -1 & 0  & 1  & 0  & 0  &  & 35 & 0  & 1  & 0  & 0  & 1  \\
12 & -1 & 1  & 0  & 1  & 1  &  & 36 & 1  & -1 & -1 & 1  & -1 \\
13 & 0  & -1 & -1 & -1 & 0  &  & 37 & 1  & -1 & 0  & -1 & -1 \\
14 & 0  & 1  & -1 & 1  & 0  &  & 38 & 1  & -1 & 1  & 1  & -1 \\
15 & 1  & 0  & 0  & 1  & 1  &  & 39 & 1  & -1 & 1  & 1  & -1 \\
16 & 1  & 1  & -1 & -1 & 1  &  & 40 & 1  & 1  & 1  & 1  & 1  \\ \cline{1-6} 
17 & -1 & -1 & 1  & -1 & -1 &  &    &    &    &    &    &    \\
18 & -1 & 0  & 0  & 1  & 0  &  &    &    &    &    &    &    \\
19 & -1 & 1  & 0  & -1 & 0  &  &    &    &    &    &    &    \\
20 & 0  & 1  & 0  & 0  & 1  &  &    &    &    &    &    &    \\
21 & 1  & -1 & 0  & -1 & -1 &  &    &    &    &    &    &    \\
22 & 1  & -1 & 1  & 1  & -1 &  &    &    &    &    &    &    \\
23 & 1  & 0  & 1  & -1 & 1  &  &    &    &    &    &    &    \\
24 & 1  & 1  & 1  & 1  & 1  &  &    &    &    &    &    &   
\end{tabular}
}
\end{table}




\section{Conclusions}
The criteria derived and presented in this chapter incorporate both the precision of the primary terms and the possibility of model misspecification; they were adapted for experiments in blocks as well.

We explored how the designs and their performances change when the prior variance scaling parameter is changed, and what happens when we consider a different number of potential terms in the full model.

Though some specific patterns can be observed in the relationship between the allocation of weights and optimal designs' performance, it still could be seen that components' functions cannot be compensated by each other and, therefore, if one is looking for an efficient design in some sense, the corresponding component should be included in the criterion with non-zero weight. In most cases it would be suggested to examine a few weight combinations and the resulting optimal designs, such that the most suitable could be chosen by the experimenter. 

In addition, as the examples demonstrated, most of the designs are highly $D$- and $L$-efficient, but the variation in efficiency in terms of other criteria can be quite large, that  is, at the stage of planning an experiment, the possible direction of model misspecification should be carefully thought through together with the main model which will be fitted. The scale of the `tuning' parameter $\tau^2$, however, does not seem to have a great influence, though, obviously, the importance of its choice should not be neglected.

Regarding the practical implementation of the algorithm, it provides the results in reasonable time even when no additional computational power is used. However, the necessity of the orthonormalisation of the candidate set of points, arising from the derivation of the bias criteria components, makes the adaptation of the generalised criteria to, for example, experimental frameworks with nested unit structures, a non-trivial problem. Ignoring such a requirement would amend the formulae of the bias components, but in general it should not be expected to considerably influence the resulting optimal designs and their performances; though the orthonormalisation step should not be blindly omitted, especially if the prediction bias is of a particular interest. 
      