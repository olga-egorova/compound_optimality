%% Log-file of the corrections made

\documentclass[11pt,a4paper,oneside]{article}
\usepackage[T1]{fontenc}
\usepackage[latin9]{inputenc}
\usepackage[USenglish]{babel}
\usepackage{graphicx} % Allows including images
%\usepackage{booktabs}
\usepackage{amsmath}
\usepackage{bm}
\usepackage{color, soul}
\usepackage{xcolor}
\usepackage{multicol}
\usepackage{parskip}
\usepackage[authoryear,round]{natbib}
\usepackage[applemac]{inputenx}
\usepackage{caption} \captionsetup[table]{skip=5pt}
%\setlength{\parskip}{\baselineskip}
\usepackage{setspace}
\usepackage{gensymb}
\usepackage{lscape}    %changing page orientation

\begin{document}
\begin{center}
\textbf{\Large{Minor corrections to the thesis}}
\end{center}

\textbf{Abstract}
\begin{enumerate}
\item ``We first present'' (was in past tense)
\item ``we also adapt the criteria for use in blocked experiments'' (was in passive voice)
\end{enumerate}

\textbf{Introduction}
\begin{enumerate}
\item Page 2. ``.. all responses are put in an $n$-dimensional vector $\bm{Y}$''
\item Page 3. Extra ``further'' removed
\end{enumerate}

\textbf{Chapter 2. Background and Related Work}
\begin{enumerate}
\item Page 9. ``.. and derived the $DP$-criterion: minimising $(F_{p,d;1-\alpha_{DP}})^{p}\vert(\bm{X'}\bm{X})^{-1}\vert$'' (was $\nu$ instead of $d$)
\item \hl{Page 10. Expanding the explanation of where L-optimality comes from}: ``If we are interested in the functions of parameters given by $\bm{L}'\bm{\beta}$, then the variance-covariance matrix of $\bm{L}'\bm{\hat{\beta}}$ is...''
\item \hl{Page 10. Explaining why Sidak's corrections were used}: ``Here we shall use Sidak's correction \citep{vsidak1967rectangular}, which is...''
\item Page 10. ``.. for example, $I$-optimality, that is minimising \textbf{average} prediction variance''
\item \hl{Page 11. Rephrasing in order to make a sentence more readable}: "The idea of compound criteria <...>  allows finding a compromise between two or more desirable properties while searching for the design, and explore how the changes in the allocation of weights between the individual criteria affect the optimal designs' performances."
\item Page 11. ``.. prior knowledge of the \textbf{experimenter}'' (misprint)
\item Page 12. ``.. it is sometimes \textbf{necessary} to allocate units in blocks''
\item Page 12. ``.. its $(i,j)^{th}$ element''
\item Page 12. ``.. and $\bm{\beta}_B$ is the vector of block effects''
\item Page 13. ``.. matrix whose elements indicate \textbf{the} treatments''
\item Page 13. ``.. the number of replications after \textbf{subtracting the number of replicates} taken for the estimation of block contrasts.''
\item  \hl{Page 15}.``.. \cite{Goos2001Doptimal} considered the three cases when D-optimal designs for split-plot experiments do not depend on the value of $\eta$; in other practical cases an estimate of the variance ratio is to be provided.'' 
\item  \hl{Page 16}. ``\cite{Arnouts2012staggered} presented a coordinate-exchange algorithm \ldots examples of D- and I-optimal designs.''
\item Page 16. ``.. assumed possible values of $\eta_i$'' (misprint)
\item Page 16. ``.. in the presence'' (misprint)
\item \hl{Page 17. ``In this thesis, due to the primary interest being the quality of inference,..''}
\item Page 18. Removed ``Therefore'' in the 4th paragraph.
\end{enumerate}

\textbf{Chapter 3. Compound criteria. Some amendments}
\begin{enumerate}
\item Page 20. Adding significance levels where they were missed: $F_{n-p-d,d;1-\alpha_{LoF}}$ instead of $F_{n-p-d,d}$, and on Figure 3.1 on page 21.
\item \hl{Page 21. Expanding interpretation of the plot in Figure 3.1}: ``Larger numbers of available residual degrees of freedom \ldots'' 
\item Page 21. ``The third component in both criteria corresponds to the DF-efficiency defined in (2.7).''
\item Pages 22 and 24. Tables 3.1 and 3.4: added columns with DF-efficiency values
\item Page 23. ``..therefore, their $DF$-efficiency values $\frac{n-d}{n}$ are below $75\%$ (i.e. when $d>10$).'' 
\item Page 24. ``..the design optimal <\ldots> (\#$12$) is also optimal with respect to the criterion with equal weights put on the $DF$- and $LoF$-components'' 
\item Page 25. ``..except for the point number $18-19$ and the point number $36-37$''
\item Page 25. ``..the resulting efficiency values. It also would be sensible..'' (breaking the sentence)
\item \hl{Page 27. Adding extra comment}: ``This particular parametrisation <\ldots> designs which are more than $50\%$ efficient.''
\item Page 27. Adding subsection 3.2.2 Optimal designs (no text amendments)
\item Pages 28 and 29.  Tables 3.6 and 3.7: added columns with DF-efficiency values
\item \hl{Page 29. Paraphrasing the last paragraph}: ``It was desired to alter the form of an efficiency's contribution..''
\end{enumerate}

\textbf{Chapter 4. Generalised Compound Criteria}
\begin{enumerate}
\item \hl{Page 31. Adding bullet points in the chapter introduction}
\item Page 32 - 33. Adding subsections 4.1.1 and 4.1.2
\item Page 32. Equation (4.3) -- $\varepsilon$ in the subscripts
\item Page 33. ``..how large the magnitude of each of the potential terms is assumed to \textbf{possibly} be..''
\item Page 34. ``..we incorporate $L$-optimality for the primary terms..''
\item Pages 35 and 36. ``..and (4.11)''
\item Page 35. Adding the reference to Section 3.1.1
\item Page 37. Adding the reference to \cite{DuMouchel1994})
\item !!!\hl{Page 37. Amending the derivation of the formula for the confidence region for $\bm{\beta}_2$ over the posterior distribution (introducing the corresponding F-quantile)}.
\item Page 39. ``Henceforth'' instead of ``Here and further on''
\item Page 40. ``The results should'' instead of ``The results would''
\item Page 43. Table 4.6, $2$nd part, column names: \textbf{L}, \textbf{LP}, \textbf{LoF(LP)} and \textbf{Bias(L)}
\item Page 44. Rephrasing: ``similar tendency seems to be true''
\item \hl{Page 49. Paragraph 3. Adding an explanation regarding large $LoF$-efficiencies} in the case of smaller $\tau^2$: ``Such consistently high performance...''
\item Page 53. ``..each factor can take five levels..''
\item Page 53. ``all third and \textbf{fourth} order terms are taken as potentially missed terms''
\item Page 53. Adding: `` Most of the replicates are at corner points...''
\item Page 58. Changing the name of Section 4.3 (to ``Blocked Experiments'') and adding  subsection 4.3.1 ``Generalised criteria''
\item Page 58. Adding ``$\bm{\beta}_{B}$ is the vector of \textbf{fixed block effects},...'' 
\item Page 59. $\bm{\beta}_{B}$ instead of $\bm{\beta}_{0}$
\item Page 59. ``Therefore, ...'' 
\item Page 60. ``\textbf{the} $D$-optimal design''
\item !!\hl{Page 64. Adding a comment regarding ignoring the orthonormalisation}: ``Ignoring such a requirement would amend the formulae of the bias components, but in general it should not be expected to considerably influence the resulting optimal designs and their performances; though the orthonormalisation step should not be blindly omitted, especially if the prediction bias is of a particular interest.''
\end{enumerate}

\textbf{Chapter 5. MSE-based Criteria}
\begin{enumerate}
\item \hl{Page 66. Adding an explanation of `pseudo-Bayesian'}: ``..(hence we shall refer to such an approach as ``pseudo-Bayesian'')..''
\item Page 67. ``.. and $\bm{u}$, $\bm{v}$ are column vectors''
\item Page 68. Adding a definition of $\bm{\tilde{\beta}}_{2i}$ after the equation (5.5): ``Here $\bm{\tilde{\beta}}_{2i}$, $i=1..N$ -- independent random vectors sampled from $\mathcal{N}(\bm{0},\tau^{2}\sigma^{2}\bm{I}_{q})$.''
\item Page 70. Emphasising the computational advantage of the trace-based criterion: ``The main advantage is the absence of the necessity of any additional numerical evaluations, and...'' 
\item Page 74. Amending the interpretation: ``..the decreased scale of the potentially missed contamination results in \textbf{an easier achievable compromise} between...'' 
\item Page 74. ``.. in the corresponding tables'' 
\item Page 74. Rephrasing: ``..in the case of $MSE(D)$-efficient designs with smaller $\tau^2$.''
\end{enumerate}

\textbf{Chapter 6. MSE-based Criteria for Blocked Experiments}
\begin{enumerate}
\item Page 78. Equation (6.2): $\bm{\beta}_B$ and $\bm{\beta}_1$ instead of $\bm{\tilde{\beta}}_b$ and $\bm{\tilde{\beta}}_p$
\item Page 82. ``in the design \#$8$''
\item \hl{Page 84. Adding a comment regarding relative efficiencies} between the designs optimal w.r.t. the continuous and point-prior-based criteria: ``As for the efficiency losses...''
\item \hl{Page 85. Reason for fixing two centre points per block in the case-study:}``... the experimenters wanted to have at least two centre points in each block to ensure representation of the conditions thought a priori most likely to be best (with dosages of $95\%$ for each supplement)''
\item Page 87. ``$LoF(DP)-$''
\item Page 88. Amending the explanation of the ``Relative efficiency'': ``The last column of this table, ``Relative Efficiency,'' is..''
\item Page 91. Misprint: ``experimenters''
\end{enumerate}

\textbf{Chapter 7. MSE-based Criteria for Multistratum Experiments}
\begin{enumerate}
\item Page 94. Changing \textbf{REML Methodology} from a subsection to section
\item \hl{Page 96}. Changing the title of Section 7.2.2: ``Yates' procedure'' 
\item Page 97. ``$m_j=\ldots$ \textbf{units}''
\item \hl{Page 98}. Adding the \textbf{Construction procedure} subsection title (no text amendments)
\item Page 103. Expanding the sentence: ``Now we need to evaluate the number of degrees of freedom for..''
\item Page 103. ``.. the factors applied at the current and \textbf{all} higher strata''
\item \hl{Page 104. Explaining why $\tau^2=\sqrt{1/q}$ was used}: ``As this number tends to become..''
\item Page 104. ``Regarding the \textbf{third-order} potential terms..''
\item Page 106. ``(design\textbf{s} \#$3$)''
\item Page 106. ``.. designs \#$5$ and \#$7$ provide the worst performance with respect to the $MSE(L)-$ and $LP$-components respectively''
\item Page 109. ``i.e. designs \#$6$ in Table 7.6''
\item Page 111. Adding the \textbf{Discussion} subsection title (no text amendments)
\end{enumerate}

\textbf{Chapter 8. Conclusions and Future Work}
\begin{enumerate}
\item \hl{Page 114. Adding a comment regarding advantages and disadvantages} of the Pareto frontier approach: ``On one hand, such an algorithm would result in a set of designs...''
\end{enumerate}

\bibliographystyle{rss}
\bibliography{thesis_bib}

\end{document}