In the context of a factorial experiment and allowing the possibility of the model assumption violation in a specified direction, we developed compound optimality criteria by incorporating the `pure error' approach, which is model-independence and hence is the most sensible one in such framework. Generalised criteria contain components which are `responsible' for the minimisation of the primary and potential terms' variances and minimisation of the predicted bias. Considering bias of the estimations of the fitted model's coefficients instead leads to the derivation of the MSE-based criteria. 

The examples for both cases demonstrate certain tendencies in the components' relationships and how the resulting designs' efficiencies respond to the changes in the allocation of weights in the criteria. Whilst most of the designs are highly and moderately efficient in terms of the primary criteria corresponding to the primary terms inference, the performance with regards to the lack-of-fit and bias can be quite sensitive to any amendments in both weight allocations and the parameters of the prior distribution. 

As for the combining several elementary criteria in one, a possible alternative could be using a Pareto Frontier approach (described in detail, for example, in \citealp{lu2011optimization}), which would allow making the choice after having observed the best designs in terms of several criteria and not by choosing the weight combination for the compound criteria first.

The need for the a priori specified model contamination form and the prior distribution parameters may become a non-straightforward problem, especially if not much is known about the process under study. It would be interesting to see how the imprecise, incorrect choice of the potential terms would influences the goodness of the resulting designs. The further step could be in introducing a way of relaxing the assumption of such fixed alternative model, however, it is not clear now how this can be done (at least analytically).

Currently the work is going on adapting the MSE-based criteria for blocked experiments; further on it would be good to extend the theory to the cases of more complicated models: non-linear and/or random effects models, which will lead to working with split-plot designs.            