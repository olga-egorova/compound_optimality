\documentclass[11pt]{article}

%%%%%%%%%%%%      PACKAGES        %%%%%%%%%%%%
\usepackage[colorlinks=true,linkcolor=blue,urlcolor=blue,citecolor=blue]{hyperref}
\usepackage{setspace}
\onehalfspacing
%\doublespacing
\usepackage{fancyhdr,afterpage}
\usepackage{lscape}
\usepackage{multirow}
\usepackage{longtable}
\usepackage{textcomp,latexsym}
\usepackage{parskip}
\usepackage[round]{natbib}
\usepackage{adjustbox,lipsum}
\usepackage[titletoc]{appendix}
    
\usepackage[round]{natbib}
\usepackage{amsmath}
\usepackage{graphicx}
\usepackage{gensymb}
\usepackage{tikz}
\usepackage{float}
%\usepackage{slashbox}
\usepackage{caption}
\usepackage{subcaption}
\usepackage{amsfonts}
\usepackage{enumerate}
\usepackage{amssymb}
\usepackage{multirow}
\usepackage{bm}
\setcounter{tocdepth}{3}
\setcounter{secnumdepth}{3}

\DeclareMathOperator*{\argmax}{arg\,max}
\DeclareMathOperator*{\argmin}{arg\,min}

\usepackage[a4paper,inner=4.2cm,outer=2.3cm,top=2.5cm,bottom=2.5cm,pdftex]{geometry} % MARGINS


\begin{document}
\tableofcontents

\newpage
\section*{Abstract}

\section{Introduction}

\textit{
	Model-independent approach to building designs in case of model uncertainty
	\begin{itemize}
		\item[-] Planning a controlled intervention -- reliance on a model (1) +
		\item[-] Active assumption of a specific model misspecification (2) =
		\item[-] Combining conflicting aims: model-based inference based on the model-independent variance estimate and distrust in the model sufficiency
		\item[-] Briefly: extension on more complex experimental structures
	\end{itemize}
}

Designed experiments are commonly conducted in order to gain understanding of the impact that different parameters of interest make on the outcome; the quantitative measure of this impact allows comparisons as well as making interpretable conclusions regarding the scale and the pattern of the relationships between process parameters and measured output. 

While the exact true nature of that relationship remains unknown, some form of approximation is needed -- and polynomial functions are able to provide any required precision for functions from a certain class of differentiability \citep{Rudin1987real}. Response Surface Methodology (RSM, \cite{Box1951Roy}) aims at optimising the approximated functions by fitting second-order polynomials, and higher desirable precision would require a polynomial of a higher order and, therefore, more experimental effort. 

Whichever the chosen, `primary',  model is, planning a controlled intervention does rely on the approximating model assumptions in two main -- and quite contradicting -- directions(?). Firstly, it is desirable to ensure the goodness of what the model is essentially fitted for, that is, for example, the precision of model parameters and/or the prediction accuracy of the untested treatment combinations. The corresponding design aims are usually reflected in utilising well-known optimality criteria (D-, A-, ?prediction) and rely on the variance estimate, the way it is obtained. Secondly, treating the chosen model as the absolute truth, especially at the designing stage is at least too optimistic and, as in many cases fitting a more suitable, but complicated model might not be feasible due to various restrictions and limitations, could be dangerous for the results credibility. So having a particular model also means that at the stage of planning it is highly desirable to include some [sensitivity towards and robustness against]  control over the model lack of fit and its effect on the inference. 

We deal with such duality [of model-dependence and accounting for its misspecification] by developing compound optimality criteria, each constructed as a weighted combination of individual criterion functions, with the two main features:
\begin{enumerate}
	\item Each corresponds to a specific desirable property:  either accounting for an aim coming from trusting the model or mitigating the effects from its potential misfit. The relative importance of the components are reflected by the assigned weights -- and we shall examine the performance of the resulting designs in terms of the individual criteria, and explore the role of weight allocation.
	\item The concept of model-independent internal variance estimation, `pure error' (\cite{GilmourTrinca2012})) underlies each of the individual criteria -- the most appropriate [sensible] strategy in the case of possible model insufficiency. 
 \end{enumerate}

The general spirit of this work aims at aligning with the concept of a good design, which, as summarised by \cite{Box1987empirical},  should ``make it possible to detect lack of fit'' and ``provide an internal estimate of error from replication'', among other properties. 

We will focus on factorial experiments with a relatively small number of runs and the fitted model being a polynomial regression. Section \ref{sec::background} provides the background on the modelling, error estimation and fundamental individual criteria. Controlling the lack-of-fit and the bias arising from the model misspecification are introduced in Section \ref{sec::criteria}, where they are combined with the primary model-driven ones in compound optimality criteria. Their adaptation to experimental frameworks with restricted randomisation is described in Section \ref{sec::ms_experiments}, as well as the design construction procedure for the multistratum experiments. Finally, a series of examples are presented in Section \ref{sec::examples} -- examining dynamics across various optimal designs,  shape of the constructed criteria and other properties, followed up by a Discussion in Section \ref{sec::discussion} with main conclusions and recommendations. 

The $\mbox{R}$ code used to search for the optimal designs are located at <...> [?github for now].

\section{Model-dependent planning}
\label{sec::background}

\textit{	
	\begin{itemize}
		\item[-] Modelling and relevant error estimation 
		\item[-]  Fundamental individual criteria
		\item[-] Combining objectives: lit.review
	\end{itemize}
}

Assuming a smooth enough relationship between $k$ experimental factors $X_1, \dots, X_k$ $\in$ $\Theta \subset \mbox{R}^{k}$ and the response variable $Y=\eta(X_1,\ldots, X_k)$ $\in$ $\mbox{R}$, a suitable polynomial model is chosen to fit data obtained from $n$ experimental runs:

\begin{equation}
\label{eq::back_model}
\bm{Y}=\bm{X\beta}+\bm{\varepsilon}.
\end{equation} 

Here $\bm{X}$ is the $n\times p$ model matrix, $\bm{Y}$ is the $n\times 1$ vector of responses; $\bm{\beta}$ is the $p\times 1$ vector of parameters corresponding to the model terms and $\bm{\varepsilon}$ are independent normally distributed random error terms with constant variance: $\bm{\varepsilon}\sim \mathcal{N}(\bm{0},\sigma^{2}\bm{I}_{n})$. 

Any inference following the model fitting relies on the error variance estimate  $\hat{\sigma}^2$. The most common one is the mean square error: $\hat{\sigma}^2_{mse}=\mbox{Residual SS}/(n-p)$ (e.g. \cite{Draper1998}), which is model-dependent with the residual degrees of freedom $\nu = n-p$ containing the number of model parameters. 

The other one is `pure' error, independent from the parametric model, which is derived from the further decomposition of the residual sum of squares into the `pure' error and `lack-of-fit' components: $\hat{\sigma}^2_{PE}=\mbox{Pure error SS}/(n-t)$, where $t$ is the number of unique treatments (combinations of factors' levels) applied and $d=n-t$ is the pure error degrees of freedom, that is the number of replicated points. In other words, the error is estimated as the mean square error from fitting the full treatment model:

\begin{equation}
\label{eq::treatment_model}
\bm{Y}=\bm{X_{t}\mu_{t}}+\bm{\varepsilon},
\end{equation} 

where $\bm{X_{t}}$ is the $n\times t$ full treatment model matrix, in which the $(i,j)^{th}$ element is equal to $1$ if treatment $j$ is applied to the $i^{th}$ unit, otherwise it is set to $0$. Then the elements of the $t$-dimensional vector $\bm{\mu_{t}}$ are the mean effects of each treatment. The vector of errors $\bm{\varepsilon}$ comprises the between-unit variation, such that $\mbox{E}(\bm{\varepsilon})=\bm{0}$, $\mbox{Var}(\bm{\varepsilon})=\sigma^2\bm{I}_{n}$.

Many authors advocate the use of the `pure' error estimate instead of the one pooled with the lack-of-fit part from the model (\ref{eq::back_model}): \cite{Cox1958planning} recommends using it for the estimation unless there are no replicate treatments, while \cite{Draper1998} argue for the reliability of `pure' error and recommend aiming for the presence of replicates at the stage of planning. \cite{Atkinson2007} also state that ``If lack-of-fit of the model is of potential interest, $\sigma^{2}$ is better estimated from replicate observations''. Finally, the work by \cite{GilmourTrinca2012} that inspired this research, comprises a thorough analysis in favour of estimating the error from the full treatment model which is true regardless of what function is used to approximate the relationship of interest.

% Atkinson -- (page 22)
%[For response surface experiments in blocks \cite{Gilmour2000PErsm} explicated the definition of pure error and its estimate which are compatible with the unblocked case: from replicates but also taking into account  block effects additive to treatment effects.] 

\subsection*{Optimality criteria}

Model-dependency at the stage of experimental planning is reflected in searching for a design that optimises a criterion function, that is a function of a design that translates a specific inference-driven objective. For example, among the most well-known, ``alphabetic'' optimality criteria, such as $D$-, $C$-, $L$-optimality and a series of others (\cite{Atkinson2007}) target the precision of parameters estimates in model (\ref{eq::back_model}); while others, like $G$- and $V$-optimality, deal with the prediction variance. Traditionally these criteria are formulated based on the mean square error estimate; citet{GilmourTrinca2012} derived the alternative pure-error based criteria, which would guarantee the presence of replicates in the resulting designs. 

Combining multiple desirable objectives in the design can be fulfilled through constructing a compound criterion. This concept is based on the notion of design efficiency, which can be defined for any design matrix $X$ and any criterion $F(X)$ as the ratio with respect to the best (e.g. maximum, without the loss of generality) value achieved at the optimal design. For example, the $D$-efficiency of design $\bm{X}$ is
\begin{equation*}
%\label{eq::D_eff}
\mbox{Eff}_{D}(X)=\frac{\vert \bm{X}'\bm{X}\vert^{1/p}}{\vert \bm{X}'_{*}\bm{X}_{*}\vert^{1/p}},
\end{equation*}   
where $\bm{X}_{*}$ is the $D$-optimum design. In this definition the power $1/p$ brings the efficiency to the scale of variances of model coefficients $\bm{\beta}_{i}, i=1\ldots p.$ The efficiency value may vary from $0$ to $1$ and is equal to $1$ if and only if the design is optimal according to the criterion of interest.

The compound criterion to be maximised among all the possible designs is obtained then as weighted product of the individual criteria efficiencies $\mbox{Eff}_{1},\ldots, \mbox{Eff}_{m}$ with the corresponding weights $\kappa_{1},\ldots ,\kappa_{m}$ (s.t. $\kappa_{k}>0$ and $\sum_{k=1}^{m}\kappa_{k}=1$):
\begin{equation}
\label{eq::compound}
\mbox{Eff}^{\kappa_{1}}_{1}(\bm{X})\times\mbox{Eff}^{\kappa_{2}}_{2}(\bm{X})\times\ldots\times\mbox{Eff}^{\kappa_{m}}_{m}(\bm{X})\rightarrow \underset{\bm{X}}\max.
\end{equation}

The choice of weights is arbitrary in general, although is driven by prior knowledge of the experimenter, the objectives of a specific experiment and the components' interpretation. In the examples considered further in this work, we chose a set of weight allocations, most of which are  classical design schemes for experiments with mixtures \citep{Cornell2011Mixtures}. 

%The most general and intuitively sensible recommendation would be to obtain optimal designs with respect to several weight allocations, and then, after examining them, choose the one to be applied. Throughout the course of this research, when considering the examples of experiments layouts, we For example, if the criterion consists of three elements, we would first obtain designs with all the weight on each component, then all combinations of distributing it equally between two components, and finally put equal weights on all of the components. Some additional schemes are included when examining particular cases.  

Some of the alternative approaches of combining several objectives would include generating the Pareto optimal set of designs -- the approach thoroughly described by \cite{Lu2011optimization}.  \cite{Stallings2015general}developed methodology for generalising eigenvalue-based criteria (e.g. $A$- and $E$-optimality) in a way that allows differing interest (expressed through the weights) among any set of estimable functions of the fitted model parameters. The introduced strategy reflects the aims of experimentation that are not traditionally accounted for but definitely are of interest. [Add more references to compound optimality?]

\section{Individual criteria construction}
\label{sec::criteria}
\textit{	
	\begin{itemize}
		\item[-] Model misspecification setup. What objectives to be reflected in criteria: PE
		\item[-] Lack-of-fit criterion (and then the traditional LoF)
		\item[-] MSE criterion
		\item[-] Combining with DP/LP -- the general compound criterion
	\end{itemize}
}

Standard design optimality theory is developed under the assumption that the primary model provides the most proper fit for the data: in many real applications this is quite a strong belief, and in reality we need to take into account at least the possibility that some misspecification is present at the planning stage.
%In Section \ref{sec::back_misspecification} a brief overview of various types of model misspecification is presented. 
%Other types of model misspecification (in background)

In this work we consider the case when the fitted polynomial model with $p$ parameters is nested within a larger model that is assumed to provide a better fit for the data:
\begin{equation}
\label{eq::full_model}
\bm{Y}=\bm{X}_p\bm{\beta}_p+\bm{X}_q\bm{\beta}_q+\bm{\varepsilon},
\end{equation}
where $\bm{X}_q$ is an $n\times q$ extension of the primary model matrix containing extra $q$  terms that we refer to as "potential terms" and that represent the fitted model disturbance, with vector $\bm{\beta}_q$ denoting the corresponding parameters. They are not of any inferential interest, and, moreover, not all of them are necessarily estimable when the experiment is relatively small, i.e. $n<p+q$ -- this is the case we mainly consider here. As usually, we do assume independent and normally distributed error terms: $\bm{\varepsilon}\sim \mathcal{N}(\bm{0},\sigma^{2}\bm{I}_{n})$. 

\subsection*{Lack-of-fit criterion}
 
Aiming towards controlling the magnitude and scope of the potential terms, we adapt Bayesian approach regarding the full model parameters.  Diffuse prior shall be put on primary terms -- an arbitrary mean and a variance going to infinity, and the prior on potential terms is a normal distribution: $\bm{\beta}_q\sim\mathcal{N}(0,\bm{\Sigma}_{0})$, with the prior variance scaled with respect to the error variation:
$\bm{\Sigma}_{0}=\sigma^{2}\tau^{2}\bm{I}_{q}$. Then, following the normality in the model (\ref{eq::full_model}), the posterior distribution of the joint vector of  coefficients $\bm{\beta}  = [\bm{\beta}^T_p, \bm{\beta}^T_q]^T$ is (\cite{Koch2007introduction} is multivariate normal \cite{DuMouchel1994}):
\begin{align*}
\bm{\beta}|\bm{Y} &\sim \mathcal{N}(\bm{b},\bm{\Sigma}), \\
\mbox{where } \bm{b} = \bm{\Sigma X}^T\bm{Y} &\mbox{ and }  \bm{\Sigma} = \sigma^{2}[\bm{X}^T\bm{X} + \bm{K}/\tau^{2}],^{-1}  \bm{X}=[\bm{X}_p, \bm{X}_q],\\
\bm{K} &= \begin{pmatrix}
\bm{0}_{p\times p} & \bm{0}_{p\times q}\\
\bm{0}_{q\times p} & \bm{I}_{q\times q}
\end{pmatrix}.
\end{align*}

The marginal posterior distribution of $\bm{\beta}_q$ is also multivariate normal with mean $\bm{b}_q$ -- the last $q$ elements of $\bm{b}$ and the covariance matrix $\bm{\Sigma}_{qq}$-- the bottom right $q \times q $ submatrix of $\bm{\Sigma}$:
\begin{align*}
\bm{\Sigma}_{qq} = \sigma^{2}[(\bm{X}^T\bm{X} +  \bm{K}/\tau^{2})^{-1}]_{[q,q]} &= 
\sigma^{2}\begin{bmatrix}
\bm{X}^T_p\bm{X}_p& \bm{X}^T_p\bm{X}_q \\
\bm{X}^T_q\bm{X}_p& \bm{X}^T_q\bm{X}_q+\bm{I}_{q}/\tau^{2}
\end{bmatrix}^{-1}_{[q,q]}\\&=
\sigma^{2}[\bm{X}^T_q\bm{X}_q+\bm{I}_{q}/\tau^{2}-\bm{X}^T_q\bm{X}_p(\bm{X}^T_p\bm{X}_p)^{-1}\bm{X}^T_p\bm{X}_q]^{-1}\\&=\sigma^{2}\left[\bm{L}+\bm{I}_{q}/\tau^{2}\right],^{-1}
\end{align*} 
where $\bm{L} = \bm{X}^T_q\bm{X}_q-\bm{X}^T_q\bm{X}_p(\bm{X}^T_p\bm{X}_p)^{-1}\bm{X}^T_p\bm{X}_q$ is known in model-sensitivity design literature as the ``dispersion matrix'' [references], which provides a point-wise measure of the distance between column vectors of $\bm{X}_q$ and linear subspace defined by the column vectors of the primary model matrix $\bm{X}_p$, that is how well each of the potential terms could be approximated by a linear span of the primary ones [?].

Reducing the primary model's lack-of-fit in the direction of the potential terms can be translated into a criterion function of the design by utilising the posterior distribution for $\bm{\beta}_q$ derived above and constructing a $(1-\alpha)\times100\%$ confidence region for the parameters depending on the model matrices and the variance estimate $s^2$ on $\nu$ degrees of freedom \citep{Draper1998}:
\begin{equation*}
(\bm{\beta}_{2}-\bm{b}_{2})^{'}(\bm{L}+\bm{I}_{q}/\tau^{2})(\bm{\beta}_{2}-\bm{b}_{2})\leq qs^{2}F_{q,\nu;1-\alpha},
\end{equation*}
where $\mathrm{F}_{q,\nu; \alpha}$ is the $\alpha$-quantile of F-distribution with $q$ and $\nu$ degrees of freedom.
Minimising the volume of the confidence region is equivalent to 
\begin{equation}
\label{eq::LoFDP_criterion}
\left|\bm{L}+\bm{I}_{q}/\tau^{2}\right|^{-1/q}F_{q,d;1-\alpha_{LoF}} \longrightarrow \mbox{ min, }  
\end{equation}  
which we refer to as ``Lack-of-fit DP-criterion'', that is directly related to [1] the lack-of-fit component in the Generalised D-optimality developed by \cite{Goos2005model} -- where the residual number of degrees of freedom $\nu$ does not depend on the design, and [2] DP-optimality \citep{GilmourTrinca2012}, with the F-quantile preserved from $\nu = d$ being the number of replicates in the design. 

Another way of formulating a criterion that we consider in this work -- minimising the average of posterior variances of linear functions of $\bm{\beta}_q$ defined by matrix $\bm{J}$, and we define the Lack-of-fit LP-criterion as mean of the squared lengths of the $(1-\alpha)\times100\%$ confidence intervals for these linear functions:
\begin{equation}
\label{eq::LoFLP_criterion}
\frac{1}{q}\mbox{trace}\left[\bm{JJ}^T\left(\bm{L}+\frac{\bm{I}_{q}}{\tau^{2}}\right)^{-1}\right]F_{1,d;1-\alpha}  \longrightarrow \mbox{ min. } 
\end{equation} 

This trace-based criterion is linked to the lack-of-fit part of the Generalised L-optimality \citep{Goos2005model}, and the pure error estimation approach retains the corresponding F-quantile. 
[Henceforth we mainly consider the case when $\bm{J}$ is the identity matrix, that is we work with the analogue of $AP$-optimality. In other words, the lack-of-fit component in the generalised $AP$-criterion  stands for the minimisation of the $\bm{L}_2$-norm of the $q$-dimensional vector of the posterior confidence intervals' lengths for the potential parameters. ]

\subsection*{MSE-based bias criterion}
Together with controlling the magnitude of model contamination, it is also desirable to "protect" the quality of inference that is to be drawn through fitting the primary model, from the potential presence of extra terms. 
% minimise its effect on the primary model inference, that is the quality of paramters' estimates .
From this point of view, the bias of the parameters' estimates $\hat{\bm{beta}}_p$ would be of substantial interest; a natural way of evaluating their quality is the matrix of  mean square error \citep{FedorovMontepiedra1997} , which is the $\bm{L}_2$-distance between the true and estimated values with respect to the probability distribution measure of $\bm{Y}$ under the assumption of model (\ref{eq::full_model}):
\begin{align}
\label{eq::MSE}
\mbox{MSE}(\bm{\hat{\beta}}_p|\bm{\beta})=&\mathtt{E}_{\bm{Y}|\bm{\beta}}[(\bm{\hat{\beta}}_p-\bm{\beta}_p)(\bm{\hat{\beta}}_p-\bm{\beta}_p)^T]\notag\\=&\sigma^2(\bm{X}_p^T\bm{X}_p)^{-1}+\bm{A}\bm{\beta}_q\bm{\beta}_q^T\bm{A}^T, 
\end{align}
where $\bm{A}=(\bm{X}_p^{T}\bm{X}_p)^{-1}\bm{X}_p^{T}\bm{X}_q$ denotes the $p \times q$ alias matrix, which elements reflect the linearity scale of the relationship between the primary (rows) and potential (columns) terms. 

The determinant-based criterion that would correspond to the overall simultaneous minimisation of the bias above is constructed as log-determinant of the MSE matrix averaged across [?] the values of the full model parameters $\bm{\beta}$:
\begin{equation}
\label{eq::MSE_det}
\mathtt{E}_{\bm{\beta}}\log(\det[\mbox{MSE}(\bm{\hat{\beta}}_p|\bm{\beta})]) \longrightarrow \mbox{ min.}
\end{equation}

Using the matrix determinant lemma \citep{Harville2006matrix} and setting $\bm{M}=\bm{X}_1^{'}\bm{X}_1$ and $\bm{\tilde{\beta}}_2=\bm{\beta}_2/\sigma,$ the determinant and the applied logarithm in (\ref{eq::MSE_det})) can be decomposed:
%\begin{equation*}
%\det[\bm{P}+\bm{uv}']=\det[\bm{P}]\det[1+\bm{v'}\bm{P}^{-1}\bm{u}], 
%\end{equation*}
%where $\bm{P}$ is an invertible square matrix and $\bm{u}$, $\bm{v}$ are column vectors, 
\begin{align}
\label{eq::mse_det_dec}
&\det[\mbox{MSE}(\bm{\hat{\beta}}_p|\bm{\beta}_q)]\notag\\&=\det[\sigma^2\bm{M}^{-1}+\bm{M}^{-1}\bm{X}_p^{T}\bm{X}_q\bm{\beta}_q\bm{\beta}_q^T\bm{X}_q^{T}\bm{X}_p\bm{M}^{-1}]\notag\\&=\sigma^{2p}\det[\bm{M}^{-1}+\bm{M}^{-1}\bm{X}_p^{'}\bm{X}_q\bm{\tilde{\beta}}_q\bm{\tilde{\beta}}_q^T\bm{X}_q^{T}\bm{X}_p\bm{M}^{-1}]\notag\\&=\sigma^{2p}\det[\bm{M}^{-1}]\det[1+\bm{\tilde{\beta}}_q^T\bm{X}_q^{T}\bm{X}_p\bm{M}^{-1}\bm{M}\bm{M}^{-1}\bm{X}_p^{T}\bm{X}_q\bm{\tilde{\beta}}_q]\notag\\&=\sigma^{2p}\det[\bm{M}^{-1}](1+\bm{\tilde{\beta}}_q^T\bm{X}_q^{T}\bm{X}_p\bm{M}^{-1}\bm{X}_p^{T}\bm{X}_q\bm{\tilde{\beta}}_q);\\[10pt]
&\log(\det[\mbox{MSE}(\bm{\hat{\beta}}_p|\bm{\beta}_q)])\notag\\&=p\log\sigma^2+\log(\det[\bm{M}^{-1}])+\log(1+\bm{\tilde{\beta}}_q^T\bm{X}_q^{T}\bm{X}_p\bm{M}^{-1}\bm{X}_p^{T}\bm{X}_q\bm{\tilde{\beta}}_q). 
\end{align}
The first summand does not depend on the design, so it will not be included in the criterion; the second one is the $D$-optimality criterion function. Therefore, the criterion in (\ref{eq::MSE_det}) which we refer to becomes
\begin{equation*}
\log(\det[\bm{M}^{-1}])+\mathtt{E}_{\bm{\tilde{\beta}}_q}\log(1+\bm{\tilde{\beta}}_q^T\bm{X}_q^{T}\bm{X}_p\bm{M}^{-1}\bm{X}_p^{T}\bm{X}_q\bm{\tilde{\beta}}_q) \longrightarrow min.
\end{equation*}
Due to the obvious lack of information regarding $\bm{\tilde{\beta}}_q$, the second term needs to be evaluated numerically. Expressing the prior variance of $\bm{\beta}_q$ as a scaled error variance $\bm{\beta}_q \sim \mathcal{N}(\bm{0},\tau^{2}\sigma^{2}\bm{I}_{q})$ means that $\bm{\tilde{\beta}}_q \sim \mathcal{N}(\bm{0},\tau^{2}\bm{I}_{q})$, and that, quite conveniently, its prior distribution does not depend on the unknown $\sigma.^2$ Then a primitive Monte-Carlo can be used to evaluate that term: drawing a sample of large size $N$ from the prior, and approximating the expectation above by the average across the sampled values of $\bm{\tilde{\beta}}_{q_i},$ $i=1,\dots, N$:
\begin{equation*}
\mathtt{E}_{\bm{\tilde{\beta}}_q}\log(1+\bm{\tilde{\beta}}_q^T\bm{X}_q^{T}\bm{X}_p\bm{M}^{-1}\bm{X}_p^{T}\bm{X}_q\bm{\tilde{\beta}}_q) \approx \frac{1}{N}\sum_{i=1}^{N}\log(1+\bm{\tilde{\beta}}_{q_i}^T\bm{X}_q^{T}\bm{X}_p\bm{M}^{-1}\bm{X}_p^{T}\bm{X}_q\bm{\tilde{\beta}}_{q_i}).
\end{equation*} 

One of the alternatives to this more computationally demanding approach is to use the point prior for $\bm{\beta}_q$, that is setting  $\bm{\beta}_q=\sigma\tau\bm{1}_q$ (where $\bm{1}_q$ is a $q$-dimensional vector of $1$s), which is the standard deviation  of the initial normal prior. Then $\bm{\tilde{\beta}}_q = \tau\bm{1}_q$ with probability $1$ and the expectation above becomes:
\begin{align*}
\mathtt{E}_{\bm{\tilde{\beta}}_q}\log(1+\bm{\tilde{\beta}}_q^T\bm{X}_q^{T}\bm{X}_p\bm{M}^{-1}\bm{X}_p^{T}\bm{X}_q\bm{\tilde{\beta}}_q)  &\approx \log(1+\tau^2\bm{1}^T_q\bm{X}_q^{T}\bm{X}_p\bm{M}^{-1}\bm{X}_p^{T}\bm{X}_q\bm{1}_q) \\&=\log(1+\tau^2\sum_{i,j=1}^{q}[\bm{X}_q^{T}\bm{X}_p\bm{M}^{-1}\bm{X}_p^{T}\bm{X}_q][i,j]),
\end{align*}
with summation of matrix elements taking less computational time even for large $q$. 
 
To infer the trace-based form of the criterion that would minimise the mean squared bias, we use the pseudo-Bayesian approach of setting priors on the model's parameters  to calculating the trace function of the MSE matrix (\ref{eq::MSE}): 
\begin{align*}
\mathtt{E}_{\bm{\beta}_q}\mbox{trace}[\mbox{MSE}(\bm{\hat{\beta}}_p\bm{\beta}_q)]&=\mbox{trace}[\mathtt{E}_{\bm{\beta}_q}\mbox{MSE}(\bm{\hat{\beta}}_p|\bm{\beta}_q)]=\\&=\mbox{trace}[\sigma^2(\bm{X}_p^{T}\bm{X}_p)^{-1} + \mathtt{E}_{\bm{\beta}_q}(\bm{A}\bm{\beta}_q\bm{\beta}_q^T\bm{A}^T)]=\\&=\mbox{trace}[\sigma^2(\bm{X}_p^{T}\bm{X}_p)^{-1}+\sigma^2\tau^2\bm{A}\bm{A}^T]=\\&=\sigma^2\mbox{trace}[(\bm{X}_p^{T}\bm{X}_p)^{-1}+\tau^2\bm{A}\bm{A}^T]\\&=\sigma^2[\mbox{trace}\{(\bm{X}_p^{T}\bm{X}_p)^{-1}\}+\tau^2\mbox{trace}\bm{A}\bm{A}^T].
\end{align*}

The operations of calculating trace and expectation are commutative, hence there is no necessity of any additional numerical evaluations, and in this case of the trace-based criterion using the point prior for $\bm{\beta}_q$ defined earlier would lead to the same resulting function. By minimising the whole function above, we simultaneously minimise both the sum of primary terms' variance and the expected squared norm of the bias vector in the direction of the potential terms. The second part here contains the scaling parameter $\tau^2$ which regulates the magnitude of the potential terms' variation relatively to the error variance.  
 
\subsection*{Compound criteria}
The compound criterion is constructed to account for the three main objectives -- precision (or accuracy?) of the primary model parameters, control for the lack-of-fit and minimising the inferential bias from the potential model contamination -- as the weighted product of efficiencies (\ref{eq::compound}) with respect to DP-criterion, and Lack-of-Fit DP functions(\ref{eq::LoFDP_criterion}, \ref{eq::LoFLP_criterion}) and the MSE-based ones. All of them are brought to the scale of efficiencies [?and $\exp$]:

\begin{align}
\label{eq::MSE_D}
\mbox{minimise }&\left[\left|\bm{X}^T_{p}\bm{X}_{p}\right|^{-1/p}F_{p,d;1-\alpha_{DP}}\right]^{\kappa_{DP}} \times \notag \\ &\left[\left|\bm{L}+\frac{\bm{I}_{q}}{\tau^{2}}\right|^{-1/q}F_{q,d;1-\alpha_{LoF}}\right]^{\kappa_{LoF}}\times \notag\\ & \left[|\bm{X}^T_{p}\bm{X}_{p}|^{-1}\exp\left(\mathtt{E}_{\bm{\tilde{\beta}}_q}\log(1+\bm{\tilde{\beta}}_q^T\bm{X}_q^{T}\bm{X}_p\bm{M}^{-1}\bm{X}_p^{T}\bm{X}_q\bm{\tilde{\beta}}_q) \right)\right]_{.}^{\kappa_{MSE}/p}
\end{align}
 
This criterion is later referred to as compound ``MSE-DP-criterion'', where $\alpha_{DP}$ and $\alpha_{LoF}$ denote the confidence intervals' probability levels for the primary $\bm{\beta}_p$ and potential coefficients $\bm{\beta}_q$ -- in the DP and LoF-DP elementary criteria; usually they are set to $0.05$ or $0.01$. As mentioned before, non-negative weights $\kappa_X$ that sum up to $1$ define the compound criterion and are chosen to reflect the experimenter's priorities. 

Similarly, we obtain the compound ``MSE-LP-criterion'' by joining the $LP$ criterion with trace-based Lack-of-fit (\ref{eq::LoFLP_criterion})) and MSE components :
\begin{align}
\label{eq::MSE_L}
\mbox{minimise} &\left[\frac{1}{p}\mbox{trace}(\bm{WX}^T_{p}\bm{X}_{p})^{-1}F_{1,d;1-\alpha_{LP}}\right]^{\kappa_{LP}}\times \notag\\& \left[\frac{1}{q}\mbox{trace}\left(\bm{L}+\frac{\bm{I}_{q}}{\tau^{2}}\right)^{-1}F_{1,d;1-\alpha_{LoF}}\right]^{\kappa_{LoF}}\times 
\notag\\& \left[\frac{1}{p}\mbox{trace}\{(\bm{X}^T_{p}\bm{X}_{p})^{-1}+\tau^2\bm{A}\bm{A}^T\}\right]^{\kappa_{MSE}}_{.}
\end{align}


\section{Restricted randomisation}
\label{sec::restricted_randomisation}
\textit{	
	\begin{itemize}
		\item[-] General framework of multistratum experiments. [add more on model uncertainty]
		\item[-] Error estimation: REML and pure error REML. Yates procedure
		\item[-] Adapting optimality criteria and design construction
	\end{itemize}
}

In a large number of industrial, engineering and laboratory-based experiments, either the nature of the study or certain restrictions result in the necessity of considering a multi-level structure of experimental units. For example, a chemical process consisting of applying treatments to the material batches of different sizes at each stage;  or one or several experimental factors' values can be changed only once per a certain amount of runs whereas values of other factors are varied between runs. Therefore, different factors are applied at different levels, and randomisation is performed at each level, thus the whole process of allocating treatments to experimental units should be amended accordingly \citep{MeadGilmour2012}.

In this work we go on to consider experimental framework comprising such a hierarchical structure of experimental units and treatments, which is referred to as multistratum experiment, and each stratum is defined as a level in the unit structure. Units are grouped into whole-plots, each of them divided into sub-plots, which contain a certain number of sub-sub-plots, and so on up to the smallest units --- runs of the experiment -- [Figure .with Hasse diagram?] . In the general case of two strata, we deal with what is called a ``split-plot'' experiment, in case of three strata a ``split-split-plot'' experiment. 
%\begin{figure}[h]
%	\begin{center}
%		\includegraphics[scale=0.85]{Hasse.jpg}      %width=\textwidth
%		\caption{Hasse diagram for factors}
%		\label{Fig::Hasse}
%	\end{center}
%\end{figure} 
Our aim is to adapt the MSE-based criteria derived before to the factorial experiments with units distributed in several strata. The most important aspects are (1) estimating the error components at all of the levels and (2) formulating the compliant criteria and provide the suitable implementation strategy for the optimal design search.

\subsection{Response-Surface Methodology for multistratum framework}
%% Model. Ratio of variances.
[OE: introduce the notion of "block"?]
Here we are considering experiments with $s$ strata in total, stratum $i$ being nested within units of the stratum $i-1$, and stratum $0$ will be seen as the whole experiment (following the notation set by \citet{Trinca2015improved}). The number of units in stratum $i$ within every unit in stratum $i-1$ is denoted by $n_i$, such that $m_{j}=\prod_{i=1}^{j}n_{i}$ is the number of units in stratum $j$ and, therefore,  $n=m_{s}=\prod_{i=1}^{s}n_{i}$ is the total number of runs. As randomisation occurs at each stratum, the model accounting for the hierarchical error structure and correlated observations can be written as
\begin{equation}
\label{eq::MS_model}
\bm{Y}=\bm{X}\bm{\beta}+\sum_{i=1}^{s}\bm{Z}_{i}\bm{\varepsilon}_{i},
\end{equation}
where $\bm{Y}$ represents the $n$-dimensional vector of responses, $\bm{X}$ stands for the $n \times p$ model matrix and $\bm{\beta}$ for the $p$-dimensional vector of corresponding model coefficients. Each row of the $n\times m_{i}$ matrix $\bm{Z}_{i}$ corresponds to a single experimental run and indicates the unit in stratum $i$ containing this run. $\bm{\varepsilon}_{i}$ is a vector of random effects occurring due to the randomisation at level $i$, and these effects are assumed to be independent and identically distributed around zero mean and variance $\sigma^{2}_{i}$. Generalised Least Squared  (GLS) estimators of $\bm{\beta}$ are calculated as
\begin{equation}
\label{eq::MS_GLS}
\bm{\hat{\beta}}=(\bm{X}'\bm{V}^{-1}\bm{X})^{-1}\bm{X}'\bm{V}^{-1}\bm{Y}.
\end{equation}

%The assumption of error normality, though quite common is required for some analysis methods, is not a necessary one at this stage.

%Inferential purposes of obtaining parameter estimates $\bm{\hat{\beta}}$ of good quality imply the necessity of evaluating the dispersion parameters $\sigma^2_{i}$ that are directly related to the estimates' variance and, therefore, responsible for measuring the uncertainty of the inference in a general sense.

\subsubsection*{Error Estimate}
An extensive amount of research has been conducted on the design of and analysis of data from split-plot and split-split-plot experiments, with one of the first works by \cite{Letsinger1996BiRandomization}, emphasising the necessity of adapting the design and analysis strategy with respect to the more complicated error structure.  In the most general case of non-orthogonal unit structures Residual Maximum Likelihood (REML) methodology has been acknowledged to be the most sensible approach to estimating the variance components by maximising the part of the likelihood function corresponding to the ``random'' part of the model; the details are provided, for example, in \cite{Searle2001generalized}. 

%In traditional REML-based methodology the estimates of $\sigma^{2}_{i}$ are obtained from model (\ref{eq::MS_model}) and then GLS estimators of $\bm{\beta}$ are calculated as
%\begin{equation}
%\label{eq::MS_GLS}
%\bm{\hat{\beta}}=(\bm{X}'\bm{V}^{-1}\bm{X})^{-1}\bm{X}'\bm{V}^{-1}\bm{Y},
%\end{equation}
%where $\bm{V}$ is the unknown variance-covariance matrix of the response vector, to which the estimates are to be plugged in:
%\begin{equation*}
%\bm{V}=\sum_{i=1}^{s}\sigma^{2}_{i}\bm{Z}_{i}\bm{Z}'_{i},
%\end{equation*}
%so that $\mbox{Var}(\bm{\hat{\beta}})=(\bm{X}'\bm{V}^{-1}\bm{X})^{-1}.$

In some case, though,  REML tends to underestimate the variance components in the higher stratum, for example \cite{Goos2006practical} considered some split-plot experiments with true whole-plot variances being non-negligibly far from zero, for which REML, however, provided zero estimates. A possible alternative of following Bayesian strategy \citep{Gilmour2009analysis}, however, this requires a careful choice of the prior which is often not possible, especially in the case of more than two strata.

In the context of model uncertainty, it would be appropriate to consider instead the direct relationship between the response and the set of treatments, similar to (\ref{eq::treatment_model}):
\begin{equation}
\label{eq::treatmentMS}
\bm{Y}=\bm{X_t}\bm{\mu_t}+\sum_{i=1}^{s}\bm{Z}_{i}\bm{\varepsilon}_{i}.
\end{equation}

Each column of $\bm{X_t}$ corresponds to a treatment defined as a combination of the experimental factors;
%in case of factorial experiment the total number of possible treatments (or candidate design points) $T$ would be the product of the number of the factors' levels. 
$\bm{Z}_{i}$ is the indicator matrix of random effects at stratum $i$.

\cite{GilmourGoos2016PEREML} introduced the notion of ``Pure Error REML'', where random effects are estimated from the treatment model (\ref{eq::treatmentMS}), and then for the sake of making inference regarding parameters of the response model (\ref{eq::MS_model}), they are substituted in the GLS estimators of the fixed effects (\ref{eq::MS_GLS}). It is also argued that some appropriate corrections are to be adapted and applied to the obtained estimates, i.e.~the ones suggested by \cite{kenward1997small}. 

A careful definition of ``pure error'' is necessary when considering the treatment and response surface models, especially in the presence of nested random block effects. \cite{Gilmour2000PErsm} discuss the pure error estimation issues in the context of blocked experiments. 

[OE: The following is to be paraphrased -- made much shorter or removed overall?]
Pure error is expected to measure the variability between experimental units regardless of the treatments applied, and therefore, the assumption of treatment-unit additivity is essential in unblocked experiments.  In the presence of one or several blocking factors, the existence of block-treatment interaction would imply that pure error in the lower strata can only be estimated from the replicates within blocks, so that inter-block information is not taken into consideration. Hence, in the blocked cases the assumption of treatment-block additivity \citep{Draper1998} is desirable. \cite{Gilmour2000PErsm} argue for the preservation of treatment-unit additivity when either fixed or random block effects are in the model in order to preserve consistency with the unblocked experiments; that would also conveniently imply treatment-block additivity. Adoption of these assumptions would then lead to the following representation of the relationship between responses and treatments in multi-stratum experiments. 

The outcome of applying treatment $i$ $(i=1\ldots T)$ to the experimental run located in the units $j_1$ of the first stratum, $j_2$ of the second stratum, $\ldots$ , and unit $j_s$ of the $s-$th stratum can be expressed as follows:
\begin{equation}
\label{eq::ms_tr}
y_{ij_{1}...j_{s}}=\mu+t_{i}+b_{j_1}+\ldots +b_{j_s},
\end{equation}
where $b_{j_s}$ is usually denoted as between-run variation. 

Under the stronger assumption of the response surface model (\ref{eq::MS_model}), the treatment effects $t_{i}$ are represented as a set of polynomial model terms.  

Based on the interrelation of the two models, and the desirability of being able to provide the possibility for testing for the lack-of-fit together with obtaining robust estimates of the variance components, the approach of using pure error is adopted, \cite{GilmourGoos2016Robust} explored several approaches to estimating the variance components. \\[5 pt]
[OE: the end of the part to be re-written/excluded]

The first approach is stratum-by-stratum data analysis, where each randomisation level is considered separately, and at each stratum $i$ units are treated as runs aggregated in $m_{i-1}$ blocks with fixed effects. The lower stratum variance is estimated from within blocks (using intra-block residual mean square), and the higher stratum (inter-block) variance from the difference between the intra- and inter-block residual mean squares scaled by the number of runs per block \citep{Hinkelmann2005Advanced}. This strategy requires only the treatment-unit additivity assumption and provides unbiased treatment effects' estimators. However, it would be desirable to make use of combining information not just from within-strata (intra-block) treatment replicates, but also from between-strata (inter-block) ones, especially in the common cases of relatively small experiments when the amount of runs available does not allow for obtaining higher stratum variance components from only considering blocks as whole units. Yates' procedure, first suggested by \cite{yates1939recovery} and described in detail by \cite{Hinkelmann2005Advanced}, provides more degrees of freedom for the variance estimate in the higher stratum -- and this is the approach we are adopting in this work. 

In the case of two levels of randomisation, variance components are estimated in two variance decomposition steps: 
\begin{enumerate}
	\item Total SS = SS(Blocks) + SS(Treatments$\vert$Blocks) + SS(Residual)\\
	From fitting this full treatment-after-blocks model the residual mean square $S^2$ is taken as the estimate of the intra-block variance. SS(Residual) is then substituted in the following partition.
	\item Total SS = SS(Treatments) + SS(Blocks$\vert$Treatments) + SS(Residual)\\
	Therefore, the sums of squares for the blocks after the treatment effects have been accounted for is obtained and the corresponding mean square $$S^2_{b}=\mbox{SS(Blocks$\vert$Treatments)}/\nu_{b}$$ is then used to get an estimate of the inter-block variance:
	$\hat{\sigma}^2_{b}=\frac{\nu_{b}(S^2_{b}-S^2)}{d}$,
	where pure error degrees of freedom for inter-block variance is $d=n-\mbox{trace}[\bm{Z}'\bm{X_t}(\bm{X_t}'\bm{X_t})^{-1}\bm{X_t}'\bm{Z}]$, $\nu_{b}=\mbox{rank}([\bm{X_t} \bm{Z}])-\mbox{rank}(\bm{X_t})$, and $\bm{X_t}$ and $\bm{Z}$ are as in (\ref{eq::treatmentMS}).
\end{enumerate}

Allocation of the degrees of freedom in practice, following Yates' procedure, is provided in the Appendix. 

[OE: to shorten]\\
\cite{GilmourGoos2016Robust} show that in addition to the treatment-unit additivity and randomisation the assumption of the experimental units being a random sample from an infinite population is necessary for the inter-block variance estimate to be unbiased. In this particular context, when the normality and independence of responses in (\ref{eq::MS_model}) is a standard assumption, although a potential presence of contamination effects in the fixed part of the model is also accounted for, Yates' approach seems to be the most appropriate technique. However, relying on the distribution assumption to the extent that makes REML the most sensible approach means that the fixed part of the fitted model is assumed to be absolutely true, that is the parameters of the population distribution are known, which does not comply with the model uncertainty framework. 

Together with the specifications regarding estimation of the variance components, it is necessary to carefully establish the way the pure error degrees of freedom should be determined and how the available remaining number of treatment degrees of freedom would be distributed between the polynomial, lack-of-fit and inter-block information components. 

\subsection{MSE-based compound criteria for multistratum experiments}
\label{sec::ch7_search}
In the presence of potential model disturbance which is expressed as additional polynomial terms, the full model for a multistratum experiment is then:
\begin{equation}
\label{eq::MS_model_full}
\bm{Y}=\bm{X}_p\bm{\beta}_p+\bm{X}_q\bm{\beta}_q+\sum_{i=1}^{s}\bm{Z}_{i}\bm{\varepsilon}_{i},
\end{equation}
where $\bm{X}_p$ is the primary terms matrix, and $bm{X}_q$ contains potential terms.

The design construction will follow the approach developed by \cite{Trinca2015improved}: an iterative, stratum-by-stratum algorithm, that treats the higher stratum units as fixed blocks; such an approach does not require any prior assumptions regarding the values of the variance components. 

At each step a candidate set of treatments applied at the current stratum is to be generated, and the point exchange algorithm is applied. The model matrix to be used in the optimisation comprises model terms from all higher strata (with the factor values for individual runs obtained at previous steps), terms from the current stratum and interactions between these two groups of terms, if there are any to be considered. The allocation of the degrees of freedom shall be implemented according to the Yates procedure, and some illustrative examples will be given later on.

\subsection*{Criteria for blocked experiments}
\label{subsec::compound_blocked}

In some experiments, where the number of runs is relatively large, and/or the variability between units is quite high, experimental units are allocated in blocks such that within each block the units are similar in some way, e.g.~the runs carried out within the same day. Such formalisation contributes to 
controlling the variability by separating variation coming from the difference between blocks and the variability arising from within the blocks \citep{Bailey2008design}.

Under the assumption of additivity of $b$ fixed block effects the polynomial model can be presented as:
\begin{equation}
\label{eq::blocked_model}
\bm{Y}=\bm{Z\beta}_B+\bm{X\beta}+\bm{\varepsilon},
\end{equation}
which, in addition to the same terms as in (\ref{eq::primary_model}) comprises $\bm{Z}$ -- the $n\times b$ matrix, such that its $(i,j)^{th}$ element is equal to $1$ if unit $i$ is in block $j$ and to $0$ otherwise, and $\bm{\beta}_B$ is the vector of block effects. 

The full information matrix has the form
$%\begin{equation*}
\bm{M_B}=
\begin{pmatrix}
\bm{Z}'\bm{Z} & \bm{Z}'\bm{X}\\
\bm{X}'\bm{Z} & \bm{X}'\bm{X}
\end{pmatrix}.     
$%\end{equation*} 

Using the rules of inverting blocked matrices (\citealp{Harville2006matrix}), we can isolate the variance of the polynomial coefficients' estimates:
\begin{align*}
\mbox{Var}(\hat{\bm{\beta}})&=\sigma^2(\bm{M_B}^{-1})_{22}=\sigma^2(\bm{X}'\bm{QX}),^{-1}\\
\mbox{where }&\bm{Q}=\bm{I}-\bm{Z}(\bm{Z}'\bm{Z})^{-1}\bm{Z}'.
\end{align*}
The $DP$- and $LP$-criteria become $DP_S$ and $LP_S$ in the context of a blocked experiment, and they can be straightforwardly defined as
\begin{align}
\label{eq::DPs_blocked}
DP_S: &(F_{p,d_B;1-\alpha_{DP}})^{p}\vert (\bm{X}'\bm{Q}\bm{X})^{-1}\vert \rightarrow \mbox{min,}\\
\label{eq::LPs_blocked}
LP_S: &F_{1,d_B;1-\alpha_{LP}}\mbox{trace}\{\bm{W}(\bm{X}'\bm{Q}\bm{X})^{-1}\} \rightarrow \mbox{min.}.
\end{align}

The number of pure error degrees of freedom is now calculated as $d_B=n-\mbox{rank}[\bm{Z}:\bm{T}]$, where $\bm{T}$ is the $n\times t$ matrix whose elements indicate the treatments (\citealp{GilmourTrinca2012}), providing the number of replications after subtracting the ones taken for the estimation of block contrasts.  

To adapt the derivation of the lack-of-fit and MSE-based criteria to the blocked experiments, we start by formulating the model comprising both primary terms and possible contamination in the form of potential terms, now for blocked experiments: 
\begin{align*}
\bm{Y}=\bm{Z\beta}_{B}+\bm{X}_{p}\bm{\beta}_{p}+\bm{X}_{q}\bm{\beta}_{q}+\bm{\varepsilon}.
\end{align*}
Denote the $n\times(b+p)$ model matrix of the block and primary terms by $\tilde{\bm{X}}_{p}=[\bm{Z},\bm{X}_{p}]$ and let $\bm{\tilde{\beta}}_p=[\bm{\beta}_{B},\bm{\beta}_{p}]'$ be the joint vector of fixed block effects and primary model terms, and by $\bm{\hat{\tilde{\beta}}}_p$ we denote the vector of the corresponding estimates. It is worth noting that the number of primary terms $p$ does not include the intercept, as it is aliased with the block effects.

\subsection{Lack-of-fit criteria}
The information matrix for the model (\ref{eq::blocked_model}), up to a multiple of $\sigma^2$, is:
\begin{align*}
\bm{M_B}=
\begin{pmatrix}
\bm{Z}^T\bm{Z} & \bm{Z}^T\bm{X}_{p} & \bm{Z}^T\bm{X}_{q}\\
\bm{X}^T_{p}\bm{Z} & \bm{X}^T_{p}\bm{X}_{p} & \bm{X}^T_{p}\bm{X}_{q}\\
\bm{X}^T_{q}\bm{Z} & \bm{X}^T_{q}\bm{X}_{p} & \bm{X}^T_{q}\bm{X}_{q}+\bm{I}_{q}/\tau^2
\end{pmatrix}
=
\begin{pmatrix}
\bm{\tilde{X}}^T_{p}\bm{\tilde{X}}_{p} & \bm{\tilde{X}}^T_{p}\bm{X}_{q}\\
\bm{X}^T_{q}\bm{\tilde{X}}_{p} & \bm{X}^T_{q}\bm{X}_{q}+\bm{I}_{q}/\tau^2
\end{pmatrix}_{.}
\end{align*} 

Assuming the same normal prior on $\bm{\beta}_q$ $\sim \mathcal{N}(\bm{0}, \tau^2\sigma^2\bm{I}_q)$, we can construct the variance-covariance matrix corresponding to the potential terms, which would be the inverse of the lower right submatrix of $\bm{M_B}$: $\bm{\tilde{\Sigma}}_{qq}=\sigma^2[\bm{M}^{-1}_B]_{22}$.
\begin{align*}
\bm{\tilde{\Sigma}}_{qq}&=\sigma^2([\bm{M_B}]_{22}-[\bm{M_B}]_{21}([\bm{M_B}]_{11})^{-1}[\bm{M_B}]_{12})^{-1}\\
&=\sigma^2(\bm{X}^T_{q}\bm{X}_{q}+\bm{I}_{q}/\tau^2-\bm{X}^T_{q}\bm{\tilde{X}}_{p}(\bm{\tilde{X}}^T_{p}\bm{\tilde{X}}_{p})^{-1}\bm{\tilde{X}}^T_{p}\bm{X}_{q})^{-1}\\&=\sigma^2\left(\bm{\tilde{L}}+\frac{\bm{I}_{q}}{\tau^2}\right), \mbox{ where }\bm{\tilde{L}}=\bm{X}^T_{q}\bm{X}_{q}-\bm{X}^T_{q}\bm{\tilde{X}}_{p}(\bm{\tilde{X}}^T_{p}\bm{\tilde{X}}_{p})^{-1}\bm{\tilde{X}}^T_{p}\bm{X}_{q}.
\end{align*}

Therefore, the lack-of-fit criteria in (\ref{eq::LoFDP_criterion}) and (\ref{eq::LoFLP_criterion}) are adjusted for blocked experiments by replacing the primary terms matrix $\bm{X}_{p}$ by the extended matrix $\bm{\tilde{X}}_{p}$ and the dispersion matrix $\bm{L}$ -- by $\bm{\tilde{L}}$ as obtained above.

\subsection{MSE-based criteria}

As for the MSE-based measure of the shift in the primary terms estimates, we first consider the overall mean square matrix  
\begin{align}
\label{eq::MSE_b}
\mbox{MSE}(\bm{\hat{\tilde{\beta}}}_p|\bm{\tilde{\beta}})=&\mathtt{E}_{\bm{Y}|\bm{\beta}}[(\bm{\hat{\tilde{\beta}}}_p-\bm{\tilde{\beta}}_p)(\bm{\hat{\tilde{\beta}}}_p-\bm{\tilde{\beta}}_p)'] \notag\\=&\sigma^2(\bm{\tilde{X}}_p^{'}\bm{\tilde{X}}_p)^{-1}+\bm{\tilde{A}}\bm{\beta}_q\bm{\beta}_q^T\bm{\tilde{A}}^T, 
\end{align}
with $\bm{\tilde{A}}=(\bm{\tilde{X}}_p^T\bm{\tilde{X}}_p)^{-1}\bm{\tilde{X}}_p^{'T}\bm{X}_q$ being the alias matrix, 
and its partition with respect to block and primary effects:
\begin{align*}
%\label{eq::mse_b_matrix}
&\mathtt{E}_{\bm{Y}|\bm{\beta}}[(\bm{\hat{\tilde{\beta}}}_p-\bm{\tilde{\beta}}_p)(\bm{\hat{\tilde{\beta}}}_p-\bm{\tilde{\beta}}_p)^T]= \notag\\ &\mathtt{E}_{\bm{Y}|\bm{\beta}}\{[\hat{\tilde{\beta}}_{p1}-\tilde{\beta}_{p1},\ldots,
\hat{\tilde{\beta}}_{pb}-\tilde{\beta}_{pb}, \hat{\tilde{\beta}}_{p b+1}-\tilde{\beta}_{p b+1},\ldots, \hat{\tilde{\beta}}_{p b+p}-\tilde{\beta}_{p b+p}]\times \notag\\ &[\hat{\tilde{\beta}}_{p 1}-\tilde{\beta}_{p1},\ldots,
\hat{\tilde{\beta}}_{pb}-\tilde{\beta}_{pb}, \hat{\tilde{\beta}}_{p b+1}-\tilde{\beta}_{p b+1},\ldots, \hat{\tilde{\beta}}_{p b+p}-\tilde{\beta}_{p b+p}]^T\}=\notag\\
&\mathtt{E}_{\bm{Y}|\bm{\beta}}\{[\bm{\hat{\beta}}_B-\bm{\beta}_B, \bm{\hat{\beta}}_p-\bm{\beta}_p][\bm{\hat{\beta}}_B-\bm{\beta}_B, \bm{\hat{\beta}}_p-\bm{\beta}_p]^T\}=\notag\\
&\begin{bmatrix}
\mathtt{E}_{\bm{Y}|\bm{\beta}}(\bm{\hat{\beta}}_B-\bm{\beta}_B)(\bm{\hat{\beta}}_B-\bm{\beta}_B)^T & \mathtt{E}_{\bm{Y}|\bm{\beta}}(\bm{\hat{\beta}}_B-\bm{\beta}_B)(\bm{\hat{\beta}}_p-\bm{\beta}_p)^T\\
\mathtt{E}_{\bm{Y}|\bm{\beta}}(\bm{\hat{\beta}}_p-\bm{\beta}_p)(\bm{\hat{\beta}}_B-\bm{\beta}_B)^T & \mathtt{E}_{\bm{Y}|\bm{\beta}}(\bm{\hat{\beta}}_p-\bm{\beta}_p)(\bm{\hat{\beta}}_p-\bm{\beta}_p)^T
\end{bmatrix}.
\end{align*}

The part  corresponding to the bias of the primary terms $\bm{\beta}_p$ is the lower right $p \times p$ submatrix, and we can extract it from the MSE expression in (\ref{eq::MSE_b}). The respective submatrix of the first summand  is
\begin{equation*}
[\sigma^2(\bm{\tilde{X}}_p^{T}\bm{\tilde{X}}_p)^{-1}]_{22}=\sigma^2(\bm{X}^T_{p}
\bm{QX}_{p})^{-1}, \mbox { where } \bm{Q}=\bm{I}-\bm{Z}(\bm{Z}^T\bm{Z})^{-1}\bm{Z}^T_{.} 
\end{equation*}

Using the matrix inversion rule for block matrices \citep{Harville2006matrix}, we now consider $\bm{\tilde{A}}$:
\begin{align*}
\bm{\tilde{A}}=&\left(
\begin{bmatrix}
\bm{Z}^T\\
\bm{X}^T_{p}
\end{bmatrix}
\begin{bmatrix}
\bm{Z} & \bm{X}_{p}
\end{bmatrix}\right)^{-1}
\begin{bmatrix}
\bm{Z}^T\\
\bm{X}^T_{p}
\end{bmatrix}\bm{X}_{q}=
\begin{pmatrix}
\bm{Z}^T\bm{Z} & \bm{Z}^T\bm{X}_{p}\\
\bm{X}^T_{p}\bm{Z} & \bm{X}^T_{p}\bm{X}_{p}
\end{pmatrix}^{-1}
\begin{bmatrix}
\bm{Z}^T\\
\bm{X}^T_{p}
\end{bmatrix}\bm{X}_{q}=
\\ &
\begin{bmatrix}
(\bm{Z}^T\bm{PZ})^{-1} & -(\bm{Z}^T\bm{PZ})^{-1}\bm{Z}^T\bm{X}_{p}(\bm{X}^T_{p}\bm{X}_{p})^{-1} \\
-(\bm{X}^T_{p}\bm{QX}_{1})^{-1}\bm{X}^T_{p}\bm{Z}(\bm{Z}^T\bm{Z})^{-1} & (\bm{X}^T_{1}\bm{QX}_{p})^{-1}
\end{bmatrix}
\begin{bmatrix}
\bm{Z}^T\\
\bm{X}^T_{p}
\end{bmatrix}\bm{X}_{q}=\\ &
\begin{bmatrix}
(\bm{Z}^T\bm{PZ})^{-1}\bm{Z}^T\bm{PX}_{q}\\
(\bm{X}^T_{p}\bm{QX}_{1})^{-1}\bm{X}^T_{p}\bm{QX}_{q}
\end{bmatrix}, 
\end{align*}
where $\bm{P}=\bm{I}-\bm{X}_{p}(\bm{X}^T_{p}\bm{X}_{p})^{-1}\bm{X}^T_{p}$; $\bm{ZZ}'$, $\bm{X}'_{1}\bm{X}_{1}$ and $\bm{Z}'\bm{PZ}$ are all invertible and, therefore, the operations are legitimate. Now consider the second summand in (\ref{eq::MSE_b}):
\begin{align*}
&\bm{\tilde{A}}\bm{\beta}_q\bm{\beta}_q^T\bm{\tilde{A}}^T=
\begin{bmatrix}
(\bm{Z}^T\bm{PZ})^{-1}\bm{Z}^T\bm{PX}_{q}\bm{\beta}_q \\
(\bm{X}^T_{p}\bm{QX}_{p})^{-1}\bm{X}^T_{p}\bm{QX}_{q}\bm{\beta}_q
\end{bmatrix}
\begin{bmatrix}
\bm{\beta}^T_q\bm{X}^T_{2}\bm{PZ}(\bm{Z}^T\bm{PZ})^{-1} & \bm{\beta}^T_q\bm{X}^T_{q}\bm{QX}_{p}(\bm{X}^T_{p}\bm{QX}_{p})^{-1}
\end{bmatrix}=\\
&\begin{bmatrix}
(\bm{Z}^T\bm{PZ})^{-1}\bm{Z}^T\bm{PX}_{q}\bm{\beta}_q\bm{\beta}^T_q\bm{X}^T_{q}\bm{PZ}(\bm{Z}^T\bm{PZ})^{-1} & (\bm{Z}^T\bm{PZ})^{-1}\bm{Z}^T\bm{PX}_{q}\bm{\beta}_q\bm{\beta}^T_q\bm{X}^T_{q}\bm{QX}_{p}(\bm{X}^T_{p}\bm{QX}_{p})^{-1} \\
(\bm{X}^T_{p}\bm{QX}_{p})^{-1}\bm{X}^T_{p}\bm{QX}_{q}\bm{\beta}_q\bm{\beta}^T_q\bm{X}^T_{q}\bm{PZ}(\bm{Z}^T\bm{PZ})^{-1} & (\bm{X}^T_{p}\bm{QX}_{p})^{-1}\bm{X}^T_{p}\bm{QX}_{q}\bm{\beta}_q\bm{\beta}^T_q\bm{X}^T_{q}\bm{QX}_{p}(\bm{X}^T_{p}\bm{QX}_{p})^{-1}
\end{bmatrix}.
\end{align*}

Then the submatrix of (\ref{eq::MSE_b}) corresponding to the primary terms is
\begin{align}
\label{eq::mesb_submatrix}
\mbox{MSE}(\bm{\hat{\tilde{\beta}}}_p|\bm{\tilde{\beta}})_{pp}=& \sigma^2(\bm{X}^T_{p}\bm{QX}_{p})^{-1}+(\bm{X}^T_{p}\bm{QX}_{p})^{-1}\bm{X}^T_{p}\bm{QX}_{q}\bm{\beta}_q\bm{\beta}^T_q\bm{X}^T_{q}\bm{QX}_{p}(\bm{X}^T_{p}\bm{QX}_{p})^{-1}\notag\\=& \sigma^2\bm{\tilde{M}}^{-1}+\bm{\tilde{M}}^{-1}\bm{X}^T_{p}\bm{QX}_{q}\bm{\beta}_q\bm{\beta}^T_q\bm{X}^T_{q}\bm{QX}_{p}\bm{\tilde{M}},^{-1}
\end{align}
where $\bm{\tilde{M}}=\bm{X}^T_{p}\bm{QX}_{p}.$

As in the unblocked case, we first look at the determinant of the corresponding submatrix (\ref{eq::mesb_submatrix}): 
\begin{align}
\label{eq::MSE_B_det}
\det[\mbox{MSE}(\bm{\hat{\tilde{\beta}}}_p|\bm{\tilde{\beta}})_{pp}]=&\det[\sigma^2\bm{\tilde{M}}^{-1}+\bm{\tilde{M}}^{-1}\bm{X}^T_{p}\bm{QX}_{q}\bm{\beta}_q\bm{\beta}^T_q\bm{X}^T_{q}\bm{QX}_{p}\bm{\tilde{M}}^{-1}]=\notag\\& \sigma^{2p}\det[\bm{\tilde{M}}^{-1}+\bm{\tilde{M}}^{-1}\bm{X}^T_{p}\bm{QX}_{q}\bm{\tilde{\beta}}_q\bm{\tilde{\beta}}^T_q\bm{X}^T_{q}\bm{QX}_{p}\bm{\tilde{M}}^{-1}]=\notag\\& \sigma^{2p}\det[\bm{\tilde{M}}^{-1}](1+\bm{\tilde{\beta}}^T_q\bm{X}^T_{q}\bm{QX}^T_{p}\bm{\tilde{M}}^{-1}\bm{X}^T_{p}\bm{QX}_{q}\bm{\tilde{\beta}}_q).
\end{align}

The $q$-dimensional random vector $\bm{\tilde{\beta}}_q$, as before, follows $\mathcal{N}(\bm{0},\tau^{2}\bm{I}_{q})$, so that this prior does not depend on the error variance $\sigma_{.}^2$. Next, taking the expectation of the logarithm of (\ref{eq::MSE_B_det}) over the set prior distribution is completely identical to the derivations leading to (\ref{eq::MSE_D}). The $MSE(D)$-component then becomes:
\begin{equation}
\label{eq::mse_b_component}
\log(\det[\bm{\tilde{M}}^{-1}])+\mathtt{E}_{\bm{\tilde{\beta}}_2}\log(1+\bm{\tilde{\beta}}_2'\bm{X}_2^{'}\bm{QX}_1\bm{\tilde{M}}^{-1}\bm{X}_1^{'}\bm{QX}_2\bm{\tilde{\beta}}_2)
\end{equation}

and the resulting determinant-based compound criterion for a blocked experiments is
\begin{align}
\label{eq::MSE_D_B}
&\left[\left|(\bm{X}^T_{p}\bm{Q}\bm{X}_{p})^{-1}\right|^{1/p}F_{p,d_B;1-\alpha_{DP}}\right]^{\kappa_{DP}} \times \notag \\ &\left[\left|\bm{\tilde{L}}+\frac{\bm{I}_{q}}{\tau^{2}}\right|^{-1/q}F_{q,d_B;1-\alpha_{LoF}}\right]^{\kappa_{LoF}}\times \\ & \left[|\bm{X}^T_{p}\bm{QX}_{p}|^{-1}\exp\left(\frac{1}{N}\sum_{i=1}^{N}\log(1+\bm{\tilde{\beta}}_{2i}'\bm{X}_q^{T}\bm{QX}_p\bm{\tilde{M}}^{-1}\bm{X}_p^{T}\bm{QX}_q\bm{\tilde{\beta}}_{2i})\right)\right]^{\kappa_{MSE}/p} \longrightarrow \mbox{ min.}\notag
\end{align}

Taking the expectation of the trace of (\ref{eq::mesb_submatrix}) makes up the trace-based component criterion:
\begin{align}
\label{eq::MSE_B_tr}
\mathtt{E}_{\beta_q}\mbox{trace}[\mbox{MSE}(\bm{\hat{\tilde{\beta}}}_p|\bm{\tilde{\beta}})_{pp}]&= \mbox{trace}[\mathtt{E}_{\beta_q}\mbox{MSE}(\bm{\hat{\tilde{\beta}}}_p|\bm{\tilde{\beta}})_{pp}]\notag \\&=\mbox{trace}[\sigma^2\bm{\tilde{M}}^{-1}_{pp}+\mathtt{E}_{\beta_q}(\bm{\tilde{A}}\bm{\beta}_q\bm{\beta}_q^T\bm{\tilde{A}})_{pp}]\notag\\&=  \sigma^2\mbox{trace}[\bm{\tilde{M}}^{-1}_{pp}+\tau^2\{\bm{\tilde{A}}\bm{\tilde{A}}^T\}_{pp}]\notag \\&=\sigma^2[\mbox{trace}(\bm{X}^T_{p}\bm{QX}_{p})^{-1}+\tau^2\mbox{trace}\{\bm{\tilde{A}}\bm{\tilde{A}}^T\}_{pp}].
\end{align}

The $MSE(LP)$-compound criterion for a blocked experiment is
\begin{align}
\label{eq::MSE_L_B}
&\left[\frac{1}{p}\mbox{trace}(\bm{WX}^T_{p}\bm{Q}\bm{X}_{p})^{-1}F_{1,d_B;1-\alpha_{LP}}\right]^{\kappa_{LP}}\times \notag \\&\left[\frac{1}{q}\mbox{trace}\left(\bm{\tilde{L}}+\bm{I}_{q}/\tau^{2}\right)^{-1}F_{1,d_B;1-\alpha_{LoF}}\right]^{\kappa_{LoF}}\times \notag \\& \left[\frac{1}{p}\mbox{trace}\{(\bm{X}^T_{p}\bm{QX}_{p})^{-1}+\tau^2[\bm{\tilde{A}}\bm{\tilde{A}}^T]_{pp}\}\right]^{\kappa_{MSE}} \longrightarrow \mbox{ min.}
\end{align}

The probability levels $\alpha$, weights $\kappa$ hold the same meanings as in the unblocked case; and as was noted before, the number of pure error degrees of freedom $d_B$ accounts for the comparisons between blocks.

%Before moving to a more detailed presentation of the design construction algorithm, recall the MSE-based criteria (\ref{eq::MSE_D}), (\ref{eq::MSE_L}), (\ref{eq::MSE_D_B}) and (\ref{eq::MSE_L_B}) that are to be used:
%
%For an unblocked experiment:
%\begin{itemize}
%\item Determinant-based $MSE(D)$ criterion:
%\begin{align}
%\label{eq::MSE_D_ms}
%\mbox{minimise }&\left[\left|\bm{X}'_{1}\bm{X}_{1}\right|^{-1/p}F_{p,d;1-\alpha_{DP}}\right]^{\kappa_{DP}} \times \notag \\ &\left[\left|\bm{L}+\frac{\bm{I}_{q}}{\tau^{2}}\right|^{-1/q}F_{q,d;1-\alpha_{LoF}}\right]^{\kappa_{LoF}}\times \notag\\ & \left[|\bm{X}'_{1}\bm{X}_{1}|^{-1}\exp\left(\frac{1}{N}\sum_{i=1}^{N}\log(1+\bm{\tilde{\beta}}_{2i}'\bm{X}_2^{'}\bm{X}_1\bm{M}^{-1}\bm{X}_1^{'}\bm{X}_2\bm{\tilde{\beta}}_{2i})\right)\right]_{.}^{\kappa_{MSE}/p}
%\end{align}
%\item Trace-based $MSE(L)$ criterion:
%\begin{align}
%\label{eq::MSE_L_ms}
%\mbox{minimise} &\left[\frac{1}{p}\mbox{trace}(\bm{WX}'_{1}\bm{X}_{1})^{-1}F_{1,d;1-\alpha_{LP}}\right]^{\kappa_{LP}}\times \notag\\& \left[\frac{1}{q}\mbox{trace}\left(\bm{L}+\frac{\bm{I}_{q}}{\tau^{2}}\right)^{-1}F_{1,d;1-\alpha_{LoF}}\right]^{\kappa_{LoF}}\times 
%\notag\\& \left[\frac{1}{p}\mbox{trace}\{(\bm{X}'_{1}\bm{X}_{1})^{-1}+\tau^2\bm{A}\bm{A}'\}\right]^{\kappa_{MSE}}_{.}
%\end{align}
%\end{itemize} 
% 
%For a blocked experiment:
%\begin{itemize}
%\item Determinant-based $MSE(D)$ criterion:
%\begin{align}
%\label{eq::MSE_D_B_ms}
%\mbox{minimise }&\left[\left|(\bm{X}'_{1}\bm{Q}\bm{X}_{1})^{-1}\right|^{1/p}F_{p,d_B;1-\alpha_{DP}}\right]^{\kappa_{DP}} \times \notag \\ &\left[\left|\bm{\tilde{L}}+\frac{\bm{I}_{q}}{\tau^{2}}\right|^{-1/q}F_{q,d_B;1-\alpha_{LoF}}\right]^{\kappa_{LoF}}\times \notag\\ & \left[|\bm{X}'_{1}\bm{QX}_{1}|^{-1}\exp\left(\frac{1}{N}\sum_{i=1}^{N}\log(1+\bm{\tilde{\beta}}_{2i}'\bm{X}_2^{'}\bm{QX}_1\bm{\tilde{M}}^{-1}\bm{X}_1^{'}\bm{QX}_2\bm{\tilde{\beta}}_{2i})\right)\right]_{.}^{\kappa_{MSE}/p}
%\end{align}
%\item Trace-based $MSE(L)$ criterion:
%\begin{align}
%\label{eq::MSE_L_B_ms}
%\mbox{minimise }&\left[\frac{1}{p}\mbox{trace}(\bm{WX}'_{1}\bm{Q}\bm{X}_{1})^{-1}F_{1,d_B;1-\alpha_{LP}}\right]^{\kappa_{LP}}\times \notag \\&\left[\frac{1}{q}\mbox{trace}\left(\bm{\tilde{L}}+\frac{\bm{I}_{q}}{\tau^{2}}\right)^{-1}F_{1,d_B;1-\alpha_{LoF}}\right]^{\kappa_{LoF}}\times \notag \\& \left[\frac{1}{p}\mbox{trace}\{(\bm{X}'_{1}\bm{QX}_{1})^{-1}+\tau^2[\bm{\tilde{A}}\bm{\tilde{A}}']_{22}\}\right]^{\kappa_{MSE}}_{.}
%\end{align}
%\end{itemize} 

Then the optimal multistratum design search is implemented following the steps below:
\begin{enumerate}
\item Starting from the first stratum, if there are any factors applied at this level, a candidate set of treatments is formed together with the fitted model matrix comprising the primary terms and the matrix of potential terms. The optimal unblocked design $X_1$ is then obtained using the usual point-exchange algorithm, by minimising (\ref{eq::MSE_D_ms}) or (\ref{eq::MSE_L_ms}). The labels of the treatments applied at the current stratum are saved at this stage as well in order to calculate the number of pure error degrees of freedom at the lower strata.

If there are no factors applied at the first stratum, we move to the second one, and conduct the optimal design search for a blocked experiment, where the number of blocks is equal to the number of units in the first stratum $n_1$, with $n_2$ runs per block; in this case criterion function (\ref{eq::MSE_D_B_ms}) or (\ref{eq::MSE_L_B_ms}) is used, and number of pure error degrees of freedom $d_B$ is calculated according to the usual blocked experiment framework. Treatment labels are saved as well. 
 
\item When moving from stratum $i-1$ to stratum $i$, all factors applied in the higher strata are now treated as ``whole-plot'' factors. The corresponding model matrix $\bm{Xw.m}$ containing terms inherited from the higher strata is expanded accordingly, as treatments applied at each unit in stratum $i-1$ are now applied to $n_i$ units of the current stratum nested within it. A similar expansion procedure is carried out for the vector (or matrix, if there are two or more higher strata with factors applied) of the treatment labels, so that for each current unit we are able to see what treatment has been applied to it at each stratum.

Blocking with no factors applied may occur at any stratum, not only at the first one. In such cases the procedure remains the same: skipping to the next stratum with some treatment applied, expanding the design matrices and vectors with treatment labels corresponding to the higher strata.

\item Once there is a model matrix with the ``whole-plot'' terms is formed, the search procedure might be started for the current stratum. The candidate set of treatments is set for the factors applied in the current stratum; parameters of interest include not only the ones formed by these factors but also their interactions with the higher strata terms. As nesting within the previous strata is treated as fixed block effects, the criteria used are the ones given in (\ref{eq::MSE_D_B_ms}) and (\ref{eq::MSE_L_B_ms}). However, there are a few features worth noting:
\begin{itemize}
\item For each design under consideration during the extensive search procedure, its model matrix $\bm{X}_1$ is now constructed by binding the ``whole-plot'' model matrix $\bm{Xw.m}$, the model comprising the terms formed from the factors applied at the current stratum $\bm{Xi.m}$, and the matrix formed of the interaction terms (if any are to be included) between the two. The same relates to the construction of the potential terms matrix $\bm{X}_2$: it needs to be recalculated every time a design point is swapped between the candidate set of the current stratum terms and the current design if it contains any interactions involving terms inherited from the previous strata. If not, it then only comprises terms from the current stratum factors.
\item Presence of the potential terms matrix in the criteria also implies that each stratum $i$ will ``have'' its own number of potential terms $q_i$ and, therefore, in the cases when the value of the variance scaling parameter $\tau^2$ depends on it, at each stratum the criterion function will be evaluated with the respective values of $\tau^2_i$ instead of some common one for all levels. In this work we consider common values of $\tau^2$, however, it is a case-sensitive parameter, and it is to be discussed in each particular case.
\item As the numbers of primary and potential terms vary from stratum to stratum, so do the significance levels $\alpha_{LP}$ and $\alpha_{LoF}$ in the case of trace-based criterion (\ref{eq::MSE_L_B_ms}):
\begin{align*}
\alpha_{LP}&=1-(1-\alpha_1)^{\frac{1}{p}},\\
\alpha_{LoF}&=1-(1-\alpha_2)^{\frac{1}{q}},
\end{align*}
as the corrected confidence levels depend on the dimension of the confidence regions (as in (\ref{eq::Sidak})). 
\item We use the same values of weights in the criteria for all strata; however, the flexibility of the algorithm allows changing weights (and even criteria) between the strata.
\end{itemize}
\item If there are at least $3$ strata with some factors applied, and when the current stratum number is $3$ or more, an additional swapping procedure is performed (the same as that described by \cite{Trinca2001multistratum}). By looking at the $i-2$ stratum units that have the same treatments applied to them, and interchange the $i-1$ stratum units within those, the performance of the design evaluated with respect to the performance at the current stratum $i$. The same swapping is performed for all the higher strata up to the first one.
\item It is all then repeated from step number $2$, until current stratum $i$ reaches the lowest stratum $s$. 
\end{enumerate}



\section{Examples}
\label{sec::examples}

Here we will study the optimal designs in terms of the criteria (\ref{eq::MSE_DP}) and (\ref{eq::MSE_LP})  in the framework a factorial experiment, with $3$ factors, each is at three levels. The small number of runs ($n=16$) allows estimation of the full-second order polynomial model ($p=10$), but we assume that the extended model, potentially providing a better fit,  contains also all third-order terms (linear-by-linear-by-linear and quadratic-by-linear interactions), $q=30$ of them in total. 

The intercept is a nuisance parameter, and so the criteria are adapted in such a way that the full information matrix in the DP- and LP-components and in the first part of the MSE-based components is replaced by the one excluding the intercept $\bm{M}_0 = \bm{X}^T_{p-1}\bm{Q}_0\bm{X}_{p-1}$, where $\bm{X}_{p-1}$ is the model matrix without the intercept, $\bm{Q}_0=\bm{I}_n-\frac{1}{n}\bm{11}'$. Otherwise the procedure of obtaining the $MSE(D)$- and $MSE(L)$-based components remains the same, and the MSE-based criteria functions that have been amended according to the intercept exclusion are referred to as MSE-DPs and MSE-LPs (similar to DPs and LPs from \cite{GilmourTrinca2012}):
\begin{align*}
\mbox{MSE-DPs:} &\left[\left|\bm{X}^T_{p-1}\bm{Q}_0\bm{X}_{p-1}\right|^{-1/(p-1)}F_{p-1,d;1-\alpha_{DP}}\right]^{\kappa_{DP}} \times \notag \\ &\left[\left|\bm{L}+\frac{\bm{I}_{q}}{\tau^{2}}\right|^{-1/q}F_{q,d;1-\alpha_{LoF}}\right]^{\kappa_{LoF}}\times \notag\\ & \left[|\bm{X}'_{p-1}\bm{Q}_0\bm{X}_{p-1}|^{-1}\exp\left(\mathtt{E}_{\bm{\tilde{\beta}}_q}\log(1+\bm{\tilde{\beta}}_q^T\bm{X}_q^{T}\bm{Q}_0\bm{X}_{p-1}\bm{M}_0^{-1}\bm{X}_{p-1}^{T}\bm{Q}_0\bm{X}_q\bm{\tilde{\beta}}_q) \right)\right]_{,}^{\kappa_{MSE}/(p-1)}\\
\mbox{MSE-LPs:} &\left[\frac{1}{p-1}\mbox{trace}(\bm{WX}^T_{p-1}\bm{Q}_{0}\bm{X}_{p-1})^{-1}F_{1,d;1-\alpha_{LP}}\right]^{\kappa_{LP}}\times \notag\\& \left[\frac{1}{q}\mbox{trace}\left(\bm{L}+\frac{\bm{I}_{q}}{\tau^{2}}\right)^{-1}F_{1,d;1-\alpha_{LoF}}\right]^{\kappa_{LoF}}\times 
\notag\\& \left[\frac{1}{p-1}\mbox{trace}[\bm{M}^{-1}+\tau^2\bm{A}\bm{A}^T]_{[p-1, p-1]}\right]_{.}^{\kappa_{MSE}}
\end{align*}

In the third component of the MSE-LP-criterion $[\bm{M}^{-1}+\tau^2\bm{A}\bm{A}^T]_{[p-1, p-1]}$ stands for the submatrix corresponding to the parameters of interest, i.e.~the first column and  row removed.

Considering two values of the variance scaling parameter $\tau^2=1$ and $\tau^2=1/q$, for each compound criterion we obtain two sets of optimal designs, their summaries are given in Tables \ref{tab::MSE(D)_ex1} and \ref{tab::MSE(L)_ex1}. 
Every row corresponds to a design that has been obtained as optimal according to the compound criterion defined by the combination of weights $\kappa$-s. We explore the distribution of degrees of freedom between the pure error and lack-of-fit components in the designs and at the optimal designs' efficiencies with respect to the individual criteria that are given in the last columns. 
%In the case of the $MSE-DP$-optimal designs, in Table \ref{tab::MSE(D)P_ex1}, the difference between their $MSE(D)$-efficiency values and the $MSE(D)$-efficiencies of the corresponding designs in Table \ref{tab::MSE(D)_ex1} indicates how much we lose in terms of the performance when the point prior is used for the $MSE(D)$-component estimation. 
Optimal designs were obtained using the point exchange algorithm (\cite{Fedorov1972theory}), with $500$ random starts; the MSE(D)-part of the compound criterion was estimated using the MC sampling, and this is the most time consuming part of the computations. When this presents a considerable challenge, we can recommend the previously mentioned alternative of imputing the point prior values of $\bm{\tilde{\beta}_{q}}$. The resulting losses in the efficiencies are quite small, and time savings are substantial -- the illustration on this example is presented in Appendix \ref{appendix::example1}.

[the Results of the smaller example are in preparation]
%%% MSE(D) designs, tau^2=1 and tau^2=1/q
\begin{table}[h]
\caption{Properties of MSE-DP-optimal designs}
\label{tab::MSE(D)_ex1}
\resizebox{\textwidth}{!}}                               \\
   & \textbf{DP}       & \textbf{LoF(DP)}    & \textbf{MSE(D)}   & \textbf{PE}        & \textbf{LoF}        & \textbf{DP}   & \textbf{LoF(DP)}   & \textbf{MSE(D)}  &  \textbf{LP}       & \textbf{LoF(LP)}   & \textbf{MSE(L)}  \\
1 & 1    & 0    & 0    & \multicolumn{1}{|r}{18} & \multicolumn{1}{r|}{1}  & 100.00 & 47.77  & 91.05  & \multicolumn{1}{|r}{96.38} & 94.92 & 10.70 \\
2 & 0    & 1    & 0    & \multicolumn{1}{|r}{8}  & \multicolumn{1}{r|}{11} & 43.70  & 100.00 & 54.74  & \multicolumn{1}{|r}{0.75}  & 89.99 & 2.08  \\
3 & 0    & 0    & 1    & \multicolumn{1}{|r}{0}  & \multicolumn{1}{r|}{19} & 0.00   & 0.00   & 100.00 & \multicolumn{1}{|r}{0.00}  & 0.00  & 22.32 \\
4 & 0.5  & 0.5  & 0    & \multicolumn{1}{|r}{11} & \multicolumn{1}{r|}{8}  & 78.50  & 87.61  & 88.56  & \multicolumn{1}{|r}{73.88} & 98.85 & 17.66 \\
5 & 0.5  & 0    & 0.5  & \multicolumn{1}{|r}{15} & \multicolumn{1}{r|}{4}  & 97.26  & 56.51  & 93.77  & \multicolumn{1}{|r}{97.55} & 50.04 & 12.74 \\
6 & 0    & 0.5  & 0.5  & \multicolumn{1}{|r}{8}  & \multicolumn{1}{r|}{11} & 64.72  & 96.84  & 87.53  & \multicolumn{1}{|r}{57.04} & 36.17 & 29.33 \\
7 & 1/3  & 1/3  & 1/3  & \multicolumn{1}{|r}{10} & \multicolumn{1}{r|}{9}  & 79.45  & 84.14  & 93.23  & \multicolumn{1}{|r}{81.06} & 43.42 & 16.71 \\
8 & 0.5  & 0.25 & 0.25 & \multicolumn{1}{|r}{13} & \multicolumn{1}{r|}{6}  & 93.38  & 64.35  & 95.55  & \multicolumn{1}{|r}{95.76} & 48.58 & 14.77 \\
9 & 0.25 & 0.5  & 0.25 & \multicolumn{1}{|r}{9} & \multicolumn{1}{r|}{10} & 69.52  & 95.76  & 87.36  & \multicolumn{1}{|r}{63.13} & 40.41 & 25.46 \\
 & & & & & & & & & & & \\
   & \multicolumn{3}{l}{\textbf{Criteria, $\bm{\tau^2=1/q$}}} & \multicolumn{2}{l}{\textbf{DoF}} & \multicolumn{6}{l}{\textbf{Efficiency,\%}}                               \\
   & \textbf{DP}       & \textbf{LoF(DP)}    & \textbf{MSE(D)}   & \textbf{PE}        & \textbf{LoF}        & \textbf{DP}   & \textbf{LoF(DP)}   & \textbf{MSE(D)}  & \textbf{LP}       & \textbf{LoF(LP)}   & \textbf{MSE(L)}  \\
1 & 1    & 0    & 0    & \multicolumn{1}{|r}{18} & \multicolumn{1}{r|}{1}  & 100.00 & 94.41  & 90.60  & \multicolumn{1}{|r}{96.42} & 98.36  & 44.94 \\
2 & 0    & 1    & 0    & \multicolumn{1}{|r}{16} & \multicolumn{1}{r|}{3}  & 39.66  & 100.00 & 37.95  & \multicolumn{1}{|r}{0.13}  & 100.00 & 0.12  \\
3 & 0    & 0    & 1    & \multicolumn{1}{|r}{0}  & \multicolumn{1}{r|}{19} & 0.00   & 0.00   & 100.00 & \multicolumn{1}{|r}{0.00}  & 0.00   & 77.97 \\
4 & 0.5  & 0.5  & 0    & \multicolumn{1}{|r}{18} & \multicolumn{1}{r|}{1}  & 100.00 & 94.41  & 90.60  & \multicolumn{1}{|r}{96.42} & 98.36  & 44.94 \\
5 & 0.5  & 0    & 0.5  & \multicolumn{1}{|r}{17} & \multicolumn{1}{r|}{2}  & 97.31  & 91.28  & 93.98  & \multicolumn{1}{|r}{96.21} & 94.32  & 50.53 \\
6 & 0    & 0.5  & 0.5  & \multicolumn{1}{|r}{15} & \multicolumn{1}{r|}{4}  & 96.10  & 92.31  & 93.29  & \multicolumn{1}{|r}{99.48} & 95.09  & 57.05 \\
7 & 1/3  & 1/3  & 1/3  & \multicolumn{1}{|r}{18} & \multicolumn{1}{r|}{1}  & 100.00 & 94.41  & 90.60  & \multicolumn{1}{|r}{96.42} & 98.36  & 44.94 \\
8 & 0.5  & 0.25 & 0.25 & \multicolumn{1}{|r}{18} & \multicolumn{1}{r|}{1}  & 100.00 & 94.41  & 90.60  & \multicolumn{1}{|r}{96.42} & 98.36  & 44.94 \\
9 & 0.25 & 0.5  & 0.25 & \multicolumn{1}{|r}{18} & \multicolumn{1}{r|}{1}  & 99.96  & 94.33  & 90.65  & \multicolumn{1}{|r}{96.24} & 98.31  & 44.84 
\end{tabular}
}
\end{table}

%The resulting designs tend to have a lot of the available degrees of freedom allocated to the pure error, except, for example, for the $MSE(D)$-optimal designs (\#$3$). However, for the smaller value of $\tau^2=1/q$ the imbalance is more extreme for the determinant-based criteria, where almost no degrees of freedom are left for lack of fit.
%
%There are no repeated designs for $\tau^2=1$, but for $\tau^2=1/q$, in Table \ref{tab::MSE(D)_ex1}, designs \#$4$, \#$7$ and \#$8$ are the same as the $DP$-optimal design, and have quite large values of other efficiencies, and design \#$9$, although it is different, has quite similar efficiency values. 
%
%
%As for the trace-based criterion and the optimal designs studied in Table \ref{tab::MSE(L)_ex1}, in general, all of them tend to have larger $LP$- and $MSE(L)$-efficiencies in case of smaller $\tau^2$, i.e. the decreased scale of the potentially missed contamination results in a more easily achievable compromise between the contradicting parts of the criteria (the same happens with the trace-based efficiencies of the MSE determinant-based optimal designs). It is also notable that, as it was observed in the case of generalised criteria, the $LoF-LP$-optimal design is also $LoF-DP$-optimal for $\tau^2=1/q$ (design \#$2$ in the corresponding tables).
%
%The $MSE(L)$-component seems to be much more sensitive to the weight allocations than the $MSE(D)$ component. For example, in the case of $\tau^2=1$ some decent efficiency values are gained only when the whole weight is on the `potential terms' criterion components, i.e.~designs \#$3$ and \#$6$. 
%
%It does not seem to work agreeably with the $LoF-LP$-component either: the $LoF-LP$-optimal design is $0.00\%$ $LP$- and $MSE(L)$-efficient in the case of $\tau^2=1$, and the efficiencies are close to $0\%$ in the case of smaller $\tau^2$. 
%% MSE(L) criteria
\begin{table}[h]
\caption{Properties of MSE(L)-optimal designs}
\label{tab::MSE(L)_ex1}
\resizebox{\textwidth}{!}}                               \\
   & \textbf{LP}       & \textbf{LoF(LP)}    & \textbf{MSE(L)}   & \textbf{PE}        & \textbf{LoF}        & \textbf{DP}   & \textbf{LoF(DP)}   & \textbf{MSE(D)}  &  \textbf{LP}       & \textbf{LoF(LP)}   & \textbf{MSE(L)}  \\
1 & 1    & 0    & 0    & \multicolumn{1}{|r}{16} & \multicolumn{1}{r|}{3} & 97.54 & 53.86 & 92.54 & \multicolumn{1}{|r}{100.00} & 96.87  & 12.08  \\
2 & 0    & 1    & 0    & \multicolumn{1}{|r}{13} & \multicolumn{1}{r|}{6} & 35.43 & 81.99 & 36.72 & \multicolumn{1}{|r}{0.00}   & 100.00 & 0.00   \\
3 & 0    & 0    & 1    & \multicolumn{1}{|r}{4} & \multicolumn{1}{r|}{15} & 18.67 & 38.43 & 51.84 & \multicolumn{1}{|r}{11.73}  & 34.79  & 100.00 \\
4 & 0.5  & 0.5  & 0    & \multicolumn{1}{|r}{15} & \multicolumn{1}{r|}{4} & 95.14 & 60.37 & 92.78 & \multicolumn{1}{|r}{99.80}  & 98.12  & 13.99  \\
5 & 0.5  & 0    & 0.5  & \multicolumn{1}{|r}{12} & \multicolumn{1}{r|}{7} & 77.77 & 72.05 & 84.91 & \multicolumn{1}{|r}{81.10}  & 98.72  & 25.19  \\
6 & 0    & 0.5  & 0.5  & \multicolumn{1}{|r}{9} & \multicolumn{1}{r|}{10} & 36.80 & 70.91 & 51.12 & \multicolumn{1}{|r}{28.13}  & 91.60  & 83.52  \\
7 & 1/3  & 1/3  & 1/3  & \multicolumn{1}{|r}{11} & \multicolumn{1}{r|}{8} & 69.53 & 73.16 & 79.71 & \multicolumn{1}{|r}{70.59}  & 97.47  & 27.98  \\
8 & 0.5  & 0.25 & 0.25 & \multicolumn{1}{|r}{12} & \multicolumn{1}{r|}{7} & 77.20 & 72.83 & 84.44 & \multicolumn{1}{|r}{81.47}  & 98.80  & 23.88  \\
9 & 0.25 & 0.5  & 0.25 & \multicolumn{1}{|r}{12} & \multicolumn{1}{r|}{7} & 70.90 & 69.80 & 78.15 & \multicolumn{1}{|r}{72.16}  & 98.49  & 26.19  \\
 & & & & & & & & & & & \\
   & \multicolumn{3}{l}{\textbf{Criteria, $\bm{\tau^2=1/q$}}} & \multicolumn{2}{l}{\textbf{DoF}} & \multicolumn{6}{l}{\textbf{Efficiency,\%}}                               \\
   & \textbf{LP}       & \textbf{LoF(LP)}    & \textbf{MSE(L)}   & \textbf{PE}        & \textbf{LoF}        & \textbf{DP}   & \textbf{LoF(DP)}   & \textbf{MSE(D)}  & \textbf{LP}       & \textbf{LoF(LP)}   & \textbf{MSE(L)}  \\
1 & 1    & 0    & 0    & \multicolumn{1}{|r}{16} & \multicolumn{1}{r|}{3} & 97.54 & 92.13 & 92.18 & \multicolumn{1}{|r}{100.00} & 95.61  & 52.55  \\
2 & 0    & 1    & 0    & \multicolumn{1}{|r}{16} & \multicolumn{1}{r|}{3} & 39.66 & 100.00 & 37.95 & \multicolumn{1}{|r}{0.13}   & 100.00 & 0.12   \\
3 & 0    & 0    & 1    & \multicolumn{1}{|r}{3} & \multicolumn{1}{r|}{16} & 0.77  & 0.93  & 83.48 & \multicolumn{1}{|r}{0.02}   & 0.01   & 100.00 \\
4 & 0.5  & 0.5  & 0    & \multicolumn{1}{|r}{17} & \multicolumn{1}{r|}{2} & 96.87 & 94.29 & 89.84 & \multicolumn{1}{|r}{97.97}  & 97.80  & 51.17  \\
5 & 0.5  & 0    & 0.5  & \multicolumn{1}{|r}{12} & \multicolumn{1}{r|}{7} & 79.66 & 87.18 & 86.32 & \multicolumn{1}{|r}{84.81}  & 88.22  & 79.60  \\
6 & 0    & 0.5  & 0.5  & \multicolumn{1}{|r}{13} & \multicolumn{1}{r|}{6} & 76.46 & 89.87 & 80.78 & \multicolumn{1}{|r}{79.23}  & 91.48  & 79.11  \\
7 & 1/3  & 1/3  & 1/3  & \multicolumn{1}{|r}{13} & \multicolumn{1}{r|}{6} & 81.53 & 90.09 & 85.52 & \multicolumn{1}{|r}{85.96}  & 91.56  & 76.65  \\
8 & 0.5  & 0.25 & 0.25 & \multicolumn{1}{|r}{15} & \multicolumn{1}{r|}{4} & 90.60 & 91.94 & 88.43 & \multicolumn{1}{|r}{95.09}  & 94.75  & 63.82  \\
9 & 0.25 & 0.5  & 0.25 & \multicolumn{1}{|r}{15} & \multicolumn{1}{r|}{4} & 84.12 & 92.88 & 83.39 & \multicolumn{1}{|r}{87.86}  & 95.51  & 72.66 
\end{tabular}
}
\end{table}

%Regarding the designs' performances with respect to the $DP$- and $LP$-components, it can be observed that the designs tend to be quite $DP$-efficient, overall more efficient in the case of $MSE(D)$-efficient designs with smaller $\tau^2$. $DP$-efficient designs are not bad in terms of $LP$-efficiency and vice versa; again, the same cannot be said for the lack-of-fit components and seems not to be true at all for the $MSE$ components, especially, for the $MSE(L)$-optimal design when $\tau^2=1$.
%   
%$LP$- and $MSE(L)$-components seem to be in a conflict, though for $\tau^2=1/q$ $MSE(L)$-efficiency values are stable across the designs, therefore, making it possible to find compromise designs which at least perform not too badly with regard to both of these criterion parts; for example, designs \#$5$ -- \#$7$ are more than $75\%$-efficient (Table \ref{tab::MSE(L)_ex1}).   

The observed relationships between the components of the criteria provide some ideas on how the designs obtained as optimal with respect to the various allocations of weights differ, what they have in common, and how the components interact when the weights are reallocated. A compromise can be found between the criterion components corresponding to the properties of the primary terms and the components responsible for the reducing the negative impact from the assumed potential model misspecification. Thereby, the careful choice of the prior parameters ($\tau^2$ in this case) seems to be of a certain importance as well. 

%% Criteria for blocked experiments as a special case of MS

\subsection{Example. Case study}
\label{subsec::case_study}
We shall consider an example of a real-life optimal design problem, and explore a range of solutions provided by the compound criteria constructed in previous sections.
A company specialising in food supplements production for small animals were to conduct an experiment to figure out whether a slight decrease in the recommended dosages of three particular products taken together would make a significant impact on the resulting ``performance'', which is expressed in terms of some continuous response. The dosage range of interest for each supplement (experimental factor) is from $90\%$ to $100\%$ of the standard recommendation; it is desired that there would be $3$ levels (i.e. taking the values of $90\%$, $95\%$ and $100\%$). Carrying out the experiment with more than $3$ levels was more complicated: measuring, for example, $92.5\%$ of the recommended dosage was quite tricky. 

The treatments were to be applied to $n = 36$ cages of animals (experimental units), which would be allocated in $b=2$ equal sized blocks. The fullest response surface model, therefore, would contain all linear, quadratic and interaction terms: the primary model contains $p=9$ terms. As it was suspected that increasing dosages beyond certain values might not be of an impact, that would be translated in a non-quadratic curvature of the fitted function -- meaning that a presence of higher order terms would provide a better fit for the data, and it was desirable to accommodate for that possibility at the design stage. In the extended model We accounted for $q=10$ potential terms ((linear-by-linear-by-linear, quadratic-by-linear and cubic) in the extended model, with the notations the same as before:
\begin{equation*}
\bm{Y}=\bm{Z\beta}_{B}+\bm{X}_{p}\bm{\beta}_{p}+\bm{X}_{q}\bm{\beta}_{q}+\bm{\varepsilon},
\end{equation*}

The design search was performed among a larger $5$-level candidate set of points, but due to the form of the $2$nd order polynomial primary model and the criteria used, the resulting designs were $3$-level ones. The experimenters also wished to have at least two centre points in each block to ensure representation of the conditions thought a priori most likely to be best (with dosages of $95\%$ for each supplement), i.e. $4$ runs in total were fixed beforehand; this constraint was built directly into the search procedure; we also obtained designs without this restriction and evaluated the efficiency losses.

The experimenters preferred using the determinant-based criterion  as the primary inferential interest is on the overall input of the model terms; we searched for exact optimal designs with respect to the compound MSE(DP)-criteria for blocked experiments (\ref{eq::MSE_B_det}). For each combination of weights we will consider two cases: $\tau^2=1$ and $\tau^2=1/q$; as for the number of Monte Carlo samples used to estimate the third criterion component, for$\tau^2=1$ we set $N=500$, and for $\tau^2=1/q$ we set $N=1000$ in order to have a sufficiently small relative estimation error ([cite the thesis?]). The search for each design was performed with $50$ random starts.
We considered three sets of weights: (1) with the weight being equally distributed between the components, (2) a bit more weight ($0.4$) put on the DP-component, with the rest allocated to the lack-of-fit component, and (3) with half of the weight on the $MSE(D)$-component with the rest of it distributed evenly among the others. 

%%% MSE(D)-optimal designs, WITHOUT centre points: DP, LoF(DP) and MSE(D)
Table \ref{tab::MSE(D)_caseCP} contains the summaries of the optimal designs; the two types of efficiencies are presented: with respect to the individual criteria with (``CP Efficiency'') and without (``No CP Efficiency'') the fixed two centre points. The former ones will be, obviously, larger, and the differences represent the magnitude of the loss by restricting the set of designs to be considered. The``Relative Efficiency'' column reflects how well the given design performs with respect to the optimal one in terms of the same compound criterion, obtained without fixing the centre points, providing a ``single-value'' measure on the efficiency reduction.  

%%% MSE(D)-optimal designs, WITH centre points: DP, LoF(DP) and MSE(D)

\begin{table}[h]
\caption{Case-study. Properties of MSE(D)-optimal blocked designs, with two centre points per block}
\label{tab::MSE(D)_caseCP}
\resizebox{\textwidth}{!}}  & \multicolumn{3}{l}{\textbf{CP Efficiency,\%}}& \multicolumn{1}{l}{\textbf{Relative}}                          \\
   & \textbf{DP}       & \textbf{LoF(DP)}    & \textbf{MSE(D)}   & \textbf{PE}        & \textbf{LoF}        & \textbf{DP}   & \textbf{LoF(DP)}   & \textbf{MSE(D)}  &  \textbf{DP}       & \textbf{LoF(DP)}   & \textbf{MSE(D)} & \textbf{Efficiency,\%} \\
1 & 1/3 & 1/3 & 1/3 & \multicolumn{1}{|r}{14} & \multicolumn{1}{r|}{11} & 88.63 & 90.03 & 99.97 & \multicolumn{1}{|r}{92.89} & 97.59 & \multicolumn{1}{r|}{100.75} & 98.18 \\
2 & 0.4 & 0.2 & 0.4 & \multicolumn{1}{|r}{14} & \multicolumn{1}{r|}{11} & 88.35 & 90.14 & 99.15 & \multicolumn{1}{|r}{92.60} & 97.70 & \multicolumn{1}{r|}{99.92} & 98.31 \\
3 & 0.25 & 0.25 & 0.5 & \multicolumn{1}{|r}{14} & \multicolumn{1}{r|}{11} & 88.63 & 90.03 & 99.75 & \multicolumn{1}{|r}{92.89} & 97.59 & \multicolumn{1}{r|}{100.53} & 98.90 \\
4 & 1 & 0 & 0 & \multicolumn{1}{|r}{20} & \multicolumn{1}{r|}{5} & 95.41 & 62.13 & 95.24 & \multicolumn{1}{|r}{100.00} & 67.34 & \multicolumn{1}{r|}{95.98} & 95.58 \\
5 & 0 & 1 & 0 & \multicolumn{1}{|r}{14} & \multicolumn{1}{r|}{11} & 66.49 & 92.26 & 79.91 & \multicolumn{1}{|r}{69.69} & 100.00 & \multicolumn{1}{r|}{80.53} & 92.26 \\
6 & 0 & 0 & 1 & \multicolumn{1}{|r}{14} & \multicolumn{1}{r|}{11} & 88.31 & 88.35 & 99.23 & \multicolumn{1}{|r}{92.55} & 95.77 & \multicolumn{1}{r|}{100.00} & 99.23 \\
 & & & & & & & & & & & & \\
   & \multicolumn{3}{l}{\textbf{Criteria, $\bm{\tau^2=1/q}$}} & \multicolumn{2}{l}{\textbf{DoF}} & \multicolumn{3}{l}{\textbf{No CP Efficiency,\%}}  & \multicolumn{3}{l}{\textbf{CP Efficiency,\%}}& \multicolumn{1}{l}{\textbf{Relative}}                          \\
   & \textbf{DP}       & \textbf{LoF(DP)}    & \textbf{MSE(D)}   & \textbf{PE}        & \textbf{LoF}        & \textbf{DP}   & \textbf{LoF(DP)}   & \textbf{MSE(D)}  &  \textbf{DP}       & \textbf{LoF(DP)}   & \textbf{MSE(D)} & \textbf{Efficiency,\%} \\
1 & 1/3 & 1/3 & 1/3 & \multicolumn{1}{|r}{18} & \multicolumn{1}{r|}{7} & 92.52 & 95.13 & 95.86 & \multicolumn{1}{|r}{96.97} & 98.00 & \multicolumn{1}{r|}{96.55} & 94.95 \\
2 & 0.4 & 0.2 & 0.4 & \multicolumn{1}{|r}{18} & \multicolumn{1}{r|}{7} & 94.19 & 92.19 & 96.54 & \multicolumn{1}{|r}{98.72} & 94.98 & \multicolumn{1}{r|}{97.24} & 95.34 \\
3 & 0.25 & 0.25 & 0.5 & \multicolumn{1}{|r}{17} & \multicolumn{1}{r|}{8} & 92.44 & 93.70 & 97.05 & \multicolumn{1}{|r}{96.88} & 96.53 & \multicolumn{1}{r|}{97.75} & 95.72 \\
4 & 1 & 0 & 0 & \multicolumn{1}{|r}{20} & \multicolumn{1}{r|}{5} & 95.41 & 90.63 & 95.66 & \multicolumn{1}{|r}{100.00} & 93.37 & \multicolumn{1}{r|}{96.35} & 95.41 \\
5 & 0 & 1 & 0 & \multicolumn{1}{|r}{22} & \multicolumn{1}{r|}{3} & 76.11 & 97.07 & 77.10 & \multicolumn{1}{|r}{79.77} & 100.00 & \multicolumn{1}{r|}{77.65} & 97.07 \\
6 & 0 & 0 & 1 & \multicolumn{1}{|r}{14} & \multicolumn{1}{r|}{11} & 88.37 & 90.37 & 99.29 & \multicolumn{1}{|r}{92.62} & 93.10 & \multicolumn{1}{r|}{100.00} & 99.29
\end{tabular}
}
\end{table}

The main feature observed is that in general individual efficiency values are quite large, both in the restricted or unrestricted cases. This might be attributed to the large number of available residual degrees of freedom ($25$), and this contributes to better compromises achievable between the three criterion components. The imbalance in the distribution of the residual degrees of freedom is not strong, however, the prevalence towards the pure error is quite consistent. 
In general, the performances are better for smaller $\tau^2$ (except for the $MSE(D)$ part), as in this case the model disturbance effect is assumed to be quite small, and the compromise might be more feasible. Overall, relative efficiencies are quite good, losses among the first three designs (optimal with respect to the compound criteria) do not exceed $1.82\%$ for $\tau^2$ and $5.05\%$ for $\tau^2=1/q$.

It is notable that designs \#$1$ and \#$3$ (for the larger $\tau^2$) are the same. Its $MSE(D)$-value turned out to be better than of the design \#$6$, which was constructed as being optimal with respect to this component. This design was chosen to be used when carrying out the experiment; it has been run successfully and useful conclusions were drawn from the obtained data. It can be found in Table \ref{tab::CS_Design} and on Figure \ref{fig::CS_design}, the points in each block have been ordered for the sake of easier perception, and, of course, they have to be and were randomised in each block before running the experiment. There are only two centre points in each block, and replicates of other points are generally split between blocks (except for the $(-1,1,1)$ point which is duplicated in the first block only).

\begin{table}[h]
\caption{Case-study. MSE(D)-optimal design \#$1$ with two centre points, $\tau^2=1$}
\begin{center}
\label{tab::CS_Design}
\scalebox{0.73}{
%\resizebox{\textwidth}{!}{%
\begin{tabular}{rrrrrrr}
-1 & -1 & -1 &  & -1 & -1 & -1 \\
-1 & -1 & 0  &  & -1 & -1 & 0  \\
-1 & -1 & 1  &  & -1 & -1 & 1  \\
-1 & 0  & -1 &  & -1 & 0  & 1  \\
-1 & 1  & -1 &  & -1 & 1  & -1 \\
-1 & 1  & 1  &  & -1 & 1  & 0  \\
-1 & 1  & 1  &  & 0  & -1 & -1 \\
0  & -1 & 1  &  & 0  & -1 & 1  \\
0  & 0  & 0  &  & 0  & 0  & 0  \\
0  & 0  & 0  &  & 0  & 0  & 0  \\
0  & 1  & -1 &  & 0  & 1  & 1  \\
1  & -1 & -1 &  & 1  & -1 & -1 \\
1  & -1 & 0  &  & 1  & -1 & 0  \\
1  & -1 & 1  &  & 1  & -1 & 1  \\
1  & 0  & 1  &  & 1  & 0  & -1 \\
1  & 1  & -1 &  & 1  & 0  & 1  \\
1  & 1  & 0  &  & 1  & 1  & -1 \\
1  & 1  & 1  &  & 1  & 1  & 1 
\end{tabular}
}
\end{center}
\end{table}

\begin{figure}[h]
\begin{center}
\includegraphics[scale=0.65]{CS_design.jpg}      %width=\textwidth
\caption{MSE(D)-optimal design \#$1$. Colours (blue and red) and symbols (`x' and `o') serve as block indicators. }
\label{fig::CS_design}
\end{center}
\end{figure} 

As for the time costs, on average an optimal design was found in $13-15$ hours, which was acceptable in this particular case. Sometimes, however, it took up to $20-24$ hours, so some extra time allowance should be accounted for when using these criteria and this search algorithm and/or the extensive sampling might be replaced by a less demanding alternative. 

%%% MSE(D)-optimal designs, WITH centre points: DP, LoF(DPs) and MSE(D) -- LoF as Ds for potential terms
\newpage
[OE: I'm not sure we need to include the following: looking at what would have happened if we could afford $5$ levels.]

Although the primary choice of the number of levels of each factor is $3$, the number of experimental runs would allow for $5$ levels so that all of the potential terms can be estimated. So we considered one more compound criterion which has a different lack-of-fit component: $DP_S$-- optimality for potential terms in the full (third-order polynomial) model, arising from the $D_S$-optimality that was suggested by \cite{Atkinson2007} (page $360$) in the context of model discrimination:
\begin{align}
\label{eq::crit2}
&\left[\left|\bm{\tilde{M}}\right|^{-1/p}\mathrm{F}(p,d,\alpha_{\!_{DP}})\right]^{\kappa_{\!_{DP}}} \times  \left[|\bm{\tilde{L}}|^{-1/q}F_{q,d_B;1-\alpha_{LoF}}\right]^{\kappa_{\!_{LoF}}}\times \notag\\ & \left[|\bm{\tilde{M}}|^{-1}\exp\left(\frac{1}{N}\sum_{i=1}^{N}\log(1+\bm{\tilde{\beta}}_{2i}'\bm{X}_q^{T}\bm{Q}\bm{X}_1\bm{\tilde{M}}^{-1}\bm{X}_1^{'}\bm{Q}\bm{X}_q\bm{\tilde{\beta}}_{2i})\right)\right]_{.}^{\kappa_{\!_{MSE}}/p}
\end{align}

This criterion further on will be referred to as the ``full'' criterion, and we shall explore how this would affect the resulting designs.

The summary of the corresponding optimal designs is given in Table \ref{tab::MSE(D)_caseCPLoF}: the `new' lack-of-fit component is denoted by DP*s. The notions of `No CP'  and `CP' efficiencies are the same as previously. For each design we calculated its $LoF(DP)$-value, so that we could assess how they would perform in terms of the criterion (\ref{eq::LoFDP_criterion}). The performances of the optimal designs are summarised in Table \ref{tab::MSE(D)_caseCPLoF}. In order to illustrate general tendencies in the designs' appearances, designs optimal with respect to the criterion with equal weights, for two values of $\tau^2$ are provided in Table \ref{tab::Full_designs}.

\begin{table}[h]
\centering
\caption{Case-study. Properties of ``Full'' MSE(D)-optimal blocked designs, with two centre points per block}
\label{tab::MSE(D)_caseCPLoF}
\resizebox{\textwidth}{!}}  & \multicolumn{4}{l}{\textbf{CP Efficiency,\%}}& \multicolumn{1}{l}{\textbf{Relative}}                          \\
   & \textbf{DP}       & \textbf{DP*s}    & \textbf{MSE(D)}   & \textbf{PE}        & \textbf{LoF}        & \textbf{DP} & \textbf{DP*s}  & \textbf{LoF(DP)}   & \textbf{MSE(D)}  &  \textbf{DP}  &  \textbf{DP*s} & \textbf{LoF(DP)}   & \textbf{MSE(D)} & \textbf{Efficiency,\%} \\
1 & 1/3 & 1/3 & 1/3 & \multicolumn{1}{|r}{14} & \multicolumn{1}{r|}{11} & 80.58 & 69.38 & 87.05 & 91.74 & \multicolumn{1}{|r}{84.46} & 79.90 & 94.36 & \multicolumn{1}{r|}{92.46} & 94.42 \\
2 & 0.4 & 0.2 & 0.4 & \multicolumn{1}{|r}{14} & \multicolumn{1}{r|}{11} & 83.90 & 61.18 & 85.36 & 57.47 & \multicolumn{1}{|r}{87.93} & 70.46 & 92.52 & \multicolumn{1}{r|}{57.92} & 95.83 \\
3 & 0.25 & 0.25 & 0.5 & \multicolumn{1}{|r}{13} & \multicolumn{1}{r|}{12} & 79.93 & 68.24 & 87.06 & 94.05 & \multicolumn{1}{|r}{83.77} & 78.59 & 94.37 & \multicolumn{1}{r|}{94.79} & 96.32 \\
4 & 1 & 0 & 0 & \multicolumn{1}{|r}{20} & \multicolumn{1}{r|}{5} & 95.41 & 0.00 & 62.13 & 95.24 & \multicolumn{1}{|r}{100.00} & 0.00 & 67.34 & \multicolumn{1}{r|}{95.98} & 95.41 \\
5 & 0 & 1 & 0 & \multicolumn{1}{|r}{14} & \multicolumn{1}{r|}{11} & 62.01 & 86.83 & 92.98 & 74.95 & \multicolumn{1}{|r}{64.99} & 100.00 & 100.78 & \multicolumn{1}{r|}{75.53} & 86.83 \\
6 & 0 & 0 & 1 & \multicolumn{1}{|r}{14} & \multicolumn{1}{r|}{11} & 88.31 & 0.00 & 88.35 & 99.23 & \multicolumn{1}{|r}{92.55} & 0.00 & 95.77 & \multicolumn{1}{r|}{100.00} & 99.23 \\
 & & & & & & & & & & & & & & \\
   & \multicolumn{3}{l}{\textbf{Criteria, $\bm{\tau^2=1/q}$}} & \multicolumn{2}{l}{\textbf{DoF}} & \multicolumn{4}{l}{\textbf{No CP Efficiency,\%}}  & \multicolumn{4}{l}{\textbf{CP Efficiency,\%}}& \multicolumn{1}{l}{\textbf{Relative}}                          \\
    & \textbf{DP}       & \textbf{DP*s}    & \textbf{MSE(D)}   & \textbf{PE}   & \textbf{LoF}        & \textbf{DP} & \textbf{DP*s}  & \textbf{LoF(DP)}   & \textbf{MSE(D)}  &  \textbf{DP} & \textbf{DP*s} & \textbf{LoF(DP)}   & \textbf{MSE(D)} & \textbf{Efficiency,\%} \\
1 & 1/3 & 1/3 & 1/3 & \multicolumn{1}{|r}{14} & \multicolumn{1}{r|}{11} & 80.14 & 69.01 & 87.70 & 91.41 & \multicolumn{1}{|r}{83.99} & 81.43 & 90.34 & \multicolumn{1}{r|}{92.07} & 93.97\\
2 & 0.4 & 0.2 & 0.4 & \multicolumn{1}{|r}{14} & \multicolumn{1}{r|}{11} & 83.76 & 61.29 & 88.28 & 94.41 & \multicolumn{1}{|r}{87.79} & 72.31 & 90.95 & \multicolumn{1}{r|}{95.09} & 96.04\\
3 & 0.25 & 0.25 & 0.5 & \multicolumn{1}{|r}{12} & \multicolumn{1}{r|}{13} & 79.66 & 64.43 & 84.44 & 94.84 & \multicolumn{1}{|r}{83.49} & 76.02 & 86.99 & \multicolumn{1}{r|}{95.52} & 95.20\\
4 & 1 & 0 & 0 & \multicolumn{1}{|r}{20} & \multicolumn{1}{r|}{5} & 95.41 & 0.00 & 90.63 & 95.66 & \multicolumn{1}{|r}{100.00} & 0.00 & 93.37 & \multicolumn{1}{r|}{96.35} &  95.41\\
5 & 0 & 1 & 0 & \multicolumn{1}{|r}{14} & \multicolumn{1}{r|}{11} & 59.81 & 84.75 & 86.15 & 72.05 & \multicolumn{1}{|r}{62.68} & 100.00 & 88.75 & \multicolumn{1}{r|}{72.57} & 84.75 \\
6 & 0 & 0 & 1 & \multicolumn{1}{|r}{14} & \multicolumn{1}{r|}{11} & 88.73 & 0.00 & 90.57 & 99.29 & \multicolumn{1}{|r}{92.99} & 0.00 & 93.30 & \multicolumn{1}{r|}{100.00} & 99.29
\end{tabular}
}
\end{table}

\begin{table}[h]
\centering
\caption{Case-study. Designs \#$1$ from Table \ref{tab::MSE(D)_caseCPLoF}, $\tau^2=1$ (left) and $\tau^2=1/q$ (right)}
\begin{center}
\label{tab::Full_designs}
\scalebox{0.73}{
\begin{tabular}{rrrrrrrr|r|rrrrlrrr}
-1 & -1 & 0 &  & -1 & -1 & -1 &  &  &  & -1 & -1 & -1 &  & -1 & -1 & -1 \\
-1 & -1 & 1 &  & -1 & -1 & 0 &  &  &  & -1 & -1 & 0.5 &  & -1 & -1 & 0.5 \\
-1 & 0 & -1 &  & -1 & -1 & 1 &  &  &  & -1 & -0.5 & 1 &  & -1 & -0.5 & 1 \\
-1 & 0.5 & 1 &  & -1 & 0.5 & 1 &  &  &  & -1 & 0 & -0.5 &  & -1 & 1 & -1 \\
-1 & 1 & -1 &  & -1 & 1 & -1 &  &  &  & -1 & 1 & -1 &  & -1 & 1 & 1 \\
-1 & 1 & 0.5 &  & -1 & 1 & 0.5 &  &  &  & -1 & 1 & 1 &  & -0.5 & -1 & -0.5 \\
-0.5 & 1 & 1 &  & -0.5 & -0.5 & 1 &  &  &  & -0.5 & 1 & 0 &  & -0.5 & 0.5 & -1 \\
0 & -1 & -1 &  & -0.5 & 1 & -0.5 &  &  &  & 0 & -1 & 1 &  & 0 & -1 & 1 \\
0 & 0 & 0 &  & 0 & -1 & -1 &  &  &  & 0 & 0 & 0 &  & 0 & 0 & 0 \\
0 & 0 & 0 &  & 0 & 0 & 0 &  &  &  & 0 & 0 & 0 &  & 0 & 0 & 0 \\
0.5 & -1 & 1 &  & 0 & 0 & 0 &  &  &  & 0 & 1 & 1 &  & 0 & 1 & 1 \\
0.5 & 1 & -1 &  & 0.5 & -1 & 1 &  &  &  & 0.5 & -0.5 & -1 &  & 0.5 & 1 & -0.5 \\
1 & -1 & -1 &  & 0.5 & 1 & -1 &  &  &  & 1 & -1 & -1 &  & 1 & -1 & -1 \\
1 & -1 & 0.5 &  & 1 & -1 & -1 &  &  &  & 1 & -1 & 0 &  & \textbf{1} & \textbf{-1} & \textbf{1} \\
1 & -0.5 & 1 &  & 1 & -1 & 0.5 &  &  &  & 1 & 0.5 & 1 &  & \textbf{1} & \textbf{-1} & \textbf{1} \\
1 & 0.5 & -1 &  & 1 & -0.5 & -0.5 &  &  &  & \textbf{1} & \textbf{1} & \textbf{-1} &  & 1 & 0 & -0.5 \\
1 & 1 & -0.5 &  & 1 & 0.5 & -1 &  &  &  & \textbf{1} & \textbf{1} & \textbf{-1} &  & 1 & 0.5 & 1 \\
1 & 1 & 1 &  & 1 & 1 & 1 &  &  &  & 1 & 1 & 0.5 &  & 1 & 1 & 0.5
\end{tabular}
}
\end{center}
\end{table}

In case of $\tau^2=1$ all pure error degrees of freedom (except for the $2$ coming from the replicated centre points) occur from $12$ points duplicated in different blocks; in case of $\tau^2=1/q$ (i.e. $\tau^2=0.1$) two `corner' points are replicated within the same block (they are highlighted in Table \ref{tab::Full_designs}). Quite a few experimental units would receive an `intermediate' $\pm 0.5$ dosage of at least one product, and as this did not comply with the demands of the experimenters, the choice was still made in favour of the three-level design in Table \ref{tab::CS_Design}. 

\section{Discussion}
\label{sec::discussion}

\section{Acknowledgements}

%%% Bibliography
\cleardoublepage
\phantomsection
\addcontentsline{toc}{chapter}{Bibliography} 
%\bibliographystyle{apalike}
\bibliographystyle{rss}
\bibliography{thesis_bib}

\end{document}