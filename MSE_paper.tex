\documentclass[11pt]{article}

%%%%%%%%%%%%      PACKAGES        %%%%%%%%%%%%
\usepackage[colorlinks=true,linkcolor=blue,urlcolor=blue,citecolor=blue]{hyperref}
\usepackage{setspace}
\onehalfspacing
%\doublespacing
\usepackage{fancyhdr,afterpage}
\usepackage{lscape}
\usepackage{multirow}
\usepackage{longtable}
\usepackage{textcomp,latexsym}
\usepackage{parskip}
\usepackage[round]{natbib}
\usepackage{adjustbox,lipsum}
\usepackage[titletoc]{appendix}
    
\usepackage[round]{natbib}
\usepackage{amsmath}
\usepackage{graphicx}
\usepackage{gensymb}
\usepackage{tikz}
\usepackage{float}
\usepackage{slashbox}
\usepackage{caption}
\usepackage{subcaption}
\usepackage{amsfonts}
\usepackage{enumerate}
\usepackage{amssymb}
\usepackage{multirow}
\usepackage{bm}
\setcounter{tocdepth}{3}
\setcounter{secnumdepth}{3}

\DeclareMathOperator*{\argmax}{arg\,max}
\DeclareMathOperator*{\argmin}{arg\,min}

\usepackage[a4paper,inner=4.2cm,outer=2.3cm,top=2.5cm,bottom=2.5cm,pdftex]{geometry} % MARGINS


\begin{document}
\section{Abstract}

\section{Introduction}

Most experiments are conducted in order to gain understanding of the impact that different parameters of interest make on the outcome; the quantitative measure of this impact allows comparisons as well as making interpretable conclusions regarding the scale and the pattern of the relationships between process parameters and measured output. 
%% Polynomial models
While the exact true nature of that relationship remains unknown, we need to have some form of approximation. Polynomial functions are well known to be able to provide an approximation of any required precision for functions from a certain class of differentiability \citep{Rudin1987real}: obviously, higher desirable precision would require a polynomial of a higher order and, therefore, more experimental effort. Response Surface Methodology (RSM) that was first introduced by \cite{Box1951Roy} is aimed at optimising the response by fitting a second order polynomial to the data obtained after an experiment on the points within the experimental region of interest has been performed. The response measured at each of $n$ experimental runs would then be presented as a linear combination of experimental factors' values, their interactions and corresponding quadratic terms; of course, some of the terms might be excluded, but, unless stated otherwise, we will consider a full second-order polynomial model with $1+2k+\frac{k(k-1)}{2}$ parameters (including the intercept), where $k$ is the number of factors. Together with the fixed polynomial part, the error term here would stand for all the variation in data that has not been captured by the model. 
%% Experimental design
At the stage of experimental planning, when there is little, if any, prior knowledge and no data available, there is still a lot that can and should be done to make sure the results of the experimentation and the following analysis are credible.

The design construction usually depends on the model in several ways. Firstly, the interest in the properties of the model parameters' estimation leads to the development of various optimality criteria (e.g.~so-called `alphabetic criteria'). Secondly, the fact of the model-dependence of the majority of the criteria implies the necessity of taking into account the possible lack-of-fit as well as the desirability of obtaining error variance estimates from replicated observations (model-independent, `pure' error). The latter two are also in the list of the properties of what would be considered as a `good design' (summarised by \cite{Box1987empirical}, Chapter $14$): it should ``make it possible to detect lack of fit'' and ``provide an internal estimate of error from replication''.

It is obvious that all the individual criteria are not interchangeable and can possibly be contradictory, however, they all are desired to be accounted for. There are a few ways of combining them, and here we will be working with the notion of a compound optimality criterion which is basically a weighted combination of the elementary criteria, where weights are generally arbitrary, and are expected to be chosen in accordance with the experimenter's beliefs and intentions. 

We will focus in this work on factorial experiments with a relatively small number of runs and the fitted model being a polynomial regression. We work here with exact designs: for each run we are to provide a point as a combination of the treatment factors' levels; therefore, finding the corresponding optimal design is essentially a matrix optimisation problem, i.e. searching for a design maximising (minimising) the criterion function. When considering examples of different compound criteria with various weight allocations, we will examine the resulting optimal designs in terms of their performances with respect to individual components. This would allow evaluating how much is ``lost'' when achieving a compromise, what component criteria contradict each others' performances, and how they are affected by changes of weight allocations.

The first and main objective of the research presented is to develop a methodology for combining several desirable data inference properties in compound optimality criteria for factorial experiments. The properties are to correspond to the precision of inference based on the primary model and to the possibility of the specified model contamination presence; the components are aimed to comply with the `pure error' strategy, where it is feasible.

The obvious following objective is to investigate the relationships between the individual components of the introduced criteria and, by considering some examples, provide some empirical recommendations for complex experimentation research that would benefit from applying these methods.

The criteria will also be adapted to be applied within experimental frameworks with restricted randomisation: blocked and multistratum experiments.

\section{Background}

Recall the polynomial regression model (\ref{eq::back_model}) for the unblocked experiment, containing initial factors, their powers and interactions is used as a good approximation to the relation between the variables:  

\begin{equation}
\label{eq::back_model}
\bm{Y}=\bm{X\beta}+\bm{\varepsilon}.
\end{equation} 

Here $\bm{X}$ is the $n\times p$ model matrix, $\bm{Y}$ is the $n\times 1$ vector of responses; $\bm{\beta}$ is the $p\times 1$ vector of parameters corresponding to the model terms and $\bm{\varepsilon}$ are independent normally distributed random error terms with constant variance, i.e. $\bm{\varepsilon}\sim \mathcal{N}(\bm{0},\sigma^{2}\bm{I}_{n})$.

\subsection{Pure error}
The error variance estimation can be obtained in two ways. The first one is the mean square error: $\hat{\sigma}^2_{mse}=\mbox{Residual SS}/(n-p)$, i.e. the residual sum of squares divided by the corresponding number of degrees of freedom (obtained from the ANOVA decomposition, the details can be found in \cite{Draper1998}). This estimator obviously depends on the model and on the number of its parameters. The other one is model-independent `pure' error, derived from the further decomposition of the residual sum of squares into the `pure' error and `lack-of-fit' components: $\hat{\sigma}^2_{PE}=\mbox{Pure error SS}/(n-t)$, where $t$ is the number of unique treatments applied and $d=n-t$ is the pure error degrees of freedom, i.e. the number of replicated points. In other words, the error is estimated by fitting the full treatment model:

\begin{equation}
\label{eq::back_trnt}
\bm{Y}=\bm{X_{t}\mu_{t}}+\bm{\varepsilon},
\end{equation} 

where $\bm{X_{t}}$ is the $n\times t$ full treatment model matrix, in which the $(i,j)^{th}$ element is equal to $1$ if treatment $j$ is applied to the $i^{th}$ unit, otherwise it is set to $0$. Then the elements of the $t$-dimensional vector $\bm{\mu_{t}}$ are the mean effects of each treatment. The vector of errors $\bm{\varepsilon}$ comprises the between-unit variation, such that $\mbox{E}(\bm{\varepsilon})=\bm{0}$, $\mbox{Var}(\bm{\varepsilon})=\sigma^2\bm{I}_{n}$.

Many authors advocate the use of the `pure' error estimate of $\sigma^2$ instead of the one pooled with the lack-of-fit part from the model (\ref{eq::back_model}): \cite{Cox1958planning} recommends using it for the estimation unless there are no replicate treatments, while \cite{Draper1998} argue for the reliability of `pure' error and recommend aiming for the presence of replicates at the stage of planning. \cite{Atkinson2007} also state that ``If lack-of-fit of the model is of potential interest, $\sigma^{2}$ is better estimated from replicate observations'' (page 22). For response surface experiments in blocks \cite{Gilmour2000PErsm} explicated the definition of pure error and its estimate which are compatible with the unblocked case: from replicates but also taking into account  block effects additive to treatment effects. Finally, the work by \cite{GilmourTrinca2012} that inspired this research, comprises a thorough analysis in favour of estimating the error from the full treatment model which is true regardless of what function is used to approximate the relationship of interest.


\subsection{Combining criteria}

The concept of the compound criteria is based on the notion of design efficiency, which can be defined for any design matrix and any criterion. For example, the $D$-efficiency of design $\bm{X}$ is
\begin{equation}
\label{eq::D_eff}
\mbox{Eff}_{D}(X)=\left[\frac{\vert \bm{X}'\bm{X}\vert}{\vert \bm{X}'_{*}\bm{X}_{*}\vert}\right]^{1/p},
\end{equation}   
where $\bm{X}_{*}$ is the $D$-optimum design. In this definition the power $1/p$ brings the efficiency to the scale of variances of model coefficients $\bm{\beta}_{i}, i=1\ldots p.$ It is obvious that the efficiency value may vary from $0$ to $1$ and is equal to $1$ if and only if the design is optimal according to the criterion of interest.

The final criterion function to be maximised among all the possible designs is obtained by combining the efficiencies for the component criteria $\mbox{Eff}_{1},\ldots, \mbox{Eff}_{m}$ with the corresponding weights $\kappa_{1},\ldots ,\kappa_{m}$ such that each $\kappa_{k}>0$ and $\sum_{k=1}^{m}\kappa_{k}=1:$

\begin{equation}
\label{eq::compound}
\mbox{Eff}^{\kappa_{1}}_{1}(\bm{X})\times\mbox{Eff}^{\kappa_{2}}_{2}(\bm{X})\times\ldots\times\mbox{Eff}^{\kappa_{m}}_{m}(\bm{X})\rightarrow \underset{\bm{X}}\max.
\end{equation}

The choice of weights is an arbitrary decision, made relying on both prior knowledge of the experimenter, the objectives of a specific experiment and the interpretation of the criterion components. The most general and intuitively sensible recommendation would be to obtain optimal designs with respect to several weight allocations, and then, after examining them, choose the one to be applied. Throughout the course of this research, when considering the examples of experiments layouts, we chose a set of weight allocations. Most of these sets were essentially classical design schemes for experiments with mixtures \citep{Cornell2011Mixtures}. For example, if the criterion consists of three elements, we would first obtain designs with all the weight on each component, then all combinations of distributing it equally between two components, and finally put equal weights on all of the components. Some additional schemes are included when examining particular cases.  

Some of the alternative approaches of considering several objectives would include the Pareto frontier approach, thoroughly described, for example, by \cite{Lu2011optimization}. When there are $l$ individual criterion functions $\phi_1(\bm{X}),\ldots\phi_{l}(\bm{X})$ to be maximised, the Pareto optimal set is generated so that for each of the designs from this set there is no other possible design that provides better criterion values for all the criteria (or, as it was defined, the Pareto optimal set is formed of designs that are not ``dominated'' by any other design).

Another approach was presented by \cite{Stallings2015general}: the authors developed methodology for generalising eigenvalue-based criteria (e.g. $A$- and $E$-optimality) in a way that allows differing interest (expressed through the weights) among any set of estimable functions of the fitted model parameters. The introduced strategy reflects the aims of experimentation that are not traditionally accounted for but definitely are of interest.

\section{Component criteria construction}
\label{ch::compound_criteria}

Standard design optimality theory is developed under the assumption that the primary model provides the most proper fit for the data: in many real applications this is quite a strong belief, and in reality we need to take into account at least the possibility that some misspecification is present at the planning stage.
%In Section \ref{sec::back_misspecification} a brief overview of various types of model misspecification is presented. 
%Other types of model misspecification (in background)

In this work we consider the case when the fitted polynomial model with $p$ parameters is nested within a larger model that is assumed to provide a better fit for the data:
\begin{equation}
\label{eq::full_model}
\bm{Y}=\bm{X}_p\bm{\beta}_p+\bm{X}_q\bm{\beta}_q+\bm{\varepsilon},
\end{equation}
where $\bm{X}_q$ is an $n\times q$ extension of the primary model matrix containing extra $q$  terms that we refer to as "potential terms" and that represent the fitted model disturbance, with vector $\bm{\beta}_q$ denoting the corresponding parameters. They are not of any inferential interest, and, moreover, not all of them are necessarily estimable when the experiment is relatively small, i.e. $n<p+q$ -- this is the case we mainly consider here. As usually, we do assume independent and normally distributed error terms: $\bm{\varepsilon}\sim \mathcal{N}(\bm{0},\sigma^{2}\bm{I}_{n})$. 

\subsection*{Lack-of-fit criterion}
 
Aiming towards controlling the magnitude and scope of the potential terms, we adapt Bayesian approach regarding the full model parameters.  Diffuse prior shall be put on primary terms -- an arbitrary mean and a variance going to infinity, and the prior on potential terms is a normal distribution: $\bm{\beta}_q\sim\mathcal{N}(0,\bm{\Sigma}_{0})$, with the prior variance scaled with respect to the error variation:
$\bm{\Sigma}_{0}=\sigma^{2}\tau^{2}\bm{I}_{q}$. Then, following the normality in the model (\ref{eq::full_model}), the posterior distribution of the joint vector of  coefficients $\bm{\beta}  = [\bm{\beta}^T_p, \bm{\beta}^T_q]^T$ is (\cite{Koch2007introduction} is multivariate normal \cite{DuMouchel1994}):
\begin{align*}
\bm{\beta}|\bm{Y} &\sim \mathcal{N}(\bm{b},\bm{\Sigma}), \\
\mbox{where } \bm{b} = \bm{\Sigma X}^T\bm{Y} &\mbox{ and }  \bm{\Sigma} = \sigma^{2}[\bm{X}^T\bm{X} + \bm{K}/\tau^{2}],^{-1}  \bm{X}=[\bm{X}_p, \bm{X}_q],\\
\bm{K} &= \begin{pmatrix}
\bm{0}_{p\times p} & \bm{0}_{p\times q}\\
\bm{0}_{q\times p} & \bm{I}_{q\times q}
\end{pmatrix}.
\end{align*}

The marginal posterior distribution of $\bm{\beta}_q$ is also multivariate normal with mean $\bm{b}_q$ -- the last $q$ elements of $\bm{b}$ and the covariance matrix $\bm{\Sigma}_{qq}$-- the bottom right $q \times q $ submatrix of $\bm{\Sigma}$:
\begin{align*}
\bm{\Sigma}_{qq} = \sigma^{2}[(\bm{X}^T\bm{X} +  \bm{K}/\tau^{2})^{-1}]_{[q,q]} &= 
\sigma^{2}\begin{bmatrix}
\bm{X}^T_p\bm{X}_p& \bm{X}^T_p\bm{X}_q \\
\bm{X}^T_q\bm{X}_p& \bm{X}^T_q\bm{X}_q+\bm{I}_{q}/\tau^{2}
\end{bmatrix}^{-1}_{[q,q]}\\&=
\sigma^{2}[\bm{X}^T_q\bm{X}_q+\bm{I}_{q}/\tau^{2}-\bm{X}^T_q\bm{X}_p(\bm{X}^T_p\bm{X}_p)^{-1}\bm{X}^T_p\bm{X}_q]^{-1}\\&=\sigma^{2}\left[\bm{L}+\bm{I}_{q}/\tau^{2}\right],^{-1}
\end{align*} 
where $\bm{L} = \bm{X}^T_q\bm{X}_q-\bm{X}^T_q\bm{X}_p(\bm{X}^T_p\bm{X}_p)^{-1}\bm{X}^T_p\bm{X}_q$ is known in model-sensitivity design literature as the ``dispersion matrix'' [references], which provides a point-wise measure of the distance between column vectors of $\bm{X}_q$ and linear subspace defined by the column vectors of the primary model matrix $\bm{X}_p$, that is how well each of the potential terms could be approximated by a linear span of the primary ones [?].

Reducing the primary model's lack-of-fit in the direction of the potential terms can be translated into a criterion function of the design by utilising the posterior distribution for $\bm{\beta}_q$ derived above and constructing a $(1-\alpha)\times100\%$ confidence region for the parameters depending on the model matrices and the variance estimate $s^2$ on $\nu$ degrees of freedom \citep{Draper1998}:
\begin{equation*}
(\bm{\beta}_{2}-\bm{b}_{2})^{'}(\bm{L}+\bm{I}_{q}/\tau^{2})(\bm{\beta}_{2}-\bm{b}_{2})\leq qs^{2}F_{q,\nu;1-\alpha},
\end{equation*}
where $\mathrm{F}_{q,\nu; \alpha}$ is the $\alpha$-quantile of F-distribution with $q$ and $\nu$ degrees of freedom.
Minimising the volume of the confidence region is equivalent to 
\begin{equation}
\label{eq::LoFDP_criterion}
\left|\bm{L}+\bm{I}_{q}/\tau^{2}\right|^{-1/q}F_{q,d;1-\alpha_{LoF}} \longrightarrow \mbox{ min, }  
\end{equation}  
which we refer to as ``Lack-of-fit DP-criterion'', that is directly related to [1] the lack-of-fit component in the Generalised D-optimality developed by \cite{Goos2005model} -- where the residual number of degrees of freedom $\nu$ does not depend on the design, and [2] DP-optimality \citep{GilmourTrinca2012}, with the F-quantile preserved from $\nu = d$ being the number of replicates in the design. 

Another way of formulating a criterion that we consider in this work -- minimising the average of posterior variances of linear functions of $\bm{\beta}_q$ defined by matrix $\bm{J}$, and we define the Lack-of-fit LP-criterion as mean of the squared lengths of the $(1-\alpha)\times100\%$ confidence intervals for these linear functions:
\begin{equation}
\label{eq::LoFLP_criterion}
\frac{1}{q}\mbox{trace}\left[\bm{JJ}^T\left(\bm{L}+\frac{\bm{I}_{q}}{\tau^{2}}\right)^{-1}\right]F_{1,d;1-\alpha}  \longrightarrow \mbox{ min. } 
\end{equation} 

This trace-based criterion is linked to the lack-of-fit part of the Generalised L-optimality \citep{Goos2005model}, and the pure error estimation approach retains the corresponding F-quantile. 
[Henceforth we mainly consider the case when $\bm{J}$ is the identity matrix, that is we work with the analogue of $AP$-optimality. In other words, the lack-of-fit component in the generalised $AP$-criterion  stands for the minimisation of the $\bm{L}_2$-norm of the $q$-dimensional vector of the posterior confidence intervals' lengths for the potential parameters. ]

\subsection*{MSE-based bias criterion}
Together with controlling the magnitude of model contamination, it is also desirable to "protect" the quality of inference that is to be drawn through fitting the primary model, from the potential presence of extra terms. 
% minimise its effect on the primary model inference, that is the quality of paramters' estimates .
From this point of view, the bias of the parameters' estimates $\hat{\bm{beta}}_p$ would be of substantial interest; a natural way of evaluating their quality is the matrix of  mean square error \citep{FedorovMontepiedra1997} , which is the $\bm{L}_2$-distance between the true and estimated values with respect to the probability distribution measure of $\bm{Y}$ under the assumption of model (\ref{eq::full_model}):
\begin{align}
\label{eq::MSE}
\mbox{MSE}(\bm{\hat{\beta}}_p|\bm{\beta})=&\mathtt{E}_{\bm{Y}|\bm{\beta}}[(\bm{\hat{\beta}}_p-\bm{\beta}_p)(\bm{\hat{\beta}}_p-\bm{\beta}_p)^T]\notag\\=&\sigma^2(\bm{X}_p^T\bm{X}_p)^{-1}+\bm{A}\bm{\beta}_q\bm{\beta}_q^T\bm{A}^T, 
\end{align}
where $\bm{A}=(\bm{X}_p^{T}\bm{X}_p)^{-1}\bm{X}_p^{T}\bm{X}_q$ denotes the $p \times q$ alias matrix, which elements reflect the linearity scale of the relationship between the primary (rows) and potential (columns) terms. 

The determinant-based criterion that would correspond to the overall simultaneous minimisation of the bias above is constructed as log-determinant of the MSE matrix averaged across [?] the values of the full model parameters $\bm{\beta}$:
\begin{equation}
\label{eq::MSE_det}
\mathtt{E}_{\bm{\beta}}\log(\det[\mbox{MSE}(\bm{\hat{\beta}}_p|\bm{\beta})]) \longrightarrow \mbox{ min.}
\end{equation}

Using the matrix determinant lemma \citep{Harville2006matrix} and setting $\bm{M}=\bm{X}_1^{'}\bm{X}_1$ and $\bm{\tilde{\beta}}_2=\bm{\beta}_2/\sigma,$ the determinant and the applied logarithm in (\ref{eq::MSE_det})) can be decomposed:
%\begin{equation*}
%\det[\bm{P}+\bm{uv}']=\det[\bm{P}]\det[1+\bm{v'}\bm{P}^{-1}\bm{u}], 
%\end{equation*}
%where $\bm{P}$ is an invertible square matrix and $\bm{u}$, $\bm{v}$ are column vectors, 
\begin{align}
\label{eq::mse_det_dec}
&\det[\mbox{MSE}(\bm{\hat{\beta}}_p|\bm{\beta}_q)]\notag\\&=\det[\sigma^2\bm{M}^{-1}+\bm{M}^{-1}\bm{X}_p^{T}\bm{X}_q\bm{\beta}_q\bm{\beta}_q^T\bm{X}_q^{T}\bm{X}_p\bm{M}^{-1}]\notag\\&=\sigma^{2p}\det[\bm{M}^{-1}+\bm{M}^{-1}\bm{X}_p^{'}\bm{X}_q\bm{\tilde{\beta}}_q\bm{\tilde{\beta}}_q^T\bm{X}_q^{T}\bm{X}_p\bm{M}^{-1}]\notag\\&=\sigma^{2p}\det[\bm{M}^{-1}]\det[1+\bm{\tilde{\beta}}_q^T\bm{X}_q^{T}\bm{X}_p\bm{M}^{-1}\bm{M}\bm{M}^{-1}\bm{X}_p^{T}\bm{X}_q\bm{\tilde{\beta}}_q]\notag\\&=\sigma^{2p}\det[\bm{M}^{-1}](1+\bm{\tilde{\beta}}_q^T\bm{X}_q^{T}\bm{X}_p\bm{M}^{-1}\bm{X}_p^{T}\bm{X}_q\bm{\tilde{\beta}}_q);\\[10pt]
&\log(\det[\mbox{MSE}(\bm{\hat{\beta}}_p|\bm{\beta}_q)])\notag\\&=p\log\sigma^2+\log(\det[\bm{M}^{-1}])+\log(1+\bm{\tilde{\beta}}_q^T\bm{X}_q^{T}\bm{X}_p\bm{M}^{-1}\bm{X}_p^{T}\bm{X}_q\bm{\tilde{\beta}}_q). 
\end{align}
The first summand does not depend on the design, so it will not be included in the criterion; the second one is the $D$-optimality criterion function. Therefore, the criterion in (\ref{eq::MSE_det}) which we refer to becomes
\begin{equation*}
\log(\det[\bm{M}^{-1}])+\mathtt{E}_{\bm{\tilde{\beta}}_q}\log(1+\bm{\tilde{\beta}}_q^T\bm{X}_q^{T}\bm{X}_p\bm{M}^{-1}\bm{X}_p^{T}\bm{X}_q\bm{\tilde{\beta}}_q) \longrightarrow min.
\end{equation*}
Due to the obvious lack of information regarding $\bm{\tilde{\beta}}_q$, the second term needs to be evaluated numerically. Expressing the prior variance of $\bm{\beta}_q$ as a scaled error variance $\bm{\beta}_q \sim \mathcal{N}(\bm{0},\tau^{2}\sigma^{2}\bm{I}_{q})$ means that $\bm{\tilde{\beta}}_q \sim \mathcal{N}(\bm{0},\tau^{2}\bm{I}_{q})$, and that, quite conveniently, its prior distribution does not depend on the unknown $\sigma.^2$ Then a primitive Monte-Carlo can be used to evaluate that term: drawing a sample of large size $N$ from the prior, and approximating the expectation above by the average across the sampled values of $\bm{\tilde{\beta}}_{q_i},$ $i=1,\dots, N$:
\begin{equation*}
\mathtt{E}_{\bm{\tilde{\beta}}_q}\log(1+\bm{\tilde{\beta}}_q^T\bm{X}_q^{T}\bm{X}_p\bm{M}^{-1}\bm{X}_p^{T}\bm{X}_q\bm{\tilde{\beta}}_q) \approx \frac{1}{N}\sum_{i=1}^{N}\log(1+\bm{\tilde{\beta}}_{q_i}^T\bm{X}_q^{T}\bm{X}_p\bm{M}^{-1}\bm{X}_p^{T}\bm{X}_q\bm{\tilde{\beta}}_{q_i}).
\end{equation*} 

One of the alternatives to this more computationally demanding approach is to use the point prior for $\bm{\beta}_q$, that is setting  $\bm{\beta}_q=\sigma\tau\bm{1}_q$ (where $\bm{1}_q$ is a $q$-dimensional vector of $1$s), which is the standard deviation  of the initial normal prior. Then $\bm{\tilde{\beta}}_q = \tau\bm{1}_q$ with probability $1$ and the expectation above becomes:
\begin{align*}
\mathtt{E}_{\bm{\tilde{\beta}}_q}\log(1+\bm{\tilde{\beta}}_q^T\bm{X}_q^{T}\bm{X}_p\bm{M}^{-1}\bm{X}_p^{T}\bm{X}_q\bm{\tilde{\beta}}_q)  &\approx \log(1+\tau^2\bm{1}^T_q\bm{X}_q^{T}\bm{X}_p\bm{M}^{-1}\bm{X}_p^{T}\bm{X}_q\bm{1}_q) \\&=\log(1+\tau^2\sum_{i,j=1}^{q}[\bm{X}_q^{T}\bm{X}_p\bm{M}^{-1}\bm{X}_p^{T}\bm{X}_q][i,j]),
\end{align*}
with summation of matrix elements taking less computational time even for large $q$. 
 
To infer the trace-based form of the criterion that would minimise the mean squared bias, we use the pseudo-Bayesian approach of setting priors on the model's parameters  to calculating the trace function of the MSE matrix (\ref{eq::MSE}): 
\begin{align*}
\mathtt{E}_{\bm{\beta}_q}\mbox{trace}[\mbox{MSE}(\bm{\hat{\beta}}_p\bm{\beta}_q)]&=\mbox{trace}[\mathtt{E}_{\bm{\beta}_q}\mbox{MSE}(\bm{\hat{\beta}}_p|\bm{\beta}_q)]=\\&=\mbox{trace}[\sigma^2(\bm{X}_p^{T}\bm{X}_p)^{-1} + \mathtt{E}_{\bm{\beta}_q}(\bm{A}\bm{\beta}_q\bm{\beta}_q^T\bm{A}^T)]=\\&=\mbox{trace}[\sigma^2(\bm{X}_p^{T}\bm{X}_p)^{-1}+\sigma^2\tau^2\bm{A}\bm{A}^T]=\\&=\sigma^2\mbox{trace}[(\bm{X}_p^{T}\bm{X}_p)^{-1}+\tau^2\bm{A}\bm{A}^T]\\&=\sigma^2[\mbox{trace}\{(\bm{X}_p^{T}\bm{X}_p)^{-1}\}+\tau^2\mbox{trace}\bm{A}\bm{A}^T].
\end{align*}

The operations of calculating trace and expectation are commutative, hence there is no necessity of any additional numerical evaluations, and in this case of the trace-based criterion using the point prior for $\bm{\beta}_q$ defined earlier would lead to the same resulting function. By minimising the whole function above, we simultaneously minimise both the sum of primary terms' variance and the expected squared norm of the bias vector in the direction of the potential terms. The second part here contains the scaling parameter $\tau^2$ which regulates the magnitude of the potential terms' variation relatively to the error variance.  
 
\subsection*{Compound criteria}
The compound criterion is constructed to account for the three main objectives -- precision (or accuracy?) of the primary model parameters, control for the lack-of-fit and minimising the inferential bias from the potential model contamination -- as the weighted product of efficiencies (\ref{eq::compound}) with respect to DP-criterion, and Lack-of-Fit DP functions(\ref{eq::LoFDP_criterion}, \ref{eq::LoFLP_criterion}) and the MSE-based ones. All of them are brought to the scale of efficiencies [?and $\exp$]:

\begin{align}
\label{eq::MSE_D}
\mbox{minimise }&\left[\left|\bm{X}^T_{p}\bm{X}_{p}\right|^{-1/p}F_{p,d;1-\alpha_{DP}}\right]^{\kappa_{DP}} \times \notag \\ &\left[\left|\bm{L}+\frac{\bm{I}_{q}}{\tau^{2}}\right|^{-1/q}F_{q,d;1-\alpha_{LoF}}\right]^{\kappa_{LoF}}\times \notag\\ & \left[|\bm{X}^T_{p}\bm{X}_{p}|^{-1}\exp\left(\mathtt{E}_{\bm{\tilde{\beta}}_q}\log(1+\bm{\tilde{\beta}}_q^T\bm{X}_q^{T}\bm{X}_p\bm{M}^{-1}\bm{X}_p^{T}\bm{X}_q\bm{\tilde{\beta}}_q) \right)\right]_{.}^{\kappa_{MSE}/p}
\end{align}
 
This criterion is later referred to as compound ``MSE-DP-criterion'', where $\alpha_{DP}$ and $\alpha_{LoF}$ denote the confidence intervals' probability levels for the primary $\bm{\beta}_p$ and potential coefficients $\bm{\beta}_q$ -- in the DP and LoF-DP elementary criteria; usually they are set to $0.05$ or $0.01$. As mentioned before, non-negative weights $\kappa_X$ that sum up to $1$ define the compound criterion and are chosen to reflect the experimenter's priorities. 

Similarly, we obtain the compound ``MSE-LP-criterion'' by joining the $LP$ criterion with trace-based Lack-of-fit (\ref{eq::LoFLP_criterion})) and MSE components :
\begin{align}
\label{eq::MSE_L}
\mbox{minimise} &\left[\frac{1}{p}\mbox{trace}(\bm{WX}^T_{p}\bm{X}_{p})^{-1}F_{1,d;1-\alpha_{LP}}\right]^{\kappa_{LP}}\times \notag\\& \left[\frac{1}{q}\mbox{trace}\left(\bm{L}+\frac{\bm{I}_{q}}{\tau^{2}}\right)^{-1}F_{1,d;1-\alpha_{LoF}}\right]^{\kappa_{LoF}}\times 
\notag\\& \left[\frac{1}{p}\mbox{trace}\{(\bm{X}^T_{p}\bm{X}_{p})^{-1}+\tau^2\bm{A}\bm{A}^T\}\right]^{\kappa_{MSE}}_{.}
\end{align}


\section{Examples}
\label{ch::compound_examples}

Here we will study the optimal designs in terms of the criteria (\ref{eq::MSE_DP}) and (\ref{eq::MSE_LP})  in the framework a factorial experiment, with $3$ factors, each is at three levels. The small number of runs ($n=16$) allows estimation of the full-second order polynomial model ($p=10$), but we assume that the extended model, potentially providing a better fit,  contains also all third-order terms (linear-by-linear-by-linear and quadratic-by-linear interactions), $q=30$ of them in total. 

The intercept is a nuisance parameter, and so the criteria are adapted in such a way that the full information matrix in the DP- and LP-components and in the first part of the MSE-based components is replaced by the one excluding the intercept $\bm{M}_0 = \bm{X}^T_{p-1}\bm{Q}_0\bm{X}_{p-1}$, where $\bm{X}_{p-1}$ is the model matrix without the intercept, $\bm{Q}_0=\bm{I}_n-\frac{1}{n}\bm{11}'$. Otherwise the procedure of obtaining the $MSE(D)$- and $MSE(L)$-based components remains the same, and the MSE-based criteria functions that have been amended according to the intercept exclusion are referred to as MSE-DPs and MSE-LPs (similar to DPs and LPs from \cite{GilmourTrinca2012}):
\begin{align*}
\mbox{MSE-DPs:} &\left[\left|\bm{X}^T_{p-1}\bm{Q}_0\bm{X}_{p-1}\right|^{-1/(p-1)}F_{p-1,d;1-\alpha_{DP}}\right]^{\kappa_{DP}} \times \notag \\ &\left[\left|\bm{L}+\frac{\bm{I}_{q}}{\tau^{2}}\right|^{-1/q}F_{q,d;1-\alpha_{LoF}}\right]^{\kappa_{LoF}}\times \notag\\ & \left[|\bm{X}'_{p-1}\bm{Q}_0\bm{X}_{p-1}|^{-1}\exp\left(\mathtt{E}_{\bm{\tilde{\beta}}_q}\log(1+\bm{\tilde{\beta}}_q^T\bm{X}_q^{T}\bm{Q}_0\bm{X}_{p-1}\bm{M}_0^{-1}\bm{X}_{p-1}^{T}\bm{Q}_0\bm{X}_q\bm{\tilde{\beta}}_q) \right)\right]_{,}^{\kappa_{MSE}/(p-1)}\\
\mbox{MSE-LPs:} &\left[\frac{1}{p-1}\mbox{trace}(\bm{WX}^T_{p-1}\bm{Q}_{0}\bm{X}_{p-1})^{-1}F_{1,d;1-\alpha_{LP}}\right]^{\kappa_{LP}}\times \notag\\& \left[\frac{1}{q}\mbox{trace}\left(\bm{L}+\frac{\bm{I}_{q}}{\tau^{2}}\right)^{-1}F_{1,d;1-\alpha_{LoF}}\right]^{\kappa_{LoF}}\times 
\notag\\& \left[\frac{1}{p-1}\mbox{trace}[\bm{M}^{-1}+\tau^2\bm{A}\bm{A}^T]_{[p-1, p-1]}\right]_{.}^{\kappa_{MSE}}
\end{align*}

In the third component of the MSE-LP-criterion $[\bm{M}^{-1}+\tau^2\bm{A}\bm{A}^T]_{[p-1, p-1]}$ stands for the submatrix corresponding to the parameters of interest, i.e.~the first column and  row removed.

Considering two values of the variance scaling parameter $\tau^2=1$ and $\tau^2=1/q$, for each compound criterion we obtain two sets of optimal designs, their summaries are given in Tables \ref{tab::MSE(D)_ex1} and \ref{tab::MSE(L)_ex1}. 
Every row corresponds to a design that has been obtained as optimal according to the compound criterion defined by the combination of weights $\kappa$-s. We explore the distribution of degrees of freedom between the pure error and lack-of-fit components in the designs and at the optimal designs' efficiencies with respect to the individual criteria that are given in the last columns. 
%In the case of the $MSE-DP$-optimal designs, in Table \ref{tab::MSE(D)P_ex1}, the difference between their $MSE(D)$-efficiency values and the $MSE(D)$-efficiencies of the corresponding designs in Table \ref{tab::MSE(D)_ex1} indicates how much we lose in terms of the performance when the point prior is used for the $MSE(D)$-component estimation. 
Optimal designs were obtained using the point exchange algorithm (\cite{Fedorov1972theory}), with $500$ random starts; the MSE(D)-part of the compound criterion was estimated using the MC sampling, and this is the most time consuming part of the computations. When this presents a considerable challenge, we can recommend the previously mentioned alternative of imputing the point prior values of $\bm{\tilde{\beta}_{q}}$. The resulting losses in the efficiencies are quite small, and time savings are substantial -- the illustration on this example is presented in Appendix \ref{appendix::example1}.

[the Results of the smaller example are in preparation]
%%% MSE(D) designs, tau^2=1 and tau^2=1/q
\begin{table}[h]
\caption{Properties of MSE-DP-optimal designs}
\label{tab::MSE(D)_ex1}
\resizebox{\textwidth}{!}}                               \\
   & \textbf{DP}       & \textbf{LoF(DP)}    & \textbf{MSE(D)}   & \textbf{PE}        & \textbf{LoF}        & \textbf{DP}   & \textbf{LoF(DP)}   & \textbf{MSE(D)}  &  \textbf{LP}       & \textbf{LoF(LP)}   & \textbf{MSE(L)}  \\
1 & 1    & 0    & 0    & \multicolumn{1}{|r}{18} & \multicolumn{1}{r|}{1}  & 100.00 & 47.77  & 91.05  & \multicolumn{1}{|r}{96.38} & 94.92 & 10.70 \\
2 & 0    & 1    & 0    & \multicolumn{1}{|r}{8}  & \multicolumn{1}{r|}{11} & 43.70  & 100.00 & 54.74  & \multicolumn{1}{|r}{0.75}  & 89.99 & 2.08  \\
3 & 0    & 0    & 1    & \multicolumn{1}{|r}{0}  & \multicolumn{1}{r|}{19} & 0.00   & 0.00   & 100.00 & \multicolumn{1}{|r}{0.00}  & 0.00  & 22.32 \\
4 & 0.5  & 0.5  & 0    & \multicolumn{1}{|r}{11} & \multicolumn{1}{r|}{8}  & 78.50  & 87.61  & 88.56  & \multicolumn{1}{|r}{73.88} & 98.85 & 17.66 \\
5 & 0.5  & 0    & 0.5  & \multicolumn{1}{|r}{15} & \multicolumn{1}{r|}{4}  & 97.26  & 56.51  & 93.77  & \multicolumn{1}{|r}{97.55} & 50.04 & 12.74 \\
6 & 0    & 0.5  & 0.5  & \multicolumn{1}{|r}{8}  & \multicolumn{1}{r|}{11} & 64.72  & 96.84  & 87.53  & \multicolumn{1}{|r}{57.04} & 36.17 & 29.33 \\
7 & 1/3  & 1/3  & 1/3  & \multicolumn{1}{|r}{10} & \multicolumn{1}{r|}{9}  & 79.45  & 84.14  & 93.23  & \multicolumn{1}{|r}{81.06} & 43.42 & 16.71 \\
8 & 0.5  & 0.25 & 0.25 & \multicolumn{1}{|r}{13} & \multicolumn{1}{r|}{6}  & 93.38  & 64.35  & 95.55  & \multicolumn{1}{|r}{95.76} & 48.58 & 14.77 \\
9 & 0.25 & 0.5  & 0.25 & \multicolumn{1}{|r}{9} & \multicolumn{1}{r|}{10} & 69.52  & 95.76  & 87.36  & \multicolumn{1}{|r}{63.13} & 40.41 & 25.46 \\
 & & & & & & & & & & & \\
   & \multicolumn{3}{l}{\textbf{Criteria, $\bm{\tau^2=1/q$}}} & \multicolumn{2}{l}{\textbf{DoF}} & \multicolumn{6}{l}{\textbf{Efficiency,\%}}                               \\
   & \textbf{DP}       & \textbf{LoF(DP)}    & \textbf{MSE(D)}   & \textbf{PE}        & \textbf{LoF}        & \textbf{DP}   & \textbf{LoF(DP)}   & \textbf{MSE(D)}  & \textbf{LP}       & \textbf{LoF(LP)}   & \textbf{MSE(L)}  \\
1 & 1    & 0    & 0    & \multicolumn{1}{|r}{18} & \multicolumn{1}{r|}{1}  & 100.00 & 94.41  & 90.60  & \multicolumn{1}{|r}{96.42} & 98.36  & 44.94 \\
2 & 0    & 1    & 0    & \multicolumn{1}{|r}{16} & \multicolumn{1}{r|}{3}  & 39.66  & 100.00 & 37.95  & \multicolumn{1}{|r}{0.13}  & 100.00 & 0.12  \\
3 & 0    & 0    & 1    & \multicolumn{1}{|r}{0}  & \multicolumn{1}{r|}{19} & 0.00   & 0.00   & 100.00 & \multicolumn{1}{|r}{0.00}  & 0.00   & 77.97 \\
4 & 0.5  & 0.5  & 0    & \multicolumn{1}{|r}{18} & \multicolumn{1}{r|}{1}  & 100.00 & 94.41  & 90.60  & \multicolumn{1}{|r}{96.42} & 98.36  & 44.94 \\
5 & 0.5  & 0    & 0.5  & \multicolumn{1}{|r}{17} & \multicolumn{1}{r|}{2}  & 97.31  & 91.28  & 93.98  & \multicolumn{1}{|r}{96.21} & 94.32  & 50.53 \\
6 & 0    & 0.5  & 0.5  & \multicolumn{1}{|r}{15} & \multicolumn{1}{r|}{4}  & 96.10  & 92.31  & 93.29  & \multicolumn{1}{|r}{99.48} & 95.09  & 57.05 \\
7 & 1/3  & 1/3  & 1/3  & \multicolumn{1}{|r}{18} & \multicolumn{1}{r|}{1}  & 100.00 & 94.41  & 90.60  & \multicolumn{1}{|r}{96.42} & 98.36  & 44.94 \\
8 & 0.5  & 0.25 & 0.25 & \multicolumn{1}{|r}{18} & \multicolumn{1}{r|}{1}  & 100.00 & 94.41  & 90.60  & \multicolumn{1}{|r}{96.42} & 98.36  & 44.94 \\
9 & 0.25 & 0.5  & 0.25 & \multicolumn{1}{|r}{18} & \multicolumn{1}{r|}{1}  & 99.96  & 94.33  & 90.65  & \multicolumn{1}{|r}{96.24} & 98.31  & 44.84 
\end{tabular}
}
\end{table}

%The resulting designs tend to have a lot of the available degrees of freedom allocated to the pure error, except, for example, for the $MSE(D)$-optimal designs (\#$3$). However, for the smaller value of $\tau^2=1/q$ the imbalance is more extreme for the determinant-based criteria, where almost no degrees of freedom are left for lack of fit.
%
%There are no repeated designs for $\tau^2=1$, but for $\tau^2=1/q$, in Table \ref{tab::MSE(D)_ex1}, designs \#$4$, \#$7$ and \#$8$ are the same as the $DP$-optimal design, and have quite large values of other efficiencies, and design \#$9$, although it is different, has quite similar efficiency values. 
%
%
%As for the trace-based criterion and the optimal designs studied in Table \ref{tab::MSE(L)_ex1}, in general, all of them tend to have larger $LP$- and $MSE(L)$-efficiencies in case of smaller $\tau^2$, i.e. the decreased scale of the potentially missed contamination results in a more easily achievable compromise between the contradicting parts of the criteria (the same happens with the trace-based efficiencies of the MSE determinant-based optimal designs). It is also notable that, as it was observed in the case of generalised criteria, the $LoF-LP$-optimal design is also $LoF-DP$-optimal for $\tau^2=1/q$ (design \#$2$ in the corresponding tables).
%
%The $MSE(L)$-component seems to be much more sensitive to the weight allocations than the $MSE(D)$ component. For example, in the case of $\tau^2=1$ some decent efficiency values are gained only when the whole weight is on the `potential terms' criterion components, i.e.~designs \#$3$ and \#$6$. 
%
%It does not seem to work agreeably with the $LoF-LP$-component either: the $LoF-LP$-optimal design is $0.00\%$ $LP$- and $MSE(L)$-efficient in the case of $\tau^2=1$, and the efficiencies are close to $0\%$ in the case of smaller $\tau^2$. 
%% MSE(L) criteria
\begin{table}[h]
\caption{Properties of MSE(L)-optimal designs}
\label{tab::MSE(L)_ex1}
\resizebox{\textwidth}{!}}                               \\
   & \textbf{LP}       & \textbf{LoF(LP)}    & \textbf{MSE(L)}   & \textbf{PE}        & \textbf{LoF}        & \textbf{DP}   & \textbf{LoF(DP)}   & \textbf{MSE(D)}  &  \textbf{LP}       & \textbf{LoF(LP)}   & \textbf{MSE(L)}  \\
1 & 1    & 0    & 0    & \multicolumn{1}{|r}{16} & \multicolumn{1}{r|}{3} & 97.54 & 53.86 & 92.54 & \multicolumn{1}{|r}{100.00} & 96.87  & 12.08  \\
2 & 0    & 1    & 0    & \multicolumn{1}{|r}{13} & \multicolumn{1}{r|}{6} & 35.43 & 81.99 & 36.72 & \multicolumn{1}{|r}{0.00}   & 100.00 & 0.00   \\
3 & 0    & 0    & 1    & \multicolumn{1}{|r}{4} & \multicolumn{1}{r|}{15} & 18.67 & 38.43 & 51.84 & \multicolumn{1}{|r}{11.73}  & 34.79  & 100.00 \\
4 & 0.5  & 0.5  & 0    & \multicolumn{1}{|r}{15} & \multicolumn{1}{r|}{4} & 95.14 & 60.37 & 92.78 & \multicolumn{1}{|r}{99.80}  & 98.12  & 13.99  \\
5 & 0.5  & 0    & 0.5  & \multicolumn{1}{|r}{12} & \multicolumn{1}{r|}{7} & 77.77 & 72.05 & 84.91 & \multicolumn{1}{|r}{81.10}  & 98.72  & 25.19  \\
6 & 0    & 0.5  & 0.5  & \multicolumn{1}{|r}{9} & \multicolumn{1}{r|}{10} & 36.80 & 70.91 & 51.12 & \multicolumn{1}{|r}{28.13}  & 91.60  & 83.52  \\
7 & 1/3  & 1/3  & 1/3  & \multicolumn{1}{|r}{11} & \multicolumn{1}{r|}{8} & 69.53 & 73.16 & 79.71 & \multicolumn{1}{|r}{70.59}  & 97.47  & 27.98  \\
8 & 0.5  & 0.25 & 0.25 & \multicolumn{1}{|r}{12} & \multicolumn{1}{r|}{7} & 77.20 & 72.83 & 84.44 & \multicolumn{1}{|r}{81.47}  & 98.80  & 23.88  \\
9 & 0.25 & 0.5  & 0.25 & \multicolumn{1}{|r}{12} & \multicolumn{1}{r|}{7} & 70.90 & 69.80 & 78.15 & \multicolumn{1}{|r}{72.16}  & 98.49  & 26.19  \\
 & & & & & & & & & & & \\
   & \multicolumn{3}{l}{\textbf{Criteria, $\bm{\tau^2=1/q$}}} & \multicolumn{2}{l}{\textbf{DoF}} & \multicolumn{6}{l}{\textbf{Efficiency,\%}}                               \\
   & \textbf{LP}       & \textbf{LoF(LP)}    & \textbf{MSE(L)}   & \textbf{PE}        & \textbf{LoF}        & \textbf{DP}   & \textbf{LoF(DP)}   & \textbf{MSE(D)}  & \textbf{LP}       & \textbf{LoF(LP)}   & \textbf{MSE(L)}  \\
1 & 1    & 0    & 0    & \multicolumn{1}{|r}{16} & \multicolumn{1}{r|}{3} & 97.54 & 92.13 & 92.18 & \multicolumn{1}{|r}{100.00} & 95.61  & 52.55  \\
2 & 0    & 1    & 0    & \multicolumn{1}{|r}{16} & \multicolumn{1}{r|}{3} & 39.66 & 100.00 & 37.95 & \multicolumn{1}{|r}{0.13}   & 100.00 & 0.12   \\
3 & 0    & 0    & 1    & \multicolumn{1}{|r}{3} & \multicolumn{1}{r|}{16} & 0.77  & 0.93  & 83.48 & \multicolumn{1}{|r}{0.02}   & 0.01   & 100.00 \\
4 & 0.5  & 0.5  & 0    & \multicolumn{1}{|r}{17} & \multicolumn{1}{r|}{2} & 96.87 & 94.29 & 89.84 & \multicolumn{1}{|r}{97.97}  & 97.80  & 51.17  \\
5 & 0.5  & 0    & 0.5  & \multicolumn{1}{|r}{12} & \multicolumn{1}{r|}{7} & 79.66 & 87.18 & 86.32 & \multicolumn{1}{|r}{84.81}  & 88.22  & 79.60  \\
6 & 0    & 0.5  & 0.5  & \multicolumn{1}{|r}{13} & \multicolumn{1}{r|}{6} & 76.46 & 89.87 & 80.78 & \multicolumn{1}{|r}{79.23}  & 91.48  & 79.11  \\
7 & 1/3  & 1/3  & 1/3  & \multicolumn{1}{|r}{13} & \multicolumn{1}{r|}{6} & 81.53 & 90.09 & 85.52 & \multicolumn{1}{|r}{85.96}  & 91.56  & 76.65  \\
8 & 0.5  & 0.25 & 0.25 & \multicolumn{1}{|r}{15} & \multicolumn{1}{r|}{4} & 90.60 & 91.94 & 88.43 & \multicolumn{1}{|r}{95.09}  & 94.75  & 63.82  \\
9 & 0.25 & 0.5  & 0.25 & \multicolumn{1}{|r}{15} & \multicolumn{1}{r|}{4} & 84.12 & 92.88 & 83.39 & \multicolumn{1}{|r}{87.86}  & 95.51  & 72.66 
\end{tabular}
}
\end{table}

%Regarding the designs' performances with respect to the $DP$- and $LP$-components, it can be observed that the designs tend to be quite $DP$-efficient, overall more efficient in the case of $MSE(D)$-efficient designs with smaller $\tau^2$. $DP$-efficient designs are not bad in terms of $LP$-efficiency and vice versa; again, the same cannot be said for the lack-of-fit components and seems not to be true at all for the $MSE$ components, especially, for the $MSE(L)$-optimal design when $\tau^2=1$.
%   
%$LP$- and $MSE(L)$-components seem to be in a conflict, though for $\tau^2=1/q$ $MSE(L)$-efficiency values are stable across the designs, therefore, making it possible to find compromise designs which at least perform not too badly with regard to both of these criterion parts; for example, designs \#$5$ -- \#$7$ are more than $75\%$-efficient (Table \ref{tab::MSE(L)_ex1}).   

The observed relationships between the components of the criteria provide some ideas on how the designs obtained as optimal with respect to the various allocations of weights differ, what they have in common, and how the components interact when the weights are reallocated. A compromise can be found between the criterion components corresponding to the properties of the primary terms and the components responsible for the reducing the negative impact from the assumed potential model misspecification. Thereby, the careful choice of the prior parameters ($\tau^2$ in this case) seems to be of a certain importance as well. 

\section{Criteria for blocked experiments}
\label{ch::compound_blocked}

In some experiments, where the number of runs is relatively large, and/or the variability between units is quite high, experimental units are allocated in blocks such that within each block the units are similar in some way, e.g.~the runs carried out within the same day. Such formalisation contributes to 
controlling the variability by separating variation coming from the difference between blocks and the variability arising from within the blocks \citep{Bailey2008design}.

Under the assumption of additivity of $b$ fixed block effects the polynomial model can be presented as:
\begin{equation}
\label{eq::blocked_model}
\bm{Y}=\bm{Z\beta}_B+\bm{X\beta}+\bm{\varepsilon},
\end{equation}
which, in addition to the same terms as in (\ref{eq::primary_model}) comprises $\bm{Z}$ -- the $n\times b$ matrix, such that its $(i,j)^{th}$ element is equal to $1$ if unit $i$ is in block $j$ and to $0$ otherwise, and $\bm{\beta}_B$ is the vector of block effects. 

The full information matrix has the form
$%\begin{equation*}
\bm{M_B}=
\begin{pmatrix}
\bm{Z}'\bm{Z} & \bm{Z}'\bm{X}\\
\bm{X}'\bm{Z} & \bm{X}'\bm{X}
\end{pmatrix}.     
$%\end{equation*} 

Using the rules of inverting blocked matrices (\citealp{Harville2006matrix}), we can isolate the variance of the polynomial coefficients' estimates:
\begin{align*}
\mbox{Var}(\hat{\bm{\beta}})&=\sigma^2(\bm{M_B}^{-1})_{22}=\sigma^2(\bm{X}'\bm{QX}),^{-1}\\
\mbox{where }&\bm{Q}=\bm{I}-\bm{Z}(\bm{Z}'\bm{Z})^{-1}\bm{Z}'.
\end{align*}
The $DP$- and $LP$-criteria become $DP_S$ and $LP_S$ in the context of a blocked experiment, and they can be straightforwardly defined as
\begin{align}
\label{eq::DPs_blocked}
DP_S: &(F_{p,d_B;1-\alpha_{DP}})^{p}\vert (\bm{X}'\bm{Q}\bm{X})^{-1}\vert \rightarrow \mbox{min,}\\
\label{eq::LPs_blocked}
LP_S: &F_{1,d_B;1-\alpha_{LP}}\mbox{trace}\{\bm{W}(\bm{X}'\bm{Q}\bm{X})^{-1}\} \rightarrow \mbox{min.}.
\end{align}

The number of pure error degrees of freedom is now calculated as $d_B=n-\mbox{rank}[\bm{Z}:\bm{T}]$, where $\bm{T}$ is the $n\times t$ matrix whose elements indicate the treatments (\citealp{GilmourTrinca2012}), providing the number of replications after subtracting the ones taken for the estimation of block contrasts.  

To adapt the derivation of the lack-of-fit and MSE-based criteria to the blocked experiments, we start by formulating the model comprising both primary terms and possible contamination in the form of potential terms, now for blocked experiments: 
\begin{align*}
\bm{Y}=\bm{Z\beta}_{B}+\bm{X}_{p}\bm{\beta}_{p}+\bm{X}_{q}\bm{\beta}_{q}+\bm{\varepsilon}.
\end{align*}
Denote the $n\times(b+p)$ model matrix of the block and primary terms by $\tilde{\bm{X}}_{p}=[\bm{Z},\bm{X}_{p}]$ and let $\bm{\tilde{\beta}}_p=[\bm{\beta}_{B},\bm{\beta}_{p}]'$ be the joint vector of fixed block effects and primary model terms, and by $\bm{\hat{\tilde{\beta}}}_p$ we denote the vector of the corresponding estimates. It is worth noting that the number of primary terms $p$ does not include the intercept, as it is aliased with the block effects.

\subsection{Lack-of-fit criteria}
The information matrix for the model (\ref{eq::blocked_model}), up to a multiple of $\sigma^2$, is:
\begin{align*}
\bm{M_B}=
\begin{pmatrix}
\bm{Z}^T\bm{Z} & \bm{Z}^T\bm{X}_{p} & \bm{Z}^T\bm{X}_{q}\\
\bm{X}^T_{p}\bm{Z} & \bm{X}^T_{p}\bm{X}_{p} & \bm{X}^T_{p}\bm{X}_{q}\\
\bm{X}^T_{q}\bm{Z} & \bm{X}^T_{q}\bm{X}_{p} & \bm{X}^T_{q}\bm{X}_{q}+\bm{I}_{q}/\tau^2
\end{pmatrix}
=
\begin{pmatrix}
\bm{\tilde{X}}^T_{p}\bm{\tilde{X}}_{p} & \bm{\tilde{X}}^T_{p}\bm{X}_{q}\\
\bm{X}^T_{q}\bm{\tilde{X}}_{p} & \bm{X}^T_{q}\bm{X}_{q}+\bm{I}_{q}/\tau^2
\end{pmatrix}_{.}
\end{align*} 

Assuming the same normal prior on $\bm{\beta}_q$ $\sim \mathcal{N}(\bm{0}, \tau^2\sigma^2\bm{I}_q)$, we can construct the variance-covariance matrix corresponding to the potential terms, which would be the inverse of the lower right submatrix of $\bm{M_B}$: $\bm{\tilde{\Sigma}}_{qq}=\sigma^2[\bm{M}^{-1}_B]_{22}$.
\begin{align*}
\bm{\tilde{\Sigma}}_{qq}&=\sigma^2([\bm{M_B}]_{22}-[\bm{M_B}]_{21}([\bm{M_B}]_{11})^{-1}[\bm{M_B}]_{12})^{-1}\\
&=\sigma^2(\bm{X}^T_{q}\bm{X}_{q}+\bm{I}_{q}/\tau^2-\bm{X}^T_{q}\bm{\tilde{X}}_{p}(\bm{\tilde{X}}^T_{p}\bm{\tilde{X}}_{p})^{-1}\bm{\tilde{X}}^T_{p}\bm{X}_{q})^{-1}\\&=\sigma^2\left(\bm{\tilde{L}}+\frac{\bm{I}_{q}}{\tau^2}\right), \mbox{ where }\bm{\tilde{L}}=\bm{X}^T_{q}\bm{X}_{q}-\bm{X}^T_{q}\bm{\tilde{X}}_{p}(\bm{\tilde{X}}^T_{p}\bm{\tilde{X}}_{p})^{-1}\bm{\tilde{X}}^T_{p}\bm{X}_{q}.
\end{align*}

Therefore, the lack-of-fit criteria in (\ref{eq::LoFDP_criterion}) and (\ref{eq::LoFLP_criterion}) are adjusted for blocked experiments by replacing the primary terms matrix $\bm{X}_{p}$ by the extended matrix $\bm{\tilde{X}}_{p}$ and the dispersion matrix $\bm{L}$ -- by $\bm{\tilde{L}}$ as obtained above.

\subsection{MSE-based criteria}

As for the MSE-based measure of the shift in the primary terms estimates, we first consider the overall mean square matrix  
\begin{align}
\label{eq::MSE_b}
\mbox{MSE}(\bm{\hat{\tilde{\beta}}}_p|\bm{\tilde{\beta}})=&\mathtt{E}_{\bm{Y}|\bm{\beta}}[(\bm{\hat{\tilde{\beta}}}_p-\bm{\tilde{\beta}}_p)(\bm{\hat{\tilde{\beta}}}_p-\bm{\tilde{\beta}}_p)'] \notag\\=&\sigma^2(\bm{\tilde{X}}_p^{'}\bm{\tilde{X}}_p)^{-1}+\bm{\tilde{A}}\bm{\beta}_q\bm{\beta}_q^T\bm{\tilde{A}}^T, 
\end{align}
with $\bm{\tilde{A}}=(\bm{\tilde{X}}_p^T\bm{\tilde{X}}_p)^{-1}\bm{\tilde{X}}_p^{'T}\bm{X}_q$ being the alias matrix, 
and its partition with respect to block and primary effects:
\begin{align*}
%\label{eq::mse_b_matrix}
&\mathtt{E}_{\bm{Y}|\bm{\beta}}[(\bm{\hat{\tilde{\beta}}}_p-\bm{\tilde{\beta}}_p)(\bm{\hat{\tilde{\beta}}}_p-\bm{\tilde{\beta}}_p)^T]= \notag\\ &\mathtt{E}_{\bm{Y}|\bm{\beta}}\{[\hat{\tilde{\beta}}_{p1}-\tilde{\beta}_{p1},\ldots,
\hat{\tilde{\beta}}_{pb}-\tilde{\beta}_{pb}, \hat{\tilde{\beta}}_{p b+1}-\tilde{\beta}_{p b+1},\ldots, \hat{\tilde{\beta}}_{p b+p}-\tilde{\beta}_{p b+p}]\times \notag\\ &[\hat{\tilde{\beta}}_{p 1}-\tilde{\beta}_{p1},\ldots,
\hat{\tilde{\beta}}_{pb}-\tilde{\beta}_{pb}, \hat{\tilde{\beta}}_{p b+1}-\tilde{\beta}_{p b+1},\ldots, \hat{\tilde{\beta}}_{p b+p}-\tilde{\beta}_{p b+p}]^T\}=\notag\\
&\mathtt{E}_{\bm{Y}|\bm{\beta}}\{[\bm{\hat{\beta}}_B-\bm{\beta}_B, \bm{\hat{\beta}}_p-\bm{\beta}_p][\bm{\hat{\beta}}_B-\bm{\beta}_B, \bm{\hat{\beta}}_p-\bm{\beta}_p]^T\}=\notag\\
&\begin{bmatrix}
\mathtt{E}_{\bm{Y}|\bm{\beta}}(\bm{\hat{\beta}}_B-\bm{\beta}_B)(\bm{\hat{\beta}}_B-\bm{\beta}_B)^T & \mathtt{E}_{\bm{Y}|\bm{\beta}}(\bm{\hat{\beta}}_B-\bm{\beta}_B)(\bm{\hat{\beta}}_p-\bm{\beta}_p)^T\\
\mathtt{E}_{\bm{Y}|\bm{\beta}}(\bm{\hat{\beta}}_p-\bm{\beta}_p)(\bm{\hat{\beta}}_B-\bm{\beta}_B)^T & \mathtt{E}_{\bm{Y}|\bm{\beta}}(\bm{\hat{\beta}}_p-\bm{\beta}_p)(\bm{\hat{\beta}}_p-\bm{\beta}_p)^T
\end{bmatrix}.
\end{align*}

The part  corresponding to the bias of the primary terms $\bm{\beta}_p$ is the lower right $p \times p$ submatrix, and we can extract it from the MSE expression in (\ref{eq::MSE_b}). The respective submatrix of the first summand  is
\begin{equation*}
[\sigma^2(\bm{\tilde{X}}_p^{T}\bm{\tilde{X}}_p)^{-1}]_{22}=\sigma^2(\bm{X}^T_{p}
\bm{QX}_{p})^{-1}, \mbox { where } \bm{Q}=\bm{I}-\bm{Z}(\bm{Z}^T\bm{Z})^{-1}\bm{Z}^T_{.} 
\end{equation*}

Using the matrix inversion rule for block matrices \citep{Harville2006matrix}, we now consider $\bm{\tilde{A}}$:
\begin{align*}
\bm{\tilde{A}}=&\left(
\begin{bmatrix}
\bm{Z}^T\\
\bm{X}^T_{p}
\end{bmatrix}
\begin{bmatrix}
\bm{Z} & \bm{X}_{p}
\end{bmatrix}\right)^{-1}
\begin{bmatrix}
\bm{Z}^T\\
\bm{X}^T_{p}
\end{bmatrix}\bm{X}_{q}=
\begin{pmatrix}
\bm{Z}^T\bm{Z} & \bm{Z}^T\bm{X}_{p}\\
\bm{X}^T_{p}\bm{Z} & \bm{X}^T_{p}\bm{X}_{p}
\end{pmatrix}^{-1}
\begin{bmatrix}
\bm{Z}^T\\
\bm{X}^T_{p}
\end{bmatrix}\bm{X}_{q}=
\\ &
\begin{bmatrix}
(\bm{Z}^T\bm{PZ})^{-1} & -(\bm{Z}^T\bm{PZ})^{-1}\bm{Z}^T\bm{X}_{p}(\bm{X}^T_{p}\bm{X}_{p})^{-1} \\
-(\bm{X}^T_{p}\bm{QX}_{1})^{-1}\bm{X}^T_{p}\bm{Z}(\bm{Z}^T\bm{Z})^{-1} & (\bm{X}^T_{1}\bm{QX}_{p})^{-1}
\end{bmatrix}
\begin{bmatrix}
\bm{Z}^T\\
\bm{X}^T_{p}
\end{bmatrix}\bm{X}_{q}=\\ &
\begin{bmatrix}
(\bm{Z}^T\bm{PZ})^{-1}\bm{Z}^T\bm{PX}_{q}\\
(\bm{X}^T_{p}\bm{QX}_{1})^{-1}\bm{X}^T_{p}\bm{QX}_{q}
\end{bmatrix}, 
\end{align*}
where $\bm{P}=\bm{I}-\bm{X}_{p}(\bm{X}^T_{p}\bm{X}_{p})^{-1}\bm{X}^T_{p}$; $\bm{ZZ}'$, $\bm{X}'_{1}\bm{X}_{1}$ and $\bm{Z}'\bm{PZ}$ are all invertible and, therefore, the operations are legitimate. Now consider the second summand in (\ref{eq::MSE_b}):
\begin{align*}
&\bm{\tilde{A}}\bm{\beta}_q\bm{\beta}_q^T\bm{\tilde{A}}^T=
\begin{bmatrix}
(\bm{Z}^T\bm{PZ})^{-1}\bm{Z}^T\bm{PX}_{q}\bm{\beta}_q \\
(\bm{X}^T_{p}\bm{QX}_{p})^{-1}\bm{X}^T_{p}\bm{QX}_{q}\bm{\beta}_q
\end{bmatrix}
\begin{bmatrix}
\bm{\beta}^T_q\bm{X}^T_{2}\bm{PZ}(\bm{Z}^T\bm{PZ})^{-1} & \bm{\beta}^T_q\bm{X}^T_{q}\bm{QX}_{p}(\bm{X}^T_{p}\bm{QX}_{p})^{-1}
\end{bmatrix}=\\
&\begin{bmatrix}
(\bm{Z}^T\bm{PZ})^{-1}\bm{Z}^T\bm{PX}_{q}\bm{\beta}_q\bm{\beta}^T_q\bm{X}^T_{q}\bm{PZ}(\bm{Z}^T\bm{PZ})^{-1} & (\bm{Z}^T\bm{PZ})^{-1}\bm{Z}^T\bm{PX}_{q}\bm{\beta}_q\bm{\beta}^T_q\bm{X}^T_{q}\bm{QX}_{p}(\bm{X}^T_{p}\bm{QX}_{p})^{-1} \\
(\bm{X}^T_{p}\bm{QX}_{p})^{-1}\bm{X}^T_{p}\bm{QX}_{q}\bm{\beta}_q\bm{\beta}^T_q\bm{X}^T_{q}\bm{PZ}(\bm{Z}^T\bm{PZ})^{-1} & (\bm{X}^T_{p}\bm{QX}_{p})^{-1}\bm{X}^T_{p}\bm{QX}_{q}\bm{\beta}_q\bm{\beta}^T_q\bm{X}^T_{q}\bm{QX}_{p}(\bm{X}^T_{p}\bm{QX}_{p})^{-1}
\end{bmatrix}.
\end{align*}

Then the submatrix of (\ref{eq::MSE_b}) corresponding to the primary terms is
\begin{align}
\label{eq::mesb_submatrix}
\mbox{MSE}(\bm{\hat{\tilde{\beta}}}_p|\bm{\tilde{\beta}})_{pp}=& \sigma^2(\bm{X}^T_{p}\bm{QX}_{p})^{-1}+(\bm{X}^T_{p}\bm{QX}_{p})^{-1}\bm{X}^T_{p}\bm{QX}_{q}\bm{\beta}_q\bm{\beta}^T_q\bm{X}^T_{q}\bm{QX}_{p}(\bm{X}^T_{p}\bm{QX}_{p})^{-1}\notag\\=& \sigma^2\bm{\tilde{M}}^{-1}+\bm{\tilde{M}}^{-1}\bm{X}^T_{p}\bm{QX}_{q}\bm{\beta}_q\bm{\beta}^T_q\bm{X}^T_{q}\bm{QX}_{p}\bm{\tilde{M}},^{-1}
\end{align}
where $\bm{\tilde{M}}=\bm{X}^T_{p}\bm{QX}_{p}.$

As in the unblocked case, we first look at the determinant of the corresponding submatrix (\ref{eq::mesb_submatrix}): 
\begin{align}
\label{eq::MSE_B_det}
\det[\mbox{MSE}(\bm{\hat{\tilde{\beta}}}_p|\bm{\tilde{\beta}})_{pp}]=&\det[\sigma^2\bm{\tilde{M}}^{-1}+\bm{\tilde{M}}^{-1}\bm{X}^T_{p}\bm{QX}_{q}\bm{\beta}_q\bm{\beta}^T_q\bm{X}^T_{q}\bm{QX}_{p}\bm{\tilde{M}}^{-1}]=\notag\\& \sigma^{2p}\det[\bm{\tilde{M}}^{-1}+\bm{\tilde{M}}^{-1}\bm{X}^T_{p}\bm{QX}_{q}\bm{\tilde{\beta}}_q\bm{\tilde{\beta}}^T_q\bm{X}^T_{q}\bm{QX}_{p}\bm{\tilde{M}}^{-1}]=\notag\\& \sigma^{2p}\det[\bm{\tilde{M}}^{-1}](1+\bm{\tilde{\beta}}^T_q\bm{X}^T_{q}\bm{QX}^T_{p}\bm{\tilde{M}}^{-1}\bm{X}^T_{p}\bm{QX}_{q}\bm{\tilde{\beta}}_q).
\end{align}

The $q$-dimensional random vector $\bm{\tilde{\beta}}_q$, as before, follows $\mathcal{N}(\bm{0},\tau^{2}\bm{I}_{q})$, so that this prior does not depend on the error variance $\sigma_{.}^2$. Next, taking the expectation of the logarithm of (\ref{eq::MSE_B_det}) over the set prior distribution is completely identical to the derivations leading to (\ref{eq::MSE_D}). The $MSE(D)$-component then becomes:
\begin{equation}
\label{eq::mse_b_component}
\log(\det[\bm{\tilde{M}}^{-1}])+\mathtt{E}_{\bm{\tilde{\beta}}_2}\log(1+\bm{\tilde{\beta}}_2'\bm{X}_2^{'}\bm{QX}_1\bm{\tilde{M}}^{-1}\bm{X}_1^{'}\bm{QX}_2\bm{\tilde{\beta}}_2)
\end{equation}

and the resulting determinant-based compound criterion for a blocked experiments is
\begin{align}
\label{eq::MSE_D_B}
&\left[\left|(\bm{X}^T_{p}\bm{Q}\bm{X}_{p})^{-1}\right|^{1/p}F_{p,d_B;1-\alpha_{DP}}\right]^{\kappa_{DP}} \times \notag \\ &\left[\left|\bm{\tilde{L}}+\frac{\bm{I}_{q}}{\tau^{2}}\right|^{-1/q}F_{q,d_B;1-\alpha_{LoF}}\right]^{\kappa_{LoF}}\times \\ & \left[|\bm{X}^T_{p}\bm{QX}_{p}|^{-1}\exp\left(\frac{1}{N}\sum_{i=1}^{N}\log(1+\bm{\tilde{\beta}}_{2i}'\bm{X}_q^{T}\bm{QX}_p\bm{\tilde{M}}^{-1}\bm{X}_p^{T}\bm{QX}_q\bm{\tilde{\beta}}_{2i})\right)\right]^{\kappa_{MSE}/p} \longrightarrow \mbox{ min.}\notag
\end{align}

Taking the expectation of the trace of (\ref{eq::mesb_submatrix}) makes up the trace-based component criterion:
\begin{align}
\label{eq::MSE_B_tr}
\mathtt{E}_{\beta_q}\mbox{trace}[\mbox{MSE}(\bm{\hat{\tilde{\beta}}}_p|\bm{\tilde{\beta}})_{pp}]&= \mbox{trace}[\mathtt{E}_{\beta_q}\mbox{MSE}(\bm{\hat{\tilde{\beta}}}_p|\bm{\tilde{\beta}})_{pp}]\notag \\&=\mbox{trace}[\sigma^2\bm{\tilde{M}}^{-1}_{pp}+\mathtt{E}_{\beta_q}(\bm{\tilde{A}}\bm{\beta}_q\bm{\beta}_q^T\bm{\tilde{A}})_{pp}]\notag\\&=  \sigma^2\mbox{trace}[\bm{\tilde{M}}^{-1}_{pp}+\tau^2\{\bm{\tilde{A}}\bm{\tilde{A}}^T\}_{pp}]\notag \\&=\sigma^2[\mbox{trace}(\bm{X}^T_{p}\bm{QX}_{p})^{-1}+\tau^2\mbox{trace}\{\bm{\tilde{A}}\bm{\tilde{A}}^T\}_{pp}].
\end{align}

The $MSE(LP)$-compound criterion for a blocked experiment is
\begin{align}
\label{eq::MSE_L_B}
&\left[\frac{1}{p}\mbox{trace}(\bm{WX}^T_{p}\bm{Q}\bm{X}_{p})^{-1}F_{1,d_B;1-\alpha_{LP}}\right]^{\kappa_{LP}}\times \notag \\&\left[\frac{1}{q}\mbox{trace}\left(\bm{\tilde{L}}+\bm{I}_{q}/\tau^{2}\right)^{-1}F_{1,d_B;1-\alpha_{LoF}}\right]^{\kappa_{LoF}}\times \notag \\& \left[\frac{1}{p}\mbox{trace}\{(\bm{X}^T_{p}\bm{QX}_{p})^{-1}+\tau^2[\bm{\tilde{A}}\bm{\tilde{A}}^T]_{pp}\}\right]^{\kappa_{MSE}} \longrightarrow \mbox{ min.}
\end{align}

The probability levels $\alpha$, weights $\kappa$ hold the same meanings as in the unblocked case; and as was noted before, the number of pure error degrees of freedom $d_B$ accounts for the comparisons between blocks.

\subsection{Example}
[A usual example or the case-study?]
\section{Future work}
\subsection{Multistratum experiments}
In a large number of industrial, engineering and laboratory-based experiments, either the nature of the process under study or certain technical restrictions result in the necessity of considering a multi-level unit structure. For example, a chemical process consisting of applying treatments to the material batches of different sizes at each stage;  or one or several experimental factors' values can be changed only once per a certain amount of runs whereas values of other factors are varied between runs. Therefore, different factors are applied at different levels, and randomisation is performed at each level, thus the whole process of allocating treatments to experimental units should be amended accordingly \citep{MeadGilmour2012}.

Experiments comprising a hierarchical structure of experimental units and treatments are referred to as multistratum experiments, and each stratum is defined as a level in the unit structure. Units are grouped into whole-plots, each of them divided into sub-plots, which contain a certain number of sub-sub-plots, and so on up to the smallest units --- runs of the experiment. The corresponding factors are often called ``very-hard-to-change'', ``hard-to-change'', ``easy-to-change'', ``very-easy-to-change'' factors; these labels are quite arbitrary and certainly determined by a particular experimental framework. In the case of two strata, we deal with what is called a ``split-plot'' experiment, in case of three strata a ``split-split-plot'' experiment. The names come from the terminology of agricultural experiments where such setups were originally implemented. \cite{Nelder1965analysis} introduced the notion of simple orthogonal block structures, comprising `chains' of nested and/or crossed experimental units together with the corresponding factors. Sometimes the ``split'' notation is used exceptionally for orthogonal block design structures, however, we employ it for general nested unit structures where the orthogonality property does not necessarily hold.
%\cite{Speed1982class}

Some visualisation tools allowing a comprehensive presentation of the outline of such experiments can be found in \cite{Goos2012Hasse}: Hasse diagrams, that were described extensively and in great detail by \cite{Bailey2008design}. Hasse diagrams are graphs containing the information about the experimental units, factors applied at every stratum, together with a scheme of the available degrees of freedom allocation, so that they appear to be a much useful tool providing a full and clear picture of a complicated structured experiment. 

A generalised example of a multistratum experiment with a simple nested unit structure is presented in Figure \ref{Fig::Hasse}. Each node of the graph corresponds to a blocking factor and edges are drawn from one factor to another that is nested within it. The ``universe'' node U stands for the whole experiment.

From here onwards we will be considering experiments with $s$ strata in total, stratum $i$ being nested within units of the stratum $i-1$, and stratum $0$ will be seen as the whole experiment (following the notation set by \citet{Trinca2015improved}). The number of units in stratum $i$ within every unit in stratum $i-1$ is denoted by $n_i$, such that $m_{j}=\prod_{i=1}^{j}n_{i}$ is the number of units in stratum $j$ and, therefore,  $n=m_{s}=\prod_{i=1}^{s}n_{i}$ is the total number of runs.

\begin{figure}[h]
\begin{center}
\includegraphics[scale=0.85]{Hasse.jpg}      %width=\textwidth
\caption{Hasse diagram for factors}
\label{Fig::Hasse}
\end{center}
\end{figure} 
 
Recall the expression for the linear polynomial hierarchical model from (\ref{eq::intro_ms}):
\begin{equation}
\label{eq::back_ms}
\bm{Y}=\bm{X}\bm{\beta}+\sum_{i=1}^{s}\bm{Z}_{i}\bm{\varepsilon}_{i},
\end{equation} 
where the first part $\bm{X\beta}$ contains the fixed effects, and $\sum_{i=1}^{s}\bm{Z}_{i}\bm{\varepsilon}_{i}$ stands for the random part, i.e.~it comprises information on variation occurring at each level of randomisation. All of the errors are assumed to be independent, having zero means and constant variances $\sigma^2_{i}$ and, as seen from the model formulation, additive.

In this case the fixed effects coefficients, $\bm{\beta}$ are usually estimated using the Generalised Least Square (GLS) formula:
\begin{equation}
\label{eq::back_gls}
\bm{\hat{\beta}}=(\bm{X}'\bm{VX})^{-1}\bm{X}'\bm{V}^{-1}Y,
\end{equation}
so that the variance of the estimators is
\begin{equation}
\label{eq::back_glsvar}
Var(\bm{\hat{\beta}})=(\bm{X}'\bm{VX})^{-1},
\end{equation}
where $\bm{V}$ is the variance-covariance matrix of the (normally distributed) responses, has a block-diagonal structure, and can be presented as:
\begin{equation}
\label{eq::back_glsV}
\bm{V}=\sum_{i=1}^{s}\sigma^2_{i}\bm{Z}_{i}\bm{Z}'_{i}=\sigma^{2}_{s}\left(\bm{I}_{n}+\sum_{i=1}^{s-1}\eta_{i}\bm{Z}_{i}\bm{Z}'_{i}\right).
\end{equation}
Here $\eta_{i}=\sigma^{2}_{i}/\sigma^{2}_{s}$ is defined as a variance ratio, denoting the magnitude of variability occurring at higher strata scaled with respect to the between-run variance.

At the analysis stage, as the variance components are unknown, their estimates are obtained and substituted in (\ref{eq::back_glsvar}) to obtain an estimated variance-covariance matrix of the parameters' estimators, $\bm{\hat{V}(\hat{\beta})}$; the estimation methods are discussed in more detail in Chapter \ref{ch::mse_ms}.

However, when an experiment is being planned, and none of the $\eta_{i}$ or $\sigma^2_{s}$ are known, there are two main approaches to deal with it. The first one is to search for designs assuming some point prior values of the $\eta$ parameter and, for these, evaluate the variance-based criteria, considering all strata simultaneously.

The case of split-plot experiments, with two strata, $\sigma^2_{s}=\sigma^2$ and $\eta=\sigma^2_{1}/\sigma^2$, has been extensively considered in the literature. \cite{Goos2001Doptimal} considered the three cases when D-optimal designs for split-plot experiments do not depend on the value of $\eta$; in other practical cases an estimate of the variance ratio is to be provided. \cite{Goos2003Doptimal} and \cite{Jones2007candidate} developed algorithms for finding $D$-optimal split-plot designs and for some examples demonstrated the robustness to the different values of $\eta$. An algorithm for constructing $D$-optimal split-split-plot designs can be found in the paper by \cite{Jones2009Doptimal}. 
%\cite{Goos2007tailor} -- split-plot for mixture experiments  

The Bayesian alternative to the `exhaustive search' approach across the unknown parameter space allows specifying a prior distribution on the unknown parameters, and then evaluating the criteria by integrating over that prior. \cite{Arnouts2012staggered} presented a coordinate-exchange algorithm for constructing D-optimal designs for a staggered experimental structure (i.e. values of hard-to-change factors' are changed at different time points), with log-normal prior distributions being put on the variance ratios. Later \cite{Arnouts2015staggered} studied the staggered structured designs in the context of response surface modelling, and considered a few examples of D- and I-optimal designs.
 
\cite{Mylona2014optimal} introduced a composite criterion, combining $D$-optimality for the fixed and variance components; in such an approach both the criterion formula and the form of the prior distributions define the way of evaluation the criterion function, which often leads to the necessity of choosing a computationally efficient numerical methodology. \cite{Gilmour2009analysis} proposed an appropriate Bayesian analysis strategy of data from multistratum experiments, that is robust to designs non-orthogonality and is shown to be more reliable than REML approach.  

However, in the general case a lot of computational effort might be required in order to be confident in the goodness of the design obtained by choosing from a set of assumed possible values of $\eta_i$, either using point priors or adapting general Bayesian strategy, especially when there are more than two strata and, therefore, the range of unknown parameters becomes multi-dimensional. The stratum-by-stratum approach (as first suggested by \cite{Trinca2001multistratum} and more recently improved by \cite{Trinca2015improved}) implies going from the highest stratum to the lowest, at each level choosing the set of treatments, and the units of higher level being treated as fixed block effects. Such methodology allows `protecting' against the case of large higher level variances, and eliminates the need for any prior information on unknown parameters. The authors later adapted the stratum-by-stratum  strategy for the inference criteria \citep{Trinca2016SPinference}, providing tools for constructing efficient designs for relatively small experiments in the presence of restricted randomisation.
\section{Acknowledgements}

%%% Bibliography
\cleardoublepage
\phantomsection
\addcontentsline{toc}{chapter}{Bibliography} 
%\bibliographystyle{apalike}
\bibliographystyle{rss}
\bibliography{thesis_bib}

\end{document}